\documentclass[a4paper,12pt]{article}

%this enables IPA characters for linguistics
%\usepackage{tipa}

%this changes the way citations look and how they are organized in the bibliography
%\usepackage[authoryear, round]{natbib}
%\bibliographystyle{apa-good-ampersand}

\usepackage[numbers]{natbib}
\bibliographystyle{unsrtnat}

%enables the use of images
\usepackage{graphicx}

%enables hyperlinks
\usepackage[colorlinks=false]{hyperref}

%enables some special characters, needed for writing Scandinavian
\usepackage[utf8]{inputenc}

%fixes the margin
\usepackage[margin=2.5cm]{geometry}

%removes page number which PDF readers have anyway.
\pagestyle{empty}

%removes ligatures
\usepackage{microtype}
\DisableLigatures{encoding = *, family = * }

%for making tables stay in place, use [H]
\usepackage{float}

\begin{document}
\noindent
Submitted to \textit{Open Differential Psychology} March 24th, 2014\\
Published in \textit{Open Differential Psychology} DATE, YEAR\\
\\
\begin{center}
\textbf{{\Huge The global hereditarian hypothesis and the National Longitudinal Survey of Freshman}}
\end{center}
\noindent\Large John Fuerst\footnote{Corresponding author: j122177@hotmail.com}\\

\noindent\Large Emil O. W. Kirkegaard\\
\\
\normalsize\textbf{Abstract}\\
We discuss the global hereditarian hypothesis of race differences in \textit{g} and test it on data from the NLSF. We find that migrants country of origin's IQ predicts GPA and SAT/ACT.

\noindent\textbf{Keywords}: National IQs, race differences, country of origin, NLSF

\section{Introduction}
In terms of race differences in \textit{g} no other pair of races is as well studied as the 1.1 \textit{d} IQ gap between African Americans and European  Americans\cite{roth2001ethnic}. For this reason, most discussion about the causes of race differences has centered on this gap\cite{rushton2005thirty}. Arthur Jensen has advanced the hypothesis that the gap is likely due to involve genetic factors, and this has been called "the hereditarian hypothesis"\cite{jensen1969much,jenseneducability,jensen1998g,rushton2005thirty}.

Since the publication of Richard Lynn and Tatu Vanhanen's 2002 book \textit{IQ and the wealth of nations} and follow-up books the discussion of racial differences has moved to the global perspective\cite{lynn2002iq,lynn2006iq,lynn2006race,lynn2008global,lynn2012intelligence}. While there was initially much hostility towards the data, even researchers previously critical of the idea has themselves done research on the data\cite{hunt2006sorry,hunt2012makes} and others have called it a "productive research paradigm"\cite{rindermann2013intelligence}. Lynn and Vanhanen have also advanced the hypothesis that not just the African American European American \textit{IQ} gap, but more generally the gaps between different races living in various countries are due in part to genetic factors. We call this "the global hereditarian hypothesis".

Much research since 2002 has centered on the country level correlations between national IQs, scholastic achievement scores, economic, health, governmental, biological, environmental variables and theories about their origin (see overview of findings in \cite{lynn2012intelligence}). We think it is time to approach the topic from another angle, that of immigrant populations in other countries. The idea is that if the global hereditarian hypothesis is true to a significant degree, for instance 50\% of the differences in national IQs are due to genetic factors, then persons who travel to other countries should be similar in their IQ to their home country, everything else equal (e.g. no selective migration and IQ-environment interaction effects). Since IQ is a known predictor and cause of many social and economic variables at the personal level, it should also be a predictor at the group level inside host countries. We call this hypothesis "the spatial transferability hypothesis". Many countries now have large and growing immigrant population which make it possible to test this hypothesis.

Other researchers have already found evidence for the hypothesis. Jones and Schneider have shown that one can predict wages among immigrant groups in the U. S. by their country of origin\cite{jones2010iq} and Vinogradov and Kolvereid have shown that one can predict self-employment in Norway among immigrant groups\cite{vinogradov2010home}.

One of us (JF) has previously shown that national IQs highly correlate with GMAT (Graduate Management Admission Test), GRE (Graduate Record Examinations), and TOEFL (Test of English as a Foreign Language) scores.\cite{JFuerst1}. As these tests have often been shown to be predictive of student performance irrespective of national citizenship\cite{talento-miller2009validity}, it is implied that national IQs predict, to some extent, migrant academic performance. JF further showed that migrant PISA scores are predictable from their country of origin's IQ and PISA score and not from the destination country\cite{JFuerst2}. Results are summarized in Table \ref{JFresults}.

\begin{table}[H]
\centering
\begin{tabular}{|l|l|}
\hline
\textbf{Variable (N)}    & \textbf{Correlation with national IQ} \\ \hline
GMAT (143)               & .720                                  \\ \hline
TOEFL (157)              & .625                                  \\ \hline
GRE-50N (173)            & .740                                  \\ \hline
GRE-M (144)              & .764                                  \\ \hline
GRE-R (144)              & .552                                  \\ \hline
GRE-T (144)              & .765                                  \\ \hline
PISA (60)		 & .589                                  \\ \hline
\end{tabular}
\caption{Correlations of national IQs with scholastic tests and English abilities.}\label{JFresults}
\end{table}

The other of us (EOWK) previously tested the hypothesis with data from Denmark and Norway\cite{kirkegaard2013DK,kirkegaard2014DK,kirkegaard2014NO}. In the first study, EOWK found a military study from 2005 which reported raw score data from testing of immigrants in 2003. EOWK then found information about the immigrant population's composition by country of origin from the official statistics agency. Then, EOWK used Lynn and Vanhanen's national IQs with the composition data to estimate the mean score of the immigrant population. The estimate came very close to the observed score from the army study, 86.3 was found while 86.7 was estimated -- a discrepancy of a mere .4 IQ.

In the second and third studies EOWK found data for criminality, fertility and country of origin from Denmark and criminality, employment rate and citizenship in Norway. EOWK then gathered information about the countries of origin in an attempt to predict these rates: IQs, GDP, Islam belief, height, murder rate, fertility. The data from Denmark and Norway were highly consistent as shown in Table \ref{kirkegaardresults}. Furthermore, the correlation between the crime rate in Denmark and Norway for the countries of origin in both samples is very high about .7-.8 (N=20).

In this paper we further test the hypothesis by examining data from The National Longitudinal Survey of Freshmen.

\begin{table}[H]
\centering
\begin{tabular}{|l|l|l|l|}
\hline
\textbf{Predictor} & \textbf{Variable} & \textbf{Denmark (N=71)} & \textbf{Norway (N=21)} \\ \hline
Islam              & Crime             & .593 to .787            & .695 to .805           \\ \hline
IQ                 & Crime             & -.467 to -.653          & -.620                  \\ \hline
GDP                & Crime             & -.287 to -.414          & -.449 to -.512         \\ \hline
Height             & Crime             & -.036 to -.218          & -.287 to -.300         \\ \hline
Murder rate        & Crime             & .058 to .242            & .059 to .101           \\ \hline
Islam              & Fertility         & .671                    &                        \\ \hline
IQ                 & Fertility         & -.514                   &                        \\ \hline
GDP                & Fertility         & -.316                   &                        \\ \hline
Fertility          & Fertility         & .504                    &                        \\ \hline
Islam              & Employment rate   &                         & -.764                  \\ \hline
IQ                 & Employment rate   &                         & .507                   \\ \hline
GDP                & Employment rate   &                         & .598                   \\ \hline
\end{tabular}
\caption{Predictors and socio-economic variables in Denmark and Norway.}\label{kirkegaardresults}
\end{table}

\section{Methods}
The National Longitudinal Survey of Freshman is described as follows\cite{NLSF}:

\begin{quote}
The National Longitudinal Survey of Freshmen (NLSF) follows a cohort of first-time freshman at selective colleges and universities through their college careers. Equal numbers of whites, blacks, Hispanics, and Asians were sampled at each of the 28 participating schools.
\end{quote}

As this sample is not representative of migrants to the U.S. and as relevant variables were self reported or interviewee assessed, this is not an ideal sample for testing our hypothesis. Nonetheless, these problematic factors will tend to attenuate any true association and, thus, make for a more robust test of the ST hypothesis.

A detailed description of variables is provided in the supplementary material.

\section{Analysis 1}
We inspected scores by nativity status and U.S. defined race. This analysis yielded few interesting results. First and second generation self defined Blacks, Hispanics, Whites, and Asians performed similar to their respective third generation racial peers; the same pattern of score differences as found in the U.S. population as a whole was found in this selective college sample; the relationship between cumulative GPA and tests scores did not vary by nativity within each racial group, though it did vary significantly between racial groupings. Results are shown in Table \ref{analysis1a} and \ref{analysis1b}.

\begin{table}[H]
\includegraphics[width=\linewidth]{analysis1a}
\caption{Results from analysis 1a.}\label{analysis1a}
\end{table}

\begin{table}[H]
\centering
\includegraphics[width=300px]{analysis1b}
\caption{Results from analysis 1b.}\label{analysis1b}
\end{table}

\section{Analysis 2}
We next looked at the association between three measures of national cognitive ability, a measure of national skin reflectance, combined first and second generation test scores, and combined first and second generation skin color ratings. With regards to migrants, we computed test scores, GPA, and skin color means separately by biological mother's and biological father's nation of origin; we then n-weight averaged the (mother and father) nation of origin means. In the vast majority of instances, both parents hailed from the same country; when not, though, we effectively split their representation. Since we were concerned with migrant scores, the U.S. was not included as a migrant sending country. When computing correlations, we weighted scores by the number of migrants per country. The correlations are shown below. All were statistically significant at the .05 level. Results are shown in Table \ref{analysis2a}. All three measures of national cognitive ability were significantly and similarly correlated with migrant test performance. The correlation between migrant skin color and national skin reflectance was .891, implying that our migrants were relatively ethnically representative of their nation of origin populations. Since our national cognitive measures were similarly predictive, for the remainder of the discussion, we simply report result based on L\&V's (2012) national IQs.

\begin{table}[H]
\centering
\includegraphics[width=350px]{analysis2a}
\caption{Results from analysis 2a.}\label{analysis2a}
\end{table}

We further looked at the association between L\&V's (2012) national IQs, migrant first and second generation test scores, and migrant second generation GPA scores. The values were weighted by the migrant sample sizes. The correlation results are shown in Table \ref{analysis2b}. The national IQ x test score association did not vary much by generation. Also, L\&V's (2012) National IQs predicted migrant national group GPA about as well as migrant test scores did on the individual level and half as well as migrant test scores did on the national level.

\begin{table}[H]
\centering
\includegraphics[width=350px]{analysis2b}
\caption{Results from analysis 2b.}\label{analysis2b}
\end{table}

\section{Analysis 3}
We looked to see if the association between national cognitive measures and second generation migrant GPA scores was mediated by second generation migrant test scores. The results for migrant GPA (dependent), L\&V's (2012) national IQ (independent), and migrant test scores (covariant) are shown below in Table \ref{analysis3a}. The national cognitive ability x GPA associations were significantly mediated by migrant test scores.

\begin{table}[H]
\centering
\includegraphics[width=350px]{analysis3a}
\caption{Results from analysis 3a.}\label{analysis3a}
\end{table}

Similarly, we checked if the association between migrant skin color and migrant test scores was mediated by national cognitive measures. (We also included national skin reflectance values as a covariate.) This was found to be the case for first, second, and the combined generation samples. The results for the combined sample are shown below in Table \ref{analysis3b}.

\begin{table}[H]
\centering
\includegraphics[width=350px]{analysis3b}
\caption{Results from analysis 3b.}\label{analysis3b}
\end{table}

\section{Analysis 4}
As a robustness check, we reran analysis 2 after taking into account parental migrant selectivity. To compute selectivity we took the difference between the parents' standardized mean educational levels as reported in the NLSF survey and the standardized average schooling years for the origin countries. For the country of origin values, we used age 25-29 data for year 1980, as this would have been the approximate cohort which birthed the NLSF students; the data came from Barro-Lee's dataset. The results are shown in Table \ref{analysis4a}. Controlling for migrant selectivity substantially increased the \textit{r} (national cognitive measures x migrant test score correlations).

\begin{table}[H]
\centering
\includegraphics[width=350px]{analysis4a}
\caption{Results from analysis 4a.}\label{analysis4a}
\end{table}

Taking into account selection also increased the National IQ x GPA association from  \textit{r}(806) = .266, \textit{p} $ < $ .01 to a first-order partial correlation of r(668) = .441, \textit{p} $ < $ .01.  Another way to approach this matter is to control for, instead of selectivity, parental educational levels.  When doing so, the National IQ x Test Score and National IQ x GPA associations, respectively \textit{r}(803) = .612, \textit{p} $ < $ .01 and \textit{r}(803) = .351, \textit{p} $ < $ .01 are strong and significant.

\section{General discussion and conclusion}
We have found more evidence in support of the spatial transferability hypothesis (ST) which posits the first and second generation migrants carry their nation and region of origin IQs with them and that these IQs have predictive value.  In the sample analyzed, National IQs predicted migrant tests scores and GPA; the association does not seem to be explainable in terms of migrant unrepresentativeness with regards to ethnicity or human capital. This finding is consistent with previous ones. Further analyses need to be conducted to see if the ST hypothesis holds up in more representative samples and in other countries.

As we found that migrant selectivity substantially moderated the association between national cognitive ability and migrant cognitive ability, future analyses should attempt to take into account migrant selectivity. We demonstrated one method by which this could be done.

We further note that NLSF is a selected sample for cognitive ability and so there is restriction of range. We did not correct for this because we did not know how strong the restriction was. This restriction of range attenuates the correlations reported here.

\section{Acknowledgments}
We thank Gerhard Meisenberg for providing multiple datasets.

\section{Detailed methods and data}
Detailed methods and data can be found in the supplementary material.

\bibliography{refs}
\end{document}
