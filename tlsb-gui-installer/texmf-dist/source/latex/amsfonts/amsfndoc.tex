%% filename: amsfndoc.tex
%% version: 2.2d
%% date: 2002/01/19
%%
%% American Mathematical Society
%% Technical Support
%% Publications Technical Group
%% 201 Charles Street
%% Providence, RI 02904
%% USA
%% tel: (401) 455-4080
%%      (800) 321-4267 (USA and Canada only)
%% fax: (401) 331-3842
%% email: tech-support@ams.org
%% 
%% Copyright 1994, 2001, 2009 American Mathematical Society.
%% 
%% Unlimited copying and redistribution of this file are permitted as
%% long as this file is not modified.  Modifications, and distribution
%% of modified versions, are permitted, but only if the resulting file
%% is renamed.
%%
%% ====================================================================
\input amsfndoc.def
\extrafonts

\maintitle User's Guide to AMSFonts Version 2.2d\\
  \fontsreleasedate<

Standard distributions of \TeX{} ordinarily come with all the fonts
specified in \filename{plain.tex}, and they may also come with a number of
additional fonts intended for use with \LaTeX.  Additional fonts
designed for use in mathematics and defined in \AmSTeX{} are not always
included among such font collections.  For this reason, the \AMS{} has
compiled a collection, known as AMSFonts, which contains fonts of
symbols and several alphabets corresponding to symbols and alphabets
used in AMS publications, including electronic journals and the MathSci
online database.

%    Get toc from .aux file
\Contents

%%%%%%%%%%%%%%%%%%%%%%%%%%%%%%%%%%%%%%%%%%%%%%%%%%%%%%%%%%%%%%%%%%%%%%%%

\section{1} Contents of the AMSFonts collection

The AMSFonts collection contains the \MF{} sources, \fn{.tfm} files and
bitmaps for the following fonts, in the sizes indicated:

\begingroup
\raggedright
\parskip=0pt
\item{\bull} The Euler family, all but \fontname{euex} in 5, 6, 7, 8, 9,
  and 10 point:
\itemitem{--} Fraktur (German), medium-weight and bold (\fontname{eufm}
  and \fontname{eufb})
\itemitem{--} ``roman'' cursive, medium-weight and bold (\fontname{eurm}
  and \fontname{eurb})
\itemitem{--} script, medium-weight and bold (\fontname{eusm} and
  \fontname{eusb})
\itemitem{--} Euler-compatible extension font (\fontname{euex}),
  in 7, 8, 9, and 10 point

\item{\bull} Additional sizes of some Computer Modern math fonts
  (the 10-point fonts are included in standard \TeX\ distributions):
\itemitem{--} bold math italic (\fontname{cmmib}), in 5, 6, 7, 8, and 9 point
\itemitem{--} bold math symbols (\fontname{cmbsy}), in 5, 6, 7, 8, and 9 point
\itemitem{--} math extension font (\fontname{cmex}), in 7, 8, and 9 point

\goodbreak
\item{\bull} Extra math symbols, in 5, 6, 7, 8, 9, and 10 point:
\itemitem{--} first series, medium-weight (\fontname{msam})
\itemitem{--} second series, including Blackboard Bold, medium-weight
  (\fontname{msbm})

\item{\bull} Cyrillic, developed at the University of Washington
\itemitem{--} lightface (\fontname{wncyr}), in 5, 6, 7, 8, 9, and 10 point
\itemitem{--} bold (\fontname{wncyb}), in 5, 6, 7, 8, 9, and 10 point
\itemitem{--} italic (\fontname{wncyi}), in 5, 6, 7, 8, 9, and 10 point
\itemitem{--} caps and small caps (\fontname{wncysc}), in 10 point
\itemitem{--} sans serif (\fontname{wncyss}), in 8, 9, and 10 point
\itemitem{--} virtual font property list (\filename{.vpl})
              files to enable the use of these fonts with alternate
              encodings and transliteration schemes

\item{\bull} Computer Modern caps and small caps (\fontname{cmcsc}),
  in 8 and 9 point
  (the 10-point font is included in standard \TeX\ distributions)

\item{\bull} The ``dummy font'', used in \AmSTeX{} for syntax checking,
  which consists only of metrics (\filename{dummy.tfm}), no character shapes

\item{\bull} Packages for using these fonts with \LaTeX{}:
\itemitem{--} \filename{amsfonts}, a package for using the fonts
  \fontname{msam}, \fontname{msbm}, and \fontname{eufm} in \LaTeX{}
\itemitem{--} \filename{amssymb}, a package for defining en masse
  (instead of selectively) command names for all the math symbols
  in the fonts \fontname{msam} and \fontname{msbm}
\itemitem{--} \filename{eucal}, a package for using the Euler script font
  \fontname{eusm}
%\itemitem{--} \filename{eurom}, a package for using the Euler cursive
%  (roman) font \fontname{eurm}
\itemitem{--} \filename{cmmib57}, a package for adapting the font
  definitions for \fn{cmmib} and \fn{cmbsy} to scale from sizes 5,7,10
  instead of sizes 5,6,7,8,9,10 when the Y\&Y/Blue Sky PostScript outline
  fonts are used
\itemitem{--} \filename{*.fd}, font definition files
\item{} For details, see Section 3 on using AMSFonts with \LaTeX{}.

\item{\bull} Macro files for using these fonts with plain \TeX{}:
\itemitem{--} \filename{amssym.tex}, a file defining names for the symbols
  in fonts \fontname{msam} and \fontname{msbm}
\itemitem{--} \filename{amssym.def}, a file that loads the fonts
  \fontname{msam}, \fontname{msbm} and \fontname{eufm} and defines some
  control sequences required by \filename{amssym.tex}
\itemitem{--} \filename{cyracc.def}, a file containing definitions needed for
  proper access to characters in the cyrillic fonts

\item{\bull} Documentation files:
\itemitem{--} \filename{amsfndoc.tex}, the source file for this User's Guide
\itemitem{--} \filename{amsfndoc.cyr}, the source file for the table showing
  cyrillic input conventions, input by \filename{amsfndoc.tex}
\itemitem{--} \filename{amsfndoc.fnt}, the source file for the tables of
  the principal 10-point fonts in the AMSFonts collection, input by
  \filename{amsfndoc.tex}; this file may also be \TeX{}ed by itself
\itemitem{--} \filename{amsfndoc.def}, the macros used to format this
  User's Guide
\itemitem{--} \filename{amsfndoc.ins}, the source file for the appendixes
   to this User's Guide, input by
  \filename{amsfndoc.tex}; this file may also be \TeX{}ed by itself
\itemitem{--} \filename{amsfonts.faq}, a file of Frequently Asked
  Questions about AMSFonts
\itemitem{--} \filename{amsfonts.bug}, a chronicle of bug fixes and list
  of known bugs still outstanding
\endgraf
\endgroup       % end \raggedright

\noindent
Each font at a particular size is provided in seven standard \TeX{}
magnifications, magsteps 0~through~5, including magstephalf (except the
{\it Textures}\slash Macintosh package, which includes
only magnifications 0 and 1; see the installation instructions).

The philosophy under which the Euler fonts were implemented was different
from that used for Computer Modern, and the result is a lower degree of
``meta-ness''.  For that reason, the appearance of these fonts is not
very good at small sizes when output on low-resolution devices, in
particular on screens.  Even so, the fonts are included in AMSFonts in all
the sizes and magnifications offered, on the assumption that the printed
output will be prepared on a device of higher resolution (at least 300dpi)
where this effect will not be noticeable.

Font charts are given in \fontchartapp.


\subsection{1.1} Font naming conventions

Developers of fonts for use with \TeX, at least those using \MF,
generally try to make the names distinctive, so that a user will know
the origin of the font by the font name.  For most of the fonts in the
AMSFonts collection, the first two letters identify the font source,
as follows:

\begingroup
\raggedright
\parskip=0pt
\item{\bull} ``{\tt cm}'': These fonts are based directly on the
  specifications for Knuth's Computer Modern fonts, as described
  in Volume~E of \CandT{} [DEK86E].
\item{\bull} ``{\tt eu}'': These are members of the Euler family,
  described below.
\item{\bull} ``{\tt ms}'': These fonts of math symbols were developed
  by or under the direction of the AMS staff.
\item{\bull} ``{\tt wn}'': These cyrillic fonts were developed at or
  under the direction of the University of Washington Humanities and
  Arts Computing Center, and are distributed with their permission.
\endgraf
\endgroup

\noindent
The font called \fn{dummy} is a special case; it was developed
as part of the Stanford University \TeX{} Project, and follows no
particular naming convention.

For information on Computer Modern fonts other than those specifically named
here, and on other fonts in general, the newsgroup {\tt comp.text.tex}
is a good source.  For some other suggestions, see \furtherinfo.


%%%%%%%%%%%%%%%%%%%%%%%%%%%%%%%%%%%%%%%%%%%%%%%%%%%%%%%%%%%%%%%%%%%%%%%%

\section{2} History of these fonts

When the AMS began using \TeX{} to produce its publications, the available
complement of symbols was found to be inadequate.  Several alphabets
used extensively as symbols were not available either.  While
development of the symbols could be undertaken by in-house personnel,
using the existing \TeX{} symbol font as a model, the creation of new
Fraktur and script alphabets required the assistance of someone with
experience in font design.


\subsection{2.1} Euler

With Donald Knuth's assistance and encouragement, Hermann Zapf, one of the
premier font designers of this century, was commissioned to create designs
for Fraktur and script, and for a somewhat experimental, upright cursive
alphabet that would represent a mathematician's handwriting on a blackboard
and that could be used in place of italic.  The designs that resulted were
named Euler, in honor of Leonhard Euler, a prominent mathematician of the
eighteenth century.  Zapf's designs were rendered in \MF{} code by graduate
students at Stanford, working under Knuth's direction; the process by which
the \MF{} fonts were implemented is described in a report by David Siegel
[DRS85].  The Euler fonts were designed to be used as math symbols; they are
not intended for setting running text.

The Fraktur face of the Euler family has been used in production by the
AMS since it became available.  However, no extensive test or use had
been made of the script or cursive until Knuth decided that they should be
used in a textbook, {\sl Concrete Mathematics}, written by him and two
co-authors [GKP88].  During the course of preparing that book, a number of
errors, particularly in spacing parameters affecting the placement of sub-
and superscripts, were discovered.  All these errors have been corrected in
the medium-weight versions of the Euler fonts (almost no boldface symbols
were used in {\sl Concrete Mathematics\/}).  Knuth also noticed that the
style of some symbols in the Computer Modern extension font, in particular
the integral sign, was too slanted to be attractive with Euler, and
consequently he prepared a new (partial) extension font for use with Euler.
Knuth described his experience with the Euler fonts in a \TUB\/ article
[DEK89].  In the article he also identified the macros he used and where they
can be obtained.

The Euler fonts are sparsely populated; only the alphabetic locations are
filled in most instances (see the font charts in \fontchartapp{} for
specifics).  For this reason, when processing the file for this User's
Guide, and in particular the font charts, warnings about ``Missing
characters'' are not a cause for concern.


\subsection{2.2} Additional Computer Modern fonts for use in math

Only the 10-point size of the Computer Modern bold math italic (which
includes Greek), symbol, and math extension fonts are included in standard
distributions of \TeX{}.  Since these symbols are often needed in
mathematics, other sizes have been constructed, using the principles
demonstrated in Knuth's {\sl Computer Modern Typefaces\/} [DEK86E], and
included in the AMSFonts collection.


\subsection{2.3} Symbols

Two fonts of ``extra'' symbols are included in the AMSFonts collection.
These are named \fontname{msam} and \fontname{msbm}, and have been
implemented in ``new'' \MF{} (\mf84); they replace earlier fonts (named
\fontname{msxm} and \fontname{msym}) that were defined in old \MF{}
(\mf79)\null.  These fonts contain symbols needed in the publishing program
of the AMS, including the MathSci online database, and include the
uppercase letters of an alphabet known as Blackboard Bold
($\Bbb A, \dots, \Bbb Z$).


\subsection{2.4} Cyrillic

Titles of books reviewed in \MR\/ are traditionally rendered in their
original language.  For books published in Russian or other Slavic
languages, this frequently requires use of the cyrillic alphabet.
A cyrillic font was developed at AMS using \MF\/79 with the \fontname{am}
fonts as a model.  This font was organized in a manner suitable for use with
the transliteration scheme adopted by {\sl MR\/} in 1980, and contained only
those letters which appear in current mathematical literature.
In particular, this meant that the letters dropped from the Russian
alphabet after the Revolution of 1917, and some letters used in non-Slavic
languages now rendered in cyrillic (such as Azerbaijani, from which no
mathematical literature is currently reviewed in {\sl MR\/}) were absent.

In 1988, the Humanities and Arts Computing Center of the University of
Washington undertook a font development project for support of scholars
in Slavic languages.  The fonts developed through this project include
several different font layouts.  One layout is based on that of the
original AMS cyrillic augmented with `{\cyr \u\i}' (cyrillic short `i'),
`{\cyr\"e}' (umlauted `e'), and several pre-Revolutionary letters.
The fonts with the AMS layout are included in the AMSFonts collection
with the permission of the University of Washington developers.
%For information on cyrillic fonts with other layouts, see \furtherinfo.
``Virtual property list'' (\fn{.vpl}) files are also included in the
collection to support several other layouts; these can be found in the
\fn{cyr-alt} area, and are accomanied by a \fn{README} file with
further information.

The cyrillic fonts are based on Computer Modern letter shapes.
Type styles include ordinary upright, bold (based on CM bold
extended), caps and small caps, italic, and upright sans serif.
The principal text fonts (upright, italic and boldface) are present in
sizes from 5 through 10 point;
sans serif is in sizes 8, 9 and 10 point;
caps and small caps are in 10 point only.
See also \typeiapp{} for information on a PostScript (Type~1) version
of the cyrillic fonts, in particular concerning the sizes provided in
that format.


\subsection{2.5} Caps/small caps

The font \fontname{cmcsc10} is referenced in \filename{plain.tex} and
should be included in all standard \TeX{} distributions.  However, Knuth
did not generate this font in any other sizes.  The AMSFonts collection
includes 8 and 9-point sizes, generated according to the same principles as
other CM fonts of these sizes.


\subsection{2.6} Dummy font

The dummy font contains no ligature or kerning information, and all
dimensions and parameter values are set to zero. This is a pseudo-font,
which has only general font metrics and no characters. No \filename{.pk}
or \filename{.gf} files are needed for this font; it is provided only in
\filename{.tfm} and \filename{.mf} form.

The dummy font is used in \AmSTeX{} to implement ``syntax checking''.
(Syntax checking is activated by the |\printoptions| command as
described in \JoT{} [MDS86].) In this mode, the dummy font replaces all
the usual ``printing'' fonts, so that \TeX{} never accumulates any text
to be set, and never tries to write out a page, but in the process of
reading the input file, checks all control sequences for syntactic
correctness. In this mode, an input file will be processed perhaps 30
percent faster than if it were actually being set. However, some errors
and conditions are not detected during a syntax check; in particular,
overfull boxes cannot be detected until setting actually occurs.


\subsection{2.6} PostScript (Type 1) implementation of AMSFonts

The AMSFonts are available in Type~1 outline form as an alternative
to the bitmap images generated by \MF.  The Type~1 files are not
part of the regular AMSFonts distribution, but can be obtained at
no charge from the AMS server.  For details, see \typeiapp.


%%%%%%%%%%%%%%%%%%%%%%%%%%%%%%%%%%%%%%%%%%%%%%%%%%%%%%%%%%%%%%%%%%%%%%%%

\section{3} How to use AMSFonts with \LaTeX{}

\subsection{3.1} General

These instructions are for current \LaTeX{} (version 2$\varepsilon$, dated
January 1995 or later). If you have version 2.09 of \LaTeX{}, dated 1993 or
earlier, you must upgrade to current \LaTeX{}.  Earlier versions of \LaTeX{}
are no longer supported for AMSFonts; however, the \LaTeX{} compatibility
mode will usually allow documents prepared for version 2.09 to be processed
with the current version.

To use the AMSFonts collection with \LaTeX{} you choose from an
assortment of \LaTeX{} ``packages'' that provide various kinds of access
to the fonts, calling the packages that you need in a given document
through standard \LaTeX{} \cn{usepackage} statements. For example, the
statement
\begintt
\usepackage{amsfonts}
\endtt
calls in the \pkg{amsfonts} package, which provides blackboard bold
and Fraktur letters and selective access to the math symbol fonts
\fn{msam} and \fn{msbm}.

These packages are currently available:

\begingroup
\raggedright
\parskip=0pt
\item{\bull} {\tt amsfonts} -- for blackboard bold letters, Fraktur letters,
  and miscellaneous symbols
\item{\bull} {\tt amssymb} -- superset of the \pkg{amsfonts} package, defines
  the full set of symbol names for the \fn{msam} and \fn{msbm} fonts
\item{\bull} {\tt eufrak} -- for Fraktur letters; redundant if \fn{amsfonts}
  is used
\item{\bull} {\tt eucal} -- Makes \cn{mathcal} use Euler script instead of
  the usual Computer Modern calligraphic alphabet
\item{\bull} {\tt euscript} -- old name of the \pkg{eucal} package, now
  obsolete but included for convenience in printing pre-existing documents
%\item{\bull} {\tt eurom} -- for Euler cursive (roman) letters
\endgraf
\endgroup

All of the above packages have a `\opt{psamsfonts}' option that should
be used if and only if your copy of the AMSFonts collection is the
Y\&Y/Blue Sky Research PostScript version. In that version, the font
files are not provided in all the sizes (5,6,7,8,9,10), but only in sizes
5,7,10, with sizes 6,8,9 produced by interpolation. In practice it's
easy to tell if you need to use the \opt{psamsfonts} option: you'll get
an error message about a missing \fn{.tfm} file:
\begingroup \parindent0pt
\begintt
! Font \U/AMSa/m/n/9=msam9 not loadable: Metric (TFM) file not found.
\endtt
\endgroup
where the mentioned font name is one of the AMS font names (\fn{msam},
\fn{msbm}, \fn{eufm}, etc.), and the font size is 6, 8, or 9.
If this happens to you, look at your \cn{usepackage} statements and
change
$$\hbox{|\usepackage{amssymb}|\qquad to\qquad
|\usepackage[psamsfonts]{amssymb}|}$$
or
$$\hbox{|\usepackage{eucal}|\qquad to\qquad
|\usepackage[psamsfonts]{eucal}|}$$
and so forth.

If you are using an AMS document class (\fn{amsart}, \fn{amsbook},
\fn{amsproc}) or an AMS author package, apply the option \fn{[psamsfonts]}
to \cn{documentclass} as well.

\Warn Adding the option \fn{[psamsfonts]} to an existing file may
  result in changed line and page breaks, owing to the fact that sizes
  6, 8, and 9 are produced by interpolation.
\endx

Use of the PostScript AMSFonts is explicitly marked in individual
documents because the interpolation process used produces character
metrics that are not identical with those of the noninterpolated font
files for the corresponding sizes. If these discrepancies were simply
ignored, {\it there would be no warning} of unexpected changes in line
or page breaks that might occur in documents exchanged between colleagues.
As it is, if you find it necessary to add or remove the \fn{psamsfonts}
option in order to print a colleague's document, you are free to go ahead
and do so, but the fact that you must make that change should be understood
as a reminder that a small possibility of changed line breaks or page
breaks does exist.

A package \pkg{cmmib57} provides analogous font
definitions for the fonts \fn{cmmib} and \fn{cmbsy} (yes, both in the
same package, despite the name), for those users who have the Y\&Y/Blue
Sky Research PostScript versions of those fonts. Typical usage is:
\begintt
\documentclass{article}
\usepackage{cmmib57}
\endtt


\subsection{3.2} Computer Modern bold math italic and symbols

The package \pkg{amsbsy} (part of the \AmS-\LaTeX{} distribution) defines
two commands to obtain bold symbols:

\begingroup
\smallskip
\description
\item {\cn{boldsymbol}} -- for bold numbers and other nonalphabetic symbols,
  as well as bold Greek letters, which cannot be made bold via the \cn{mathbf}
  command, and bold math italic letters
\item {\cn{pmb}} -- ``poor man's bold'', which overlays multiple copies
  of the same symbol with slight offsets, for cases where \cn{boldsymbol}
  does not work, e.g., a bold font is not available
\enddescription
\endgroup

\noindent
\cn{boldsymbol} can also be used to obtain bold letters from the Euler fonts.

These commands are valid in math mode only.  For example,
\begintt
$$\boldsymbol{\beta} \pmb{\boxdot}
\boldsymbol{\Omega} \boldsymbol+ \pmb{\mathbb{R}}$$
\endtt
(Since this User's Guide is not prepared with \LaTeX{}, getting output
for this expression is left as an exercise to the user.)


\subsection{3.3} Blackboard bold letters (uppercase only)

The \pkg{amsfonts} package defines a ``math alphabet'' command \cn{mathbb}
for printing letters of the blackboard bold alphabet that resides in the
\fn{msbm} font. This alphabet is restricted to uppercase only (no
lowercase, no numerals). The suggested method for defining a \cn{R}
command to print a blackboard bold R is as follows:
\begintt
\usepackage{amsfonts}
\newcommand{\R}{\mathbb{R}}
\endtt


\subsection{3.4} Extra math symbols

The \pkg{amssymb} package defines math symbol commands for all the extra
math symbols in the \fn{msam} and \fn{msbm} fonts, as listed in the
table in Section 7.  Thus if you want to use the \cn{blacktriangle}
$\blacktriangle$ and \cn{nsubseteq} $\nsubseteq$ symbols, the easiest
way is to put the statement
\begintt
\usepackage{amssymb}
\endtt
in the preamble of your document.

If you want more selective access to the math symbols in those fonts
you can use the \pkg{amsfonts} package instead and define math symbol
commands individually using \LaTeX{}'s \cn{DeclareMathSymbol}
(cf.\ [LFG] or [GMS94]):
\begintt
\usepackage{amsfonts}
\DeclareMathSymbol{\blacktriangle}{\mathord}{AMSa}{"4E}
\DeclareMathSymbol{\nsubseteq}{\mathrel}{AMSb}{"2A}
\endtt
This alternative might be useful to you if adding the \pkg{amssymb}
package to your document leads to an error message of the form
\begintt
! TeX capacity exceeded, sorry (hash size=3000)
\endtt
This could happen if you have an older version of \TeX{} with a
relatively low limit on the number of commands that can be defined in a
single document. (But in that case, note that there may be configuration
options for increasing that limit; check the documentation for your
\TeX{} system.)

Since \cn{DeclareMathSymbol} is used in the \pkg{amssymb} package, the
definitions for particular symbols can be borrowed from there (file
\fn{amssymb.sty}).  Alternatively, the values can be obtained from the
tables in Section 7.2, as follows:

\begingroup
\raggedright
\item{\bull} First digit identifies font:
\itemitem{1} AMSa
\itemitem{2} AMSb
\item{\bull} Second digit identifies class:
\itemitem{0} \cn{mathord}
\itemitem{2} \cn{mathbin}
\itemitem{3} \cn{mathrel}
\item{\bull} Third and fourth digits identify (hex) location in font.
\endgraf
\endgroup


\subsection{3.5} Euler Fraktur letters

A math alphabet command \cn{mathfrak} for using Fraktur letters such as
$\mathfrak{A}\;\mathfrak{m}\;\mathfrak{g}\;\mathfrak{H}$ in math can be
obtained by using any of the packages \pkg{amssymb}, \pkg{amsfonts}, or
\pkg{eufrak}.


\subsection{3.6} Euler script letters (uppercase only)

The main purpose of the \pkg{eucal} package is to change \LaTeX{}'s
\cn{mathcal} command so that it produces Euler script instead of
Computer Modern calligraphic letters:
\ctab{\hfil#\tabskip1em&
  #ABCDEFGHIJKLMNOPQRSTUVWXYZ\hfil}
CM calligraphic:& \tensy\\
Euler script:& \teneusm
\endctab
There is also an option \opt{mathscr} for the \pkg{eucal} package that
causes Euler script to be associated to a \cn{mathscr} command, leaving
the \cn{mathcal} command unaffected. This imitates the behavior of the
predecessor package \pkg{euscript}.


%\subsection{3.7} Euler cursive (roman) letters
%
%Math alphabet commands \cn{matheurm} and \cn{matheurb} are defined by
%the package \pkg{eurom} for medium weight and bold forms of the Euler
%cursive letters respectively, for example, \matheurm{A}, \matheurb{A},
%\matheurm{B}, \matheurb{B}.


\subsection{3.7} University of Washington Cyrillic fonts

There is no AMS package for \LaTeX{} at the present time to support the
use of Cyrillic languages in \LaTeX{} documents with the \fn{wncy*}
fonts.  Producing a proper Cyrillic package involves rather difficult
questions of input and output encodings, for which it would be useful to
rely on general mechanisms provided by \LaTeX{}.

If you require these fonts, you can consult the \LaTeX{} documentation
[LFG], [L94] and use directly the commands described there, such as
\cn{DeclareFontFamily}, \cn{DeclareFontShape}, and \cn{symbol}.  A scheme
for accessing letters that don't correspond to the 26-letter Latin alphabet
will be needed; the file \fn{cyracc.def} can be used as a starting point.

%%%%%%%%%%%%%%%%%%%%%%%%%%%%%%%%%%%%%%%%%%%%%%%%%%%%%%%%%%%%%%%%%%%%%%%%

\section{4} How to use AMSFonts with \AmSTeX{}

In \JoT{}, Michael Spivak describes various fonts that are used in
mathematics in addition to the fonts provided with the standard
distributions of \TeX.  Two references in particular are of interest with
respect to AMSFonts: the section {\bf Fonts in math mode} in Chapter 19,
and Appendix G, {\bf Further fonts}.  The first describes the use of
letters from alphabets, including Fraktur, and the second, mostly
nonalphabetic symbols.

Instructions for using the fonts of the AMSFonts collection with \AmSTeX{}
are also given in the {\sl User's Guide to \AmSTeX{} Version~2.2\/} [AMS01]
and in Appendix~G of editions of \JoT{} [MDS90] dated 1990 or later.

Additional fonts to be used with \AmSTeX{} should be specified at the top
of the document input file, in what is known as the ``preamble''.  The
arrangement of commands at the top of an input file is the following:
\begintt
\input amstex
\documentstyle{...}
|<preamble commands>
\endtt

\AmSTeX{} provides a simple method for accessing most of the fonts in the
AMSFonts collection.  The two extra symbol fonts and Euler Fraktur are
loaded automatically by the preprint style (\filename{amsppt.sty}).
If you are using \AmSTeX{}, but not the preprint style, the method used to
load these fonts and define the associated symbol names depends on how many
symbols will be needed.  If a lot of the symbols will be needed, or
you aren't worried about memory space and just want to do what is easiest,
all three fonts will be loaded and the symbol names defined if you type the
command |\UseAMSsymbols| in the preamble.  This will load the file
\filename{amssym.tex}, in which all the symbol names (more than 200 of them)
are defined.  If only a few symbols from these fonts are needed, the
commands |\loadmsam|, |\loadmsbm|, and |\loadeufm| will load the
medium-weight versions of the two extra symbol fonts and Euler Fraktur
respectively.  The command |\newsymbol| can then be used to define just
those symbols that are needed; its use is described in Section 7,
{\bf Using the extra symbols}.%
\footnote{*}{Additional fonts from the AMSFonts collection can be accessed
  easily in \AmSTeX{}.
  However, users should be aware that \TeX{} limits the number of
  math mode font families to 16, of which 11 are predefined in \AmSTeX{}.
  Only those additional families should be activated that will actually
  be used in a document, to avoid exceeding the limit.}

Two sizes of fonts, suitable for body text and for passages requiring smaller
type (e.g.\ abstracts and footnotes), are incorporated in the preprint style
\filename{amsppt.sty}.  These are accessed through the control sequences
|\tenpoint| and |\eightpoint|, which are ordinarily referred to only by
higher-level commands that identify the kind of text being input (e.g.\
|\title|, |\abstract|, |\footnote|).  Most fonts in the AMSFonts collection
have |\load...|\ instructions defined in \AmSTeX{} and will be accessed
properly for use with the preprint style when the |\load| instructions are
included in the preamble of the document input.  If you are not using the
preprint style, you can use the font definitions in \filename{amsppt.sty} as a
model.

If you are using the PostScript version of the AMSFonts developed by
Y\&Y/Blue Sky Research, only selected sizes (5, 7 and 10) are provided, and
the other sizes are produced by interpolation.  With the \AmSTeX{} preprint
style, use the command \cn{PSAMSFonts}, placed after the \cn{documentstyle}
line and before \cn{topmatter} and \cn{document}, to access these fonts
correctly.  Papers and monographs submitted to the AMS for publication are
required to use this command.

\Warn Adding the command \cn{PSAMSFonts} to an existing file may result
  in changed line and page breaks, owing to the fact that sizes
  6, 8, and 9 are produced by interpolation.
\endx

\subsection{4.1} Euler

The Euler fonts are defined only in math mode, in sizes appropriate for
text and two orders of sub- and superscripts.  They can be activated by
invoking the proper |\load| instructions before the |\documentstyle|
command, in the preamble of a paper in which the fonts are to be used.
(The medium-weight Fraktur font is activated automatically by the preprint
style.)  The Euler fonts can be activated by the following commands:

\begingroup
\smallskip
\parskip=0pt
\def\1 #1 {\item{}{\tt\bs#1}\qquad\ignorespaces}
\1 loadeufm Euler Fraktur medium (automatic with the preprint style)
\1 loadeufb Euler Fraktur bold
\1 loadeurm Euler cursive medium
\1 loadeurb Euler cursive bold
\1 loadeusm Euler script medium
\1 loadeusb Euler script bold
\endgraf
\endgroup

After the \fontname{eufm} font has been loaded, the medium-weight Fraktur
letters can be produced by typing |\frak| followed by the desired letter.
For example, |$\frak g \frak A$| yields $\frak g \frak A$. \AmSTeX{} 2.1 also
defines |\eufm|, |\eufb|, |\eurm|, |\eurb|, |\eusm| and |\eusb|.


\subsection{4.2} Computer Modern bold math italic and symbols

The Computer Modern bold math italic (\fontname{cmmib}) and bold math
symbol (\fontname{cmbsy}) fonts can both be loaded in \AmSTeX{} by the
command |\loadbold|; there are no predefined commands to load them separately.
|\loadbold| must be invoked in the preamble of the document input file.

A rather elaborate mechanism has been defined in \AmSTeX{} to simplify
access to bold letters and symbols, in math mode only.  Three control
sequences are available, each of which affects a particular class of
characters:

\begingroup
\smallskip
\parskip=0pt
\setbox0=\hbox{\tt xboldsymbol}
\def\1 #1 {\item{}\hbox to\wd0{\tt\bs#1\hfil}\qquad\ignorespaces}
\1 bold         for a single letter or numeral
\1 boldkey      for other symbols that appear on the keyboard
\1 boldsymbol   for a symbol specified by a single control sequence
\endgraf
\endgroup
\noindent
These facilities are described in more detail in the {\sl User's Guide to
\AmSTeX{} Version~2.2\/} [AMS01] and editions of \Joy\/ published in 1990
or later [MDS90].


\subsection{4.3} Computer Modern math extension font

Smaller sizes of the math extension font are appropriate for use in text
smaller than ten-point and in sub- and superscripts.  They are provided
automatically for these environments in the preprint style.  If you are
not using the preprint style, you can use the font definitions in either
\filename{amsppt.sty} or Appendix~E of \TB\/ [DEK86A] as a model.


\subsection{4.4} Extra symbols

The medium-weight versions of the two extra symbol fonts are available
automatically, including all the symbol names, if you are using the preprint
style or if you have specified |\input amssym|.  If you wish to load these
fonts separately, use the appropriate control sequence |\loadmsam| or
|\loadmsbm| in the preamble of your document.  If you load the fonts
separately, a few symbols will be defined when one of the fonts is loaded,
but most must be defined using the |\newsymbol| command before they can be
used.  See Section 7, {\bf Using the extra symbols}, for
information on both the symbol names and on using |\newsymbol| to define them.


\subsection{4.5} Cyrillic

Cyrillic is not referred to in the \AmSTeX{} files as distributed.
The cyrillic fonts included in AMSFonts are intended for use mainly in text,
not as symbols in math.  Detailed instructions for loading and using cyrillic
appear below in Section 6, {\bf Using cyrillic}.


\subsection{4.6} Caps/small caps

Caps/small caps are loaded automatically by the \AmSTeX{} preprint style
for use in ten-point and eight-point text.  If you are not using the preprint
style, you can use the font definitions in either \filename{amsppt.sty}
or Appendix~E of \TB\/ [DEK86A] as a model.


\subsection{4.7} Dummy font

No special action is needed to use the dummy font with \AmSTeX{}.
It is already built into the syntax checking procedure.


%%%%%%%%%%%%%%%%%%%%%%%%%%%%%%%%%%%%%%%%%%%%%%%%%%%%%%%%%%%%%%%%%%%%%%%%

\section{5} How to use AMSFonts with plain \TeX{} or other macro packages

If you are not using \LaTeX{} or \AmSTeX{} then there are too many
variables for us to provide much specific guidance. It will be necessary
to assume that you either have some experience with \TeX{} macros or
have a \TeX nician available to help you. However, some general
guidelines may be helpful.

Two models for defining fonts should be accessible to most users:
\item{\bull} Appendix E of \TB\/ contains size-specific font definitions
  for \hbox{|\tenpoint|}, \hbox{|\ninepoint|} and |\eightpoint| that permit
  size-switching, including support of mathematics.
\item{\bull} \filename{amsppt.sty}, the file of macros supporting the
  \AmSTeX{} preprint style, contains similar font definitions, |\tenpoint|
  and |\eightpoint|.

The font-size-switching facilities of \LaTeX{} are not recommended as a
model because they include many features (such as loading fonts on demand)
that make them too complex to be copied easily for uses outside of
\LaTeX{} except by someone with substantial \TeX{} expertise.

Before attempting to load all available fonts into every \TeX{} job,
determine (if you can) how many fonts can be accommodated by the
implementation of \TeX{} you are using.  It is generally a good idea to
load seldom-used fonts selectively.


\subsection{5.1} Euler

The following commands will load the medium-weight Euler Fraktur font, and
can be used as a model for accessing the other Euler fonts.
\begintt
\font\teneufm=eufm10
\font\seveneufm=eufm7
\font\fiveeufm=eufm5
\newfam\eufmfam
\textfont\eufmfam=\teneufm
\scriptfont\eufmfam=\seveneufm
\scriptscriptfont\eufmfam=\fiveeufm
\def\eufm#1{{\fam\eufmfam\relax#1}}
\endtt

Individual letters in the Euler fonts are accessible by the ordinary
letters on your keyboard, once the font has been loaded and named by
a control sequence equivalent to |\eufm|.

The medium-weight Fraktur font, \fontname{eufm}, can also be loaded by
|\input amssym.def|; this loads the two extra symbol fonts as well.


\subsection{5.2} Computer Modern bold math italic and symbols

The \fontname{cmmib} and \fontname{cmbsy} fonts can be loaded and made
accessible to math in ten-point environments by the following code:
\begintt
\font\tencmmib=cmmib10  \skewchar\tencmmib='177
\font\sevencmmib=cmmib7 \skewchar\sevencmmib='177
\font\fivecmmib=cmmib5  \skewchar\fivecmmib='177
\newfam\cmmibfam
\textfont\cmmibfam=\tencmmib \scriptfont\cmmibfam=\sevencmmib
 \scriptscriptfont\cmmibfam=\fivecmmib

\font\tencmbsy=cmbsy10  \skewchar\tencmbsy='60
\font\sevencmbsy=cmbsy7 \skewchar\sevencmbsy='60
\font\fivecmbsy=cmbsy5  \skewchar\fivecmbsy='60
\newfam\cmbsyfam
\textfont\cmbsyfam=\tencmbsy \scriptfont\cmbsyfam=\sevencmbsy
 \scriptscriptfont\cmbsyfam=\fivecmbsy
\endtt
The \TeX{} primitive |\mathchar| must be used to access individual characters
from a font in math mode.
|\mathchar|, like the |\char| primitive, requires that you know the position in
the font of the character you are accessing.  However, |\mathchar| also
requires that you specify the ``class'' and the family of the math character
being accessed.  See Chapter 17 of \TB{} for more details on the use of
|\mathchar|, as well as |\mathchardef|, which will allow you to define your own
macro names for individual characters in these fonts.

\Note The file \filename{amssym.def} contains a convenient macro,
|\hexnumber@|, to determine the family number of the font being accessed
through |\mathchar|.  For example, the |\mathchar| statement to properly access
the bold alpha in the \fn{cmmib} font would be:
\begintt
\mathchar"0\hexnumber@\cmmibfam0B
\endtt
\endx


\subsection{5.3} Computer Modern math extension font

The 10-point \fontname{cmex} font is loaded by \filename{plain.tex}.
To install the 7-point size appropriate for sub- and superscripts in
a ten-point math environment, include the following code in your file:
\begintt
\font\sevenex=cmex7
\scriptfont3=\sevenex \scriptscriptfont3=\sevenex
\endtt
To use other sizes implies the use of switchable-size fonts,
which may be implemented according to the models cited
at the beginning of this section.


\subsection{5.4} Extra symbols

Detailed instructions for accessing the \fontname{msam} and \fontname{msbm}
fonts are given in Section 7, {\bf Using the extra symbols}.


\subsection{5.5} Cyrillic

See Section 6, {\bf Using cyrillic}, for instructions.


\subsection{5.6} Caps/small caps

The 10-point \fontname{cmcsc} font is loaded by \filename{plain.tex}.
To use the smaller versions implies the use of switchable-size fonts,
which may be implemented according to the models cited at the beginning
of this section.


\subsection{5.7} Dummy font

The dummy font was designed to be used for syntax checking.  The general
technique is described in Appendix~D of \TB, p.~401.  This has been
implemented in the file \filename{amstex.tex}, which can be used as a model.


%%%%%%%%%%%%%%%%%%%%%%%%%%%%%%%%%%%%%%%%%%%%%%%%%%%%%%%%%%%%%%%%%%%%%%%%

\section{6} Using cyrillic

\def\2#1{${}\mapsto{}${\cyr#1}}

\newcount\cyrtablefigno  \cyrtablefigno=1
\def\cyrtablefig{Figure~\number\cyrtablefigno}

The cyrillic fonts in the AMSFonts collection have been designed so that
input using the transliteration conventions of {\sl Mathematical
Reviews\/} will be converted directly to cyrillic text. Other
transliteration schemes exist, as well as methods for keying directly
from the keyboard to access cyrillic characters. We have included two
sets of virtual fonts in this distribution, which provide access to the
characters of the cyrillic fonts through the KOI-8 or Alternativnyj
Variant (AV) encodings. At the present time macro support is provided
only for the {\sl Mathematical Reviews\/} transliteration scheme; to use
one of the other encodings, you must seek support from other sources.
Also, the instructions in this section are unsuitable for \LaTeX{}; they
apply only to \AmSTeX{}, plain \TeX{}, and other macro packages that use
plain \TeX{} font loading methods.

The following cyrillic fonts are included:

\item{} \fontname{wncyr} (upright), in sizes 5, 6, 7, 8, 9, and 10 point
\item{} \fontname{wncyb} (bold), in the same range of sizes as \fontname{wncyr}
\item{} \fontname{wncyi} (italic), in the same range of sizes as
  \fontname{wncyr}
\item{} \fontname{wncysc} (caps and small caps), in size 10 point
\item{} \fontname{wncyss} (upright sans serif), in sizes 8, 9, and 10 point

\noindent
(The Y\&Y/Blue Sky PostScript collection contains only 10-point outlines.
Other sizes must be obtained by scaling.)

The file \filename{cyracc.def}, which is included in the AMSFonts collection,
must be input to any document using the cyrillic fonts as defined with the
AMS layout.  Since the cyrillic alphabet contains more letters than the
roman alphabet, some cyrillic letters are accessed by combinations of roman
letters, accented letters, or control sequences.  \filename{cyracc.def}
contains the definitions of these accents and control sequences.
If this file is not input, some cyrillic letters will be inaccessible.


\subsection{6.1} Making cyrillic available to a document

If you are using plain \TeX{}, include the following instructions near the
top of the document input file to make the 10-point cyrillic font available
for use in text (see below for cyrillic in math):
\begintt
\input cyracc.def
\font\tencyr=wncyr10
\def\cyr{\tencyr\cyracc}
\endtt
If you require cyrillic text in more than one size, you must take a
different approach in defining |\cyr|.  An appropriate model appears in
Appendix~E of \TB\/ [DEK86A], pages 414--15.  The definition of |\cyr|
should be incorporated into size-specific macros such as |\tenpoint| and
|\eightpoint| similarly to what is done there for |\bf|.  Don't forget to
include the command |\cyracc| in the definition.

If you are using \AmSTeX{} and the preprint style, include the following
instructions in the preamble of your document input file to make
cyrillic available in 10-point and 8-point text:
\begintt
\input cyracc.def
\catcode`\@=11
\font@\tencyr=wncyr10
\font@\eightcyr=wncyr8
\catcode`\@=13
\addto\tenpoint{\def\cyr{\tencyr\cyracc}}
\addto\eightpoint{\def\cyr{\eightcyr\cyracc}}
\endtt
(The |\font@| command not only loads the fonts, but also makes them behave
properly during syntax checking.)
If you are not using the preprint style, you can use the font definitions
in either \filename{amsppt.sty} or \TB\/ Appendix~E as a model.

The macro definitions in \filename{cyracc.def} govern the behavior of
cyrillic-specific control sequences, including accents, in cyrillic and
noncyrillic text.  Definitions governing noncyrillic text are activated
as soon as \filename{cyracc.def} is |\input|.  This will permit text input
according to the scheme shown in \cyrtablefig{} to be typeset in
transliterated form, according to the {\sl MR\/} conventions.  To produce
actual cyrillic text, enclose the cyrillic input in a group that begins with
the instruction |\cyr| {\sl inside\/} the group, as
\begintt
... {\cyr ...} ...
\endtt
Enclosing in braces both the |\cyr| and the text to be set in cyrillic type
(in the same way that an italic phrase would be indicated in a roman text)
is particularly important for two reasons.  First, like |\it|, |\cyr| must
be explicitly terminated to return to roman text.  And second, unlike |\it|,
the special cyrillic control sequences invoked by |\cyracc| are interpreted
differently by \TeX{} depending on whether they are in a cyrillic or a
noncyrillic environment.  The ``cyrillic'' interpretation is not turned
off simply by invoking |\rm|.  Failure to follow this practice will yield
gibberish.


\subsection{6.2} Cyrillic input

The table in \cyrtablefig{} follows the alphabetical order of the table
published in the 1983 MR author index.  The three paired columns
contain: (1)~Cyrillic; (2)~Input; (3)~Transliteration.

The letters in the Cyrillic columns will appear in the
typeset output when the corresponding codes from the Input columns
are used in the |{\cyr ...}| context described above.  The roman
letters in the Transliteration columns will appear in the output when
the corresponding codes from the Input columns are used in a noncyrillic
environment, i.e., have not been preceded by |\cyr|.

\penalty0
\topinsert
\begingroup
\input amsfndoc.cyr
\endgroup
\endinsert

\penalty0
\indent
Several points should be noted here.

\nobreak
\item{\bull} Input codes for uppercase cyrillic which consist of more than
  one letter, e.g. |Zh|\2{Zh}, can also be input in all caps, e.g. |ZH|\2{ZH},
  if the context is entirely in caps.

\item{\bull} Particular care is necessary when the letter t\2{t} is followed
  by s\2{s}.  The control sequence |\cydot| (``cyrillic dot'')
  is provided as a separator to keep those letters distinct:
  |t\cydot s|~(t\cydot s)\2{t\cydot s}.
  Otherwise, they will be combined as ts\2{ts}.

\item{} The t\cydot s pair appears, for example, in the word
  |sovet\cydot ski\u\i|
  (sovet\cydot ski\u\i)\2{sovet\cydot ski\u\i}
  and is not uncommon in the suffix of reflexive verbs, e.g.
  \hbox{|nakhodyat\cydot sya|}
  (nakhodyat\cydot sya)\2{nakhodyat\cydot sya}.

\item{\bull} Because there is not a one-to-one correspondence between
  cyrillic and roman letters, some cyrillic letters have been placed
  in locations where a roman letter does not have a cyrillic
  counterpart.  A user who is aware of this fact may be able to
  detect input keying that does not conform to the recommendations
  shown in \cyrtablefig, and correct it more easily than otherwise.
  The following nonstandard assignments have been made:\newline
\indent |c|\2{c}; |h|\2{h}; |q|\2{q}; |w|\2{w}; |x|\2{x}.

\item{\bull} Some very strange effects can occur in cyrillic text
  hyphenated by the default English hyphenation rules; in particular,
  a cyrillic letter input as a group of letters can be decomposed.
  (Most multiple-letter input groups are converted to a single cyrillic
  letter by way of \TeX's ligaturing mechanism.)  For example,
  |shch|\2{shch} might, in especially unlucky circumstances, be
  decomposed as {\cyr s-hch}, {\cyr sh-ch} or {\cyr shc-h}.
  In other words, if there is any chance that cyrillic text might fall
  into a position where hyphenation could occur, the results should be
  checked very carefully, and discretionary hyphens used as appropriate.

\item{\bull} Hyphenation patterns do not exist for the AMS cyrillic font
  when the input conventions shown here are used.  Furthermore, it is
  probably impracticable to attempt to develop such rules, since the
  rules to recognize control sequences and complicated ligatures, both
  used extensively by the AMS cyrillic input conventions, are not easily
  specified to \TeX's hyphenation mechanism.  Another approach to
  hyphenation, requiring some changes to the cyrillic \filename{.tfm}
  files, has been described by Dimitri Vulis in a \TUB\/ article [DLV89].


\subsection{6.3} Cyrillic in math

Although the cyrillic fonts are intended for use as text, individual
letters are sometimes requested in math; for example, {\cyr SH} may be
used to represent the Shafarevich group.  When cyrillic is needed in
math mode, replace the definition of |\cyr| shown previously (which
will work only for text) by the following instructions (which will
support the use of cyrillic in both text and math):
\begintt
\newfam\cyrfam
\font\tencyr=wncyr10
\font\sevencyr=wncyr7
\font\fivecyr=wncyr5
\def\cyr{\fam\cyrfam\tencyr\cyracc}
\textfont\cyrfam=\tencyr \scriptfont\cyrfam=\sevencyr
  \scriptscriptfont\cyrfam=\fivecyr
\endtt
If only the 10-point cyrillic font has been accessed, the references
to |\sevencyr| and |\fivecyr| can be changed to |\tencyr| to save memory.
When using \AmSTeX{} and the preprint style, use |\font@| instead of |\font|,
remembering to change the |\catcode| of the |@| appropriately, and embed the
font family specifications in |\addto\tenpoint|, as shown above.

If other base text sizes are used besides ten point, the suggestions given
above in Section 6.1, {\bf Making cyrillic available}, apply here as well.

%%%%%%%%%%%%%%%%%%%%%%%%%%%%%%%%%%%%%%%%%%%%%%%%%%%%%%%%%%%%%%%%%%%%%%%%

\makeatletter

%  Define macros for presentation of tables of symbols.
\def\BBB#1{\par\bigbreak
  \leavevmode\llap{$\bullet$\enspace}{\bf#1}}
\def\ttcs#1{\leavevmode\hbox{\tt\bs\ignorespaces#1\unskip}}
\newdimen\biggest
\setbox0\hbox{$\dashrightarrow$}\biggest=\wd0
\def\1#1{\hbox to\biggest{\hfil$\csname#1\endcsname$\hfil}\ \ %
  \ttcs{#1}}

\def\getID@#1{\edef\next@{\expandafter\meaning\csname#1\endcsname}%
 \expandafter\getID@@\next@0\getID@@}
\def\getID@@#1"#2#3#4#5#6\getID@@{\def\next@{#6}%
  \ifx\next@\empty
   \def\next@{#2}%
    \ifx\next@\msafam@
     \def\ID@{10#3#4}%
    \else
     \def\ID@{20#3#4}%
    \fi
  \else
   \def\next@{#3}%
    \ifx\next@\msafam@
     \def\ID@{1#2#4#5}%
    \else
     \def\ID@{2#2#4#5}%
    \fi
  \fi}
\def\2#1{\hbox to.5\hsize
  {\hbox to\biggest{\hfill$\csname#1\endcsname$\hfill}\ \ %
    \getID@{#1}{\tt\ID@}\ \ \ttcs{#1}\hfill}}
\def\3#1#2{\hbox to.5\hsize
  {\hbox to\biggest{\hfil$\csname#1\endcsname$\hfil}\ \ %
    \getID@{#1}{\tt\ID@}\ \ \ttcs{#1}, \ttcs{#2}\hss}}
\def\4#1{\hbox to.5\hsize
  {\hbox to\biggest{\hfill$\csname#1\endcsname$\hfill}\ \ %
    \getID@{#1}{\tt\ID@}\ \ \ttcs{#1}\ \ {\eightpoint(U)}\hfill}}

\makeatother

%%%%%%%%%%%%%%%%%%%%%%%%%%%%%%%%%%%%%%%%%%%%%%%%%%%%%%%%%%%%%%%%%%%%%%%%

\section{7} Using the extra symbols

Most users of the extra symbol fonts will probably want to make them
accessible to their \TeX{} jobs with the least possible fuss.  For \AmSTeX{}
users, these fonts are available automatically with the preprint style, and
other methods of loading them for use with \AmSTeX{} are described above.
To load these fonts with \LaTeX{} or \AmS-\LaTeX{}, see Section 3.4,
{\bf Extra math symbols}.

If you are not using \AmSTeX{} or \LaTeX{}, the easiest method of loading
these fonts and defining the control sequences for accessing the symbols is
to place the command
\begintt
\input amssym.tex
\endtt
at the top of your input file.  This will load the fonts \fontname{msam},
\fontname{msbm}, and \fontname{eufm} in sizes 10, 7, and 5 point, suitable
for use in ordinary ten-point math environments, and define the names of
all the symbols in these fonts.  However, this assigns more than 200
control sequence names, so if you are limited for space, an alternative
method may be preferred.

If you type just |\input amssym.def| (or |\usepackage{amsfonts}| for
\LaTeX), the fonts will be loaded, but only the
names of the few special symbols listed below will be defined.

First there are four symbols that are normally used outside of math mode:
$$\vcenter{\halign to\hsize{\1{#}\hfil\tabskip\centering&
   \hbox to.5\hsize{\1{#}\hfil}\tabskip0pt\cr
checkmark&circledR\cr
maltese&yen\cr}}
$$
These symbols, like \P, \S, \dag, and \ddag, can also be used in math mode;
they will change sizes correctly in subscripts and superscripts.

Next are four symbols that are ``delimiters'' (although there are
no larger versions obtainable with \ttcs{left} and \ttcs{right}), so they
must be used in math mode:
$$\vcenter{\halign to\hsize{\1{#}\hfil\tabskip\centering&
   \hbox to.5\hsize{\1{#}\hfil}\tabskip0pt\cr
 ulcorner&urcorner\cr
 llcorner&lrcorner\cr}}$$

Finally, two dashed arrows are constructed from symbols in this family
(note that one of them has two names; it can be accessed by either one):
$$\vcenter{\halign to\hsize{\1{#}\hfil\tabskip\centering&
   \hbox to.5\hsize{\1{#}\hfil}\tabskip0pt\cr
 \omit\hbox to.5\hsize{\hbox to\biggest{\hfil$\dashrightarrow$\hfil}\ \ %
    \ttcs{dashrightarrow}, \ttcs{dasharrow}\hss}&dashleftarrow\cr}}$$

The Blackboard Bold letters $\Bbb A,\dots,\Bbb Z$ can be accessed by typing
(in math mode) |\Bbb A|$,\dots,$|\Bbb Z| in plain \TeX{} or \AmSTeX, or
|\mathbb{A}|$,\dots,$|\mathbb{Z}| in \LaTeX.

Wider versions of the \filename{plain.tex} |\widehat| and |\widetilde|
are now available.

Letters in the \fontname{eufm} font can be accessed (in math mode) by typing,
for example, |\frak A \frak g| to get $\frak A \frak g$
(|\mathfrak{Ag}| in \LaTeX). For the other Euler
fonts, see the various ``Euler'' subsections under the sections for
different macro packages (\LaTeX{}, \AmSTeX{}, plain \TeX{}).

\subsection{7.1} The {\tt\bs newsymbol} command (\AmSTeX{} or plain \TeX{})

All other symbols of the \fontname{msam} and \fontname{msbm} fonts must be
named by control sequences so that they can be used (in math mode only) when
the fonts are loaded.  If you are very short on space for control sequence
names, and need only a few of these symbols, you can omit the loading of
\filename{amssym.tex} and instead assign only the names you will need by
using the command |\newsymbol| for each symbol you need, to create a
control sequence that will properly produce that symbol.  The control
sequence can be either the ``standard'' name, as listed below, or one
of your own choosing.

The list of symbols below shows for each symbol the symbol itself, a
four-character~``ID'', and the ``standard'' name of the symbol.
(The first character of the ID identifies the font family in which a
symbol resides.  Symbols from the \fontname{msam} family have {\tt1} as the
first character; symbols from the \fontname{msbm} family have {\tt2} as the
first character.)
For example, the symbol $\nleqslant$ appears as
\medskip
\noindent\kern\parindent\2{nleqslant}
\medskip
\noindent
To produce a control sequence with this name, the instruction
\begintt
\newsymbol\nleqslant 230A
\endtt
appears in the file \filename{amssym.tex}.  This same instruction can
be typed by a user who is not using the \AmSTeX{} preprint style and has
chosen not to load all the symbols, and thereafter the control sequence
|\nleqslant| will produce the symbol $\nleqslant$ (in math mode), and will
act properly as a ``binary relation''.

A few symbols in these fonts replace symbols defined in \filename{plain.tex}
by combinations of symbols available in the Computer Modern fonts.  These
are |\angle|~($\angle$) and |\hbar|~($\hbar$) from the group
``Miscellaneous symbols'', and |\rightleftharpoons|~($\rightleftharpoons$)
from the group ``Arrows'' below.  The new symbols will
change sizes correctly in subscripts and superscripts, provided that you
are using appropriate redefinitions.  In order to use |\newsymbol| to
replace an existing definition, the name must first be ``undefined''.
Here are the lines you must put in your file if you are not using the
\AmSTeX{} preprint style or |\input amssym| (which perform the redefinition
automatically):
\begintt
\undefine\angle
\newsymbol\angle 105C
\undefine\hbar
\newsymbol\hbar 207E
\undefine\rightleftharpoons
\newsymbol\rightleftharpoons 130A
\endtt
\noindent
These symbols are flagged in the tables below with a ``{\eightpoint(U)}'',
as a reminder that they must be undefined.

\subsection{7.2} The extra symbols

Note in the tables that some symbols are shown with two names; in such a
case, either one can be used to access the symbol.

\BBB{Lowercase Greek letters}
$$\halign{\hbox to.5\hsize{\2{#}}&\2{#}\cr
digamma&varkappa\cr}$$

\BBB{Hebrew letters}
$$\halign{\hbox to.5\hsize{\2{#}}&\2{#}\cr
beth&gimel\cr
daleth\cr
}$$

\BBB{Miscellaneous symbols}
$$\halign{\hbox to.5\hsize{\2{#}}&\2{#}\cr
\omit\4{hbar}&backprime\cr
hslash&varnothing\cr
vartriangle&blacktriangle\cr
triangledown&blacktriangledown\cr
square&blacksquare\cr
lozenge&blacklozenge\cr
circledS&bigstar\cr
\omit\4{angle}&sphericalangle\cr
measuredangle&\omit\cr
nexists&complement\cr
mho&eth\cr
Finv&diagup\cr
Game&diagdown\cr
Bbbk&\omit\cr
}$$

\BBB{Binary operators}
$$\halign{\hbox to.5\hsize{\2{#}}&\2{#}\cr
dotplus&ltimes\cr
smallsetminus&rtimes\cr
\omit\3{Cap}{doublecap}&leftthreetimes\cr
\omit\3{Cup}{doublecup}&rightthreetimes\cr
barwedge&curlywedge\cr
veebar&curlyvee\cr
%                               %%%%%%%%%%
%\noalign{\newpage}
%                               %%%%%%%%%%
doublebarwedge\cr
boxminus&circleddash\cr
boxtimes&circledast\cr
boxdot&circledcirc\cr
boxplus&centerdot\cr
divideontimes&intercal\cr}
$$

\BBB{Binary relations}
$$\halign{\hbox to.5\hsize{\2{#}}&\2{#}\cr
leqq&geqq\cr
leqslant&geqslant\cr
eqslantless&eqslantgtr\cr
lesssim&gtrsim\cr
lessapprox&gtrapprox\cr
approxeq&eqsim\cr
lessdot&gtrdot\cr
\omit\3{lll}{llless}&\omit\3{ggg}{gggtr}\cr
lessgtr&gtrless\cr
lesseqgtr&gtreqless\cr
lesseqqgtr&gtreqqless\cr
\omit\3{doteqdot}{Doteq}&eqcirc\cr
risingdotseq&circeq\cr
fallingdotseq&triangleq\cr
backsim&thicksim\cr
backsimeq&thickapprox\cr
subseteqq&supseteqq\cr
Subset&Supset\cr
sqsubset&sqsupset\cr
preccurlyeq&succcurlyeq\cr
curlyeqprec&curlyeqsucc\cr
precsim&succsim\cr
precapprox&succapprox\cr
vartriangleleft&vartriangleright\cr
trianglelefteq&trianglerighteq\cr
vDash&Vdash\cr
Vvdash\cr
smallsmile&shortmid\cr
smallfrown&shortparallel\cr
bumpeq&between\cr
Bumpeq&pitchfork\cr
varpropto&backepsilon\cr
blacktriangleleft&blacktriangleright\cr
therefore&because\cr}$$
\bigbreak
\BBB{Negated relations}
$$\halign{\hbox to.5\hsize{\2{#}}&\2{#}\cr
nless&ngtr\cr
nleq&ngeq\cr
nleqslant&ngeqslant\cr
nleqq&ngeqq\cr
lneq&gneq\cr
lneqq&gneqq\cr
lvertneqq&gvertneqq\cr
lnsim&gnsim\cr
lnapprox&gnapprox\cr
%                               %%%%%%%%%%
%\noalign{\newpage}
%                               %%%%%%%%%%
nprec&nsucc\cr
npreceq&nsucceq\cr
precneqq&succneqq\cr
precnsim&succnsim\cr
precnapprox&succnapprox\cr
nsim&ncong\cr
nshortmid&nshortparallel\cr
nmid&nparallel\cr
nvdash&nvDash\cr
nVdash&nVDash\cr
ntriangleleft&ntriangleright\cr
ntrianglelefteq&ntrianglerighteq\cr
nsubseteq&nsupseteq\cr
nsubseteqq&nsupseteqq\cr
subsetneq&supsetneq\cr
varsubsetneq&varsupsetneq\cr
subsetneqq&supsetneqq\cr
varsubsetneqq&varsupsetneqq\cr}$$

\overfullrule=0pt

\BBB{Arrows}
$$\halign{\hbox to.5\hsize{\2{#}}&\2{#}\cr
leftleftarrows&rightrightarrows\cr
leftrightarrows&rightleftarrows\cr
Lleftarrow&Rrightarrow\cr
twoheadleftarrow&twoheadrightarrow\cr
leftarrowtail&rightarrowtail\cr
looparrowleft&looparrowright\cr
leftrightharpoons&\omit\4{rightleftharpoons}\cr
curvearrowleft&curvearrowright\cr
circlearrowleft&circlearrowright\cr
Lsh&Rsh\cr
upuparrows&downdownarrows\cr
upharpoonleft&\omit\3{upharpoonright}{restriction}\cr
downharpoonleft&downharpoonright\cr
multimap&rightsquigarrow\cr
leftrightsquigarrow\cr}$$

\BBB{Negated arrows}
$$\halign{\hbox to.5\hsize{\2{#}}&\2{#}\cr
nleftarrow&nrightarrow\cr
nLeftarrow&nRightarrow\cr
nleftrightarrow&nLeftrightarrow\cr}$$


%%%%%%%%%%%%%%%%%%%%%%%%%%%%%%%%%%%%%%%%%%%%%%%%%%%%%%%%%%%%%%%%%%%%%%%%

\References{DEK86A}

\bibitem [AMS01] {\sl User's Guide to \AmSTeX{} Version~2.2, August 2001},
  \AMS, Providence, RI, 2001; distributed with \AmSTeX{} Version~2.2.

\bibitem [AMSPS] {\sl AMSFonts in Type 1 (PostScript) form}, AMS web
  server, URL: |http://www.ams.org/tex/type1-fonts.html|.

\bibitem [DEK86A] Donald E. Knuth, {\sl The \TeX book},
  Volume~A of \CandT, \AW{} Publishing Co.,
  Reading, MA, 1986.

\bibitem [DEK86E] \bysame {\sl Computer Modern Typefaces},
  Volume~E of \CandT, \AW{} Publishing Co.,
  Reading, MA, 1986.

\bibitem [DEK89] \bysame ``Typesetting Concrete Mathematics'',
  {\sl \TUB\/} {\bf10} (1989), no.~1, 31--36; erratum,
  {\sl \TUB\/} {\bf10} (1989), no.~3, 342.

\bibitem [DLV89] Dimitri Vulis, ``Notes on Russian \TeX'',
  {\sl \TUB\/} {\bf10} (1989), no.~3, 332--36.

\bibitem [DRS86] David R Siegel, {\sl The Euler Project at Stanford},
  Computer Science Department, Stanford University, 1985.

\bibitem [GKP88] Ronald L. Graham, Donald E. Knuth, and Oren Patashnik,
  {\sl Concrete Mathematics}, \AW{} Publishing Co.,
  Reading, MA, 1988.

\bibitem[GMS94] {\sl The \LaTeX{} companion}, Michel Goossens, Frank
  Mittelbach, and Alexander Samarin, \AW{} Publishing Company,
  Reading, MA, 1994.  Chapter~8 describing AMS-\LaTeX{} is obsolete;
  a replacement chapter is available in PDF form from
  |http://www.ctan.org/tex-archive/info/companion-rev/ch8.pdf|.

\bibitem[L94] {\sl \LaTeX: A document preparation system}, Second edition,
  Leslie Lamport, \AW{} Publishing Company, Reading, MA, 1994.

\bibitem[LFG] ``\LaTeX{}2e font selection'', \fn{fntguide.tex} in the
  \LaTeX{} distribution.

\bibitem [MDS86] M. D. Spivak, \JoT, \AMS, Providence, RI, 1986.

\bibitem [MDS90] \bysame \JoT, $2^{\rm nd}$ (revised) edition,
  \AMS, Providence, RI, 1990.

\endReferences


%%%%%%%%%%%%%%%%%%%%%%%%%%%%%%%%%%%%%%%%%%%%%%%%%%%%%%%%%%%%%%%%%%%%%%%%
\begingroup \def\bye{\par\endinput}
\input amsfndoc.ins
\endgroup

%%%%%%%%%%%%%%%%%%%%%%%%%%%%%%%%%%%%%%%%%%%%%%%%%%%%%%%%%%%%%%%%%%%%%%%%
\appendix{D}{Font charts}
\begingroup \def\bye{\par\endinput}
\input amsfndoc.fnt
\endgroup

\bye
