% \iffalse meta-comment
%
% File: iflang.dtx
% Version: 2016/05/16 v1.6
% Info: Checks for the current language
%
% Copyright (C) 2007 by
%    Heiko Oberdiek <heiko.oberdiek at googlemail.com>
%    2016
%    https://github.com/ho-tex/oberdiek/issues
%
% This work may be distributed and/or modified under the
% conditions of the LaTeX Project Public License, either
% version 1.3c of this license or (at your option) any later
% version. This version of this license is in
%    http://www.latex-project.org/lppl/lppl-1-3c.txt
% and the latest version of this license is in
%    http://www.latex-project.org/lppl.txt
% and version 1.3 or later is part of all distributions of
% LaTeX version 2005/12/01 or later.
%
% This work has the LPPL maintenance status "maintained".
%
% This Current Maintainer of this work is Heiko Oberdiek.
%
% The Base Interpreter refers to any `TeX-Format',
% because some files are installed in TDS:tex/generic//.
%
% This work consists of the main source file iflang.dtx
% and the derived files
%    iflang.sty, iflang.pdf, iflang.ins, iflang.drv, iflang-test1.tex,
%    iflang-test2.tex, iflang-test3.tex, iflang-test4.tex,
%    iflang-test5.tex.
%
% Distribution:
%    CTAN:macros/latex/contrib/oberdiek/iflang.dtx
%    CTAN:macros/latex/contrib/oberdiek/iflang.pdf
%
% Unpacking:
%    (a) If iflang.ins is present:
%           tex iflang.ins
%    (b) Without iflang.ins:
%           tex iflang.dtx
%    (c) If you insist on using LaTeX
%           latex \let\install=y% \iffalse meta-comment
%
% File: iflang.dtx
% Version: 2016/05/16 v1.6
% Info: Checks for the current language
%
% Copyright (C) 2007 by
%    Heiko Oberdiek <heiko.oberdiek at googlemail.com>
%    2016
%    https://github.com/ho-tex/oberdiek/issues
%
% This work may be distributed and/or modified under the
% conditions of the LaTeX Project Public License, either
% version 1.3c of this license or (at your option) any later
% version. This version of this license is in
%    http://www.latex-project.org/lppl/lppl-1-3c.txt
% and the latest version of this license is in
%    http://www.latex-project.org/lppl.txt
% and version 1.3 or later is part of all distributions of
% LaTeX version 2005/12/01 or later.
%
% This work has the LPPL maintenance status "maintained".
%
% This Current Maintainer of this work is Heiko Oberdiek.
%
% The Base Interpreter refers to any `TeX-Format',
% because some files are installed in TDS:tex/generic//.
%
% This work consists of the main source file iflang.dtx
% and the derived files
%    iflang.sty, iflang.pdf, iflang.ins, iflang.drv, iflang-test1.tex,
%    iflang-test2.tex, iflang-test3.tex, iflang-test4.tex,
%    iflang-test5.tex.
%
% Distribution:
%    CTAN:macros/latex/contrib/oberdiek/iflang.dtx
%    CTAN:macros/latex/contrib/oberdiek/iflang.pdf
%
% Unpacking:
%    (a) If iflang.ins is present:
%           tex iflang.ins
%    (b) Without iflang.ins:
%           tex iflang.dtx
%    (c) If you insist on using LaTeX
%           latex \let\install=y% \iffalse meta-comment
%
% File: iflang.dtx
% Version: 2016/05/16 v1.6
% Info: Checks for the current language
%
% Copyright (C) 2007 by
%    Heiko Oberdiek <heiko.oberdiek at googlemail.com>
%    2016
%    https://github.com/ho-tex/oberdiek/issues
%
% This work may be distributed and/or modified under the
% conditions of the LaTeX Project Public License, either
% version 1.3c of this license or (at your option) any later
% version. This version of this license is in
%    http://www.latex-project.org/lppl/lppl-1-3c.txt
% and the latest version of this license is in
%    http://www.latex-project.org/lppl.txt
% and version 1.3 or later is part of all distributions of
% LaTeX version 2005/12/01 or later.
%
% This work has the LPPL maintenance status "maintained".
%
% This Current Maintainer of this work is Heiko Oberdiek.
%
% The Base Interpreter refers to any `TeX-Format',
% because some files are installed in TDS:tex/generic//.
%
% This work consists of the main source file iflang.dtx
% and the derived files
%    iflang.sty, iflang.pdf, iflang.ins, iflang.drv, iflang-test1.tex,
%    iflang-test2.tex, iflang-test3.tex, iflang-test4.tex,
%    iflang-test5.tex.
%
% Distribution:
%    CTAN:macros/latex/contrib/oberdiek/iflang.dtx
%    CTAN:macros/latex/contrib/oberdiek/iflang.pdf
%
% Unpacking:
%    (a) If iflang.ins is present:
%           tex iflang.ins
%    (b) Without iflang.ins:
%           tex iflang.dtx
%    (c) If you insist on using LaTeX
%           latex \let\install=y% \iffalse meta-comment
%
% File: iflang.dtx
% Version: 2016/05/16 v1.6
% Info: Checks for the current language
%
% Copyright (C) 2007 by
%    Heiko Oberdiek <heiko.oberdiek at googlemail.com>
%    2016
%    https://github.com/ho-tex/oberdiek/issues
%
% This work may be distributed and/or modified under the
% conditions of the LaTeX Project Public License, either
% version 1.3c of this license or (at your option) any later
% version. This version of this license is in
%    http://www.latex-project.org/lppl/lppl-1-3c.txt
% and the latest version of this license is in
%    http://www.latex-project.org/lppl.txt
% and version 1.3 or later is part of all distributions of
% LaTeX version 2005/12/01 or later.
%
% This work has the LPPL maintenance status "maintained".
%
% This Current Maintainer of this work is Heiko Oberdiek.
%
% The Base Interpreter refers to any `TeX-Format',
% because some files are installed in TDS:tex/generic//.
%
% This work consists of the main source file iflang.dtx
% and the derived files
%    iflang.sty, iflang.pdf, iflang.ins, iflang.drv, iflang-test1.tex,
%    iflang-test2.tex, iflang-test3.tex, iflang-test4.tex,
%    iflang-test5.tex.
%
% Distribution:
%    CTAN:macros/latex/contrib/oberdiek/iflang.dtx
%    CTAN:macros/latex/contrib/oberdiek/iflang.pdf
%
% Unpacking:
%    (a) If iflang.ins is present:
%           tex iflang.ins
%    (b) Without iflang.ins:
%           tex iflang.dtx
%    (c) If you insist on using LaTeX
%           latex \let\install=y\input{iflang.dtx}
%        (quote the arguments according to the demands of your shell)
%
% Documentation:
%    (a) If iflang.drv is present:
%           latex iflang.drv
%    (b) Without iflang.drv:
%           latex iflang.dtx; ...
%    The class ltxdoc loads the configuration file ltxdoc.cfg
%    if available. Here you can specify further options, e.g.
%    use A4 as paper format:
%       \PassOptionsToClass{a4paper}{article}
%
%    Programm calls to get the documentation (example):
%       pdflatex iflang.dtx
%       makeindex -s gind.ist iflang.idx
%       pdflatex iflang.dtx
%       makeindex -s gind.ist iflang.idx
%       pdflatex iflang.dtx
%
% Installation:
%    TDS:tex/generic/oberdiek/iflang.sty
%    TDS:doc/latex/oberdiek/iflang.pdf
%    TDS:doc/latex/oberdiek/test/iflang-test1.tex
%    TDS:doc/latex/oberdiek/test/iflang-test2.tex
%    TDS:doc/latex/oberdiek/test/iflang-test3.tex
%    TDS:doc/latex/oberdiek/test/iflang-test4.tex
%    TDS:doc/latex/oberdiek/test/iflang-test5.tex
%    TDS:source/latex/oberdiek/iflang.dtx
%
%<*ignore>
\begingroup
  \catcode123=1 %
  \catcode125=2 %
  \def\x{LaTeX2e}%
\expandafter\endgroup
\ifcase 0\ifx\install y1\fi\expandafter
         \ifx\csname processbatchFile\endcsname\relax\else1\fi
         \ifx\fmtname\x\else 1\fi\relax
\else\csname fi\endcsname
%</ignore>
%<*install>
\input docstrip.tex
\Msg{************************************************************************}
\Msg{* Installation}
\Msg{* Package: iflang 2016/05/16 v1.6 Checks for the current language (HO)}
\Msg{************************************************************************}

\keepsilent
\askforoverwritefalse

\let\MetaPrefix\relax
\preamble

This is a generated file.

Project: iflang
Version: 2016/05/16 v1.6

Copyright (C) 2007 by
   Heiko Oberdiek <heiko.oberdiek at googlemail.com>

This work may be distributed and/or modified under the
conditions of the LaTeX Project Public License, either
version 1.3c of this license or (at your option) any later
version. This version of this license is in
   http://www.latex-project.org/lppl/lppl-1-3c.txt
and the latest version of this license is in
   http://www.latex-project.org/lppl.txt
and version 1.3 or later is part of all distributions of
LaTeX version 2005/12/01 or later.

This work has the LPPL maintenance status "maintained".

This Current Maintainer of this work is Heiko Oberdiek.

The Base Interpreter refers to any `TeX-Format',
because some files are installed in TDS:tex/generic//.

This work consists of the main source file iflang.dtx
and the derived files
   iflang.sty, iflang.pdf, iflang.ins, iflang.drv, iflang-test1.tex,
   iflang-test2.tex, iflang-test3.tex, iflang-test4.tex,
   iflang-test5.tex.

\endpreamble
\let\MetaPrefix\DoubleperCent

\generate{%
  \file{iflang.ins}{\from{iflang.dtx}{install}}%
  \file{iflang.drv}{\from{iflang.dtx}{driver}}%
  \usedir{tex/generic/oberdiek}%
  \file{iflang.sty}{\from{iflang.dtx}{package}}%
  \usedir{doc/latex/oberdiek/test}%
  \file{iflang-test1.tex}{\from{iflang.dtx}{test1}}%
  \file{iflang-test2.tex}{\from{iflang.dtx}{test2}}%
  \file{iflang-test3.tex}{\from{iflang.dtx}{test3}}%
  \file{iflang-test4.tex}{\from{iflang.dtx}{test4}}%
  \file{iflang-test5.tex}{\from{iflang.dtx}{test5}}%
  \nopreamble
  \nopostamble
  \usedir{source/latex/oberdiek/catalogue}%
  \file{iflang.xml}{\from{iflang.dtx}{catalogue}}%
}

\catcode32=13\relax% active space
\let =\space%
\Msg{************************************************************************}
\Msg{*}
\Msg{* To finish the installation you have to move the following}
\Msg{* file into a directory searched by TeX:}
\Msg{*}
\Msg{*     iflang.sty}
\Msg{*}
\Msg{* To produce the documentation run the file `iflang.drv'}
\Msg{* through LaTeX.}
\Msg{*}
\Msg{* Happy TeXing!}
\Msg{*}
\Msg{************************************************************************}

\endbatchfile
%</install>
%<*ignore>
\fi
%</ignore>
%<*driver>
\NeedsTeXFormat{LaTeX2e}
\ProvidesFile{iflang.drv}%
  [2016/05/16 v1.6 Checks for the current language (HO)]%
\documentclass{ltxdoc}
\usepackage{holtxdoc}[2011/11/22]
\begin{document}
  \DocInput{iflang.dtx}%
\end{document}
%</driver>
% \fi
%
%
% \CharacterTable
%  {Upper-case    \A\B\C\D\E\F\G\H\I\J\K\L\M\N\O\P\Q\R\S\T\U\V\W\X\Y\Z
%   Lower-case    \a\b\c\d\e\f\g\h\i\j\k\l\m\n\o\p\q\r\s\t\u\v\w\x\y\z
%   Digits        \0\1\2\3\4\5\6\7\8\9
%   Exclamation   \!     Double quote  \"     Hash (number) \#
%   Dollar        \$     Percent       \%     Ampersand     \&
%   Acute accent  \'     Left paren    \(     Right paren   \)
%   Asterisk      \*     Plus          \+     Comma         \,
%   Minus         \-     Point         \.     Solidus       \/
%   Colon         \:     Semicolon     \;     Less than     \<
%   Equals        \=     Greater than  \>     Question mark \?
%   Commercial at \@     Left bracket  \[     Backslash     \\
%   Right bracket \]     Circumflex    \^     Underscore    \_
%   Grave accent  \`     Left brace    \{     Vertical bar  \|
%   Right brace   \}     Tilde         \~}
%
% \GetFileInfo{iflang.drv}
%
% \title{The \xpackage{iflang} package}
% \date{2016/05/16 v1.6}
% \author{Heiko Oberdiek\thanks
% {Please report any issues at https://github.com/ho-tex/oberdiek/issues}\\
% \xemail{heiko.oberdiek at googlemail.com}}
%
% \maketitle
%
% \begin{abstract}
% This package provides expandible checks for the current language
% based on macro \cs{languagename} or hyphenation patterns.
% \end{abstract}
%
% \tableofcontents
%
% \section{Documentation}
%
% Package \xpackage{babel} defines \cs{iflanguagename}. As first
% argument it takes a language name and executes the second or
% third argument depending on the current language. This
% language test is based on hypenation patterns. However, it
% is possible that different languages or dialects share the
% same patterns. In such cases \cs{iflanguagename} fails.
%
% However, package \xpackage{babel} and some other packages
% such as \xpackage{german} or \xpackage{ngerman} store
% the language name in the macro \cs{languagename} if
% \cs{selectlanguage} is called.
%
% \begin{declcs}{IfLanguageName} \M{lang} \M{then} \M{else}
% \end{declcs}
% Makro \cs{IfLanguageName} compares language \meta{lang} with
% the current setting of macro \cs{languagename}. If both
% contains the same name then the \meta{then} part is called,
% otherwise the \meta{else} part.
%
% The macro is expandable. Thus it can be safely used inside \cs{edef}
% or \cs{csname}. If case of errors like an undefined \cs{languagename}
% the \meta{else} part is executed.
%
% Note: Macro \cs{IfLanguageName} relies on the fact, that
% \cs{languagename} is set correctly:
% \begin{description}
% \item[Package \xpackage{babel}:]
%    \mbox{}\\
%    Full support of \cs{languagename} in its language switching commands.
% \item[Format based on \xpackage{babel} (\xfile{language.dat}):]
%    \mbox{}\\
%    If package \xpackage{babel} is not used (or not yet loaded),
%    then \xpackage{babel}'s
%    \xfile{hyphen.cfg} has set \cs{languagename} to the last
%    language in \xfile{language.dat}, but \cs{language} (current
%    patterns) is zero and points to the first language.
%    Thus the value of \cs{languagename} is basically garbage.
%    Package \xpackage{iflang} warns if \cs{languagename} and
%    \cs{language} do not fit.
%    This can be fixed by loading package \xpackage{babel} previously.
% \item[Format based on \eTeX's \xfile{etex.src} (\xfile{language.def}):]
%    \mbox{}\\
%    Unhappily it does not support
%    \cs{languagename}. Thus this package hooks into \cs{uselanguage}
%    to get \cs{languagename} defined and updated there.
%    At package loading time the changed \cs{uselanguage} has not
%    been called yet. Thus package \xpackage{iflang} tries
%    |USenglish|. This is the definite default language of \xfile{etex.src}.
%    If the current patterns suit this default language, an
%    undefined \cs{languagename} is set to this language. Otherwise
%    a \cs{languagename} remains undefined and a warning is given.
% \end{description}
%
% \begin{declcs}{IfLanguagePatterns} \M{lang} \M{then} \M{else}
% \end{declcs}
% This macro behaves similar to \cs{IfLanguageName}. But the
% language test is based on the current pattern in force (\cs{language}).
% Also this macro is expandable, in case of errors the \meta{else} part
% is called.
%
% The following naming convention for the pattern are supported:
% \begin{description}
% \item[\xpackage{babel}/\xfile{language.dat}]:
%    \cs{l@}\meta{language}
% \item[\xpackage{etex.src}/\xfile{language.def}]:
%    \cs{lang@}\meta{language}
% \end{description}
% Package \xpackage{iflang} looks for \cs{uselanguage} (defined in
% \xfile{etex.src}) to find out the naming convention in use.
%
% \StopEventually{
% }
%
% \section{Implementation}
%
%    \begin{macrocode}
%<*package>
%    \end{macrocode}
%
% \subsection{Reload check and package identification}
%    Reload check, especially if the package is not used with \LaTeX.
%    \begin{macrocode}
\begingroup\catcode61\catcode48\catcode32=10\relax%
  \catcode13=5 % ^^M
  \endlinechar=13 %
  \catcode35=6 % #
  \catcode39=12 % '
  \catcode44=12 % ,
  \catcode45=12 % -
  \catcode46=12 % .
  \catcode58=12 % :
  \catcode64=11 % @
  \catcode123=1 % {
  \catcode125=2 % }
  \expandafter\let\expandafter\x\csname ver@iflang.sty\endcsname
  \ifx\x\relax % plain-TeX, first loading
  \else
    \def\empty{}%
    \ifx\x\empty % LaTeX, first loading,
      % variable is initialized, but \ProvidesPackage not yet seen
    \else
      \expandafter\ifx\csname PackageInfo\endcsname\relax
        \def\x#1#2{%
          \immediate\write-1{Package #1 Info: #2.}%
        }%
      \else
        \def\x#1#2{\PackageInfo{#1}{#2, stopped}}%
      \fi
      \x{iflang}{The package is already loaded}%
      \aftergroup\endinput
    \fi
  \fi
\endgroup%
%    \end{macrocode}
%    Package identification:
%    \begin{macrocode}
\begingroup\catcode61\catcode48\catcode32=10\relax%
  \catcode13=5 % ^^M
  \endlinechar=13 %
  \catcode35=6 % #
  \catcode39=12 % '
  \catcode40=12 % (
  \catcode41=12 % )
  \catcode44=12 % ,
  \catcode45=12 % -
  \catcode46=12 % .
  \catcode47=12 % /
  \catcode58=12 % :
  \catcode64=11 % @
  \catcode91=12 % [
  \catcode93=12 % ]
  \catcode123=1 % {
  \catcode125=2 % }
  \expandafter\ifx\csname ProvidesPackage\endcsname\relax
    \def\x#1#2#3[#4]{\endgroup
      \immediate\write-1{Package: #3 #4}%
      \xdef#1{#4}%
    }%
  \else
    \def\x#1#2[#3]{\endgroup
      #2[{#3}]%
      \ifx#1\@undefined
        \xdef#1{#3}%
      \fi
      \ifx#1\relax
        \xdef#1{#3}%
      \fi
    }%
  \fi
\expandafter\x\csname ver@iflang.sty\endcsname
\ProvidesPackage{iflang}%
  [2016/05/16 v1.6 Checks for the current language (HO)]%
%    \end{macrocode}
%
%    \begin{macrocode}
\begingroup\catcode61\catcode48\catcode32=10\relax%
  \catcode13=5 % ^^M
  \endlinechar=13 %
  \catcode123=1 % {
  \catcode125=2 % }
  \catcode64=11 % @
  \def\x{\endgroup
    \expandafter\edef\csname IfLang@AtEnd\endcsname{%
      \endlinechar=\the\endlinechar\relax
      \catcode13=\the\catcode13\relax
      \catcode32=\the\catcode32\relax
      \catcode35=\the\catcode35\relax
      \catcode61=\the\catcode61\relax
      \catcode64=\the\catcode64\relax
      \catcode123=\the\catcode123\relax
      \catcode125=\the\catcode125\relax
    }%
  }%
\x\catcode61\catcode48\catcode32=10\relax%
\catcode13=5 % ^^M
\endlinechar=13 %
\catcode35=6 % #
\catcode64=11 % @
\catcode123=1 % {
\catcode125=2 % }
\def\TMP@EnsureCode#1#2{%
  \edef\IfLang@AtEnd{%
    \IfLang@AtEnd
    \catcode#1=\the\catcode#1\relax
  }%
  \catcode#1=#2\relax
}
\TMP@EnsureCode{39}{12}% '
\TMP@EnsureCode{40}{12}% (
\TMP@EnsureCode{41}{12}% )
\TMP@EnsureCode{44}{12}% ,
\TMP@EnsureCode{46}{12}% .
\TMP@EnsureCode{47}{12}% /
\TMP@EnsureCode{58}{12}% :
\TMP@EnsureCode{91}{12}% [
\TMP@EnsureCode{93}{12}% ]
\edef\IfLang@AtEnd{\IfLang@AtEnd\noexpand\endinput}
%    \end{macrocode}
%
% \subsection{Tools}
%
% \subsubsection{Provide some basic macros of \LaTeX}
%
%    \begin{macro}{\@firstoftwo}
%    \begin{macrocode}
\expandafter\ifx\csname @firstoftwo\endcsname\relax
  \long\def\@firstoftwo#1#2{#1}%
\fi
%    \end{macrocode}
%    \end{macro}
%    \begin{macro}{\@secondoftwo}
%    \begin{macrocode}
\expandafter\ifx\csname @secondoftwo\endcsname\relax
  \long\def\@secondoftwo#1#2{#2}%
\fi
%    \end{macrocode}
%    \end{macro}
%
% \subsubsection{Expandible existence check for macros}
%
%    \begin{macro}{\IfLang@IfDefined}
%    \begin{macrocode}
\begingroup\expandafter\expandafter\expandafter\endgroup
\expandafter\ifx\csname ifcsname\endcsname\relax
  \expandafter\@firstoftwo
\else
  \expandafter\@secondoftwo
\fi
{%
  \def\IfLang@IfDefined#1{%
    \expandafter\ifx\csname#1\endcsname\relax
      \expandafter\@secondoftwo
    \else
      \expandafter\@firstoftwo
    \fi
  }%
}{%
  \def\IfLang@IfDefined#1{%
    \ifnum\ifcsname#1\endcsname
            \expandafter\ifx\csname#1\endcsname\relax
              1%
            \else
              0%
            \fi
          \else
            1%
          \fi
          =0 %
      \expandafter\@firstoftwo
    \else
      \expandafter\@secondoftwo
    \fi
  }%
}
%    \end{macrocode}
%    \end{macro}
%
% \subsubsection{Macros for messages}
%
%    \begin{macrocode}
\begingroup\expandafter\expandafter\expandafter\endgroup
\expandafter\ifx\csname RequirePackage\endcsname\relax
  \input infwarerr.sty\relax
  \input pdftexcmds.sty\relax
\else
  \RequirePackage{infwarerr}[2007/09/09]%
  \RequirePackage{pdftexcmds}[2016/05/16]%
\fi
%    \end{macrocode}
%
% \subsubsection{Support for \xfile{etex.src}}
%
%    \begin{macro}{\IfLang@prefix}
%    \begin{macrocode}
\begingroup\expandafter\expandafter\expandafter\endgroup
\expandafter\ifx\csname uselanguage\endcsname\relax
  \@PackageInfoNoLine{iflang}{%
    Naming convention for patterns: babel%
  }%
  \def\IfLang@prefix{l@}%
\else
  \@PackageInfoNoLine{iflang}{%
    Naming convention for patterns: etex.src%
  }%
  \def\IfLang@prefix{lang@}%
  \let\IfLang@OrgUseLanguage\uselanguage
  \def\uselanguage#1{%
    \edef\languagename{#1}%
    \IfLang@OrgUseLanguage{#1}%
  }%
%    \end{macrocode}
%    \end{macro}
%    The first \cs{uselanguage} that is executed as last
%    line in \xfile{language.def} cannot patched this way.
%    However, \xfile{language.def} is very strict. It forces
%    the first added and used language to be |USenglish|.
%    Thus, if \cs{languagename} is not defined, we can
%    quite safely assume |USenglish|. As additional
%    safety precaution the actual used patterns are checked.
%    \begin{macrocode}
  \begingroup\expandafter\expandafter\expandafter\endgroup
  \expandafter\ifx\csname languagename\endcsname\relax
    \begingroup\expandafter\expandafter\expandafter\endgroup
    \expandafter\ifx\csname lang@USenglish\endcsname\relax
      \@PackageWarningNoLine{iflang}{%
        \string\lang@USenglish\space is missing%
      }%
    \else
      \ifnum\lang@USenglish=\language
        \def\languagename{USenglish}%
      \else
        \@PackageWarningNoLine{iflang}{%
          \string\languagename\space is not set,\MessageBreak
          current language is unknown%
        }%
      \fi
    \fi
  \fi
\fi
\begingroup\expandafter\expandafter\expandafter\endgroup
\expandafter\ifx\csname languagename\endcsname\relax
  \@PackageInfoNoLine{iflang}{%
    \string\languagename\space is not set%
  }%
\fi
%    \end{macrocode}
%
% \subsection{\cs{IfLanguagePatterns}}
%
%    \begin{macro}{\IfLanguagePatterns}
%    \begin{macrocode}
\def\IfLanguagePatterns#1{%
  \ifnum\IfLang@IfDefined{\IfLang@prefix#1}{%
          \ifnum\csname\IfLang@prefix#1\endcsname=\language
            0%
          \else
            1%
          \fi
        }{1}=0 %
    \expandafter\@firstoftwo
  \else
    \expandafter\@secondoftwo
  \fi
}
%    \end{macrocode}
%    \end{macro}
%
% \subsection{\cs{IfLanguageName}}
%
%    \begin{macrocode}
\begingroup\expandafter\expandafter\expandafter\endgroup
\expandafter\ifx\csname pdf@strcmp\endcsname\relax
  \expandafter\@firstoftwo
\else
  \expandafter\@secondoftwo
\fi
{%
%    \end{macrocode}
%    We do not have \cs{pdf@strcmp} (and \cs{pdfstrcmp}).
%    Thus we must define
%    our own expandable string comparison. The following
%    implementation is based on a \TeX\ pearl from David Kastrup, presented
%    at the conference Bacho\TeX~2005:
%    \url{http://www-stary.gust.org.pl/pearls/2005/david-kastrup/bachotex2005-david-kastrup-pearl1.pdf}
%
%    The orignal code allows macros inside the second string.
%    Because also \cs{languagename} might consists of further
%    macros, we need a variant that allows macros in the first
%    string, too.
%    \begin{macrocode}
  \def\IfLang@StrNil{\relax}%
  \def\IfLang@StrEqual#1{%
    \number\IfLang@StrEqualStart{}{}#1\IfLang@StrNil
  }%
  \def\IfLang@StrEqualStart#1#2#3{%
    \ifx#3\IfLang@StrNil
      \IfLang@StrEqualStop
    \fi
    \ifcat\noexpand#3\relax
      \IfLang@StrExpand{#1}{#2}#3%
    \fi
    \IfLang@StrEqualStart{\if#3#1}{#2\fi}%
  }%
  \def\IfLang@StrEqualStop\fi#1\IfLang@StrEqualStart#2#3#4{%
    \fi
    #2#4\relax'#313 %
  }%
  \def\IfLang@StrExpand#1#2#3\fi\IfLang@StrEqualStart#4#5{%
    \fi
    \IfLang@@StrExpand{#1}{#2}#3%
  }%
  \def\IfLang@@StrExpand#1#2#3\IfLang@StrNil{%
    \expandafter\IfLang@@@StrExpand#3\IfLang@StrNil{#1}{#2}%
  }%
  \def\IfLang@@@StrExpand#1\IfLang@StrNil#2#3{%
    \IfLang@StrEqualStart{#2}{#3}#1\IfLang@StrNil
  }%
%    \end{macrocode}
%    \begin{macro}{\IfLanguageName}
%    \begin{macrocode}
  \def\IfLanguageName#1{%
    \ifnum\IfLang@IfDefined{languagename}{%
            \if\expandafter\IfLang@StrEqual\expandafter%
                           {\languagename}{#1}%
              0%
            \else
              1%
            \fi
          }{1}=0 %
      \expandafter\@firstoftwo
    \else
      \expandafter\@secondoftwo
    \fi
  }%
%    \end{macrocode}
%    \end{macro}
%    \begin{macrocode}
}{%
%    \end{macrocode}
%    \begin{macro}{\IfLanguageName}
%    \begin{macrocode}
  \def\IfLanguageName#1{%
    \ifnum\IfLang@IfDefined{languagename}{%
            \pdf@strcmp{#1}{\languagename}%
          }{1}=0 %
      \expandafter\@firstoftwo
    \else
      \expandafter\@secondoftwo
    \fi
  }%
%    \end{macrocode}
%    \end{macro}
%    \begin{macrocode}
}
%    \end{macrocode}
%
% \subsection{Check plausibility of \cs{languagename}}
%
%    \begin{macrocode}
\begingroup\expandafter\expandafter\expandafter\endgroup
\expandafter\ifx\csname languagename\endcsname\relax
\else
  \IfLanguagePatterns{\languagename}{}{%
    \@PackageWarningNoLine{iflang}{%
      Mismatch between \string\language\space
      (patterns)\MessageBreak
      and setting of \string\languagename
    }%
  }%
\fi
%    \end{macrocode}
%
%    \begin{macrocode}
\IfLang@AtEnd%
%</package>
%    \end{macrocode}
%
% \section{Test}
%
% \subsection{Catcode checks for loading}
%
%    \begin{macrocode}
%<*test1>
%    \end{macrocode}
%    \begin{macrocode}
\catcode`\{=1 %
\catcode`\}=2 %
\catcode`\#=6 %
\catcode`\@=11 %
\expandafter\ifx\csname count@\endcsname\relax
  \countdef\count@=255 %
\fi
\expandafter\ifx\csname @gobble\endcsname\relax
  \long\def\@gobble#1{}%
\fi
\expandafter\ifx\csname @firstofone\endcsname\relax
  \long\def\@firstofone#1{#1}%
\fi
\expandafter\ifx\csname loop\endcsname\relax
  \expandafter\@firstofone
\else
  \expandafter\@gobble
\fi
{%
  \def\loop#1\repeat{%
    \def\body{#1}%
    \iterate
  }%
  \def\iterate{%
    \body
      \let\next\iterate
    \else
      \let\next\relax
    \fi
    \next
  }%
  \let\repeat=\fi
}%
\def\RestoreCatcodes{}
\count@=0 %
\loop
  \edef\RestoreCatcodes{%
    \RestoreCatcodes
    \catcode\the\count@=\the\catcode\count@\relax
  }%
\ifnum\count@<255 %
  \advance\count@ 1 %
\repeat

\def\RangeCatcodeInvalid#1#2{%
  \count@=#1\relax
  \loop
    \catcode\count@=15 %
  \ifnum\count@<#2\relax
    \advance\count@ 1 %
  \repeat
}
\def\RangeCatcodeCheck#1#2#3{%
  \count@=#1\relax
  \loop
    \ifnum#3=\catcode\count@
    \else
      \errmessage{%
        Character \the\count@\space
        with wrong catcode \the\catcode\count@\space
        instead of \number#3%
      }%
    \fi
  \ifnum\count@<#2\relax
    \advance\count@ 1 %
  \repeat
}
\def\space{ }
\expandafter\ifx\csname LoadCommand\endcsname\relax
  \def\LoadCommand{\input iflang.sty\relax}%
\fi
\def\Test{%
  \RangeCatcodeInvalid{0}{47}%
  \RangeCatcodeInvalid{58}{64}%
  \RangeCatcodeInvalid{91}{96}%
  \RangeCatcodeInvalid{123}{255}%
  \catcode`\@=12 %
  \catcode`\\=0 %
  \catcode`\%=14 %
  \LoadCommand
  \RangeCatcodeCheck{0}{36}{15}%
  \RangeCatcodeCheck{37}{37}{14}%
  \RangeCatcodeCheck{38}{47}{15}%
  \RangeCatcodeCheck{48}{57}{12}%
  \RangeCatcodeCheck{58}{63}{15}%
  \RangeCatcodeCheck{64}{64}{12}%
  \RangeCatcodeCheck{65}{90}{11}%
  \RangeCatcodeCheck{91}{91}{15}%
  \RangeCatcodeCheck{92}{92}{0}%
  \RangeCatcodeCheck{93}{96}{15}%
  \RangeCatcodeCheck{97}{122}{11}%
  \RangeCatcodeCheck{123}{255}{15}%
  \RestoreCatcodes
}
\Test
\csname @@end\endcsname
\end
%    \end{macrocode}
%    \begin{macrocode}
%</test1>
%    \end{macrocode}
%
% \subsection{Test with \LaTeX}
%
%    \begin{macrocode}
%<*test2|test3>
%    \end{macrocode}
%    \begin{macrocode}
\NeedsTeXFormat{LaTeX2e}
%<test3>\let\pdfstrcmp\relax
\nofiles
\documentclass{minimal}
\usepackage{qstest}
\IncludeTests{*}
\LogTests{log}{*}{*}
\usepackage[english,naustrian,ngerman]{babel}
\usepackage{iflang}
\begin{document}
\begin{qstest}{IfLanguagePatterns}{language, pattern}
  \def\test#1#2{%
    \Expect*{\IfLanguagePatterns{#1}{true}{false}}{#2}%
  }%
  \test{ngerman}{true}%
  \test{naustrian}{true}%
  \test{english}{false}%
  \test{foobar}{false}%
\end{qstest}
\begin{qstest}{IfLanguageName}{language, name}
  \def\test#1#2{%
    \Expect*{\IfLanguageName{#1}{true}{false}}{#2}%
  }%
  \test{ngerman}{true}%
  \test{naustrian}{false}%
  \selectlanguage{naustrian}%
  \test{ngerman}{false}%
  \test{naustrian}{true}%
  \test{foobar}{false}%
  %
  \def\languagename{naustrian}%
  \test{naustrian}{true}%
  \test{ngerman}{false}%
  %
  \edef\languagename{\string naustrian}%
  \test{naustrian}{true}%
  \test{ngerman}{false}%
  %
  \def\languagename{naustrian}%
  \makeatletter
  \@onelevel@sanitize\languagename
  \test{naustrian}{true}%
  \test{ngerman}{false}%
  %
  \def\languagename{naustrian}%
  \def\xaustrian{naustrian}%
  \def\xgerman{ngerman}%
  \test{\xaustrian}{true}%
  \test{\xgerman}{false}%
  %
  \def\languagename{\xaustrian}%
  \test{naustrian}{true}%
  \test{ngerman}{false}%
  \test{\xaustrian}{true}%
  \test{\xgerman}{false}%
  \test{\languagename}{true}%
  \test{\languagename\space}{false}%
  %
  \def\languagename{\empty\xaustrian\empty}%
  \test{naustrian}{true}%
  \test{ngerman}{false}%
  \test{\empty\xaustrian\empty}{true}%
  \test{\empty\xgerman\empty}{false}%
\end{qstest}
\begin{qstest}{IfDefined}{defined}
  \makeatletter
  \let\foobar\relax
  \Expect*{\IfLang@IfDefined{foobar}{true}{false}}{false}%
  \Expect*{\ifx\foobar\relax true\else false\fi}{true}%
  \let\foobar\UNDEFINED
  \Expect*{\IfLang@IfDefined{foobar}{true}{false}}{false}%
  \Expect*{\ifx\foobar\relax true\else false\fi}{false}%
  \Expect*{\ifx\foobar\UNDEFINED true\else false\fi}{true}%
\end{qstest}
\end{document}
%    \end{macrocode}
%    \begin{macrocode}
%</test2|test3>
%    \end{macrocode}
%
% \subsection{Test with \plainTeX\ and \eTeX}
%
%    \begin{macrocode}
%<*test4>
%    \end{macrocode}
%    \begin{macrocode}
%% Format `etex' based on `language.def'
\input iflang.sty
\catcode64=12

\def\TestGeneric#1#2#3{%
  \begingroup
    \edef\x{#1{#2}{true}{false}}%
    \edef\y{#3}%
    \ifx\x\y
    \else
      \errmessage{Failed test: \string#1{#2} <> #3}%
    \fi
  \endgroup
}
\def\TestPatterns{\TestGeneric\IfLanguagePatterns}
\def\TestName{\TestGeneric\IfLanguageName}

\TestPatterns{USenglish}{true}
\TestPatterns{ngerman}{false}

\TestName{USenglish}{true}
\TestName{ngerman}{false}

\uselanguage{ngerman}

\TestPatterns{USenglish}{false}
\TestPatterns{ngerman}{true}

\TestName{USenglish}{false}
\TestName{ngerman}{true}

\csname @@end\endcsname
\end
%    \end{macrocode}
%    \begin{macrocode}
%</test4>
%    \end{macrocode}
%
% \subsection{Test with \plainTeX\ and without \eTeX/\pdfTeX}
%
%    \begin{macrocode}
%<*test5>
%    \end{macrocode}
%    \begin{macrocode}
%% Format `tex' (vanilla plain-TeX)
\let\ifcsname\UNDEFINED
\let\pdfstrcmp\UNDEFINED
\input iflang.sty
\catcode64=11

\def\TestDefined#1{%
  \IfLang@IfDefined{foobar}{}{}%
  \ifx\foobar#1%
  \else
    \errmessage{Failed test: \string\foobar <> \string#1}%
  \fi
}
\let\foobar\relax
\TestDefined\relax
\let\foobar\UNDEFINED
\TestDefined\relax

\def\strip@prefix#1>{}
\def\@onelevel@sanitize#1{%
  \edef#1{\expandafter\strip@prefix\meaning#1}%
}
\def\TestCompare#1#2#3{%
  \begingroup
    \edef\x{%
      \if\IfLang@StrEqual{#1}{#2}%
        true%
      \else
        false%
      \fi
    }%
    \def\expect{#3}%
    \ifx\x\expect
    \else
      \def\a{#1}%
      \@onelevel@sanitize\a
      \def\b{#2}%
      \@onelevel@sanitize\b
      \errmessage{Failed test: `\a'='\b' <> \expect}%
    \fi
  \endgroup
}
\TestCompare{junk}{junk}{true}
\TestCompare{}{}{true}
\TestCompare{a}{b}{false}
\TestCompare{aa}{bb}{false}
\def\a{ax}
\def\b{bx}
\def\c{\a\b}
\def\d{\c\b}
\def\exch#1#2{#2#1}
\def\gobble#1{}
\TestCompare{\gobble a}{}{true}
\TestCompare{}{\gobble a}{true}
\TestCompare{a\exch xyb}{ayxb}{true}
\TestCompare{\c}{\c}{true}
\TestCompare{\d}{\c\b}{true}

\csname @@end\endcsname
\end
%    \end{macrocode}
%    \begin{macrocode}
%</test5>
%    \end{macrocode}
%
% \section{Installation}
%
% \subsection{Download}
%
% \paragraph{Package.} This package is available on
% CTAN\footnote{\url{http://ctan.org/pkg/iflang}}:
% \begin{description}
% \item[\CTAN{macros/latex/contrib/oberdiek/iflang.dtx}] The source file.
% \item[\CTAN{macros/latex/contrib/oberdiek/iflang.pdf}] Documentation.
% \end{description}
%
%
% \paragraph{Bundle.} All the packages of the bundle `oberdiek'
% are also available in a TDS compliant ZIP archive. There
% the packages are already unpacked and the documentation files
% are generated. The files and directories obey the TDS standard.
% \begin{description}
% \item[\CTAN{install/macros/latex/contrib/oberdiek.tds.zip}]
% \end{description}
% \emph{TDS} refers to the standard ``A Directory Structure
% for \TeX\ Files'' (\CTAN{tds/tds.pdf}). Directories
% with \xfile{texmf} in their name are usually organized this way.
%
% \subsection{Bundle installation}
%
% \paragraph{Unpacking.} Unpack the \xfile{oberdiek.tds.zip} in the
% TDS tree (also known as \xfile{texmf} tree) of your choice.
% Example (linux):
% \begin{quote}
%   |unzip oberdiek.tds.zip -d ~/texmf|
% \end{quote}
%
% \paragraph{Script installation.}
% Check the directory \xfile{TDS:scripts/oberdiek/} for
% scripts that need further installation steps.
% Package \xpackage{attachfile2} comes with the Perl script
% \xfile{pdfatfi.pl} that should be installed in such a way
% that it can be called as \texttt{pdfatfi}.
% Example (linux):
% \begin{quote}
%   |chmod +x scripts/oberdiek/pdfatfi.pl|\\
%   |cp scripts/oberdiek/pdfatfi.pl /usr/local/bin/|
% \end{quote}
%
% \subsection{Package installation}
%
% \paragraph{Unpacking.} The \xfile{.dtx} file is a self-extracting
% \docstrip\ archive. The files are extracted by running the
% \xfile{.dtx} through \plainTeX:
% \begin{quote}
%   \verb|tex iflang.dtx|
% \end{quote}
%
% \paragraph{TDS.} Now the different files must be moved into
% the different directories in your installation TDS tree
% (also known as \xfile{texmf} tree):
% \begin{quote}
% \def\t{^^A
% \begin{tabular}{@{}>{\ttfamily}l@{ $\rightarrow$ }>{\ttfamily}l@{}}
%   iflang.sty & tex/generic/oberdiek/iflang.sty\\
%   iflang.pdf & doc/latex/oberdiek/iflang.pdf\\
%   test/iflang-test1.tex & doc/latex/oberdiek/test/iflang-test1.tex\\
%   test/iflang-test2.tex & doc/latex/oberdiek/test/iflang-test2.tex\\
%   test/iflang-test3.tex & doc/latex/oberdiek/test/iflang-test3.tex\\
%   test/iflang-test4.tex & doc/latex/oberdiek/test/iflang-test4.tex\\
%   test/iflang-test5.tex & doc/latex/oberdiek/test/iflang-test5.tex\\
%   iflang.dtx & source/latex/oberdiek/iflang.dtx\\
% \end{tabular}^^A
% }^^A
% \sbox0{\t}^^A
% \ifdim\wd0>\linewidth
%   \begingroup
%     \advance\linewidth by\leftmargin
%     \advance\linewidth by\rightmargin
%   \edef\x{\endgroup
%     \def\noexpand\lw{\the\linewidth}^^A
%   }\x
%   \def\lwbox{^^A
%     \leavevmode
%     \hbox to \linewidth{^^A
%       \kern-\leftmargin\relax
%       \hss
%       \usebox0
%       \hss
%       \kern-\rightmargin\relax
%     }^^A
%   }^^A
%   \ifdim\wd0>\lw
%     \sbox0{\small\t}^^A
%     \ifdim\wd0>\linewidth
%       \ifdim\wd0>\lw
%         \sbox0{\footnotesize\t}^^A
%         \ifdim\wd0>\linewidth
%           \ifdim\wd0>\lw
%             \sbox0{\scriptsize\t}^^A
%             \ifdim\wd0>\linewidth
%               \ifdim\wd0>\lw
%                 \sbox0{\tiny\t}^^A
%                 \ifdim\wd0>\linewidth
%                   \lwbox
%                 \else
%                   \usebox0
%                 \fi
%               \else
%                 \lwbox
%               \fi
%             \else
%               \usebox0
%             \fi
%           \else
%             \lwbox
%           \fi
%         \else
%           \usebox0
%         \fi
%       \else
%         \lwbox
%       \fi
%     \else
%       \usebox0
%     \fi
%   \else
%     \lwbox
%   \fi
% \else
%   \usebox0
% \fi
% \end{quote}
% If you have a \xfile{docstrip.cfg} that configures and enables \docstrip's
% TDS installing feature, then some files can already be in the right
% place, see the documentation of \docstrip.
%
% \subsection{Refresh file name databases}
%
% If your \TeX~distribution
% (\teTeX, \mikTeX, \dots) relies on file name databases, you must refresh
% these. For example, \teTeX\ users run \verb|texhash| or
% \verb|mktexlsr|.
%
% \subsection{Some details for the interested}
%
% \paragraph{Attached source.}
%
% The PDF documentation on CTAN also includes the
% \xfile{.dtx} source file. It can be extracted by
% AcrobatReader 6 or higher. Another option is \textsf{pdftk},
% e.g. unpack the file into the current directory:
% \begin{quote}
%   \verb|pdftk iflang.pdf unpack_files output .|
% \end{quote}
%
% \paragraph{Unpacking with \LaTeX.}
% The \xfile{.dtx} chooses its action depending on the format:
% \begin{description}
% \item[\plainTeX:] Run \docstrip\ and extract the files.
% \item[\LaTeX:] Generate the documentation.
% \end{description}
% If you insist on using \LaTeX\ for \docstrip\ (really,
% \docstrip\ does not need \LaTeX), then inform the autodetect routine
% about your intention:
% \begin{quote}
%   \verb|latex \let\install=y\input{iflang.dtx}|
% \end{quote}
% Do not forget to quote the argument according to the demands
% of your shell.
%
% \paragraph{Generating the documentation.}
% You can use both the \xfile{.dtx} or the \xfile{.drv} to generate
% the documentation. The process can be configured by the
% configuration file \xfile{ltxdoc.cfg}. For instance, put this
% line into this file, if you want to have A4 as paper format:
% \begin{quote}
%   \verb|\PassOptionsToClass{a4paper}{article}|
% \end{quote}
% An example follows how to generate the
% documentation with pdf\LaTeX:
% \begin{quote}
%\begin{verbatim}
%pdflatex iflang.dtx
%makeindex -s gind.ist iflang.idx
%pdflatex iflang.dtx
%makeindex -s gind.ist iflang.idx
%pdflatex iflang.dtx
%\end{verbatim}
% \end{quote}
%
% \section{Catalogue}
%
% The following XML file can be used as source for the
% \href{http://mirror.ctan.org/help/Catalogue/catalogue.html}{\TeX\ Catalogue}.
% The elements \texttt{caption} and \texttt{description} are imported
% from the original XML file from the Catalogue.
% The name of the XML file in the Catalogue is \xfile{iflang.xml}.
%    \begin{macrocode}
%<*catalogue>
<?xml version='1.0' encoding='us-ascii'?>
<!DOCTYPE entry SYSTEM 'catalogue.dtd'>
<entry datestamp='$Date$' modifier='$Author$' id='iflang'>
  <name>iflang</name>
  <caption>Expandable checks for the current language.</caption>
  <authorref id='auth:oberdiek'/>
  <copyright owner='Heiko Oberdiek' year='2007'/>
  <license type='lppl1.3'/>
  <version number='1.6'/>
  <description>
    This package provides expandable checks for the current language
    based on macro <tt>\languagename</tt> or hyphenation patterns.
    <p/>
    The package is part of the <xref refid='oberdiek'>oberdiek</xref> bundle.
  </description>
  <documentation details='Package documentation'
      href='ctan:/macros/latex/contrib/oberdiek/iflang.pdf'/>
  <ctan file='true' path='/macros/latex/contrib/oberdiek/iflang.dtx'/>
  <miktex location='oberdiek'/>
  <texlive location='oberdiek'/>
  <install path='/macros/latex/contrib/oberdiek/oberdiek.tds.zip'/>
</entry>
%</catalogue>
%    \end{macrocode}
%
% \section{Acknowledgement}
% I wish to thank:
% \begin{description}
% \item[Markus Kohm] Useful hints for version 1.2.
% \end{description}
%
% \begin{History}
%   \begin{Version}{2007/04/10 v1.0}
%   \item
%     First public version.
%   \end{Version}
%   \begin{Version}{2007/04/11 v1.1}
%   \item
%     Line ends sanitized.
%   \end{Version}
%   \begin{Version}{2007/04/12 v1.2}
%   \item
%     Initialization of \cs{languagename} in case of \xfile{etex.src}.
%   \item
%     Some sanity tests added.
%   \item
%     Documentation improved.
%   \end{Version}
%   \begin{Version}{2007/04/26 v1.3}
%   \item
%     Use of package \xpackage{infwarerr}.
%   \end{Version}
%   \begin{Version}{2007/09/09 v1.4}
%   \item
%     Bug fix: \cs{IfLang@StrEqual} $\rightarrow$ \cs{IfLangStrEqual}
%     (Gabriele Balducci).
%   \item
%     Catcode section rewritten.
%   \end{Version}
%   \begin{Version}{2007/11/11 v1.5}
%   \item
%     Use of package \xpackage{pdftexcmds} for \LuaTeX\ support.
%   \end{Version}
%   \begin{Version}{2016/05/16 v1.6}
%   \item
%     Documentation updates.
%   \end{Version}
% \end{History}
%
% \PrintIndex
%
% \Finale
\endinput

%        (quote the arguments according to the demands of your shell)
%
% Documentation:
%    (a) If iflang.drv is present:
%           latex iflang.drv
%    (b) Without iflang.drv:
%           latex iflang.dtx; ...
%    The class ltxdoc loads the configuration file ltxdoc.cfg
%    if available. Here you can specify further options, e.g.
%    use A4 as paper format:
%       \PassOptionsToClass{a4paper}{article}
%
%    Programm calls to get the documentation (example):
%       pdflatex iflang.dtx
%       makeindex -s gind.ist iflang.idx
%       pdflatex iflang.dtx
%       makeindex -s gind.ist iflang.idx
%       pdflatex iflang.dtx
%
% Installation:
%    TDS:tex/generic/oberdiek/iflang.sty
%    TDS:doc/latex/oberdiek/iflang.pdf
%    TDS:doc/latex/oberdiek/test/iflang-test1.tex
%    TDS:doc/latex/oberdiek/test/iflang-test2.tex
%    TDS:doc/latex/oberdiek/test/iflang-test3.tex
%    TDS:doc/latex/oberdiek/test/iflang-test4.tex
%    TDS:doc/latex/oberdiek/test/iflang-test5.tex
%    TDS:source/latex/oberdiek/iflang.dtx
%
%<*ignore>
\begingroup
  \catcode123=1 %
  \catcode125=2 %
  \def\x{LaTeX2e}%
\expandafter\endgroup
\ifcase 0\ifx\install y1\fi\expandafter
         \ifx\csname processbatchFile\endcsname\relax\else1\fi
         \ifx\fmtname\x\else 1\fi\relax
\else\csname fi\endcsname
%</ignore>
%<*install>
\input docstrip.tex
\Msg{************************************************************************}
\Msg{* Installation}
\Msg{* Package: iflang 2016/05/16 v1.6 Checks for the current language (HO)}
\Msg{************************************************************************}

\keepsilent
\askforoverwritefalse

\let\MetaPrefix\relax
\preamble

This is a generated file.

Project: iflang
Version: 2016/05/16 v1.6

Copyright (C) 2007 by
   Heiko Oberdiek <heiko.oberdiek at googlemail.com>

This work may be distributed and/or modified under the
conditions of the LaTeX Project Public License, either
version 1.3c of this license or (at your option) any later
version. This version of this license is in
   http://www.latex-project.org/lppl/lppl-1-3c.txt
and the latest version of this license is in
   http://www.latex-project.org/lppl.txt
and version 1.3 or later is part of all distributions of
LaTeX version 2005/12/01 or later.

This work has the LPPL maintenance status "maintained".

This Current Maintainer of this work is Heiko Oberdiek.

The Base Interpreter refers to any `TeX-Format',
because some files are installed in TDS:tex/generic//.

This work consists of the main source file iflang.dtx
and the derived files
   iflang.sty, iflang.pdf, iflang.ins, iflang.drv, iflang-test1.tex,
   iflang-test2.tex, iflang-test3.tex, iflang-test4.tex,
   iflang-test5.tex.

\endpreamble
\let\MetaPrefix\DoubleperCent

\generate{%
  \file{iflang.ins}{\from{iflang.dtx}{install}}%
  \file{iflang.drv}{\from{iflang.dtx}{driver}}%
  \usedir{tex/generic/oberdiek}%
  \file{iflang.sty}{\from{iflang.dtx}{package}}%
  \usedir{doc/latex/oberdiek/test}%
  \file{iflang-test1.tex}{\from{iflang.dtx}{test1}}%
  \file{iflang-test2.tex}{\from{iflang.dtx}{test2}}%
  \file{iflang-test3.tex}{\from{iflang.dtx}{test3}}%
  \file{iflang-test4.tex}{\from{iflang.dtx}{test4}}%
  \file{iflang-test5.tex}{\from{iflang.dtx}{test5}}%
  \nopreamble
  \nopostamble
  \usedir{source/latex/oberdiek/catalogue}%
  \file{iflang.xml}{\from{iflang.dtx}{catalogue}}%
}

\catcode32=13\relax% active space
\let =\space%
\Msg{************************************************************************}
\Msg{*}
\Msg{* To finish the installation you have to move the following}
\Msg{* file into a directory searched by TeX:}
\Msg{*}
\Msg{*     iflang.sty}
\Msg{*}
\Msg{* To produce the documentation run the file `iflang.drv'}
\Msg{* through LaTeX.}
\Msg{*}
\Msg{* Happy TeXing!}
\Msg{*}
\Msg{************************************************************************}

\endbatchfile
%</install>
%<*ignore>
\fi
%</ignore>
%<*driver>
\NeedsTeXFormat{LaTeX2e}
\ProvidesFile{iflang.drv}%
  [2016/05/16 v1.6 Checks for the current language (HO)]%
\documentclass{ltxdoc}
\usepackage{holtxdoc}[2011/11/22]
\begin{document}
  \DocInput{iflang.dtx}%
\end{document}
%</driver>
% \fi
%
%
% \CharacterTable
%  {Upper-case    \A\B\C\D\E\F\G\H\I\J\K\L\M\N\O\P\Q\R\S\T\U\V\W\X\Y\Z
%   Lower-case    \a\b\c\d\e\f\g\h\i\j\k\l\m\n\o\p\q\r\s\t\u\v\w\x\y\z
%   Digits        \0\1\2\3\4\5\6\7\8\9
%   Exclamation   \!     Double quote  \"     Hash (number) \#
%   Dollar        \$     Percent       \%     Ampersand     \&
%   Acute accent  \'     Left paren    \(     Right paren   \)
%   Asterisk      \*     Plus          \+     Comma         \,
%   Minus         \-     Point         \.     Solidus       \/
%   Colon         \:     Semicolon     \;     Less than     \<
%   Equals        \=     Greater than  \>     Question mark \?
%   Commercial at \@     Left bracket  \[     Backslash     \\
%   Right bracket \]     Circumflex    \^     Underscore    \_
%   Grave accent  \`     Left brace    \{     Vertical bar  \|
%   Right brace   \}     Tilde         \~}
%
% \GetFileInfo{iflang.drv}
%
% \title{The \xpackage{iflang} package}
% \date{2016/05/16 v1.6}
% \author{Heiko Oberdiek\thanks
% {Please report any issues at https://github.com/ho-tex/oberdiek/issues}\\
% \xemail{heiko.oberdiek at googlemail.com}}
%
% \maketitle
%
% \begin{abstract}
% This package provides expandible checks for the current language
% based on macro \cs{languagename} or hyphenation patterns.
% \end{abstract}
%
% \tableofcontents
%
% \section{Documentation}
%
% Package \xpackage{babel} defines \cs{iflanguagename}. As first
% argument it takes a language name and executes the second or
% third argument depending on the current language. This
% language test is based on hypenation patterns. However, it
% is possible that different languages or dialects share the
% same patterns. In such cases \cs{iflanguagename} fails.
%
% However, package \xpackage{babel} and some other packages
% such as \xpackage{german} or \xpackage{ngerman} store
% the language name in the macro \cs{languagename} if
% \cs{selectlanguage} is called.
%
% \begin{declcs}{IfLanguageName} \M{lang} \M{then} \M{else}
% \end{declcs}
% Makro \cs{IfLanguageName} compares language \meta{lang} with
% the current setting of macro \cs{languagename}. If both
% contains the same name then the \meta{then} part is called,
% otherwise the \meta{else} part.
%
% The macro is expandable. Thus it can be safely used inside \cs{edef}
% or \cs{csname}. If case of errors like an undefined \cs{languagename}
% the \meta{else} part is executed.
%
% Note: Macro \cs{IfLanguageName} relies on the fact, that
% \cs{languagename} is set correctly:
% \begin{description}
% \item[Package \xpackage{babel}:]
%    \mbox{}\\
%    Full support of \cs{languagename} in its language switching commands.
% \item[Format based on \xpackage{babel} (\xfile{language.dat}):]
%    \mbox{}\\
%    If package \xpackage{babel} is not used (or not yet loaded),
%    then \xpackage{babel}'s
%    \xfile{hyphen.cfg} has set \cs{languagename} to the last
%    language in \xfile{language.dat}, but \cs{language} (current
%    patterns) is zero and points to the first language.
%    Thus the value of \cs{languagename} is basically garbage.
%    Package \xpackage{iflang} warns if \cs{languagename} and
%    \cs{language} do not fit.
%    This can be fixed by loading package \xpackage{babel} previously.
% \item[Format based on \eTeX's \xfile{etex.src} (\xfile{language.def}):]
%    \mbox{}\\
%    Unhappily it does not support
%    \cs{languagename}. Thus this package hooks into \cs{uselanguage}
%    to get \cs{languagename} defined and updated there.
%    At package loading time the changed \cs{uselanguage} has not
%    been called yet. Thus package \xpackage{iflang} tries
%    |USenglish|. This is the definite default language of \xfile{etex.src}.
%    If the current patterns suit this default language, an
%    undefined \cs{languagename} is set to this language. Otherwise
%    a \cs{languagename} remains undefined and a warning is given.
% \end{description}
%
% \begin{declcs}{IfLanguagePatterns} \M{lang} \M{then} \M{else}
% \end{declcs}
% This macro behaves similar to \cs{IfLanguageName}. But the
% language test is based on the current pattern in force (\cs{language}).
% Also this macro is expandable, in case of errors the \meta{else} part
% is called.
%
% The following naming convention for the pattern are supported:
% \begin{description}
% \item[\xpackage{babel}/\xfile{language.dat}]:
%    \cs{l@}\meta{language}
% \item[\xpackage{etex.src}/\xfile{language.def}]:
%    \cs{lang@}\meta{language}
% \end{description}
% Package \xpackage{iflang} looks for \cs{uselanguage} (defined in
% \xfile{etex.src}) to find out the naming convention in use.
%
% \StopEventually{
% }
%
% \section{Implementation}
%
%    \begin{macrocode}
%<*package>
%    \end{macrocode}
%
% \subsection{Reload check and package identification}
%    Reload check, especially if the package is not used with \LaTeX.
%    \begin{macrocode}
\begingroup\catcode61\catcode48\catcode32=10\relax%
  \catcode13=5 % ^^M
  \endlinechar=13 %
  \catcode35=6 % #
  \catcode39=12 % '
  \catcode44=12 % ,
  \catcode45=12 % -
  \catcode46=12 % .
  \catcode58=12 % :
  \catcode64=11 % @
  \catcode123=1 % {
  \catcode125=2 % }
  \expandafter\let\expandafter\x\csname ver@iflang.sty\endcsname
  \ifx\x\relax % plain-TeX, first loading
  \else
    \def\empty{}%
    \ifx\x\empty % LaTeX, first loading,
      % variable is initialized, but \ProvidesPackage not yet seen
    \else
      \expandafter\ifx\csname PackageInfo\endcsname\relax
        \def\x#1#2{%
          \immediate\write-1{Package #1 Info: #2.}%
        }%
      \else
        \def\x#1#2{\PackageInfo{#1}{#2, stopped}}%
      \fi
      \x{iflang}{The package is already loaded}%
      \aftergroup\endinput
    \fi
  \fi
\endgroup%
%    \end{macrocode}
%    Package identification:
%    \begin{macrocode}
\begingroup\catcode61\catcode48\catcode32=10\relax%
  \catcode13=5 % ^^M
  \endlinechar=13 %
  \catcode35=6 % #
  \catcode39=12 % '
  \catcode40=12 % (
  \catcode41=12 % )
  \catcode44=12 % ,
  \catcode45=12 % -
  \catcode46=12 % .
  \catcode47=12 % /
  \catcode58=12 % :
  \catcode64=11 % @
  \catcode91=12 % [
  \catcode93=12 % ]
  \catcode123=1 % {
  \catcode125=2 % }
  \expandafter\ifx\csname ProvidesPackage\endcsname\relax
    \def\x#1#2#3[#4]{\endgroup
      \immediate\write-1{Package: #3 #4}%
      \xdef#1{#4}%
    }%
  \else
    \def\x#1#2[#3]{\endgroup
      #2[{#3}]%
      \ifx#1\@undefined
        \xdef#1{#3}%
      \fi
      \ifx#1\relax
        \xdef#1{#3}%
      \fi
    }%
  \fi
\expandafter\x\csname ver@iflang.sty\endcsname
\ProvidesPackage{iflang}%
  [2016/05/16 v1.6 Checks for the current language (HO)]%
%    \end{macrocode}
%
%    \begin{macrocode}
\begingroup\catcode61\catcode48\catcode32=10\relax%
  \catcode13=5 % ^^M
  \endlinechar=13 %
  \catcode123=1 % {
  \catcode125=2 % }
  \catcode64=11 % @
  \def\x{\endgroup
    \expandafter\edef\csname IfLang@AtEnd\endcsname{%
      \endlinechar=\the\endlinechar\relax
      \catcode13=\the\catcode13\relax
      \catcode32=\the\catcode32\relax
      \catcode35=\the\catcode35\relax
      \catcode61=\the\catcode61\relax
      \catcode64=\the\catcode64\relax
      \catcode123=\the\catcode123\relax
      \catcode125=\the\catcode125\relax
    }%
  }%
\x\catcode61\catcode48\catcode32=10\relax%
\catcode13=5 % ^^M
\endlinechar=13 %
\catcode35=6 % #
\catcode64=11 % @
\catcode123=1 % {
\catcode125=2 % }
\def\TMP@EnsureCode#1#2{%
  \edef\IfLang@AtEnd{%
    \IfLang@AtEnd
    \catcode#1=\the\catcode#1\relax
  }%
  \catcode#1=#2\relax
}
\TMP@EnsureCode{39}{12}% '
\TMP@EnsureCode{40}{12}% (
\TMP@EnsureCode{41}{12}% )
\TMP@EnsureCode{44}{12}% ,
\TMP@EnsureCode{46}{12}% .
\TMP@EnsureCode{47}{12}% /
\TMP@EnsureCode{58}{12}% :
\TMP@EnsureCode{91}{12}% [
\TMP@EnsureCode{93}{12}% ]
\edef\IfLang@AtEnd{\IfLang@AtEnd\noexpand\endinput}
%    \end{macrocode}
%
% \subsection{Tools}
%
% \subsubsection{Provide some basic macros of \LaTeX}
%
%    \begin{macro}{\@firstoftwo}
%    \begin{macrocode}
\expandafter\ifx\csname @firstoftwo\endcsname\relax
  \long\def\@firstoftwo#1#2{#1}%
\fi
%    \end{macrocode}
%    \end{macro}
%    \begin{macro}{\@secondoftwo}
%    \begin{macrocode}
\expandafter\ifx\csname @secondoftwo\endcsname\relax
  \long\def\@secondoftwo#1#2{#2}%
\fi
%    \end{macrocode}
%    \end{macro}
%
% \subsubsection{Expandible existence check for macros}
%
%    \begin{macro}{\IfLang@IfDefined}
%    \begin{macrocode}
\begingroup\expandafter\expandafter\expandafter\endgroup
\expandafter\ifx\csname ifcsname\endcsname\relax
  \expandafter\@firstoftwo
\else
  \expandafter\@secondoftwo
\fi
{%
  \def\IfLang@IfDefined#1{%
    \expandafter\ifx\csname#1\endcsname\relax
      \expandafter\@secondoftwo
    \else
      \expandafter\@firstoftwo
    \fi
  }%
}{%
  \def\IfLang@IfDefined#1{%
    \ifnum\ifcsname#1\endcsname
            \expandafter\ifx\csname#1\endcsname\relax
              1%
            \else
              0%
            \fi
          \else
            1%
          \fi
          =0 %
      \expandafter\@firstoftwo
    \else
      \expandafter\@secondoftwo
    \fi
  }%
}
%    \end{macrocode}
%    \end{macro}
%
% \subsubsection{Macros for messages}
%
%    \begin{macrocode}
\begingroup\expandafter\expandafter\expandafter\endgroup
\expandafter\ifx\csname RequirePackage\endcsname\relax
  \input infwarerr.sty\relax
  \input pdftexcmds.sty\relax
\else
  \RequirePackage{infwarerr}[2007/09/09]%
  \RequirePackage{pdftexcmds}[2016/05/16]%
\fi
%    \end{macrocode}
%
% \subsubsection{Support for \xfile{etex.src}}
%
%    \begin{macro}{\IfLang@prefix}
%    \begin{macrocode}
\begingroup\expandafter\expandafter\expandafter\endgroup
\expandafter\ifx\csname uselanguage\endcsname\relax
  \@PackageInfoNoLine{iflang}{%
    Naming convention for patterns: babel%
  }%
  \def\IfLang@prefix{l@}%
\else
  \@PackageInfoNoLine{iflang}{%
    Naming convention for patterns: etex.src%
  }%
  \def\IfLang@prefix{lang@}%
  \let\IfLang@OrgUseLanguage\uselanguage
  \def\uselanguage#1{%
    \edef\languagename{#1}%
    \IfLang@OrgUseLanguage{#1}%
  }%
%    \end{macrocode}
%    \end{macro}
%    The first \cs{uselanguage} that is executed as last
%    line in \xfile{language.def} cannot patched this way.
%    However, \xfile{language.def} is very strict. It forces
%    the first added and used language to be |USenglish|.
%    Thus, if \cs{languagename} is not defined, we can
%    quite safely assume |USenglish|. As additional
%    safety precaution the actual used patterns are checked.
%    \begin{macrocode}
  \begingroup\expandafter\expandafter\expandafter\endgroup
  \expandafter\ifx\csname languagename\endcsname\relax
    \begingroup\expandafter\expandafter\expandafter\endgroup
    \expandafter\ifx\csname lang@USenglish\endcsname\relax
      \@PackageWarningNoLine{iflang}{%
        \string\lang@USenglish\space is missing%
      }%
    \else
      \ifnum\lang@USenglish=\language
        \def\languagename{USenglish}%
      \else
        \@PackageWarningNoLine{iflang}{%
          \string\languagename\space is not set,\MessageBreak
          current language is unknown%
        }%
      \fi
    \fi
  \fi
\fi
\begingroup\expandafter\expandafter\expandafter\endgroup
\expandafter\ifx\csname languagename\endcsname\relax
  \@PackageInfoNoLine{iflang}{%
    \string\languagename\space is not set%
  }%
\fi
%    \end{macrocode}
%
% \subsection{\cs{IfLanguagePatterns}}
%
%    \begin{macro}{\IfLanguagePatterns}
%    \begin{macrocode}
\def\IfLanguagePatterns#1{%
  \ifnum\IfLang@IfDefined{\IfLang@prefix#1}{%
          \ifnum\csname\IfLang@prefix#1\endcsname=\language
            0%
          \else
            1%
          \fi
        }{1}=0 %
    \expandafter\@firstoftwo
  \else
    \expandafter\@secondoftwo
  \fi
}
%    \end{macrocode}
%    \end{macro}
%
% \subsection{\cs{IfLanguageName}}
%
%    \begin{macrocode}
\begingroup\expandafter\expandafter\expandafter\endgroup
\expandafter\ifx\csname pdf@strcmp\endcsname\relax
  \expandafter\@firstoftwo
\else
  \expandafter\@secondoftwo
\fi
{%
%    \end{macrocode}
%    We do not have \cs{pdf@strcmp} (and \cs{pdfstrcmp}).
%    Thus we must define
%    our own expandable string comparison. The following
%    implementation is based on a \TeX\ pearl from David Kastrup, presented
%    at the conference Bacho\TeX~2005:
%    \url{http://www-stary.gust.org.pl/pearls/2005/david-kastrup/bachotex2005-david-kastrup-pearl1.pdf}
%
%    The orignal code allows macros inside the second string.
%    Because also \cs{languagename} might consists of further
%    macros, we need a variant that allows macros in the first
%    string, too.
%    \begin{macrocode}
  \def\IfLang@StrNil{\relax}%
  \def\IfLang@StrEqual#1{%
    \number\IfLang@StrEqualStart{}{}#1\IfLang@StrNil
  }%
  \def\IfLang@StrEqualStart#1#2#3{%
    \ifx#3\IfLang@StrNil
      \IfLang@StrEqualStop
    \fi
    \ifcat\noexpand#3\relax
      \IfLang@StrExpand{#1}{#2}#3%
    \fi
    \IfLang@StrEqualStart{\if#3#1}{#2\fi}%
  }%
  \def\IfLang@StrEqualStop\fi#1\IfLang@StrEqualStart#2#3#4{%
    \fi
    #2#4\relax'#313 %
  }%
  \def\IfLang@StrExpand#1#2#3\fi\IfLang@StrEqualStart#4#5{%
    \fi
    \IfLang@@StrExpand{#1}{#2}#3%
  }%
  \def\IfLang@@StrExpand#1#2#3\IfLang@StrNil{%
    \expandafter\IfLang@@@StrExpand#3\IfLang@StrNil{#1}{#2}%
  }%
  \def\IfLang@@@StrExpand#1\IfLang@StrNil#2#3{%
    \IfLang@StrEqualStart{#2}{#3}#1\IfLang@StrNil
  }%
%    \end{macrocode}
%    \begin{macro}{\IfLanguageName}
%    \begin{macrocode}
  \def\IfLanguageName#1{%
    \ifnum\IfLang@IfDefined{languagename}{%
            \if\expandafter\IfLang@StrEqual\expandafter%
                           {\languagename}{#1}%
              0%
            \else
              1%
            \fi
          }{1}=0 %
      \expandafter\@firstoftwo
    \else
      \expandafter\@secondoftwo
    \fi
  }%
%    \end{macrocode}
%    \end{macro}
%    \begin{macrocode}
}{%
%    \end{macrocode}
%    \begin{macro}{\IfLanguageName}
%    \begin{macrocode}
  \def\IfLanguageName#1{%
    \ifnum\IfLang@IfDefined{languagename}{%
            \pdf@strcmp{#1}{\languagename}%
          }{1}=0 %
      \expandafter\@firstoftwo
    \else
      \expandafter\@secondoftwo
    \fi
  }%
%    \end{macrocode}
%    \end{macro}
%    \begin{macrocode}
}
%    \end{macrocode}
%
% \subsection{Check plausibility of \cs{languagename}}
%
%    \begin{macrocode}
\begingroup\expandafter\expandafter\expandafter\endgroup
\expandafter\ifx\csname languagename\endcsname\relax
\else
  \IfLanguagePatterns{\languagename}{}{%
    \@PackageWarningNoLine{iflang}{%
      Mismatch between \string\language\space
      (patterns)\MessageBreak
      and setting of \string\languagename
    }%
  }%
\fi
%    \end{macrocode}
%
%    \begin{macrocode}
\IfLang@AtEnd%
%</package>
%    \end{macrocode}
%
% \section{Test}
%
% \subsection{Catcode checks for loading}
%
%    \begin{macrocode}
%<*test1>
%    \end{macrocode}
%    \begin{macrocode}
\catcode`\{=1 %
\catcode`\}=2 %
\catcode`\#=6 %
\catcode`\@=11 %
\expandafter\ifx\csname count@\endcsname\relax
  \countdef\count@=255 %
\fi
\expandafter\ifx\csname @gobble\endcsname\relax
  \long\def\@gobble#1{}%
\fi
\expandafter\ifx\csname @firstofone\endcsname\relax
  \long\def\@firstofone#1{#1}%
\fi
\expandafter\ifx\csname loop\endcsname\relax
  \expandafter\@firstofone
\else
  \expandafter\@gobble
\fi
{%
  \def\loop#1\repeat{%
    \def\body{#1}%
    \iterate
  }%
  \def\iterate{%
    \body
      \let\next\iterate
    \else
      \let\next\relax
    \fi
    \next
  }%
  \let\repeat=\fi
}%
\def\RestoreCatcodes{}
\count@=0 %
\loop
  \edef\RestoreCatcodes{%
    \RestoreCatcodes
    \catcode\the\count@=\the\catcode\count@\relax
  }%
\ifnum\count@<255 %
  \advance\count@ 1 %
\repeat

\def\RangeCatcodeInvalid#1#2{%
  \count@=#1\relax
  \loop
    \catcode\count@=15 %
  \ifnum\count@<#2\relax
    \advance\count@ 1 %
  \repeat
}
\def\RangeCatcodeCheck#1#2#3{%
  \count@=#1\relax
  \loop
    \ifnum#3=\catcode\count@
    \else
      \errmessage{%
        Character \the\count@\space
        with wrong catcode \the\catcode\count@\space
        instead of \number#3%
      }%
    \fi
  \ifnum\count@<#2\relax
    \advance\count@ 1 %
  \repeat
}
\def\space{ }
\expandafter\ifx\csname LoadCommand\endcsname\relax
  \def\LoadCommand{\input iflang.sty\relax}%
\fi
\def\Test{%
  \RangeCatcodeInvalid{0}{47}%
  \RangeCatcodeInvalid{58}{64}%
  \RangeCatcodeInvalid{91}{96}%
  \RangeCatcodeInvalid{123}{255}%
  \catcode`\@=12 %
  \catcode`\\=0 %
  \catcode`\%=14 %
  \LoadCommand
  \RangeCatcodeCheck{0}{36}{15}%
  \RangeCatcodeCheck{37}{37}{14}%
  \RangeCatcodeCheck{38}{47}{15}%
  \RangeCatcodeCheck{48}{57}{12}%
  \RangeCatcodeCheck{58}{63}{15}%
  \RangeCatcodeCheck{64}{64}{12}%
  \RangeCatcodeCheck{65}{90}{11}%
  \RangeCatcodeCheck{91}{91}{15}%
  \RangeCatcodeCheck{92}{92}{0}%
  \RangeCatcodeCheck{93}{96}{15}%
  \RangeCatcodeCheck{97}{122}{11}%
  \RangeCatcodeCheck{123}{255}{15}%
  \RestoreCatcodes
}
\Test
\csname @@end\endcsname
\end
%    \end{macrocode}
%    \begin{macrocode}
%</test1>
%    \end{macrocode}
%
% \subsection{Test with \LaTeX}
%
%    \begin{macrocode}
%<*test2|test3>
%    \end{macrocode}
%    \begin{macrocode}
\NeedsTeXFormat{LaTeX2e}
%<test3>\let\pdfstrcmp\relax
\nofiles
\documentclass{minimal}
\usepackage{qstest}
\IncludeTests{*}
\LogTests{log}{*}{*}
\usepackage[english,naustrian,ngerman]{babel}
\usepackage{iflang}
\begin{document}
\begin{qstest}{IfLanguagePatterns}{language, pattern}
  \def\test#1#2{%
    \Expect*{\IfLanguagePatterns{#1}{true}{false}}{#2}%
  }%
  \test{ngerman}{true}%
  \test{naustrian}{true}%
  \test{english}{false}%
  \test{foobar}{false}%
\end{qstest}
\begin{qstest}{IfLanguageName}{language, name}
  \def\test#1#2{%
    \Expect*{\IfLanguageName{#1}{true}{false}}{#2}%
  }%
  \test{ngerman}{true}%
  \test{naustrian}{false}%
  \selectlanguage{naustrian}%
  \test{ngerman}{false}%
  \test{naustrian}{true}%
  \test{foobar}{false}%
  %
  \def\languagename{naustrian}%
  \test{naustrian}{true}%
  \test{ngerman}{false}%
  %
  \edef\languagename{\string naustrian}%
  \test{naustrian}{true}%
  \test{ngerman}{false}%
  %
  \def\languagename{naustrian}%
  \makeatletter
  \@onelevel@sanitize\languagename
  \test{naustrian}{true}%
  \test{ngerman}{false}%
  %
  \def\languagename{naustrian}%
  \def\xaustrian{naustrian}%
  \def\xgerman{ngerman}%
  \test{\xaustrian}{true}%
  \test{\xgerman}{false}%
  %
  \def\languagename{\xaustrian}%
  \test{naustrian}{true}%
  \test{ngerman}{false}%
  \test{\xaustrian}{true}%
  \test{\xgerman}{false}%
  \test{\languagename}{true}%
  \test{\languagename\space}{false}%
  %
  \def\languagename{\empty\xaustrian\empty}%
  \test{naustrian}{true}%
  \test{ngerman}{false}%
  \test{\empty\xaustrian\empty}{true}%
  \test{\empty\xgerman\empty}{false}%
\end{qstest}
\begin{qstest}{IfDefined}{defined}
  \makeatletter
  \let\foobar\relax
  \Expect*{\IfLang@IfDefined{foobar}{true}{false}}{false}%
  \Expect*{\ifx\foobar\relax true\else false\fi}{true}%
  \let\foobar\UNDEFINED
  \Expect*{\IfLang@IfDefined{foobar}{true}{false}}{false}%
  \Expect*{\ifx\foobar\relax true\else false\fi}{false}%
  \Expect*{\ifx\foobar\UNDEFINED true\else false\fi}{true}%
\end{qstest}
\end{document}
%    \end{macrocode}
%    \begin{macrocode}
%</test2|test3>
%    \end{macrocode}
%
% \subsection{Test with \plainTeX\ and \eTeX}
%
%    \begin{macrocode}
%<*test4>
%    \end{macrocode}
%    \begin{macrocode}
%% Format `etex' based on `language.def'
\input iflang.sty
\catcode64=12

\def\TestGeneric#1#2#3{%
  \begingroup
    \edef\x{#1{#2}{true}{false}}%
    \edef\y{#3}%
    \ifx\x\y
    \else
      \errmessage{Failed test: \string#1{#2} <> #3}%
    \fi
  \endgroup
}
\def\TestPatterns{\TestGeneric\IfLanguagePatterns}
\def\TestName{\TestGeneric\IfLanguageName}

\TestPatterns{USenglish}{true}
\TestPatterns{ngerman}{false}

\TestName{USenglish}{true}
\TestName{ngerman}{false}

\uselanguage{ngerman}

\TestPatterns{USenglish}{false}
\TestPatterns{ngerman}{true}

\TestName{USenglish}{false}
\TestName{ngerman}{true}

\csname @@end\endcsname
\end
%    \end{macrocode}
%    \begin{macrocode}
%</test4>
%    \end{macrocode}
%
% \subsection{Test with \plainTeX\ and without \eTeX/\pdfTeX}
%
%    \begin{macrocode}
%<*test5>
%    \end{macrocode}
%    \begin{macrocode}
%% Format `tex' (vanilla plain-TeX)
\let\ifcsname\UNDEFINED
\let\pdfstrcmp\UNDEFINED
\input iflang.sty
\catcode64=11

\def\TestDefined#1{%
  \IfLang@IfDefined{foobar}{}{}%
  \ifx\foobar#1%
  \else
    \errmessage{Failed test: \string\foobar <> \string#1}%
  \fi
}
\let\foobar\relax
\TestDefined\relax
\let\foobar\UNDEFINED
\TestDefined\relax

\def\strip@prefix#1>{}
\def\@onelevel@sanitize#1{%
  \edef#1{\expandafter\strip@prefix\meaning#1}%
}
\def\TestCompare#1#2#3{%
  \begingroup
    \edef\x{%
      \if\IfLang@StrEqual{#1}{#2}%
        true%
      \else
        false%
      \fi
    }%
    \def\expect{#3}%
    \ifx\x\expect
    \else
      \def\a{#1}%
      \@onelevel@sanitize\a
      \def\b{#2}%
      \@onelevel@sanitize\b
      \errmessage{Failed test: `\a'='\b' <> \expect}%
    \fi
  \endgroup
}
\TestCompare{junk}{junk}{true}
\TestCompare{}{}{true}
\TestCompare{a}{b}{false}
\TestCompare{aa}{bb}{false}
\def\a{ax}
\def\b{bx}
\def\c{\a\b}
\def\d{\c\b}
\def\exch#1#2{#2#1}
\def\gobble#1{}
\TestCompare{\gobble a}{}{true}
\TestCompare{}{\gobble a}{true}
\TestCompare{a\exch xyb}{ayxb}{true}
\TestCompare{\c}{\c}{true}
\TestCompare{\d}{\c\b}{true}

\csname @@end\endcsname
\end
%    \end{macrocode}
%    \begin{macrocode}
%</test5>
%    \end{macrocode}
%
% \section{Installation}
%
% \subsection{Download}
%
% \paragraph{Package.} This package is available on
% CTAN\footnote{\url{http://ctan.org/pkg/iflang}}:
% \begin{description}
% \item[\CTAN{macros/latex/contrib/oberdiek/iflang.dtx}] The source file.
% \item[\CTAN{macros/latex/contrib/oberdiek/iflang.pdf}] Documentation.
% \end{description}
%
%
% \paragraph{Bundle.} All the packages of the bundle `oberdiek'
% are also available in a TDS compliant ZIP archive. There
% the packages are already unpacked and the documentation files
% are generated. The files and directories obey the TDS standard.
% \begin{description}
% \item[\CTAN{install/macros/latex/contrib/oberdiek.tds.zip}]
% \end{description}
% \emph{TDS} refers to the standard ``A Directory Structure
% for \TeX\ Files'' (\CTAN{tds/tds.pdf}). Directories
% with \xfile{texmf} in their name are usually organized this way.
%
% \subsection{Bundle installation}
%
% \paragraph{Unpacking.} Unpack the \xfile{oberdiek.tds.zip} in the
% TDS tree (also known as \xfile{texmf} tree) of your choice.
% Example (linux):
% \begin{quote}
%   |unzip oberdiek.tds.zip -d ~/texmf|
% \end{quote}
%
% \paragraph{Script installation.}
% Check the directory \xfile{TDS:scripts/oberdiek/} for
% scripts that need further installation steps.
% Package \xpackage{attachfile2} comes with the Perl script
% \xfile{pdfatfi.pl} that should be installed in such a way
% that it can be called as \texttt{pdfatfi}.
% Example (linux):
% \begin{quote}
%   |chmod +x scripts/oberdiek/pdfatfi.pl|\\
%   |cp scripts/oberdiek/pdfatfi.pl /usr/local/bin/|
% \end{quote}
%
% \subsection{Package installation}
%
% \paragraph{Unpacking.} The \xfile{.dtx} file is a self-extracting
% \docstrip\ archive. The files are extracted by running the
% \xfile{.dtx} through \plainTeX:
% \begin{quote}
%   \verb|tex iflang.dtx|
% \end{quote}
%
% \paragraph{TDS.} Now the different files must be moved into
% the different directories in your installation TDS tree
% (also known as \xfile{texmf} tree):
% \begin{quote}
% \def\t{^^A
% \begin{tabular}{@{}>{\ttfamily}l@{ $\rightarrow$ }>{\ttfamily}l@{}}
%   iflang.sty & tex/generic/oberdiek/iflang.sty\\
%   iflang.pdf & doc/latex/oberdiek/iflang.pdf\\
%   test/iflang-test1.tex & doc/latex/oberdiek/test/iflang-test1.tex\\
%   test/iflang-test2.tex & doc/latex/oberdiek/test/iflang-test2.tex\\
%   test/iflang-test3.tex & doc/latex/oberdiek/test/iflang-test3.tex\\
%   test/iflang-test4.tex & doc/latex/oberdiek/test/iflang-test4.tex\\
%   test/iflang-test5.tex & doc/latex/oberdiek/test/iflang-test5.tex\\
%   iflang.dtx & source/latex/oberdiek/iflang.dtx\\
% \end{tabular}^^A
% }^^A
% \sbox0{\t}^^A
% \ifdim\wd0>\linewidth
%   \begingroup
%     \advance\linewidth by\leftmargin
%     \advance\linewidth by\rightmargin
%   \edef\x{\endgroup
%     \def\noexpand\lw{\the\linewidth}^^A
%   }\x
%   \def\lwbox{^^A
%     \leavevmode
%     \hbox to \linewidth{^^A
%       \kern-\leftmargin\relax
%       \hss
%       \usebox0
%       \hss
%       \kern-\rightmargin\relax
%     }^^A
%   }^^A
%   \ifdim\wd0>\lw
%     \sbox0{\small\t}^^A
%     \ifdim\wd0>\linewidth
%       \ifdim\wd0>\lw
%         \sbox0{\footnotesize\t}^^A
%         \ifdim\wd0>\linewidth
%           \ifdim\wd0>\lw
%             \sbox0{\scriptsize\t}^^A
%             \ifdim\wd0>\linewidth
%               \ifdim\wd0>\lw
%                 \sbox0{\tiny\t}^^A
%                 \ifdim\wd0>\linewidth
%                   \lwbox
%                 \else
%                   \usebox0
%                 \fi
%               \else
%                 \lwbox
%               \fi
%             \else
%               \usebox0
%             \fi
%           \else
%             \lwbox
%           \fi
%         \else
%           \usebox0
%         \fi
%       \else
%         \lwbox
%       \fi
%     \else
%       \usebox0
%     \fi
%   \else
%     \lwbox
%   \fi
% \else
%   \usebox0
% \fi
% \end{quote}
% If you have a \xfile{docstrip.cfg} that configures and enables \docstrip's
% TDS installing feature, then some files can already be in the right
% place, see the documentation of \docstrip.
%
% \subsection{Refresh file name databases}
%
% If your \TeX~distribution
% (\teTeX, \mikTeX, \dots) relies on file name databases, you must refresh
% these. For example, \teTeX\ users run \verb|texhash| or
% \verb|mktexlsr|.
%
% \subsection{Some details for the interested}
%
% \paragraph{Attached source.}
%
% The PDF documentation on CTAN also includes the
% \xfile{.dtx} source file. It can be extracted by
% AcrobatReader 6 or higher. Another option is \textsf{pdftk},
% e.g. unpack the file into the current directory:
% \begin{quote}
%   \verb|pdftk iflang.pdf unpack_files output .|
% \end{quote}
%
% \paragraph{Unpacking with \LaTeX.}
% The \xfile{.dtx} chooses its action depending on the format:
% \begin{description}
% \item[\plainTeX:] Run \docstrip\ and extract the files.
% \item[\LaTeX:] Generate the documentation.
% \end{description}
% If you insist on using \LaTeX\ for \docstrip\ (really,
% \docstrip\ does not need \LaTeX), then inform the autodetect routine
% about your intention:
% \begin{quote}
%   \verb|latex \let\install=y% \iffalse meta-comment
%
% File: iflang.dtx
% Version: 2016/05/16 v1.6
% Info: Checks for the current language
%
% Copyright (C) 2007 by
%    Heiko Oberdiek <heiko.oberdiek at googlemail.com>
%    2016
%    https://github.com/ho-tex/oberdiek/issues
%
% This work may be distributed and/or modified under the
% conditions of the LaTeX Project Public License, either
% version 1.3c of this license or (at your option) any later
% version. This version of this license is in
%    http://www.latex-project.org/lppl/lppl-1-3c.txt
% and the latest version of this license is in
%    http://www.latex-project.org/lppl.txt
% and version 1.3 or later is part of all distributions of
% LaTeX version 2005/12/01 or later.
%
% This work has the LPPL maintenance status "maintained".
%
% This Current Maintainer of this work is Heiko Oberdiek.
%
% The Base Interpreter refers to any `TeX-Format',
% because some files are installed in TDS:tex/generic//.
%
% This work consists of the main source file iflang.dtx
% and the derived files
%    iflang.sty, iflang.pdf, iflang.ins, iflang.drv, iflang-test1.tex,
%    iflang-test2.tex, iflang-test3.tex, iflang-test4.tex,
%    iflang-test5.tex.
%
% Distribution:
%    CTAN:macros/latex/contrib/oberdiek/iflang.dtx
%    CTAN:macros/latex/contrib/oberdiek/iflang.pdf
%
% Unpacking:
%    (a) If iflang.ins is present:
%           tex iflang.ins
%    (b) Without iflang.ins:
%           tex iflang.dtx
%    (c) If you insist on using LaTeX
%           latex \let\install=y\input{iflang.dtx}
%        (quote the arguments according to the demands of your shell)
%
% Documentation:
%    (a) If iflang.drv is present:
%           latex iflang.drv
%    (b) Without iflang.drv:
%           latex iflang.dtx; ...
%    The class ltxdoc loads the configuration file ltxdoc.cfg
%    if available. Here you can specify further options, e.g.
%    use A4 as paper format:
%       \PassOptionsToClass{a4paper}{article}
%
%    Programm calls to get the documentation (example):
%       pdflatex iflang.dtx
%       makeindex -s gind.ist iflang.idx
%       pdflatex iflang.dtx
%       makeindex -s gind.ist iflang.idx
%       pdflatex iflang.dtx
%
% Installation:
%    TDS:tex/generic/oberdiek/iflang.sty
%    TDS:doc/latex/oberdiek/iflang.pdf
%    TDS:doc/latex/oberdiek/test/iflang-test1.tex
%    TDS:doc/latex/oberdiek/test/iflang-test2.tex
%    TDS:doc/latex/oberdiek/test/iflang-test3.tex
%    TDS:doc/latex/oberdiek/test/iflang-test4.tex
%    TDS:doc/latex/oberdiek/test/iflang-test5.tex
%    TDS:source/latex/oberdiek/iflang.dtx
%
%<*ignore>
\begingroup
  \catcode123=1 %
  \catcode125=2 %
  \def\x{LaTeX2e}%
\expandafter\endgroup
\ifcase 0\ifx\install y1\fi\expandafter
         \ifx\csname processbatchFile\endcsname\relax\else1\fi
         \ifx\fmtname\x\else 1\fi\relax
\else\csname fi\endcsname
%</ignore>
%<*install>
\input docstrip.tex
\Msg{************************************************************************}
\Msg{* Installation}
\Msg{* Package: iflang 2016/05/16 v1.6 Checks for the current language (HO)}
\Msg{************************************************************************}

\keepsilent
\askforoverwritefalse

\let\MetaPrefix\relax
\preamble

This is a generated file.

Project: iflang
Version: 2016/05/16 v1.6

Copyright (C) 2007 by
   Heiko Oberdiek <heiko.oberdiek at googlemail.com>

This work may be distributed and/or modified under the
conditions of the LaTeX Project Public License, either
version 1.3c of this license or (at your option) any later
version. This version of this license is in
   http://www.latex-project.org/lppl/lppl-1-3c.txt
and the latest version of this license is in
   http://www.latex-project.org/lppl.txt
and version 1.3 or later is part of all distributions of
LaTeX version 2005/12/01 or later.

This work has the LPPL maintenance status "maintained".

This Current Maintainer of this work is Heiko Oberdiek.

The Base Interpreter refers to any `TeX-Format',
because some files are installed in TDS:tex/generic//.

This work consists of the main source file iflang.dtx
and the derived files
   iflang.sty, iflang.pdf, iflang.ins, iflang.drv, iflang-test1.tex,
   iflang-test2.tex, iflang-test3.tex, iflang-test4.tex,
   iflang-test5.tex.

\endpreamble
\let\MetaPrefix\DoubleperCent

\generate{%
  \file{iflang.ins}{\from{iflang.dtx}{install}}%
  \file{iflang.drv}{\from{iflang.dtx}{driver}}%
  \usedir{tex/generic/oberdiek}%
  \file{iflang.sty}{\from{iflang.dtx}{package}}%
  \usedir{doc/latex/oberdiek/test}%
  \file{iflang-test1.tex}{\from{iflang.dtx}{test1}}%
  \file{iflang-test2.tex}{\from{iflang.dtx}{test2}}%
  \file{iflang-test3.tex}{\from{iflang.dtx}{test3}}%
  \file{iflang-test4.tex}{\from{iflang.dtx}{test4}}%
  \file{iflang-test5.tex}{\from{iflang.dtx}{test5}}%
  \nopreamble
  \nopostamble
  \usedir{source/latex/oberdiek/catalogue}%
  \file{iflang.xml}{\from{iflang.dtx}{catalogue}}%
}

\catcode32=13\relax% active space
\let =\space%
\Msg{************************************************************************}
\Msg{*}
\Msg{* To finish the installation you have to move the following}
\Msg{* file into a directory searched by TeX:}
\Msg{*}
\Msg{*     iflang.sty}
\Msg{*}
\Msg{* To produce the documentation run the file `iflang.drv'}
\Msg{* through LaTeX.}
\Msg{*}
\Msg{* Happy TeXing!}
\Msg{*}
\Msg{************************************************************************}

\endbatchfile
%</install>
%<*ignore>
\fi
%</ignore>
%<*driver>
\NeedsTeXFormat{LaTeX2e}
\ProvidesFile{iflang.drv}%
  [2016/05/16 v1.6 Checks for the current language (HO)]%
\documentclass{ltxdoc}
\usepackage{holtxdoc}[2011/11/22]
\begin{document}
  \DocInput{iflang.dtx}%
\end{document}
%</driver>
% \fi
%
%
% \CharacterTable
%  {Upper-case    \A\B\C\D\E\F\G\H\I\J\K\L\M\N\O\P\Q\R\S\T\U\V\W\X\Y\Z
%   Lower-case    \a\b\c\d\e\f\g\h\i\j\k\l\m\n\o\p\q\r\s\t\u\v\w\x\y\z
%   Digits        \0\1\2\3\4\5\6\7\8\9
%   Exclamation   \!     Double quote  \"     Hash (number) \#
%   Dollar        \$     Percent       \%     Ampersand     \&
%   Acute accent  \'     Left paren    \(     Right paren   \)
%   Asterisk      \*     Plus          \+     Comma         \,
%   Minus         \-     Point         \.     Solidus       \/
%   Colon         \:     Semicolon     \;     Less than     \<
%   Equals        \=     Greater than  \>     Question mark \?
%   Commercial at \@     Left bracket  \[     Backslash     \\
%   Right bracket \]     Circumflex    \^     Underscore    \_
%   Grave accent  \`     Left brace    \{     Vertical bar  \|
%   Right brace   \}     Tilde         \~}
%
% \GetFileInfo{iflang.drv}
%
% \title{The \xpackage{iflang} package}
% \date{2016/05/16 v1.6}
% \author{Heiko Oberdiek\thanks
% {Please report any issues at https://github.com/ho-tex/oberdiek/issues}\\
% \xemail{heiko.oberdiek at googlemail.com}}
%
% \maketitle
%
% \begin{abstract}
% This package provides expandible checks for the current language
% based on macro \cs{languagename} or hyphenation patterns.
% \end{abstract}
%
% \tableofcontents
%
% \section{Documentation}
%
% Package \xpackage{babel} defines \cs{iflanguagename}. As first
% argument it takes a language name and executes the second or
% third argument depending on the current language. This
% language test is based on hypenation patterns. However, it
% is possible that different languages or dialects share the
% same patterns. In such cases \cs{iflanguagename} fails.
%
% However, package \xpackage{babel} and some other packages
% such as \xpackage{german} or \xpackage{ngerman} store
% the language name in the macro \cs{languagename} if
% \cs{selectlanguage} is called.
%
% \begin{declcs}{IfLanguageName} \M{lang} \M{then} \M{else}
% \end{declcs}
% Makro \cs{IfLanguageName} compares language \meta{lang} with
% the current setting of macro \cs{languagename}. If both
% contains the same name then the \meta{then} part is called,
% otherwise the \meta{else} part.
%
% The macro is expandable. Thus it can be safely used inside \cs{edef}
% or \cs{csname}. If case of errors like an undefined \cs{languagename}
% the \meta{else} part is executed.
%
% Note: Macro \cs{IfLanguageName} relies on the fact, that
% \cs{languagename} is set correctly:
% \begin{description}
% \item[Package \xpackage{babel}:]
%    \mbox{}\\
%    Full support of \cs{languagename} in its language switching commands.
% \item[Format based on \xpackage{babel} (\xfile{language.dat}):]
%    \mbox{}\\
%    If package \xpackage{babel} is not used (or not yet loaded),
%    then \xpackage{babel}'s
%    \xfile{hyphen.cfg} has set \cs{languagename} to the last
%    language in \xfile{language.dat}, but \cs{language} (current
%    patterns) is zero and points to the first language.
%    Thus the value of \cs{languagename} is basically garbage.
%    Package \xpackage{iflang} warns if \cs{languagename} and
%    \cs{language} do not fit.
%    This can be fixed by loading package \xpackage{babel} previously.
% \item[Format based on \eTeX's \xfile{etex.src} (\xfile{language.def}):]
%    \mbox{}\\
%    Unhappily it does not support
%    \cs{languagename}. Thus this package hooks into \cs{uselanguage}
%    to get \cs{languagename} defined and updated there.
%    At package loading time the changed \cs{uselanguage} has not
%    been called yet. Thus package \xpackage{iflang} tries
%    |USenglish|. This is the definite default language of \xfile{etex.src}.
%    If the current patterns suit this default language, an
%    undefined \cs{languagename} is set to this language. Otherwise
%    a \cs{languagename} remains undefined and a warning is given.
% \end{description}
%
% \begin{declcs}{IfLanguagePatterns} \M{lang} \M{then} \M{else}
% \end{declcs}
% This macro behaves similar to \cs{IfLanguageName}. But the
% language test is based on the current pattern in force (\cs{language}).
% Also this macro is expandable, in case of errors the \meta{else} part
% is called.
%
% The following naming convention for the pattern are supported:
% \begin{description}
% \item[\xpackage{babel}/\xfile{language.dat}]:
%    \cs{l@}\meta{language}
% \item[\xpackage{etex.src}/\xfile{language.def}]:
%    \cs{lang@}\meta{language}
% \end{description}
% Package \xpackage{iflang} looks for \cs{uselanguage} (defined in
% \xfile{etex.src}) to find out the naming convention in use.
%
% \StopEventually{
% }
%
% \section{Implementation}
%
%    \begin{macrocode}
%<*package>
%    \end{macrocode}
%
% \subsection{Reload check and package identification}
%    Reload check, especially if the package is not used with \LaTeX.
%    \begin{macrocode}
\begingroup\catcode61\catcode48\catcode32=10\relax%
  \catcode13=5 % ^^M
  \endlinechar=13 %
  \catcode35=6 % #
  \catcode39=12 % '
  \catcode44=12 % ,
  \catcode45=12 % -
  \catcode46=12 % .
  \catcode58=12 % :
  \catcode64=11 % @
  \catcode123=1 % {
  \catcode125=2 % }
  \expandafter\let\expandafter\x\csname ver@iflang.sty\endcsname
  \ifx\x\relax % plain-TeX, first loading
  \else
    \def\empty{}%
    \ifx\x\empty % LaTeX, first loading,
      % variable is initialized, but \ProvidesPackage not yet seen
    \else
      \expandafter\ifx\csname PackageInfo\endcsname\relax
        \def\x#1#2{%
          \immediate\write-1{Package #1 Info: #2.}%
        }%
      \else
        \def\x#1#2{\PackageInfo{#1}{#2, stopped}}%
      \fi
      \x{iflang}{The package is already loaded}%
      \aftergroup\endinput
    \fi
  \fi
\endgroup%
%    \end{macrocode}
%    Package identification:
%    \begin{macrocode}
\begingroup\catcode61\catcode48\catcode32=10\relax%
  \catcode13=5 % ^^M
  \endlinechar=13 %
  \catcode35=6 % #
  \catcode39=12 % '
  \catcode40=12 % (
  \catcode41=12 % )
  \catcode44=12 % ,
  \catcode45=12 % -
  \catcode46=12 % .
  \catcode47=12 % /
  \catcode58=12 % :
  \catcode64=11 % @
  \catcode91=12 % [
  \catcode93=12 % ]
  \catcode123=1 % {
  \catcode125=2 % }
  \expandafter\ifx\csname ProvidesPackage\endcsname\relax
    \def\x#1#2#3[#4]{\endgroup
      \immediate\write-1{Package: #3 #4}%
      \xdef#1{#4}%
    }%
  \else
    \def\x#1#2[#3]{\endgroup
      #2[{#3}]%
      \ifx#1\@undefined
        \xdef#1{#3}%
      \fi
      \ifx#1\relax
        \xdef#1{#3}%
      \fi
    }%
  \fi
\expandafter\x\csname ver@iflang.sty\endcsname
\ProvidesPackage{iflang}%
  [2016/05/16 v1.6 Checks for the current language (HO)]%
%    \end{macrocode}
%
%    \begin{macrocode}
\begingroup\catcode61\catcode48\catcode32=10\relax%
  \catcode13=5 % ^^M
  \endlinechar=13 %
  \catcode123=1 % {
  \catcode125=2 % }
  \catcode64=11 % @
  \def\x{\endgroup
    \expandafter\edef\csname IfLang@AtEnd\endcsname{%
      \endlinechar=\the\endlinechar\relax
      \catcode13=\the\catcode13\relax
      \catcode32=\the\catcode32\relax
      \catcode35=\the\catcode35\relax
      \catcode61=\the\catcode61\relax
      \catcode64=\the\catcode64\relax
      \catcode123=\the\catcode123\relax
      \catcode125=\the\catcode125\relax
    }%
  }%
\x\catcode61\catcode48\catcode32=10\relax%
\catcode13=5 % ^^M
\endlinechar=13 %
\catcode35=6 % #
\catcode64=11 % @
\catcode123=1 % {
\catcode125=2 % }
\def\TMP@EnsureCode#1#2{%
  \edef\IfLang@AtEnd{%
    \IfLang@AtEnd
    \catcode#1=\the\catcode#1\relax
  }%
  \catcode#1=#2\relax
}
\TMP@EnsureCode{39}{12}% '
\TMP@EnsureCode{40}{12}% (
\TMP@EnsureCode{41}{12}% )
\TMP@EnsureCode{44}{12}% ,
\TMP@EnsureCode{46}{12}% .
\TMP@EnsureCode{47}{12}% /
\TMP@EnsureCode{58}{12}% :
\TMP@EnsureCode{91}{12}% [
\TMP@EnsureCode{93}{12}% ]
\edef\IfLang@AtEnd{\IfLang@AtEnd\noexpand\endinput}
%    \end{macrocode}
%
% \subsection{Tools}
%
% \subsubsection{Provide some basic macros of \LaTeX}
%
%    \begin{macro}{\@firstoftwo}
%    \begin{macrocode}
\expandafter\ifx\csname @firstoftwo\endcsname\relax
  \long\def\@firstoftwo#1#2{#1}%
\fi
%    \end{macrocode}
%    \end{macro}
%    \begin{macro}{\@secondoftwo}
%    \begin{macrocode}
\expandafter\ifx\csname @secondoftwo\endcsname\relax
  \long\def\@secondoftwo#1#2{#2}%
\fi
%    \end{macrocode}
%    \end{macro}
%
% \subsubsection{Expandible existence check for macros}
%
%    \begin{macro}{\IfLang@IfDefined}
%    \begin{macrocode}
\begingroup\expandafter\expandafter\expandafter\endgroup
\expandafter\ifx\csname ifcsname\endcsname\relax
  \expandafter\@firstoftwo
\else
  \expandafter\@secondoftwo
\fi
{%
  \def\IfLang@IfDefined#1{%
    \expandafter\ifx\csname#1\endcsname\relax
      \expandafter\@secondoftwo
    \else
      \expandafter\@firstoftwo
    \fi
  }%
}{%
  \def\IfLang@IfDefined#1{%
    \ifnum\ifcsname#1\endcsname
            \expandafter\ifx\csname#1\endcsname\relax
              1%
            \else
              0%
            \fi
          \else
            1%
          \fi
          =0 %
      \expandafter\@firstoftwo
    \else
      \expandafter\@secondoftwo
    \fi
  }%
}
%    \end{macrocode}
%    \end{macro}
%
% \subsubsection{Macros for messages}
%
%    \begin{macrocode}
\begingroup\expandafter\expandafter\expandafter\endgroup
\expandafter\ifx\csname RequirePackage\endcsname\relax
  \input infwarerr.sty\relax
  \input pdftexcmds.sty\relax
\else
  \RequirePackage{infwarerr}[2007/09/09]%
  \RequirePackage{pdftexcmds}[2016/05/16]%
\fi
%    \end{macrocode}
%
% \subsubsection{Support for \xfile{etex.src}}
%
%    \begin{macro}{\IfLang@prefix}
%    \begin{macrocode}
\begingroup\expandafter\expandafter\expandafter\endgroup
\expandafter\ifx\csname uselanguage\endcsname\relax
  \@PackageInfoNoLine{iflang}{%
    Naming convention for patterns: babel%
  }%
  \def\IfLang@prefix{l@}%
\else
  \@PackageInfoNoLine{iflang}{%
    Naming convention for patterns: etex.src%
  }%
  \def\IfLang@prefix{lang@}%
  \let\IfLang@OrgUseLanguage\uselanguage
  \def\uselanguage#1{%
    \edef\languagename{#1}%
    \IfLang@OrgUseLanguage{#1}%
  }%
%    \end{macrocode}
%    \end{macro}
%    The first \cs{uselanguage} that is executed as last
%    line in \xfile{language.def} cannot patched this way.
%    However, \xfile{language.def} is very strict. It forces
%    the first added and used language to be |USenglish|.
%    Thus, if \cs{languagename} is not defined, we can
%    quite safely assume |USenglish|. As additional
%    safety precaution the actual used patterns are checked.
%    \begin{macrocode}
  \begingroup\expandafter\expandafter\expandafter\endgroup
  \expandafter\ifx\csname languagename\endcsname\relax
    \begingroup\expandafter\expandafter\expandafter\endgroup
    \expandafter\ifx\csname lang@USenglish\endcsname\relax
      \@PackageWarningNoLine{iflang}{%
        \string\lang@USenglish\space is missing%
      }%
    \else
      \ifnum\lang@USenglish=\language
        \def\languagename{USenglish}%
      \else
        \@PackageWarningNoLine{iflang}{%
          \string\languagename\space is not set,\MessageBreak
          current language is unknown%
        }%
      \fi
    \fi
  \fi
\fi
\begingroup\expandafter\expandafter\expandafter\endgroup
\expandafter\ifx\csname languagename\endcsname\relax
  \@PackageInfoNoLine{iflang}{%
    \string\languagename\space is not set%
  }%
\fi
%    \end{macrocode}
%
% \subsection{\cs{IfLanguagePatterns}}
%
%    \begin{macro}{\IfLanguagePatterns}
%    \begin{macrocode}
\def\IfLanguagePatterns#1{%
  \ifnum\IfLang@IfDefined{\IfLang@prefix#1}{%
          \ifnum\csname\IfLang@prefix#1\endcsname=\language
            0%
          \else
            1%
          \fi
        }{1}=0 %
    \expandafter\@firstoftwo
  \else
    \expandafter\@secondoftwo
  \fi
}
%    \end{macrocode}
%    \end{macro}
%
% \subsection{\cs{IfLanguageName}}
%
%    \begin{macrocode}
\begingroup\expandafter\expandafter\expandafter\endgroup
\expandafter\ifx\csname pdf@strcmp\endcsname\relax
  \expandafter\@firstoftwo
\else
  \expandafter\@secondoftwo
\fi
{%
%    \end{macrocode}
%    We do not have \cs{pdf@strcmp} (and \cs{pdfstrcmp}).
%    Thus we must define
%    our own expandable string comparison. The following
%    implementation is based on a \TeX\ pearl from David Kastrup, presented
%    at the conference Bacho\TeX~2005:
%    \url{http://www-stary.gust.org.pl/pearls/2005/david-kastrup/bachotex2005-david-kastrup-pearl1.pdf}
%
%    The orignal code allows macros inside the second string.
%    Because also \cs{languagename} might consists of further
%    macros, we need a variant that allows macros in the first
%    string, too.
%    \begin{macrocode}
  \def\IfLang@StrNil{\relax}%
  \def\IfLang@StrEqual#1{%
    \number\IfLang@StrEqualStart{}{}#1\IfLang@StrNil
  }%
  \def\IfLang@StrEqualStart#1#2#3{%
    \ifx#3\IfLang@StrNil
      \IfLang@StrEqualStop
    \fi
    \ifcat\noexpand#3\relax
      \IfLang@StrExpand{#1}{#2}#3%
    \fi
    \IfLang@StrEqualStart{\if#3#1}{#2\fi}%
  }%
  \def\IfLang@StrEqualStop\fi#1\IfLang@StrEqualStart#2#3#4{%
    \fi
    #2#4\relax'#313 %
  }%
  \def\IfLang@StrExpand#1#2#3\fi\IfLang@StrEqualStart#4#5{%
    \fi
    \IfLang@@StrExpand{#1}{#2}#3%
  }%
  \def\IfLang@@StrExpand#1#2#3\IfLang@StrNil{%
    \expandafter\IfLang@@@StrExpand#3\IfLang@StrNil{#1}{#2}%
  }%
  \def\IfLang@@@StrExpand#1\IfLang@StrNil#2#3{%
    \IfLang@StrEqualStart{#2}{#3}#1\IfLang@StrNil
  }%
%    \end{macrocode}
%    \begin{macro}{\IfLanguageName}
%    \begin{macrocode}
  \def\IfLanguageName#1{%
    \ifnum\IfLang@IfDefined{languagename}{%
            \if\expandafter\IfLang@StrEqual\expandafter%
                           {\languagename}{#1}%
              0%
            \else
              1%
            \fi
          }{1}=0 %
      \expandafter\@firstoftwo
    \else
      \expandafter\@secondoftwo
    \fi
  }%
%    \end{macrocode}
%    \end{macro}
%    \begin{macrocode}
}{%
%    \end{macrocode}
%    \begin{macro}{\IfLanguageName}
%    \begin{macrocode}
  \def\IfLanguageName#1{%
    \ifnum\IfLang@IfDefined{languagename}{%
            \pdf@strcmp{#1}{\languagename}%
          }{1}=0 %
      \expandafter\@firstoftwo
    \else
      \expandafter\@secondoftwo
    \fi
  }%
%    \end{macrocode}
%    \end{macro}
%    \begin{macrocode}
}
%    \end{macrocode}
%
% \subsection{Check plausibility of \cs{languagename}}
%
%    \begin{macrocode}
\begingroup\expandafter\expandafter\expandafter\endgroup
\expandafter\ifx\csname languagename\endcsname\relax
\else
  \IfLanguagePatterns{\languagename}{}{%
    \@PackageWarningNoLine{iflang}{%
      Mismatch between \string\language\space
      (patterns)\MessageBreak
      and setting of \string\languagename
    }%
  }%
\fi
%    \end{macrocode}
%
%    \begin{macrocode}
\IfLang@AtEnd%
%</package>
%    \end{macrocode}
%
% \section{Test}
%
% \subsection{Catcode checks for loading}
%
%    \begin{macrocode}
%<*test1>
%    \end{macrocode}
%    \begin{macrocode}
\catcode`\{=1 %
\catcode`\}=2 %
\catcode`\#=6 %
\catcode`\@=11 %
\expandafter\ifx\csname count@\endcsname\relax
  \countdef\count@=255 %
\fi
\expandafter\ifx\csname @gobble\endcsname\relax
  \long\def\@gobble#1{}%
\fi
\expandafter\ifx\csname @firstofone\endcsname\relax
  \long\def\@firstofone#1{#1}%
\fi
\expandafter\ifx\csname loop\endcsname\relax
  \expandafter\@firstofone
\else
  \expandafter\@gobble
\fi
{%
  \def\loop#1\repeat{%
    \def\body{#1}%
    \iterate
  }%
  \def\iterate{%
    \body
      \let\next\iterate
    \else
      \let\next\relax
    \fi
    \next
  }%
  \let\repeat=\fi
}%
\def\RestoreCatcodes{}
\count@=0 %
\loop
  \edef\RestoreCatcodes{%
    \RestoreCatcodes
    \catcode\the\count@=\the\catcode\count@\relax
  }%
\ifnum\count@<255 %
  \advance\count@ 1 %
\repeat

\def\RangeCatcodeInvalid#1#2{%
  \count@=#1\relax
  \loop
    \catcode\count@=15 %
  \ifnum\count@<#2\relax
    \advance\count@ 1 %
  \repeat
}
\def\RangeCatcodeCheck#1#2#3{%
  \count@=#1\relax
  \loop
    \ifnum#3=\catcode\count@
    \else
      \errmessage{%
        Character \the\count@\space
        with wrong catcode \the\catcode\count@\space
        instead of \number#3%
      }%
    \fi
  \ifnum\count@<#2\relax
    \advance\count@ 1 %
  \repeat
}
\def\space{ }
\expandafter\ifx\csname LoadCommand\endcsname\relax
  \def\LoadCommand{\input iflang.sty\relax}%
\fi
\def\Test{%
  \RangeCatcodeInvalid{0}{47}%
  \RangeCatcodeInvalid{58}{64}%
  \RangeCatcodeInvalid{91}{96}%
  \RangeCatcodeInvalid{123}{255}%
  \catcode`\@=12 %
  \catcode`\\=0 %
  \catcode`\%=14 %
  \LoadCommand
  \RangeCatcodeCheck{0}{36}{15}%
  \RangeCatcodeCheck{37}{37}{14}%
  \RangeCatcodeCheck{38}{47}{15}%
  \RangeCatcodeCheck{48}{57}{12}%
  \RangeCatcodeCheck{58}{63}{15}%
  \RangeCatcodeCheck{64}{64}{12}%
  \RangeCatcodeCheck{65}{90}{11}%
  \RangeCatcodeCheck{91}{91}{15}%
  \RangeCatcodeCheck{92}{92}{0}%
  \RangeCatcodeCheck{93}{96}{15}%
  \RangeCatcodeCheck{97}{122}{11}%
  \RangeCatcodeCheck{123}{255}{15}%
  \RestoreCatcodes
}
\Test
\csname @@end\endcsname
\end
%    \end{macrocode}
%    \begin{macrocode}
%</test1>
%    \end{macrocode}
%
% \subsection{Test with \LaTeX}
%
%    \begin{macrocode}
%<*test2|test3>
%    \end{macrocode}
%    \begin{macrocode}
\NeedsTeXFormat{LaTeX2e}
%<test3>\let\pdfstrcmp\relax
\nofiles
\documentclass{minimal}
\usepackage{qstest}
\IncludeTests{*}
\LogTests{log}{*}{*}
\usepackage[english,naustrian,ngerman]{babel}
\usepackage{iflang}
\begin{document}
\begin{qstest}{IfLanguagePatterns}{language, pattern}
  \def\test#1#2{%
    \Expect*{\IfLanguagePatterns{#1}{true}{false}}{#2}%
  }%
  \test{ngerman}{true}%
  \test{naustrian}{true}%
  \test{english}{false}%
  \test{foobar}{false}%
\end{qstest}
\begin{qstest}{IfLanguageName}{language, name}
  \def\test#1#2{%
    \Expect*{\IfLanguageName{#1}{true}{false}}{#2}%
  }%
  \test{ngerman}{true}%
  \test{naustrian}{false}%
  \selectlanguage{naustrian}%
  \test{ngerman}{false}%
  \test{naustrian}{true}%
  \test{foobar}{false}%
  %
  \def\languagename{naustrian}%
  \test{naustrian}{true}%
  \test{ngerman}{false}%
  %
  \edef\languagename{\string naustrian}%
  \test{naustrian}{true}%
  \test{ngerman}{false}%
  %
  \def\languagename{naustrian}%
  \makeatletter
  \@onelevel@sanitize\languagename
  \test{naustrian}{true}%
  \test{ngerman}{false}%
  %
  \def\languagename{naustrian}%
  \def\xaustrian{naustrian}%
  \def\xgerman{ngerman}%
  \test{\xaustrian}{true}%
  \test{\xgerman}{false}%
  %
  \def\languagename{\xaustrian}%
  \test{naustrian}{true}%
  \test{ngerman}{false}%
  \test{\xaustrian}{true}%
  \test{\xgerman}{false}%
  \test{\languagename}{true}%
  \test{\languagename\space}{false}%
  %
  \def\languagename{\empty\xaustrian\empty}%
  \test{naustrian}{true}%
  \test{ngerman}{false}%
  \test{\empty\xaustrian\empty}{true}%
  \test{\empty\xgerman\empty}{false}%
\end{qstest}
\begin{qstest}{IfDefined}{defined}
  \makeatletter
  \let\foobar\relax
  \Expect*{\IfLang@IfDefined{foobar}{true}{false}}{false}%
  \Expect*{\ifx\foobar\relax true\else false\fi}{true}%
  \let\foobar\UNDEFINED
  \Expect*{\IfLang@IfDefined{foobar}{true}{false}}{false}%
  \Expect*{\ifx\foobar\relax true\else false\fi}{false}%
  \Expect*{\ifx\foobar\UNDEFINED true\else false\fi}{true}%
\end{qstest}
\end{document}
%    \end{macrocode}
%    \begin{macrocode}
%</test2|test3>
%    \end{macrocode}
%
% \subsection{Test with \plainTeX\ and \eTeX}
%
%    \begin{macrocode}
%<*test4>
%    \end{macrocode}
%    \begin{macrocode}
%% Format `etex' based on `language.def'
\input iflang.sty
\catcode64=12

\def\TestGeneric#1#2#3{%
  \begingroup
    \edef\x{#1{#2}{true}{false}}%
    \edef\y{#3}%
    \ifx\x\y
    \else
      \errmessage{Failed test: \string#1{#2} <> #3}%
    \fi
  \endgroup
}
\def\TestPatterns{\TestGeneric\IfLanguagePatterns}
\def\TestName{\TestGeneric\IfLanguageName}

\TestPatterns{USenglish}{true}
\TestPatterns{ngerman}{false}

\TestName{USenglish}{true}
\TestName{ngerman}{false}

\uselanguage{ngerman}

\TestPatterns{USenglish}{false}
\TestPatterns{ngerman}{true}

\TestName{USenglish}{false}
\TestName{ngerman}{true}

\csname @@end\endcsname
\end
%    \end{macrocode}
%    \begin{macrocode}
%</test4>
%    \end{macrocode}
%
% \subsection{Test with \plainTeX\ and without \eTeX/\pdfTeX}
%
%    \begin{macrocode}
%<*test5>
%    \end{macrocode}
%    \begin{macrocode}
%% Format `tex' (vanilla plain-TeX)
\let\ifcsname\UNDEFINED
\let\pdfstrcmp\UNDEFINED
\input iflang.sty
\catcode64=11

\def\TestDefined#1{%
  \IfLang@IfDefined{foobar}{}{}%
  \ifx\foobar#1%
  \else
    \errmessage{Failed test: \string\foobar <> \string#1}%
  \fi
}
\let\foobar\relax
\TestDefined\relax
\let\foobar\UNDEFINED
\TestDefined\relax

\def\strip@prefix#1>{}
\def\@onelevel@sanitize#1{%
  \edef#1{\expandafter\strip@prefix\meaning#1}%
}
\def\TestCompare#1#2#3{%
  \begingroup
    \edef\x{%
      \if\IfLang@StrEqual{#1}{#2}%
        true%
      \else
        false%
      \fi
    }%
    \def\expect{#3}%
    \ifx\x\expect
    \else
      \def\a{#1}%
      \@onelevel@sanitize\a
      \def\b{#2}%
      \@onelevel@sanitize\b
      \errmessage{Failed test: `\a'='\b' <> \expect}%
    \fi
  \endgroup
}
\TestCompare{junk}{junk}{true}
\TestCompare{}{}{true}
\TestCompare{a}{b}{false}
\TestCompare{aa}{bb}{false}
\def\a{ax}
\def\b{bx}
\def\c{\a\b}
\def\d{\c\b}
\def\exch#1#2{#2#1}
\def\gobble#1{}
\TestCompare{\gobble a}{}{true}
\TestCompare{}{\gobble a}{true}
\TestCompare{a\exch xyb}{ayxb}{true}
\TestCompare{\c}{\c}{true}
\TestCompare{\d}{\c\b}{true}

\csname @@end\endcsname
\end
%    \end{macrocode}
%    \begin{macrocode}
%</test5>
%    \end{macrocode}
%
% \section{Installation}
%
% \subsection{Download}
%
% \paragraph{Package.} This package is available on
% CTAN\footnote{\url{http://ctan.org/pkg/iflang}}:
% \begin{description}
% \item[\CTAN{macros/latex/contrib/oberdiek/iflang.dtx}] The source file.
% \item[\CTAN{macros/latex/contrib/oberdiek/iflang.pdf}] Documentation.
% \end{description}
%
%
% \paragraph{Bundle.} All the packages of the bundle `oberdiek'
% are also available in a TDS compliant ZIP archive. There
% the packages are already unpacked and the documentation files
% are generated. The files and directories obey the TDS standard.
% \begin{description}
% \item[\CTAN{install/macros/latex/contrib/oberdiek.tds.zip}]
% \end{description}
% \emph{TDS} refers to the standard ``A Directory Structure
% for \TeX\ Files'' (\CTAN{tds/tds.pdf}). Directories
% with \xfile{texmf} in their name are usually organized this way.
%
% \subsection{Bundle installation}
%
% \paragraph{Unpacking.} Unpack the \xfile{oberdiek.tds.zip} in the
% TDS tree (also known as \xfile{texmf} tree) of your choice.
% Example (linux):
% \begin{quote}
%   |unzip oberdiek.tds.zip -d ~/texmf|
% \end{quote}
%
% \paragraph{Script installation.}
% Check the directory \xfile{TDS:scripts/oberdiek/} for
% scripts that need further installation steps.
% Package \xpackage{attachfile2} comes with the Perl script
% \xfile{pdfatfi.pl} that should be installed in such a way
% that it can be called as \texttt{pdfatfi}.
% Example (linux):
% \begin{quote}
%   |chmod +x scripts/oberdiek/pdfatfi.pl|\\
%   |cp scripts/oberdiek/pdfatfi.pl /usr/local/bin/|
% \end{quote}
%
% \subsection{Package installation}
%
% \paragraph{Unpacking.} The \xfile{.dtx} file is a self-extracting
% \docstrip\ archive. The files are extracted by running the
% \xfile{.dtx} through \plainTeX:
% \begin{quote}
%   \verb|tex iflang.dtx|
% \end{quote}
%
% \paragraph{TDS.} Now the different files must be moved into
% the different directories in your installation TDS tree
% (also known as \xfile{texmf} tree):
% \begin{quote}
% \def\t{^^A
% \begin{tabular}{@{}>{\ttfamily}l@{ $\rightarrow$ }>{\ttfamily}l@{}}
%   iflang.sty & tex/generic/oberdiek/iflang.sty\\
%   iflang.pdf & doc/latex/oberdiek/iflang.pdf\\
%   test/iflang-test1.tex & doc/latex/oberdiek/test/iflang-test1.tex\\
%   test/iflang-test2.tex & doc/latex/oberdiek/test/iflang-test2.tex\\
%   test/iflang-test3.tex & doc/latex/oberdiek/test/iflang-test3.tex\\
%   test/iflang-test4.tex & doc/latex/oberdiek/test/iflang-test4.tex\\
%   test/iflang-test5.tex & doc/latex/oberdiek/test/iflang-test5.tex\\
%   iflang.dtx & source/latex/oberdiek/iflang.dtx\\
% \end{tabular}^^A
% }^^A
% \sbox0{\t}^^A
% \ifdim\wd0>\linewidth
%   \begingroup
%     \advance\linewidth by\leftmargin
%     \advance\linewidth by\rightmargin
%   \edef\x{\endgroup
%     \def\noexpand\lw{\the\linewidth}^^A
%   }\x
%   \def\lwbox{^^A
%     \leavevmode
%     \hbox to \linewidth{^^A
%       \kern-\leftmargin\relax
%       \hss
%       \usebox0
%       \hss
%       \kern-\rightmargin\relax
%     }^^A
%   }^^A
%   \ifdim\wd0>\lw
%     \sbox0{\small\t}^^A
%     \ifdim\wd0>\linewidth
%       \ifdim\wd0>\lw
%         \sbox0{\footnotesize\t}^^A
%         \ifdim\wd0>\linewidth
%           \ifdim\wd0>\lw
%             \sbox0{\scriptsize\t}^^A
%             \ifdim\wd0>\linewidth
%               \ifdim\wd0>\lw
%                 \sbox0{\tiny\t}^^A
%                 \ifdim\wd0>\linewidth
%                   \lwbox
%                 \else
%                   \usebox0
%                 \fi
%               \else
%                 \lwbox
%               \fi
%             \else
%               \usebox0
%             \fi
%           \else
%             \lwbox
%           \fi
%         \else
%           \usebox0
%         \fi
%       \else
%         \lwbox
%       \fi
%     \else
%       \usebox0
%     \fi
%   \else
%     \lwbox
%   \fi
% \else
%   \usebox0
% \fi
% \end{quote}
% If you have a \xfile{docstrip.cfg} that configures and enables \docstrip's
% TDS installing feature, then some files can already be in the right
% place, see the documentation of \docstrip.
%
% \subsection{Refresh file name databases}
%
% If your \TeX~distribution
% (\teTeX, \mikTeX, \dots) relies on file name databases, you must refresh
% these. For example, \teTeX\ users run \verb|texhash| or
% \verb|mktexlsr|.
%
% \subsection{Some details for the interested}
%
% \paragraph{Attached source.}
%
% The PDF documentation on CTAN also includes the
% \xfile{.dtx} source file. It can be extracted by
% AcrobatReader 6 or higher. Another option is \textsf{pdftk},
% e.g. unpack the file into the current directory:
% \begin{quote}
%   \verb|pdftk iflang.pdf unpack_files output .|
% \end{quote}
%
% \paragraph{Unpacking with \LaTeX.}
% The \xfile{.dtx} chooses its action depending on the format:
% \begin{description}
% \item[\plainTeX:] Run \docstrip\ and extract the files.
% \item[\LaTeX:] Generate the documentation.
% \end{description}
% If you insist on using \LaTeX\ for \docstrip\ (really,
% \docstrip\ does not need \LaTeX), then inform the autodetect routine
% about your intention:
% \begin{quote}
%   \verb|latex \let\install=y\input{iflang.dtx}|
% \end{quote}
% Do not forget to quote the argument according to the demands
% of your shell.
%
% \paragraph{Generating the documentation.}
% You can use both the \xfile{.dtx} or the \xfile{.drv} to generate
% the documentation. The process can be configured by the
% configuration file \xfile{ltxdoc.cfg}. For instance, put this
% line into this file, if you want to have A4 as paper format:
% \begin{quote}
%   \verb|\PassOptionsToClass{a4paper}{article}|
% \end{quote}
% An example follows how to generate the
% documentation with pdf\LaTeX:
% \begin{quote}
%\begin{verbatim}
%pdflatex iflang.dtx
%makeindex -s gind.ist iflang.idx
%pdflatex iflang.dtx
%makeindex -s gind.ist iflang.idx
%pdflatex iflang.dtx
%\end{verbatim}
% \end{quote}
%
% \section{Catalogue}
%
% The following XML file can be used as source for the
% \href{http://mirror.ctan.org/help/Catalogue/catalogue.html}{\TeX\ Catalogue}.
% The elements \texttt{caption} and \texttt{description} are imported
% from the original XML file from the Catalogue.
% The name of the XML file in the Catalogue is \xfile{iflang.xml}.
%    \begin{macrocode}
%<*catalogue>
<?xml version='1.0' encoding='us-ascii'?>
<!DOCTYPE entry SYSTEM 'catalogue.dtd'>
<entry datestamp='$Date$' modifier='$Author$' id='iflang'>
  <name>iflang</name>
  <caption>Expandable checks for the current language.</caption>
  <authorref id='auth:oberdiek'/>
  <copyright owner='Heiko Oberdiek' year='2007'/>
  <license type='lppl1.3'/>
  <version number='1.6'/>
  <description>
    This package provides expandable checks for the current language
    based on macro <tt>\languagename</tt> or hyphenation patterns.
    <p/>
    The package is part of the <xref refid='oberdiek'>oberdiek</xref> bundle.
  </description>
  <documentation details='Package documentation'
      href='ctan:/macros/latex/contrib/oberdiek/iflang.pdf'/>
  <ctan file='true' path='/macros/latex/contrib/oberdiek/iflang.dtx'/>
  <miktex location='oberdiek'/>
  <texlive location='oberdiek'/>
  <install path='/macros/latex/contrib/oberdiek/oberdiek.tds.zip'/>
</entry>
%</catalogue>
%    \end{macrocode}
%
% \section{Acknowledgement}
% I wish to thank:
% \begin{description}
% \item[Markus Kohm] Useful hints for version 1.2.
% \end{description}
%
% \begin{History}
%   \begin{Version}{2007/04/10 v1.0}
%   \item
%     First public version.
%   \end{Version}
%   \begin{Version}{2007/04/11 v1.1}
%   \item
%     Line ends sanitized.
%   \end{Version}
%   \begin{Version}{2007/04/12 v1.2}
%   \item
%     Initialization of \cs{languagename} in case of \xfile{etex.src}.
%   \item
%     Some sanity tests added.
%   \item
%     Documentation improved.
%   \end{Version}
%   \begin{Version}{2007/04/26 v1.3}
%   \item
%     Use of package \xpackage{infwarerr}.
%   \end{Version}
%   \begin{Version}{2007/09/09 v1.4}
%   \item
%     Bug fix: \cs{IfLang@StrEqual} $\rightarrow$ \cs{IfLangStrEqual}
%     (Gabriele Balducci).
%   \item
%     Catcode section rewritten.
%   \end{Version}
%   \begin{Version}{2007/11/11 v1.5}
%   \item
%     Use of package \xpackage{pdftexcmds} for \LuaTeX\ support.
%   \end{Version}
%   \begin{Version}{2016/05/16 v1.6}
%   \item
%     Documentation updates.
%   \end{Version}
% \end{History}
%
% \PrintIndex
%
% \Finale
\endinput
|
% \end{quote}
% Do not forget to quote the argument according to the demands
% of your shell.
%
% \paragraph{Generating the documentation.}
% You can use both the \xfile{.dtx} or the \xfile{.drv} to generate
% the documentation. The process can be configured by the
% configuration file \xfile{ltxdoc.cfg}. For instance, put this
% line into this file, if you want to have A4 as paper format:
% \begin{quote}
%   \verb|\PassOptionsToClass{a4paper}{article}|
% \end{quote}
% An example follows how to generate the
% documentation with pdf\LaTeX:
% \begin{quote}
%\begin{verbatim}
%pdflatex iflang.dtx
%makeindex -s gind.ist iflang.idx
%pdflatex iflang.dtx
%makeindex -s gind.ist iflang.idx
%pdflatex iflang.dtx
%\end{verbatim}
% \end{quote}
%
% \section{Catalogue}
%
% The following XML file can be used as source for the
% \href{http://mirror.ctan.org/help/Catalogue/catalogue.html}{\TeX\ Catalogue}.
% The elements \texttt{caption} and \texttt{description} are imported
% from the original XML file from the Catalogue.
% The name of the XML file in the Catalogue is \xfile{iflang.xml}.
%    \begin{macrocode}
%<*catalogue>
<?xml version='1.0' encoding='us-ascii'?>
<!DOCTYPE entry SYSTEM 'catalogue.dtd'>
<entry datestamp='$Date$' modifier='$Author$' id='iflang'>
  <name>iflang</name>
  <caption>Expandable checks for the current language.</caption>
  <authorref id='auth:oberdiek'/>
  <copyright owner='Heiko Oberdiek' year='2007'/>
  <license type='lppl1.3'/>
  <version number='1.6'/>
  <description>
    This package provides expandable checks for the current language
    based on macro <tt>\languagename</tt> or hyphenation patterns.
    <p/>
    The package is part of the <xref refid='oberdiek'>oberdiek</xref> bundle.
  </description>
  <documentation details='Package documentation'
      href='ctan:/macros/latex/contrib/oberdiek/iflang.pdf'/>
  <ctan file='true' path='/macros/latex/contrib/oberdiek/iflang.dtx'/>
  <miktex location='oberdiek'/>
  <texlive location='oberdiek'/>
  <install path='/macros/latex/contrib/oberdiek/oberdiek.tds.zip'/>
</entry>
%</catalogue>
%    \end{macrocode}
%
% \section{Acknowledgement}
% I wish to thank:
% \begin{description}
% \item[Markus Kohm] Useful hints for version 1.2.
% \end{description}
%
% \begin{History}
%   \begin{Version}{2007/04/10 v1.0}
%   \item
%     First public version.
%   \end{Version}
%   \begin{Version}{2007/04/11 v1.1}
%   \item
%     Line ends sanitized.
%   \end{Version}
%   \begin{Version}{2007/04/12 v1.2}
%   \item
%     Initialization of \cs{languagename} in case of \xfile{etex.src}.
%   \item
%     Some sanity tests added.
%   \item
%     Documentation improved.
%   \end{Version}
%   \begin{Version}{2007/04/26 v1.3}
%   \item
%     Use of package \xpackage{infwarerr}.
%   \end{Version}
%   \begin{Version}{2007/09/09 v1.4}
%   \item
%     Bug fix: \cs{IfLang@StrEqual} $\rightarrow$ \cs{IfLangStrEqual}
%     (Gabriele Balducci).
%   \item
%     Catcode section rewritten.
%   \end{Version}
%   \begin{Version}{2007/11/11 v1.5}
%   \item
%     Use of package \xpackage{pdftexcmds} for \LuaTeX\ support.
%   \end{Version}
%   \begin{Version}{2016/05/16 v1.6}
%   \item
%     Documentation updates.
%   \end{Version}
% \end{History}
%
% \PrintIndex
%
% \Finale
\endinput

%        (quote the arguments according to the demands of your shell)
%
% Documentation:
%    (a) If iflang.drv is present:
%           latex iflang.drv
%    (b) Without iflang.drv:
%           latex iflang.dtx; ...
%    The class ltxdoc loads the configuration file ltxdoc.cfg
%    if available. Here you can specify further options, e.g.
%    use A4 as paper format:
%       \PassOptionsToClass{a4paper}{article}
%
%    Programm calls to get the documentation (example):
%       pdflatex iflang.dtx
%       makeindex -s gind.ist iflang.idx
%       pdflatex iflang.dtx
%       makeindex -s gind.ist iflang.idx
%       pdflatex iflang.dtx
%
% Installation:
%    TDS:tex/generic/oberdiek/iflang.sty
%    TDS:doc/latex/oberdiek/iflang.pdf
%    TDS:doc/latex/oberdiek/test/iflang-test1.tex
%    TDS:doc/latex/oberdiek/test/iflang-test2.tex
%    TDS:doc/latex/oberdiek/test/iflang-test3.tex
%    TDS:doc/latex/oberdiek/test/iflang-test4.tex
%    TDS:doc/latex/oberdiek/test/iflang-test5.tex
%    TDS:source/latex/oberdiek/iflang.dtx
%
%<*ignore>
\begingroup
  \catcode123=1 %
  \catcode125=2 %
  \def\x{LaTeX2e}%
\expandafter\endgroup
\ifcase 0\ifx\install y1\fi\expandafter
         \ifx\csname processbatchFile\endcsname\relax\else1\fi
         \ifx\fmtname\x\else 1\fi\relax
\else\csname fi\endcsname
%</ignore>
%<*install>
\input docstrip.tex
\Msg{************************************************************************}
\Msg{* Installation}
\Msg{* Package: iflang 2016/05/16 v1.6 Checks for the current language (HO)}
\Msg{************************************************************************}

\keepsilent
\askforoverwritefalse

\let\MetaPrefix\relax
\preamble

This is a generated file.

Project: iflang
Version: 2016/05/16 v1.6

Copyright (C) 2007 by
   Heiko Oberdiek <heiko.oberdiek at googlemail.com>

This work may be distributed and/or modified under the
conditions of the LaTeX Project Public License, either
version 1.3c of this license or (at your option) any later
version. This version of this license is in
   http://www.latex-project.org/lppl/lppl-1-3c.txt
and the latest version of this license is in
   http://www.latex-project.org/lppl.txt
and version 1.3 or later is part of all distributions of
LaTeX version 2005/12/01 or later.

This work has the LPPL maintenance status "maintained".

This Current Maintainer of this work is Heiko Oberdiek.

The Base Interpreter refers to any `TeX-Format',
because some files are installed in TDS:tex/generic//.

This work consists of the main source file iflang.dtx
and the derived files
   iflang.sty, iflang.pdf, iflang.ins, iflang.drv, iflang-test1.tex,
   iflang-test2.tex, iflang-test3.tex, iflang-test4.tex,
   iflang-test5.tex.

\endpreamble
\let\MetaPrefix\DoubleperCent

\generate{%
  \file{iflang.ins}{\from{iflang.dtx}{install}}%
  \file{iflang.drv}{\from{iflang.dtx}{driver}}%
  \usedir{tex/generic/oberdiek}%
  \file{iflang.sty}{\from{iflang.dtx}{package}}%
  \usedir{doc/latex/oberdiek/test}%
  \file{iflang-test1.tex}{\from{iflang.dtx}{test1}}%
  \file{iflang-test2.tex}{\from{iflang.dtx}{test2}}%
  \file{iflang-test3.tex}{\from{iflang.dtx}{test3}}%
  \file{iflang-test4.tex}{\from{iflang.dtx}{test4}}%
  \file{iflang-test5.tex}{\from{iflang.dtx}{test5}}%
  \nopreamble
  \nopostamble
  \usedir{source/latex/oberdiek/catalogue}%
  \file{iflang.xml}{\from{iflang.dtx}{catalogue}}%
}

\catcode32=13\relax% active space
\let =\space%
\Msg{************************************************************************}
\Msg{*}
\Msg{* To finish the installation you have to move the following}
\Msg{* file into a directory searched by TeX:}
\Msg{*}
\Msg{*     iflang.sty}
\Msg{*}
\Msg{* To produce the documentation run the file `iflang.drv'}
\Msg{* through LaTeX.}
\Msg{*}
\Msg{* Happy TeXing!}
\Msg{*}
\Msg{************************************************************************}

\endbatchfile
%</install>
%<*ignore>
\fi
%</ignore>
%<*driver>
\NeedsTeXFormat{LaTeX2e}
\ProvidesFile{iflang.drv}%
  [2016/05/16 v1.6 Checks for the current language (HO)]%
\documentclass{ltxdoc}
\usepackage{holtxdoc}[2011/11/22]
\begin{document}
  \DocInput{iflang.dtx}%
\end{document}
%</driver>
% \fi
%
%
% \CharacterTable
%  {Upper-case    \A\B\C\D\E\F\G\H\I\J\K\L\M\N\O\P\Q\R\S\T\U\V\W\X\Y\Z
%   Lower-case    \a\b\c\d\e\f\g\h\i\j\k\l\m\n\o\p\q\r\s\t\u\v\w\x\y\z
%   Digits        \0\1\2\3\4\5\6\7\8\9
%   Exclamation   \!     Double quote  \"     Hash (number) \#
%   Dollar        \$     Percent       \%     Ampersand     \&
%   Acute accent  \'     Left paren    \(     Right paren   \)
%   Asterisk      \*     Plus          \+     Comma         \,
%   Minus         \-     Point         \.     Solidus       \/
%   Colon         \:     Semicolon     \;     Less than     \<
%   Equals        \=     Greater than  \>     Question mark \?
%   Commercial at \@     Left bracket  \[     Backslash     \\
%   Right bracket \]     Circumflex    \^     Underscore    \_
%   Grave accent  \`     Left brace    \{     Vertical bar  \|
%   Right brace   \}     Tilde         \~}
%
% \GetFileInfo{iflang.drv}
%
% \title{The \xpackage{iflang} package}
% \date{2016/05/16 v1.6}
% \author{Heiko Oberdiek\thanks
% {Please report any issues at https://github.com/ho-tex/oberdiek/issues}\\
% \xemail{heiko.oberdiek at googlemail.com}}
%
% \maketitle
%
% \begin{abstract}
% This package provides expandible checks for the current language
% based on macro \cs{languagename} or hyphenation patterns.
% \end{abstract}
%
% \tableofcontents
%
% \section{Documentation}
%
% Package \xpackage{babel} defines \cs{iflanguagename}. As first
% argument it takes a language name and executes the second or
% third argument depending on the current language. This
% language test is based on hypenation patterns. However, it
% is possible that different languages or dialects share the
% same patterns. In such cases \cs{iflanguagename} fails.
%
% However, package \xpackage{babel} and some other packages
% such as \xpackage{german} or \xpackage{ngerman} store
% the language name in the macro \cs{languagename} if
% \cs{selectlanguage} is called.
%
% \begin{declcs}{IfLanguageName} \M{lang} \M{then} \M{else}
% \end{declcs}
% Makro \cs{IfLanguageName} compares language \meta{lang} with
% the current setting of macro \cs{languagename}. If both
% contains the same name then the \meta{then} part is called,
% otherwise the \meta{else} part.
%
% The macro is expandable. Thus it can be safely used inside \cs{edef}
% or \cs{csname}. If case of errors like an undefined \cs{languagename}
% the \meta{else} part is executed.
%
% Note: Macro \cs{IfLanguageName} relies on the fact, that
% \cs{languagename} is set correctly:
% \begin{description}
% \item[Package \xpackage{babel}:]
%    \mbox{}\\
%    Full support of \cs{languagename} in its language switching commands.
% \item[Format based on \xpackage{babel} (\xfile{language.dat}):]
%    \mbox{}\\
%    If package \xpackage{babel} is not used (or not yet loaded),
%    then \xpackage{babel}'s
%    \xfile{hyphen.cfg} has set \cs{languagename} to the last
%    language in \xfile{language.dat}, but \cs{language} (current
%    patterns) is zero and points to the first language.
%    Thus the value of \cs{languagename} is basically garbage.
%    Package \xpackage{iflang} warns if \cs{languagename} and
%    \cs{language} do not fit.
%    This can be fixed by loading package \xpackage{babel} previously.
% \item[Format based on \eTeX's \xfile{etex.src} (\xfile{language.def}):]
%    \mbox{}\\
%    Unhappily it does not support
%    \cs{languagename}. Thus this package hooks into \cs{uselanguage}
%    to get \cs{languagename} defined and updated there.
%    At package loading time the changed \cs{uselanguage} has not
%    been called yet. Thus package \xpackage{iflang} tries
%    |USenglish|. This is the definite default language of \xfile{etex.src}.
%    If the current patterns suit this default language, an
%    undefined \cs{languagename} is set to this language. Otherwise
%    a \cs{languagename} remains undefined and a warning is given.
% \end{description}
%
% \begin{declcs}{IfLanguagePatterns} \M{lang} \M{then} \M{else}
% \end{declcs}
% This macro behaves similar to \cs{IfLanguageName}. But the
% language test is based on the current pattern in force (\cs{language}).
% Also this macro is expandable, in case of errors the \meta{else} part
% is called.
%
% The following naming convention for the pattern are supported:
% \begin{description}
% \item[\xpackage{babel}/\xfile{language.dat}]:
%    \cs{l@}\meta{language}
% \item[\xpackage{etex.src}/\xfile{language.def}]:
%    \cs{lang@}\meta{language}
% \end{description}
% Package \xpackage{iflang} looks for \cs{uselanguage} (defined in
% \xfile{etex.src}) to find out the naming convention in use.
%
% \StopEventually{
% }
%
% \section{Implementation}
%
%    \begin{macrocode}
%<*package>
%    \end{macrocode}
%
% \subsection{Reload check and package identification}
%    Reload check, especially if the package is not used with \LaTeX.
%    \begin{macrocode}
\begingroup\catcode61\catcode48\catcode32=10\relax%
  \catcode13=5 % ^^M
  \endlinechar=13 %
  \catcode35=6 % #
  \catcode39=12 % '
  \catcode44=12 % ,
  \catcode45=12 % -
  \catcode46=12 % .
  \catcode58=12 % :
  \catcode64=11 % @
  \catcode123=1 % {
  \catcode125=2 % }
  \expandafter\let\expandafter\x\csname ver@iflang.sty\endcsname
  \ifx\x\relax % plain-TeX, first loading
  \else
    \def\empty{}%
    \ifx\x\empty % LaTeX, first loading,
      % variable is initialized, but \ProvidesPackage not yet seen
    \else
      \expandafter\ifx\csname PackageInfo\endcsname\relax
        \def\x#1#2{%
          \immediate\write-1{Package #1 Info: #2.}%
        }%
      \else
        \def\x#1#2{\PackageInfo{#1}{#2, stopped}}%
      \fi
      \x{iflang}{The package is already loaded}%
      \aftergroup\endinput
    \fi
  \fi
\endgroup%
%    \end{macrocode}
%    Package identification:
%    \begin{macrocode}
\begingroup\catcode61\catcode48\catcode32=10\relax%
  \catcode13=5 % ^^M
  \endlinechar=13 %
  \catcode35=6 % #
  \catcode39=12 % '
  \catcode40=12 % (
  \catcode41=12 % )
  \catcode44=12 % ,
  \catcode45=12 % -
  \catcode46=12 % .
  \catcode47=12 % /
  \catcode58=12 % :
  \catcode64=11 % @
  \catcode91=12 % [
  \catcode93=12 % ]
  \catcode123=1 % {
  \catcode125=2 % }
  \expandafter\ifx\csname ProvidesPackage\endcsname\relax
    \def\x#1#2#3[#4]{\endgroup
      \immediate\write-1{Package: #3 #4}%
      \xdef#1{#4}%
    }%
  \else
    \def\x#1#2[#3]{\endgroup
      #2[{#3}]%
      \ifx#1\@undefined
        \xdef#1{#3}%
      \fi
      \ifx#1\relax
        \xdef#1{#3}%
      \fi
    }%
  \fi
\expandafter\x\csname ver@iflang.sty\endcsname
\ProvidesPackage{iflang}%
  [2016/05/16 v1.6 Checks for the current language (HO)]%
%    \end{macrocode}
%
%    \begin{macrocode}
\begingroup\catcode61\catcode48\catcode32=10\relax%
  \catcode13=5 % ^^M
  \endlinechar=13 %
  \catcode123=1 % {
  \catcode125=2 % }
  \catcode64=11 % @
  \def\x{\endgroup
    \expandafter\edef\csname IfLang@AtEnd\endcsname{%
      \endlinechar=\the\endlinechar\relax
      \catcode13=\the\catcode13\relax
      \catcode32=\the\catcode32\relax
      \catcode35=\the\catcode35\relax
      \catcode61=\the\catcode61\relax
      \catcode64=\the\catcode64\relax
      \catcode123=\the\catcode123\relax
      \catcode125=\the\catcode125\relax
    }%
  }%
\x\catcode61\catcode48\catcode32=10\relax%
\catcode13=5 % ^^M
\endlinechar=13 %
\catcode35=6 % #
\catcode64=11 % @
\catcode123=1 % {
\catcode125=2 % }
\def\TMP@EnsureCode#1#2{%
  \edef\IfLang@AtEnd{%
    \IfLang@AtEnd
    \catcode#1=\the\catcode#1\relax
  }%
  \catcode#1=#2\relax
}
\TMP@EnsureCode{39}{12}% '
\TMP@EnsureCode{40}{12}% (
\TMP@EnsureCode{41}{12}% )
\TMP@EnsureCode{44}{12}% ,
\TMP@EnsureCode{46}{12}% .
\TMP@EnsureCode{47}{12}% /
\TMP@EnsureCode{58}{12}% :
\TMP@EnsureCode{91}{12}% [
\TMP@EnsureCode{93}{12}% ]
\edef\IfLang@AtEnd{\IfLang@AtEnd\noexpand\endinput}
%    \end{macrocode}
%
% \subsection{Tools}
%
% \subsubsection{Provide some basic macros of \LaTeX}
%
%    \begin{macro}{\@firstoftwo}
%    \begin{macrocode}
\expandafter\ifx\csname @firstoftwo\endcsname\relax
  \long\def\@firstoftwo#1#2{#1}%
\fi
%    \end{macrocode}
%    \end{macro}
%    \begin{macro}{\@secondoftwo}
%    \begin{macrocode}
\expandafter\ifx\csname @secondoftwo\endcsname\relax
  \long\def\@secondoftwo#1#2{#2}%
\fi
%    \end{macrocode}
%    \end{macro}
%
% \subsubsection{Expandible existence check for macros}
%
%    \begin{macro}{\IfLang@IfDefined}
%    \begin{macrocode}
\begingroup\expandafter\expandafter\expandafter\endgroup
\expandafter\ifx\csname ifcsname\endcsname\relax
  \expandafter\@firstoftwo
\else
  \expandafter\@secondoftwo
\fi
{%
  \def\IfLang@IfDefined#1{%
    \expandafter\ifx\csname#1\endcsname\relax
      \expandafter\@secondoftwo
    \else
      \expandafter\@firstoftwo
    \fi
  }%
}{%
  \def\IfLang@IfDefined#1{%
    \ifnum\ifcsname#1\endcsname
            \expandafter\ifx\csname#1\endcsname\relax
              1%
            \else
              0%
            \fi
          \else
            1%
          \fi
          =0 %
      \expandafter\@firstoftwo
    \else
      \expandafter\@secondoftwo
    \fi
  }%
}
%    \end{macrocode}
%    \end{macro}
%
% \subsubsection{Macros for messages}
%
%    \begin{macrocode}
\begingroup\expandafter\expandafter\expandafter\endgroup
\expandafter\ifx\csname RequirePackage\endcsname\relax
  \input infwarerr.sty\relax
  \input pdftexcmds.sty\relax
\else
  \RequirePackage{infwarerr}[2007/09/09]%
  \RequirePackage{pdftexcmds}[2016/05/16]%
\fi
%    \end{macrocode}
%
% \subsubsection{Support for \xfile{etex.src}}
%
%    \begin{macro}{\IfLang@prefix}
%    \begin{macrocode}
\begingroup\expandafter\expandafter\expandafter\endgroup
\expandafter\ifx\csname uselanguage\endcsname\relax
  \@PackageInfoNoLine{iflang}{%
    Naming convention for patterns: babel%
  }%
  \def\IfLang@prefix{l@}%
\else
  \@PackageInfoNoLine{iflang}{%
    Naming convention for patterns: etex.src%
  }%
  \def\IfLang@prefix{lang@}%
  \let\IfLang@OrgUseLanguage\uselanguage
  \def\uselanguage#1{%
    \edef\languagename{#1}%
    \IfLang@OrgUseLanguage{#1}%
  }%
%    \end{macrocode}
%    \end{macro}
%    The first \cs{uselanguage} that is executed as last
%    line in \xfile{language.def} cannot patched this way.
%    However, \xfile{language.def} is very strict. It forces
%    the first added and used language to be |USenglish|.
%    Thus, if \cs{languagename} is not defined, we can
%    quite safely assume |USenglish|. As additional
%    safety precaution the actual used patterns are checked.
%    \begin{macrocode}
  \begingroup\expandafter\expandafter\expandafter\endgroup
  \expandafter\ifx\csname languagename\endcsname\relax
    \begingroup\expandafter\expandafter\expandafter\endgroup
    \expandafter\ifx\csname lang@USenglish\endcsname\relax
      \@PackageWarningNoLine{iflang}{%
        \string\lang@USenglish\space is missing%
      }%
    \else
      \ifnum\lang@USenglish=\language
        \def\languagename{USenglish}%
      \else
        \@PackageWarningNoLine{iflang}{%
          \string\languagename\space is not set,\MessageBreak
          current language is unknown%
        }%
      \fi
    \fi
  \fi
\fi
\begingroup\expandafter\expandafter\expandafter\endgroup
\expandafter\ifx\csname languagename\endcsname\relax
  \@PackageInfoNoLine{iflang}{%
    \string\languagename\space is not set%
  }%
\fi
%    \end{macrocode}
%
% \subsection{\cs{IfLanguagePatterns}}
%
%    \begin{macro}{\IfLanguagePatterns}
%    \begin{macrocode}
\def\IfLanguagePatterns#1{%
  \ifnum\IfLang@IfDefined{\IfLang@prefix#1}{%
          \ifnum\csname\IfLang@prefix#1\endcsname=\language
            0%
          \else
            1%
          \fi
        }{1}=0 %
    \expandafter\@firstoftwo
  \else
    \expandafter\@secondoftwo
  \fi
}
%    \end{macrocode}
%    \end{macro}
%
% \subsection{\cs{IfLanguageName}}
%
%    \begin{macrocode}
\begingroup\expandafter\expandafter\expandafter\endgroup
\expandafter\ifx\csname pdf@strcmp\endcsname\relax
  \expandafter\@firstoftwo
\else
  \expandafter\@secondoftwo
\fi
{%
%    \end{macrocode}
%    We do not have \cs{pdf@strcmp} (and \cs{pdfstrcmp}).
%    Thus we must define
%    our own expandable string comparison. The following
%    implementation is based on a \TeX\ pearl from David Kastrup, presented
%    at the conference Bacho\TeX~2005:
%    \url{http://www-stary.gust.org.pl/pearls/2005/david-kastrup/bachotex2005-david-kastrup-pearl1.pdf}
%
%    The orignal code allows macros inside the second string.
%    Because also \cs{languagename} might consists of further
%    macros, we need a variant that allows macros in the first
%    string, too.
%    \begin{macrocode}
  \def\IfLang@StrNil{\relax}%
  \def\IfLang@StrEqual#1{%
    \number\IfLang@StrEqualStart{}{}#1\IfLang@StrNil
  }%
  \def\IfLang@StrEqualStart#1#2#3{%
    \ifx#3\IfLang@StrNil
      \IfLang@StrEqualStop
    \fi
    \ifcat\noexpand#3\relax
      \IfLang@StrExpand{#1}{#2}#3%
    \fi
    \IfLang@StrEqualStart{\if#3#1}{#2\fi}%
  }%
  \def\IfLang@StrEqualStop\fi#1\IfLang@StrEqualStart#2#3#4{%
    \fi
    #2#4\relax'#313 %
  }%
  \def\IfLang@StrExpand#1#2#3\fi\IfLang@StrEqualStart#4#5{%
    \fi
    \IfLang@@StrExpand{#1}{#2}#3%
  }%
  \def\IfLang@@StrExpand#1#2#3\IfLang@StrNil{%
    \expandafter\IfLang@@@StrExpand#3\IfLang@StrNil{#1}{#2}%
  }%
  \def\IfLang@@@StrExpand#1\IfLang@StrNil#2#3{%
    \IfLang@StrEqualStart{#2}{#3}#1\IfLang@StrNil
  }%
%    \end{macrocode}
%    \begin{macro}{\IfLanguageName}
%    \begin{macrocode}
  \def\IfLanguageName#1{%
    \ifnum\IfLang@IfDefined{languagename}{%
            \if\expandafter\IfLang@StrEqual\expandafter%
                           {\languagename}{#1}%
              0%
            \else
              1%
            \fi
          }{1}=0 %
      \expandafter\@firstoftwo
    \else
      \expandafter\@secondoftwo
    \fi
  }%
%    \end{macrocode}
%    \end{macro}
%    \begin{macrocode}
}{%
%    \end{macrocode}
%    \begin{macro}{\IfLanguageName}
%    \begin{macrocode}
  \def\IfLanguageName#1{%
    \ifnum\IfLang@IfDefined{languagename}{%
            \pdf@strcmp{#1}{\languagename}%
          }{1}=0 %
      \expandafter\@firstoftwo
    \else
      \expandafter\@secondoftwo
    \fi
  }%
%    \end{macrocode}
%    \end{macro}
%    \begin{macrocode}
}
%    \end{macrocode}
%
% \subsection{Check plausibility of \cs{languagename}}
%
%    \begin{macrocode}
\begingroup\expandafter\expandafter\expandafter\endgroup
\expandafter\ifx\csname languagename\endcsname\relax
\else
  \IfLanguagePatterns{\languagename}{}{%
    \@PackageWarningNoLine{iflang}{%
      Mismatch between \string\language\space
      (patterns)\MessageBreak
      and setting of \string\languagename
    }%
  }%
\fi
%    \end{macrocode}
%
%    \begin{macrocode}
\IfLang@AtEnd%
%</package>
%    \end{macrocode}
%
% \section{Test}
%
% \subsection{Catcode checks for loading}
%
%    \begin{macrocode}
%<*test1>
%    \end{macrocode}
%    \begin{macrocode}
\catcode`\{=1 %
\catcode`\}=2 %
\catcode`\#=6 %
\catcode`\@=11 %
\expandafter\ifx\csname count@\endcsname\relax
  \countdef\count@=255 %
\fi
\expandafter\ifx\csname @gobble\endcsname\relax
  \long\def\@gobble#1{}%
\fi
\expandafter\ifx\csname @firstofone\endcsname\relax
  \long\def\@firstofone#1{#1}%
\fi
\expandafter\ifx\csname loop\endcsname\relax
  \expandafter\@firstofone
\else
  \expandafter\@gobble
\fi
{%
  \def\loop#1\repeat{%
    \def\body{#1}%
    \iterate
  }%
  \def\iterate{%
    \body
      \let\next\iterate
    \else
      \let\next\relax
    \fi
    \next
  }%
  \let\repeat=\fi
}%
\def\RestoreCatcodes{}
\count@=0 %
\loop
  \edef\RestoreCatcodes{%
    \RestoreCatcodes
    \catcode\the\count@=\the\catcode\count@\relax
  }%
\ifnum\count@<255 %
  \advance\count@ 1 %
\repeat

\def\RangeCatcodeInvalid#1#2{%
  \count@=#1\relax
  \loop
    \catcode\count@=15 %
  \ifnum\count@<#2\relax
    \advance\count@ 1 %
  \repeat
}
\def\RangeCatcodeCheck#1#2#3{%
  \count@=#1\relax
  \loop
    \ifnum#3=\catcode\count@
    \else
      \errmessage{%
        Character \the\count@\space
        with wrong catcode \the\catcode\count@\space
        instead of \number#3%
      }%
    \fi
  \ifnum\count@<#2\relax
    \advance\count@ 1 %
  \repeat
}
\def\space{ }
\expandafter\ifx\csname LoadCommand\endcsname\relax
  \def\LoadCommand{\input iflang.sty\relax}%
\fi
\def\Test{%
  \RangeCatcodeInvalid{0}{47}%
  \RangeCatcodeInvalid{58}{64}%
  \RangeCatcodeInvalid{91}{96}%
  \RangeCatcodeInvalid{123}{255}%
  \catcode`\@=12 %
  \catcode`\\=0 %
  \catcode`\%=14 %
  \LoadCommand
  \RangeCatcodeCheck{0}{36}{15}%
  \RangeCatcodeCheck{37}{37}{14}%
  \RangeCatcodeCheck{38}{47}{15}%
  \RangeCatcodeCheck{48}{57}{12}%
  \RangeCatcodeCheck{58}{63}{15}%
  \RangeCatcodeCheck{64}{64}{12}%
  \RangeCatcodeCheck{65}{90}{11}%
  \RangeCatcodeCheck{91}{91}{15}%
  \RangeCatcodeCheck{92}{92}{0}%
  \RangeCatcodeCheck{93}{96}{15}%
  \RangeCatcodeCheck{97}{122}{11}%
  \RangeCatcodeCheck{123}{255}{15}%
  \RestoreCatcodes
}
\Test
\csname @@end\endcsname
\end
%    \end{macrocode}
%    \begin{macrocode}
%</test1>
%    \end{macrocode}
%
% \subsection{Test with \LaTeX}
%
%    \begin{macrocode}
%<*test2|test3>
%    \end{macrocode}
%    \begin{macrocode}
\NeedsTeXFormat{LaTeX2e}
%<test3>\let\pdfstrcmp\relax
\nofiles
\documentclass{minimal}
\usepackage{qstest}
\IncludeTests{*}
\LogTests{log}{*}{*}
\usepackage[english,naustrian,ngerman]{babel}
\usepackage{iflang}
\begin{document}
\begin{qstest}{IfLanguagePatterns}{language, pattern}
  \def\test#1#2{%
    \Expect*{\IfLanguagePatterns{#1}{true}{false}}{#2}%
  }%
  \test{ngerman}{true}%
  \test{naustrian}{true}%
  \test{english}{false}%
  \test{foobar}{false}%
\end{qstest}
\begin{qstest}{IfLanguageName}{language, name}
  \def\test#1#2{%
    \Expect*{\IfLanguageName{#1}{true}{false}}{#2}%
  }%
  \test{ngerman}{true}%
  \test{naustrian}{false}%
  \selectlanguage{naustrian}%
  \test{ngerman}{false}%
  \test{naustrian}{true}%
  \test{foobar}{false}%
  %
  \def\languagename{naustrian}%
  \test{naustrian}{true}%
  \test{ngerman}{false}%
  %
  \edef\languagename{\string naustrian}%
  \test{naustrian}{true}%
  \test{ngerman}{false}%
  %
  \def\languagename{naustrian}%
  \makeatletter
  \@onelevel@sanitize\languagename
  \test{naustrian}{true}%
  \test{ngerman}{false}%
  %
  \def\languagename{naustrian}%
  \def\xaustrian{naustrian}%
  \def\xgerman{ngerman}%
  \test{\xaustrian}{true}%
  \test{\xgerman}{false}%
  %
  \def\languagename{\xaustrian}%
  \test{naustrian}{true}%
  \test{ngerman}{false}%
  \test{\xaustrian}{true}%
  \test{\xgerman}{false}%
  \test{\languagename}{true}%
  \test{\languagename\space}{false}%
  %
  \def\languagename{\empty\xaustrian\empty}%
  \test{naustrian}{true}%
  \test{ngerman}{false}%
  \test{\empty\xaustrian\empty}{true}%
  \test{\empty\xgerman\empty}{false}%
\end{qstest}
\begin{qstest}{IfDefined}{defined}
  \makeatletter
  \let\foobar\relax
  \Expect*{\IfLang@IfDefined{foobar}{true}{false}}{false}%
  \Expect*{\ifx\foobar\relax true\else false\fi}{true}%
  \let\foobar\UNDEFINED
  \Expect*{\IfLang@IfDefined{foobar}{true}{false}}{false}%
  \Expect*{\ifx\foobar\relax true\else false\fi}{false}%
  \Expect*{\ifx\foobar\UNDEFINED true\else false\fi}{true}%
\end{qstest}
\end{document}
%    \end{macrocode}
%    \begin{macrocode}
%</test2|test3>
%    \end{macrocode}
%
% \subsection{Test with \plainTeX\ and \eTeX}
%
%    \begin{macrocode}
%<*test4>
%    \end{macrocode}
%    \begin{macrocode}
%% Format `etex' based on `language.def'
\input iflang.sty
\catcode64=12

\def\TestGeneric#1#2#3{%
  \begingroup
    \edef\x{#1{#2}{true}{false}}%
    \edef\y{#3}%
    \ifx\x\y
    \else
      \errmessage{Failed test: \string#1{#2} <> #3}%
    \fi
  \endgroup
}
\def\TestPatterns{\TestGeneric\IfLanguagePatterns}
\def\TestName{\TestGeneric\IfLanguageName}

\TestPatterns{USenglish}{true}
\TestPatterns{ngerman}{false}

\TestName{USenglish}{true}
\TestName{ngerman}{false}

\uselanguage{ngerman}

\TestPatterns{USenglish}{false}
\TestPatterns{ngerman}{true}

\TestName{USenglish}{false}
\TestName{ngerman}{true}

\csname @@end\endcsname
\end
%    \end{macrocode}
%    \begin{macrocode}
%</test4>
%    \end{macrocode}
%
% \subsection{Test with \plainTeX\ and without \eTeX/\pdfTeX}
%
%    \begin{macrocode}
%<*test5>
%    \end{macrocode}
%    \begin{macrocode}
%% Format `tex' (vanilla plain-TeX)
\let\ifcsname\UNDEFINED
\let\pdfstrcmp\UNDEFINED
\input iflang.sty
\catcode64=11

\def\TestDefined#1{%
  \IfLang@IfDefined{foobar}{}{}%
  \ifx\foobar#1%
  \else
    \errmessage{Failed test: \string\foobar <> \string#1}%
  \fi
}
\let\foobar\relax
\TestDefined\relax
\let\foobar\UNDEFINED
\TestDefined\relax

\def\strip@prefix#1>{}
\def\@onelevel@sanitize#1{%
  \edef#1{\expandafter\strip@prefix\meaning#1}%
}
\def\TestCompare#1#2#3{%
  \begingroup
    \edef\x{%
      \if\IfLang@StrEqual{#1}{#2}%
        true%
      \else
        false%
      \fi
    }%
    \def\expect{#3}%
    \ifx\x\expect
    \else
      \def\a{#1}%
      \@onelevel@sanitize\a
      \def\b{#2}%
      \@onelevel@sanitize\b
      \errmessage{Failed test: `\a'='\b' <> \expect}%
    \fi
  \endgroup
}
\TestCompare{junk}{junk}{true}
\TestCompare{}{}{true}
\TestCompare{a}{b}{false}
\TestCompare{aa}{bb}{false}
\def\a{ax}
\def\b{bx}
\def\c{\a\b}
\def\d{\c\b}
\def\exch#1#2{#2#1}
\def\gobble#1{}
\TestCompare{\gobble a}{}{true}
\TestCompare{}{\gobble a}{true}
\TestCompare{a\exch xyb}{ayxb}{true}
\TestCompare{\c}{\c}{true}
\TestCompare{\d}{\c\b}{true}

\csname @@end\endcsname
\end
%    \end{macrocode}
%    \begin{macrocode}
%</test5>
%    \end{macrocode}
%
% \section{Installation}
%
% \subsection{Download}
%
% \paragraph{Package.} This package is available on
% CTAN\footnote{\url{http://ctan.org/pkg/iflang}}:
% \begin{description}
% \item[\CTAN{macros/latex/contrib/oberdiek/iflang.dtx}] The source file.
% \item[\CTAN{macros/latex/contrib/oberdiek/iflang.pdf}] Documentation.
% \end{description}
%
%
% \paragraph{Bundle.} All the packages of the bundle `oberdiek'
% are also available in a TDS compliant ZIP archive. There
% the packages are already unpacked and the documentation files
% are generated. The files and directories obey the TDS standard.
% \begin{description}
% \item[\CTAN{install/macros/latex/contrib/oberdiek.tds.zip}]
% \end{description}
% \emph{TDS} refers to the standard ``A Directory Structure
% for \TeX\ Files'' (\CTAN{tds/tds.pdf}). Directories
% with \xfile{texmf} in their name are usually organized this way.
%
% \subsection{Bundle installation}
%
% \paragraph{Unpacking.} Unpack the \xfile{oberdiek.tds.zip} in the
% TDS tree (also known as \xfile{texmf} tree) of your choice.
% Example (linux):
% \begin{quote}
%   |unzip oberdiek.tds.zip -d ~/texmf|
% \end{quote}
%
% \paragraph{Script installation.}
% Check the directory \xfile{TDS:scripts/oberdiek/} for
% scripts that need further installation steps.
% Package \xpackage{attachfile2} comes with the Perl script
% \xfile{pdfatfi.pl} that should be installed in such a way
% that it can be called as \texttt{pdfatfi}.
% Example (linux):
% \begin{quote}
%   |chmod +x scripts/oberdiek/pdfatfi.pl|\\
%   |cp scripts/oberdiek/pdfatfi.pl /usr/local/bin/|
% \end{quote}
%
% \subsection{Package installation}
%
% \paragraph{Unpacking.} The \xfile{.dtx} file is a self-extracting
% \docstrip\ archive. The files are extracted by running the
% \xfile{.dtx} through \plainTeX:
% \begin{quote}
%   \verb|tex iflang.dtx|
% \end{quote}
%
% \paragraph{TDS.} Now the different files must be moved into
% the different directories in your installation TDS tree
% (also known as \xfile{texmf} tree):
% \begin{quote}
% \def\t{^^A
% \begin{tabular}{@{}>{\ttfamily}l@{ $\rightarrow$ }>{\ttfamily}l@{}}
%   iflang.sty & tex/generic/oberdiek/iflang.sty\\
%   iflang.pdf & doc/latex/oberdiek/iflang.pdf\\
%   test/iflang-test1.tex & doc/latex/oberdiek/test/iflang-test1.tex\\
%   test/iflang-test2.tex & doc/latex/oberdiek/test/iflang-test2.tex\\
%   test/iflang-test3.tex & doc/latex/oberdiek/test/iflang-test3.tex\\
%   test/iflang-test4.tex & doc/latex/oberdiek/test/iflang-test4.tex\\
%   test/iflang-test5.tex & doc/latex/oberdiek/test/iflang-test5.tex\\
%   iflang.dtx & source/latex/oberdiek/iflang.dtx\\
% \end{tabular}^^A
% }^^A
% \sbox0{\t}^^A
% \ifdim\wd0>\linewidth
%   \begingroup
%     \advance\linewidth by\leftmargin
%     \advance\linewidth by\rightmargin
%   \edef\x{\endgroup
%     \def\noexpand\lw{\the\linewidth}^^A
%   }\x
%   \def\lwbox{^^A
%     \leavevmode
%     \hbox to \linewidth{^^A
%       \kern-\leftmargin\relax
%       \hss
%       \usebox0
%       \hss
%       \kern-\rightmargin\relax
%     }^^A
%   }^^A
%   \ifdim\wd0>\lw
%     \sbox0{\small\t}^^A
%     \ifdim\wd0>\linewidth
%       \ifdim\wd0>\lw
%         \sbox0{\footnotesize\t}^^A
%         \ifdim\wd0>\linewidth
%           \ifdim\wd0>\lw
%             \sbox0{\scriptsize\t}^^A
%             \ifdim\wd0>\linewidth
%               \ifdim\wd0>\lw
%                 \sbox0{\tiny\t}^^A
%                 \ifdim\wd0>\linewidth
%                   \lwbox
%                 \else
%                   \usebox0
%                 \fi
%               \else
%                 \lwbox
%               \fi
%             \else
%               \usebox0
%             \fi
%           \else
%             \lwbox
%           \fi
%         \else
%           \usebox0
%         \fi
%       \else
%         \lwbox
%       \fi
%     \else
%       \usebox0
%     \fi
%   \else
%     \lwbox
%   \fi
% \else
%   \usebox0
% \fi
% \end{quote}
% If you have a \xfile{docstrip.cfg} that configures and enables \docstrip's
% TDS installing feature, then some files can already be in the right
% place, see the documentation of \docstrip.
%
% \subsection{Refresh file name databases}
%
% If your \TeX~distribution
% (\teTeX, \mikTeX, \dots) relies on file name databases, you must refresh
% these. For example, \teTeX\ users run \verb|texhash| or
% \verb|mktexlsr|.
%
% \subsection{Some details for the interested}
%
% \paragraph{Attached source.}
%
% The PDF documentation on CTAN also includes the
% \xfile{.dtx} source file. It can be extracted by
% AcrobatReader 6 or higher. Another option is \textsf{pdftk},
% e.g. unpack the file into the current directory:
% \begin{quote}
%   \verb|pdftk iflang.pdf unpack_files output .|
% \end{quote}
%
% \paragraph{Unpacking with \LaTeX.}
% The \xfile{.dtx} chooses its action depending on the format:
% \begin{description}
% \item[\plainTeX:] Run \docstrip\ and extract the files.
% \item[\LaTeX:] Generate the documentation.
% \end{description}
% If you insist on using \LaTeX\ for \docstrip\ (really,
% \docstrip\ does not need \LaTeX), then inform the autodetect routine
% about your intention:
% \begin{quote}
%   \verb|latex \let\install=y% \iffalse meta-comment
%
% File: iflang.dtx
% Version: 2016/05/16 v1.6
% Info: Checks for the current language
%
% Copyright (C) 2007 by
%    Heiko Oberdiek <heiko.oberdiek at googlemail.com>
%    2016
%    https://github.com/ho-tex/oberdiek/issues
%
% This work may be distributed and/or modified under the
% conditions of the LaTeX Project Public License, either
% version 1.3c of this license or (at your option) any later
% version. This version of this license is in
%    http://www.latex-project.org/lppl/lppl-1-3c.txt
% and the latest version of this license is in
%    http://www.latex-project.org/lppl.txt
% and version 1.3 or later is part of all distributions of
% LaTeX version 2005/12/01 or later.
%
% This work has the LPPL maintenance status "maintained".
%
% This Current Maintainer of this work is Heiko Oberdiek.
%
% The Base Interpreter refers to any `TeX-Format',
% because some files are installed in TDS:tex/generic//.
%
% This work consists of the main source file iflang.dtx
% and the derived files
%    iflang.sty, iflang.pdf, iflang.ins, iflang.drv, iflang-test1.tex,
%    iflang-test2.tex, iflang-test3.tex, iflang-test4.tex,
%    iflang-test5.tex.
%
% Distribution:
%    CTAN:macros/latex/contrib/oberdiek/iflang.dtx
%    CTAN:macros/latex/contrib/oberdiek/iflang.pdf
%
% Unpacking:
%    (a) If iflang.ins is present:
%           tex iflang.ins
%    (b) Without iflang.ins:
%           tex iflang.dtx
%    (c) If you insist on using LaTeX
%           latex \let\install=y% \iffalse meta-comment
%
% File: iflang.dtx
% Version: 2016/05/16 v1.6
% Info: Checks for the current language
%
% Copyright (C) 2007 by
%    Heiko Oberdiek <heiko.oberdiek at googlemail.com>
%    2016
%    https://github.com/ho-tex/oberdiek/issues
%
% This work may be distributed and/or modified under the
% conditions of the LaTeX Project Public License, either
% version 1.3c of this license or (at your option) any later
% version. This version of this license is in
%    http://www.latex-project.org/lppl/lppl-1-3c.txt
% and the latest version of this license is in
%    http://www.latex-project.org/lppl.txt
% and version 1.3 or later is part of all distributions of
% LaTeX version 2005/12/01 or later.
%
% This work has the LPPL maintenance status "maintained".
%
% This Current Maintainer of this work is Heiko Oberdiek.
%
% The Base Interpreter refers to any `TeX-Format',
% because some files are installed in TDS:tex/generic//.
%
% This work consists of the main source file iflang.dtx
% and the derived files
%    iflang.sty, iflang.pdf, iflang.ins, iflang.drv, iflang-test1.tex,
%    iflang-test2.tex, iflang-test3.tex, iflang-test4.tex,
%    iflang-test5.tex.
%
% Distribution:
%    CTAN:macros/latex/contrib/oberdiek/iflang.dtx
%    CTAN:macros/latex/contrib/oberdiek/iflang.pdf
%
% Unpacking:
%    (a) If iflang.ins is present:
%           tex iflang.ins
%    (b) Without iflang.ins:
%           tex iflang.dtx
%    (c) If you insist on using LaTeX
%           latex \let\install=y\input{iflang.dtx}
%        (quote the arguments according to the demands of your shell)
%
% Documentation:
%    (a) If iflang.drv is present:
%           latex iflang.drv
%    (b) Without iflang.drv:
%           latex iflang.dtx; ...
%    The class ltxdoc loads the configuration file ltxdoc.cfg
%    if available. Here you can specify further options, e.g.
%    use A4 as paper format:
%       \PassOptionsToClass{a4paper}{article}
%
%    Programm calls to get the documentation (example):
%       pdflatex iflang.dtx
%       makeindex -s gind.ist iflang.idx
%       pdflatex iflang.dtx
%       makeindex -s gind.ist iflang.idx
%       pdflatex iflang.dtx
%
% Installation:
%    TDS:tex/generic/oberdiek/iflang.sty
%    TDS:doc/latex/oberdiek/iflang.pdf
%    TDS:doc/latex/oberdiek/test/iflang-test1.tex
%    TDS:doc/latex/oberdiek/test/iflang-test2.tex
%    TDS:doc/latex/oberdiek/test/iflang-test3.tex
%    TDS:doc/latex/oberdiek/test/iflang-test4.tex
%    TDS:doc/latex/oberdiek/test/iflang-test5.tex
%    TDS:source/latex/oberdiek/iflang.dtx
%
%<*ignore>
\begingroup
  \catcode123=1 %
  \catcode125=2 %
  \def\x{LaTeX2e}%
\expandafter\endgroup
\ifcase 0\ifx\install y1\fi\expandafter
         \ifx\csname processbatchFile\endcsname\relax\else1\fi
         \ifx\fmtname\x\else 1\fi\relax
\else\csname fi\endcsname
%</ignore>
%<*install>
\input docstrip.tex
\Msg{************************************************************************}
\Msg{* Installation}
\Msg{* Package: iflang 2016/05/16 v1.6 Checks for the current language (HO)}
\Msg{************************************************************************}

\keepsilent
\askforoverwritefalse

\let\MetaPrefix\relax
\preamble

This is a generated file.

Project: iflang
Version: 2016/05/16 v1.6

Copyright (C) 2007 by
   Heiko Oberdiek <heiko.oberdiek at googlemail.com>

This work may be distributed and/or modified under the
conditions of the LaTeX Project Public License, either
version 1.3c of this license or (at your option) any later
version. This version of this license is in
   http://www.latex-project.org/lppl/lppl-1-3c.txt
and the latest version of this license is in
   http://www.latex-project.org/lppl.txt
and version 1.3 or later is part of all distributions of
LaTeX version 2005/12/01 or later.

This work has the LPPL maintenance status "maintained".

This Current Maintainer of this work is Heiko Oberdiek.

The Base Interpreter refers to any `TeX-Format',
because some files are installed in TDS:tex/generic//.

This work consists of the main source file iflang.dtx
and the derived files
   iflang.sty, iflang.pdf, iflang.ins, iflang.drv, iflang-test1.tex,
   iflang-test2.tex, iflang-test3.tex, iflang-test4.tex,
   iflang-test5.tex.

\endpreamble
\let\MetaPrefix\DoubleperCent

\generate{%
  \file{iflang.ins}{\from{iflang.dtx}{install}}%
  \file{iflang.drv}{\from{iflang.dtx}{driver}}%
  \usedir{tex/generic/oberdiek}%
  \file{iflang.sty}{\from{iflang.dtx}{package}}%
  \usedir{doc/latex/oberdiek/test}%
  \file{iflang-test1.tex}{\from{iflang.dtx}{test1}}%
  \file{iflang-test2.tex}{\from{iflang.dtx}{test2}}%
  \file{iflang-test3.tex}{\from{iflang.dtx}{test3}}%
  \file{iflang-test4.tex}{\from{iflang.dtx}{test4}}%
  \file{iflang-test5.tex}{\from{iflang.dtx}{test5}}%
  \nopreamble
  \nopostamble
  \usedir{source/latex/oberdiek/catalogue}%
  \file{iflang.xml}{\from{iflang.dtx}{catalogue}}%
}

\catcode32=13\relax% active space
\let =\space%
\Msg{************************************************************************}
\Msg{*}
\Msg{* To finish the installation you have to move the following}
\Msg{* file into a directory searched by TeX:}
\Msg{*}
\Msg{*     iflang.sty}
\Msg{*}
\Msg{* To produce the documentation run the file `iflang.drv'}
\Msg{* through LaTeX.}
\Msg{*}
\Msg{* Happy TeXing!}
\Msg{*}
\Msg{************************************************************************}

\endbatchfile
%</install>
%<*ignore>
\fi
%</ignore>
%<*driver>
\NeedsTeXFormat{LaTeX2e}
\ProvidesFile{iflang.drv}%
  [2016/05/16 v1.6 Checks for the current language (HO)]%
\documentclass{ltxdoc}
\usepackage{holtxdoc}[2011/11/22]
\begin{document}
  \DocInput{iflang.dtx}%
\end{document}
%</driver>
% \fi
%
%
% \CharacterTable
%  {Upper-case    \A\B\C\D\E\F\G\H\I\J\K\L\M\N\O\P\Q\R\S\T\U\V\W\X\Y\Z
%   Lower-case    \a\b\c\d\e\f\g\h\i\j\k\l\m\n\o\p\q\r\s\t\u\v\w\x\y\z
%   Digits        \0\1\2\3\4\5\6\7\8\9
%   Exclamation   \!     Double quote  \"     Hash (number) \#
%   Dollar        \$     Percent       \%     Ampersand     \&
%   Acute accent  \'     Left paren    \(     Right paren   \)
%   Asterisk      \*     Plus          \+     Comma         \,
%   Minus         \-     Point         \.     Solidus       \/
%   Colon         \:     Semicolon     \;     Less than     \<
%   Equals        \=     Greater than  \>     Question mark \?
%   Commercial at \@     Left bracket  \[     Backslash     \\
%   Right bracket \]     Circumflex    \^     Underscore    \_
%   Grave accent  \`     Left brace    \{     Vertical bar  \|
%   Right brace   \}     Tilde         \~}
%
% \GetFileInfo{iflang.drv}
%
% \title{The \xpackage{iflang} package}
% \date{2016/05/16 v1.6}
% \author{Heiko Oberdiek\thanks
% {Please report any issues at https://github.com/ho-tex/oberdiek/issues}\\
% \xemail{heiko.oberdiek at googlemail.com}}
%
% \maketitle
%
% \begin{abstract}
% This package provides expandible checks for the current language
% based on macro \cs{languagename} or hyphenation patterns.
% \end{abstract}
%
% \tableofcontents
%
% \section{Documentation}
%
% Package \xpackage{babel} defines \cs{iflanguagename}. As first
% argument it takes a language name and executes the second or
% third argument depending on the current language. This
% language test is based on hypenation patterns. However, it
% is possible that different languages or dialects share the
% same patterns. In such cases \cs{iflanguagename} fails.
%
% However, package \xpackage{babel} and some other packages
% such as \xpackage{german} or \xpackage{ngerman} store
% the language name in the macro \cs{languagename} if
% \cs{selectlanguage} is called.
%
% \begin{declcs}{IfLanguageName} \M{lang} \M{then} \M{else}
% \end{declcs}
% Makro \cs{IfLanguageName} compares language \meta{lang} with
% the current setting of macro \cs{languagename}. If both
% contains the same name then the \meta{then} part is called,
% otherwise the \meta{else} part.
%
% The macro is expandable. Thus it can be safely used inside \cs{edef}
% or \cs{csname}. If case of errors like an undefined \cs{languagename}
% the \meta{else} part is executed.
%
% Note: Macro \cs{IfLanguageName} relies on the fact, that
% \cs{languagename} is set correctly:
% \begin{description}
% \item[Package \xpackage{babel}:]
%    \mbox{}\\
%    Full support of \cs{languagename} in its language switching commands.
% \item[Format based on \xpackage{babel} (\xfile{language.dat}):]
%    \mbox{}\\
%    If package \xpackage{babel} is not used (or not yet loaded),
%    then \xpackage{babel}'s
%    \xfile{hyphen.cfg} has set \cs{languagename} to the last
%    language in \xfile{language.dat}, but \cs{language} (current
%    patterns) is zero and points to the first language.
%    Thus the value of \cs{languagename} is basically garbage.
%    Package \xpackage{iflang} warns if \cs{languagename} and
%    \cs{language} do not fit.
%    This can be fixed by loading package \xpackage{babel} previously.
% \item[Format based on \eTeX's \xfile{etex.src} (\xfile{language.def}):]
%    \mbox{}\\
%    Unhappily it does not support
%    \cs{languagename}. Thus this package hooks into \cs{uselanguage}
%    to get \cs{languagename} defined and updated there.
%    At package loading time the changed \cs{uselanguage} has not
%    been called yet. Thus package \xpackage{iflang} tries
%    |USenglish|. This is the definite default language of \xfile{etex.src}.
%    If the current patterns suit this default language, an
%    undefined \cs{languagename} is set to this language. Otherwise
%    a \cs{languagename} remains undefined and a warning is given.
% \end{description}
%
% \begin{declcs}{IfLanguagePatterns} \M{lang} \M{then} \M{else}
% \end{declcs}
% This macro behaves similar to \cs{IfLanguageName}. But the
% language test is based on the current pattern in force (\cs{language}).
% Also this macro is expandable, in case of errors the \meta{else} part
% is called.
%
% The following naming convention for the pattern are supported:
% \begin{description}
% \item[\xpackage{babel}/\xfile{language.dat}]:
%    \cs{l@}\meta{language}
% \item[\xpackage{etex.src}/\xfile{language.def}]:
%    \cs{lang@}\meta{language}
% \end{description}
% Package \xpackage{iflang} looks for \cs{uselanguage} (defined in
% \xfile{etex.src}) to find out the naming convention in use.
%
% \StopEventually{
% }
%
% \section{Implementation}
%
%    \begin{macrocode}
%<*package>
%    \end{macrocode}
%
% \subsection{Reload check and package identification}
%    Reload check, especially if the package is not used with \LaTeX.
%    \begin{macrocode}
\begingroup\catcode61\catcode48\catcode32=10\relax%
  \catcode13=5 % ^^M
  \endlinechar=13 %
  \catcode35=6 % #
  \catcode39=12 % '
  \catcode44=12 % ,
  \catcode45=12 % -
  \catcode46=12 % .
  \catcode58=12 % :
  \catcode64=11 % @
  \catcode123=1 % {
  \catcode125=2 % }
  \expandafter\let\expandafter\x\csname ver@iflang.sty\endcsname
  \ifx\x\relax % plain-TeX, first loading
  \else
    \def\empty{}%
    \ifx\x\empty % LaTeX, first loading,
      % variable is initialized, but \ProvidesPackage not yet seen
    \else
      \expandafter\ifx\csname PackageInfo\endcsname\relax
        \def\x#1#2{%
          \immediate\write-1{Package #1 Info: #2.}%
        }%
      \else
        \def\x#1#2{\PackageInfo{#1}{#2, stopped}}%
      \fi
      \x{iflang}{The package is already loaded}%
      \aftergroup\endinput
    \fi
  \fi
\endgroup%
%    \end{macrocode}
%    Package identification:
%    \begin{macrocode}
\begingroup\catcode61\catcode48\catcode32=10\relax%
  \catcode13=5 % ^^M
  \endlinechar=13 %
  \catcode35=6 % #
  \catcode39=12 % '
  \catcode40=12 % (
  \catcode41=12 % )
  \catcode44=12 % ,
  \catcode45=12 % -
  \catcode46=12 % .
  \catcode47=12 % /
  \catcode58=12 % :
  \catcode64=11 % @
  \catcode91=12 % [
  \catcode93=12 % ]
  \catcode123=1 % {
  \catcode125=2 % }
  \expandafter\ifx\csname ProvidesPackage\endcsname\relax
    \def\x#1#2#3[#4]{\endgroup
      \immediate\write-1{Package: #3 #4}%
      \xdef#1{#4}%
    }%
  \else
    \def\x#1#2[#3]{\endgroup
      #2[{#3}]%
      \ifx#1\@undefined
        \xdef#1{#3}%
      \fi
      \ifx#1\relax
        \xdef#1{#3}%
      \fi
    }%
  \fi
\expandafter\x\csname ver@iflang.sty\endcsname
\ProvidesPackage{iflang}%
  [2016/05/16 v1.6 Checks for the current language (HO)]%
%    \end{macrocode}
%
%    \begin{macrocode}
\begingroup\catcode61\catcode48\catcode32=10\relax%
  \catcode13=5 % ^^M
  \endlinechar=13 %
  \catcode123=1 % {
  \catcode125=2 % }
  \catcode64=11 % @
  \def\x{\endgroup
    \expandafter\edef\csname IfLang@AtEnd\endcsname{%
      \endlinechar=\the\endlinechar\relax
      \catcode13=\the\catcode13\relax
      \catcode32=\the\catcode32\relax
      \catcode35=\the\catcode35\relax
      \catcode61=\the\catcode61\relax
      \catcode64=\the\catcode64\relax
      \catcode123=\the\catcode123\relax
      \catcode125=\the\catcode125\relax
    }%
  }%
\x\catcode61\catcode48\catcode32=10\relax%
\catcode13=5 % ^^M
\endlinechar=13 %
\catcode35=6 % #
\catcode64=11 % @
\catcode123=1 % {
\catcode125=2 % }
\def\TMP@EnsureCode#1#2{%
  \edef\IfLang@AtEnd{%
    \IfLang@AtEnd
    \catcode#1=\the\catcode#1\relax
  }%
  \catcode#1=#2\relax
}
\TMP@EnsureCode{39}{12}% '
\TMP@EnsureCode{40}{12}% (
\TMP@EnsureCode{41}{12}% )
\TMP@EnsureCode{44}{12}% ,
\TMP@EnsureCode{46}{12}% .
\TMP@EnsureCode{47}{12}% /
\TMP@EnsureCode{58}{12}% :
\TMP@EnsureCode{91}{12}% [
\TMP@EnsureCode{93}{12}% ]
\edef\IfLang@AtEnd{\IfLang@AtEnd\noexpand\endinput}
%    \end{macrocode}
%
% \subsection{Tools}
%
% \subsubsection{Provide some basic macros of \LaTeX}
%
%    \begin{macro}{\@firstoftwo}
%    \begin{macrocode}
\expandafter\ifx\csname @firstoftwo\endcsname\relax
  \long\def\@firstoftwo#1#2{#1}%
\fi
%    \end{macrocode}
%    \end{macro}
%    \begin{macro}{\@secondoftwo}
%    \begin{macrocode}
\expandafter\ifx\csname @secondoftwo\endcsname\relax
  \long\def\@secondoftwo#1#2{#2}%
\fi
%    \end{macrocode}
%    \end{macro}
%
% \subsubsection{Expandible existence check for macros}
%
%    \begin{macro}{\IfLang@IfDefined}
%    \begin{macrocode}
\begingroup\expandafter\expandafter\expandafter\endgroup
\expandafter\ifx\csname ifcsname\endcsname\relax
  \expandafter\@firstoftwo
\else
  \expandafter\@secondoftwo
\fi
{%
  \def\IfLang@IfDefined#1{%
    \expandafter\ifx\csname#1\endcsname\relax
      \expandafter\@secondoftwo
    \else
      \expandafter\@firstoftwo
    \fi
  }%
}{%
  \def\IfLang@IfDefined#1{%
    \ifnum\ifcsname#1\endcsname
            \expandafter\ifx\csname#1\endcsname\relax
              1%
            \else
              0%
            \fi
          \else
            1%
          \fi
          =0 %
      \expandafter\@firstoftwo
    \else
      \expandafter\@secondoftwo
    \fi
  }%
}
%    \end{macrocode}
%    \end{macro}
%
% \subsubsection{Macros for messages}
%
%    \begin{macrocode}
\begingroup\expandafter\expandafter\expandafter\endgroup
\expandafter\ifx\csname RequirePackage\endcsname\relax
  \input infwarerr.sty\relax
  \input pdftexcmds.sty\relax
\else
  \RequirePackage{infwarerr}[2007/09/09]%
  \RequirePackage{pdftexcmds}[2016/05/16]%
\fi
%    \end{macrocode}
%
% \subsubsection{Support for \xfile{etex.src}}
%
%    \begin{macro}{\IfLang@prefix}
%    \begin{macrocode}
\begingroup\expandafter\expandafter\expandafter\endgroup
\expandafter\ifx\csname uselanguage\endcsname\relax
  \@PackageInfoNoLine{iflang}{%
    Naming convention for patterns: babel%
  }%
  \def\IfLang@prefix{l@}%
\else
  \@PackageInfoNoLine{iflang}{%
    Naming convention for patterns: etex.src%
  }%
  \def\IfLang@prefix{lang@}%
  \let\IfLang@OrgUseLanguage\uselanguage
  \def\uselanguage#1{%
    \edef\languagename{#1}%
    \IfLang@OrgUseLanguage{#1}%
  }%
%    \end{macrocode}
%    \end{macro}
%    The first \cs{uselanguage} that is executed as last
%    line in \xfile{language.def} cannot patched this way.
%    However, \xfile{language.def} is very strict. It forces
%    the first added and used language to be |USenglish|.
%    Thus, if \cs{languagename} is not defined, we can
%    quite safely assume |USenglish|. As additional
%    safety precaution the actual used patterns are checked.
%    \begin{macrocode}
  \begingroup\expandafter\expandafter\expandafter\endgroup
  \expandafter\ifx\csname languagename\endcsname\relax
    \begingroup\expandafter\expandafter\expandafter\endgroup
    \expandafter\ifx\csname lang@USenglish\endcsname\relax
      \@PackageWarningNoLine{iflang}{%
        \string\lang@USenglish\space is missing%
      }%
    \else
      \ifnum\lang@USenglish=\language
        \def\languagename{USenglish}%
      \else
        \@PackageWarningNoLine{iflang}{%
          \string\languagename\space is not set,\MessageBreak
          current language is unknown%
        }%
      \fi
    \fi
  \fi
\fi
\begingroup\expandafter\expandafter\expandafter\endgroup
\expandafter\ifx\csname languagename\endcsname\relax
  \@PackageInfoNoLine{iflang}{%
    \string\languagename\space is not set%
  }%
\fi
%    \end{macrocode}
%
% \subsection{\cs{IfLanguagePatterns}}
%
%    \begin{macro}{\IfLanguagePatterns}
%    \begin{macrocode}
\def\IfLanguagePatterns#1{%
  \ifnum\IfLang@IfDefined{\IfLang@prefix#1}{%
          \ifnum\csname\IfLang@prefix#1\endcsname=\language
            0%
          \else
            1%
          \fi
        }{1}=0 %
    \expandafter\@firstoftwo
  \else
    \expandafter\@secondoftwo
  \fi
}
%    \end{macrocode}
%    \end{macro}
%
% \subsection{\cs{IfLanguageName}}
%
%    \begin{macrocode}
\begingroup\expandafter\expandafter\expandafter\endgroup
\expandafter\ifx\csname pdf@strcmp\endcsname\relax
  \expandafter\@firstoftwo
\else
  \expandafter\@secondoftwo
\fi
{%
%    \end{macrocode}
%    We do not have \cs{pdf@strcmp} (and \cs{pdfstrcmp}).
%    Thus we must define
%    our own expandable string comparison. The following
%    implementation is based on a \TeX\ pearl from David Kastrup, presented
%    at the conference Bacho\TeX~2005:
%    \url{http://www-stary.gust.org.pl/pearls/2005/david-kastrup/bachotex2005-david-kastrup-pearl1.pdf}
%
%    The orignal code allows macros inside the second string.
%    Because also \cs{languagename} might consists of further
%    macros, we need a variant that allows macros in the first
%    string, too.
%    \begin{macrocode}
  \def\IfLang@StrNil{\relax}%
  \def\IfLang@StrEqual#1{%
    \number\IfLang@StrEqualStart{}{}#1\IfLang@StrNil
  }%
  \def\IfLang@StrEqualStart#1#2#3{%
    \ifx#3\IfLang@StrNil
      \IfLang@StrEqualStop
    \fi
    \ifcat\noexpand#3\relax
      \IfLang@StrExpand{#1}{#2}#3%
    \fi
    \IfLang@StrEqualStart{\if#3#1}{#2\fi}%
  }%
  \def\IfLang@StrEqualStop\fi#1\IfLang@StrEqualStart#2#3#4{%
    \fi
    #2#4\relax'#313 %
  }%
  \def\IfLang@StrExpand#1#2#3\fi\IfLang@StrEqualStart#4#5{%
    \fi
    \IfLang@@StrExpand{#1}{#2}#3%
  }%
  \def\IfLang@@StrExpand#1#2#3\IfLang@StrNil{%
    \expandafter\IfLang@@@StrExpand#3\IfLang@StrNil{#1}{#2}%
  }%
  \def\IfLang@@@StrExpand#1\IfLang@StrNil#2#3{%
    \IfLang@StrEqualStart{#2}{#3}#1\IfLang@StrNil
  }%
%    \end{macrocode}
%    \begin{macro}{\IfLanguageName}
%    \begin{macrocode}
  \def\IfLanguageName#1{%
    \ifnum\IfLang@IfDefined{languagename}{%
            \if\expandafter\IfLang@StrEqual\expandafter%
                           {\languagename}{#1}%
              0%
            \else
              1%
            \fi
          }{1}=0 %
      \expandafter\@firstoftwo
    \else
      \expandafter\@secondoftwo
    \fi
  }%
%    \end{macrocode}
%    \end{macro}
%    \begin{macrocode}
}{%
%    \end{macrocode}
%    \begin{macro}{\IfLanguageName}
%    \begin{macrocode}
  \def\IfLanguageName#1{%
    \ifnum\IfLang@IfDefined{languagename}{%
            \pdf@strcmp{#1}{\languagename}%
          }{1}=0 %
      \expandafter\@firstoftwo
    \else
      \expandafter\@secondoftwo
    \fi
  }%
%    \end{macrocode}
%    \end{macro}
%    \begin{macrocode}
}
%    \end{macrocode}
%
% \subsection{Check plausibility of \cs{languagename}}
%
%    \begin{macrocode}
\begingroup\expandafter\expandafter\expandafter\endgroup
\expandafter\ifx\csname languagename\endcsname\relax
\else
  \IfLanguagePatterns{\languagename}{}{%
    \@PackageWarningNoLine{iflang}{%
      Mismatch between \string\language\space
      (patterns)\MessageBreak
      and setting of \string\languagename
    }%
  }%
\fi
%    \end{macrocode}
%
%    \begin{macrocode}
\IfLang@AtEnd%
%</package>
%    \end{macrocode}
%
% \section{Test}
%
% \subsection{Catcode checks for loading}
%
%    \begin{macrocode}
%<*test1>
%    \end{macrocode}
%    \begin{macrocode}
\catcode`\{=1 %
\catcode`\}=2 %
\catcode`\#=6 %
\catcode`\@=11 %
\expandafter\ifx\csname count@\endcsname\relax
  \countdef\count@=255 %
\fi
\expandafter\ifx\csname @gobble\endcsname\relax
  \long\def\@gobble#1{}%
\fi
\expandafter\ifx\csname @firstofone\endcsname\relax
  \long\def\@firstofone#1{#1}%
\fi
\expandafter\ifx\csname loop\endcsname\relax
  \expandafter\@firstofone
\else
  \expandafter\@gobble
\fi
{%
  \def\loop#1\repeat{%
    \def\body{#1}%
    \iterate
  }%
  \def\iterate{%
    \body
      \let\next\iterate
    \else
      \let\next\relax
    \fi
    \next
  }%
  \let\repeat=\fi
}%
\def\RestoreCatcodes{}
\count@=0 %
\loop
  \edef\RestoreCatcodes{%
    \RestoreCatcodes
    \catcode\the\count@=\the\catcode\count@\relax
  }%
\ifnum\count@<255 %
  \advance\count@ 1 %
\repeat

\def\RangeCatcodeInvalid#1#2{%
  \count@=#1\relax
  \loop
    \catcode\count@=15 %
  \ifnum\count@<#2\relax
    \advance\count@ 1 %
  \repeat
}
\def\RangeCatcodeCheck#1#2#3{%
  \count@=#1\relax
  \loop
    \ifnum#3=\catcode\count@
    \else
      \errmessage{%
        Character \the\count@\space
        with wrong catcode \the\catcode\count@\space
        instead of \number#3%
      }%
    \fi
  \ifnum\count@<#2\relax
    \advance\count@ 1 %
  \repeat
}
\def\space{ }
\expandafter\ifx\csname LoadCommand\endcsname\relax
  \def\LoadCommand{\input iflang.sty\relax}%
\fi
\def\Test{%
  \RangeCatcodeInvalid{0}{47}%
  \RangeCatcodeInvalid{58}{64}%
  \RangeCatcodeInvalid{91}{96}%
  \RangeCatcodeInvalid{123}{255}%
  \catcode`\@=12 %
  \catcode`\\=0 %
  \catcode`\%=14 %
  \LoadCommand
  \RangeCatcodeCheck{0}{36}{15}%
  \RangeCatcodeCheck{37}{37}{14}%
  \RangeCatcodeCheck{38}{47}{15}%
  \RangeCatcodeCheck{48}{57}{12}%
  \RangeCatcodeCheck{58}{63}{15}%
  \RangeCatcodeCheck{64}{64}{12}%
  \RangeCatcodeCheck{65}{90}{11}%
  \RangeCatcodeCheck{91}{91}{15}%
  \RangeCatcodeCheck{92}{92}{0}%
  \RangeCatcodeCheck{93}{96}{15}%
  \RangeCatcodeCheck{97}{122}{11}%
  \RangeCatcodeCheck{123}{255}{15}%
  \RestoreCatcodes
}
\Test
\csname @@end\endcsname
\end
%    \end{macrocode}
%    \begin{macrocode}
%</test1>
%    \end{macrocode}
%
% \subsection{Test with \LaTeX}
%
%    \begin{macrocode}
%<*test2|test3>
%    \end{macrocode}
%    \begin{macrocode}
\NeedsTeXFormat{LaTeX2e}
%<test3>\let\pdfstrcmp\relax
\nofiles
\documentclass{minimal}
\usepackage{qstest}
\IncludeTests{*}
\LogTests{log}{*}{*}
\usepackage[english,naustrian,ngerman]{babel}
\usepackage{iflang}
\begin{document}
\begin{qstest}{IfLanguagePatterns}{language, pattern}
  \def\test#1#2{%
    \Expect*{\IfLanguagePatterns{#1}{true}{false}}{#2}%
  }%
  \test{ngerman}{true}%
  \test{naustrian}{true}%
  \test{english}{false}%
  \test{foobar}{false}%
\end{qstest}
\begin{qstest}{IfLanguageName}{language, name}
  \def\test#1#2{%
    \Expect*{\IfLanguageName{#1}{true}{false}}{#2}%
  }%
  \test{ngerman}{true}%
  \test{naustrian}{false}%
  \selectlanguage{naustrian}%
  \test{ngerman}{false}%
  \test{naustrian}{true}%
  \test{foobar}{false}%
  %
  \def\languagename{naustrian}%
  \test{naustrian}{true}%
  \test{ngerman}{false}%
  %
  \edef\languagename{\string naustrian}%
  \test{naustrian}{true}%
  \test{ngerman}{false}%
  %
  \def\languagename{naustrian}%
  \makeatletter
  \@onelevel@sanitize\languagename
  \test{naustrian}{true}%
  \test{ngerman}{false}%
  %
  \def\languagename{naustrian}%
  \def\xaustrian{naustrian}%
  \def\xgerman{ngerman}%
  \test{\xaustrian}{true}%
  \test{\xgerman}{false}%
  %
  \def\languagename{\xaustrian}%
  \test{naustrian}{true}%
  \test{ngerman}{false}%
  \test{\xaustrian}{true}%
  \test{\xgerman}{false}%
  \test{\languagename}{true}%
  \test{\languagename\space}{false}%
  %
  \def\languagename{\empty\xaustrian\empty}%
  \test{naustrian}{true}%
  \test{ngerman}{false}%
  \test{\empty\xaustrian\empty}{true}%
  \test{\empty\xgerman\empty}{false}%
\end{qstest}
\begin{qstest}{IfDefined}{defined}
  \makeatletter
  \let\foobar\relax
  \Expect*{\IfLang@IfDefined{foobar}{true}{false}}{false}%
  \Expect*{\ifx\foobar\relax true\else false\fi}{true}%
  \let\foobar\UNDEFINED
  \Expect*{\IfLang@IfDefined{foobar}{true}{false}}{false}%
  \Expect*{\ifx\foobar\relax true\else false\fi}{false}%
  \Expect*{\ifx\foobar\UNDEFINED true\else false\fi}{true}%
\end{qstest}
\end{document}
%    \end{macrocode}
%    \begin{macrocode}
%</test2|test3>
%    \end{macrocode}
%
% \subsection{Test with \plainTeX\ and \eTeX}
%
%    \begin{macrocode}
%<*test4>
%    \end{macrocode}
%    \begin{macrocode}
%% Format `etex' based on `language.def'
\input iflang.sty
\catcode64=12

\def\TestGeneric#1#2#3{%
  \begingroup
    \edef\x{#1{#2}{true}{false}}%
    \edef\y{#3}%
    \ifx\x\y
    \else
      \errmessage{Failed test: \string#1{#2} <> #3}%
    \fi
  \endgroup
}
\def\TestPatterns{\TestGeneric\IfLanguagePatterns}
\def\TestName{\TestGeneric\IfLanguageName}

\TestPatterns{USenglish}{true}
\TestPatterns{ngerman}{false}

\TestName{USenglish}{true}
\TestName{ngerman}{false}

\uselanguage{ngerman}

\TestPatterns{USenglish}{false}
\TestPatterns{ngerman}{true}

\TestName{USenglish}{false}
\TestName{ngerman}{true}

\csname @@end\endcsname
\end
%    \end{macrocode}
%    \begin{macrocode}
%</test4>
%    \end{macrocode}
%
% \subsection{Test with \plainTeX\ and without \eTeX/\pdfTeX}
%
%    \begin{macrocode}
%<*test5>
%    \end{macrocode}
%    \begin{macrocode}
%% Format `tex' (vanilla plain-TeX)
\let\ifcsname\UNDEFINED
\let\pdfstrcmp\UNDEFINED
\input iflang.sty
\catcode64=11

\def\TestDefined#1{%
  \IfLang@IfDefined{foobar}{}{}%
  \ifx\foobar#1%
  \else
    \errmessage{Failed test: \string\foobar <> \string#1}%
  \fi
}
\let\foobar\relax
\TestDefined\relax
\let\foobar\UNDEFINED
\TestDefined\relax

\def\strip@prefix#1>{}
\def\@onelevel@sanitize#1{%
  \edef#1{\expandafter\strip@prefix\meaning#1}%
}
\def\TestCompare#1#2#3{%
  \begingroup
    \edef\x{%
      \if\IfLang@StrEqual{#1}{#2}%
        true%
      \else
        false%
      \fi
    }%
    \def\expect{#3}%
    \ifx\x\expect
    \else
      \def\a{#1}%
      \@onelevel@sanitize\a
      \def\b{#2}%
      \@onelevel@sanitize\b
      \errmessage{Failed test: `\a'='\b' <> \expect}%
    \fi
  \endgroup
}
\TestCompare{junk}{junk}{true}
\TestCompare{}{}{true}
\TestCompare{a}{b}{false}
\TestCompare{aa}{bb}{false}
\def\a{ax}
\def\b{bx}
\def\c{\a\b}
\def\d{\c\b}
\def\exch#1#2{#2#1}
\def\gobble#1{}
\TestCompare{\gobble a}{}{true}
\TestCompare{}{\gobble a}{true}
\TestCompare{a\exch xyb}{ayxb}{true}
\TestCompare{\c}{\c}{true}
\TestCompare{\d}{\c\b}{true}

\csname @@end\endcsname
\end
%    \end{macrocode}
%    \begin{macrocode}
%</test5>
%    \end{macrocode}
%
% \section{Installation}
%
% \subsection{Download}
%
% \paragraph{Package.} This package is available on
% CTAN\footnote{\url{http://ctan.org/pkg/iflang}}:
% \begin{description}
% \item[\CTAN{macros/latex/contrib/oberdiek/iflang.dtx}] The source file.
% \item[\CTAN{macros/latex/contrib/oberdiek/iflang.pdf}] Documentation.
% \end{description}
%
%
% \paragraph{Bundle.} All the packages of the bundle `oberdiek'
% are also available in a TDS compliant ZIP archive. There
% the packages are already unpacked and the documentation files
% are generated. The files and directories obey the TDS standard.
% \begin{description}
% \item[\CTAN{install/macros/latex/contrib/oberdiek.tds.zip}]
% \end{description}
% \emph{TDS} refers to the standard ``A Directory Structure
% for \TeX\ Files'' (\CTAN{tds/tds.pdf}). Directories
% with \xfile{texmf} in their name are usually organized this way.
%
% \subsection{Bundle installation}
%
% \paragraph{Unpacking.} Unpack the \xfile{oberdiek.tds.zip} in the
% TDS tree (also known as \xfile{texmf} tree) of your choice.
% Example (linux):
% \begin{quote}
%   |unzip oberdiek.tds.zip -d ~/texmf|
% \end{quote}
%
% \paragraph{Script installation.}
% Check the directory \xfile{TDS:scripts/oberdiek/} for
% scripts that need further installation steps.
% Package \xpackage{attachfile2} comes with the Perl script
% \xfile{pdfatfi.pl} that should be installed in such a way
% that it can be called as \texttt{pdfatfi}.
% Example (linux):
% \begin{quote}
%   |chmod +x scripts/oberdiek/pdfatfi.pl|\\
%   |cp scripts/oberdiek/pdfatfi.pl /usr/local/bin/|
% \end{quote}
%
% \subsection{Package installation}
%
% \paragraph{Unpacking.} The \xfile{.dtx} file is a self-extracting
% \docstrip\ archive. The files are extracted by running the
% \xfile{.dtx} through \plainTeX:
% \begin{quote}
%   \verb|tex iflang.dtx|
% \end{quote}
%
% \paragraph{TDS.} Now the different files must be moved into
% the different directories in your installation TDS tree
% (also known as \xfile{texmf} tree):
% \begin{quote}
% \def\t{^^A
% \begin{tabular}{@{}>{\ttfamily}l@{ $\rightarrow$ }>{\ttfamily}l@{}}
%   iflang.sty & tex/generic/oberdiek/iflang.sty\\
%   iflang.pdf & doc/latex/oberdiek/iflang.pdf\\
%   test/iflang-test1.tex & doc/latex/oberdiek/test/iflang-test1.tex\\
%   test/iflang-test2.tex & doc/latex/oberdiek/test/iflang-test2.tex\\
%   test/iflang-test3.tex & doc/latex/oberdiek/test/iflang-test3.tex\\
%   test/iflang-test4.tex & doc/latex/oberdiek/test/iflang-test4.tex\\
%   test/iflang-test5.tex & doc/latex/oberdiek/test/iflang-test5.tex\\
%   iflang.dtx & source/latex/oberdiek/iflang.dtx\\
% \end{tabular}^^A
% }^^A
% \sbox0{\t}^^A
% \ifdim\wd0>\linewidth
%   \begingroup
%     \advance\linewidth by\leftmargin
%     \advance\linewidth by\rightmargin
%   \edef\x{\endgroup
%     \def\noexpand\lw{\the\linewidth}^^A
%   }\x
%   \def\lwbox{^^A
%     \leavevmode
%     \hbox to \linewidth{^^A
%       \kern-\leftmargin\relax
%       \hss
%       \usebox0
%       \hss
%       \kern-\rightmargin\relax
%     }^^A
%   }^^A
%   \ifdim\wd0>\lw
%     \sbox0{\small\t}^^A
%     \ifdim\wd0>\linewidth
%       \ifdim\wd0>\lw
%         \sbox0{\footnotesize\t}^^A
%         \ifdim\wd0>\linewidth
%           \ifdim\wd0>\lw
%             \sbox0{\scriptsize\t}^^A
%             \ifdim\wd0>\linewidth
%               \ifdim\wd0>\lw
%                 \sbox0{\tiny\t}^^A
%                 \ifdim\wd0>\linewidth
%                   \lwbox
%                 \else
%                   \usebox0
%                 \fi
%               \else
%                 \lwbox
%               \fi
%             \else
%               \usebox0
%             \fi
%           \else
%             \lwbox
%           \fi
%         \else
%           \usebox0
%         \fi
%       \else
%         \lwbox
%       \fi
%     \else
%       \usebox0
%     \fi
%   \else
%     \lwbox
%   \fi
% \else
%   \usebox0
% \fi
% \end{quote}
% If you have a \xfile{docstrip.cfg} that configures and enables \docstrip's
% TDS installing feature, then some files can already be in the right
% place, see the documentation of \docstrip.
%
% \subsection{Refresh file name databases}
%
% If your \TeX~distribution
% (\teTeX, \mikTeX, \dots) relies on file name databases, you must refresh
% these. For example, \teTeX\ users run \verb|texhash| or
% \verb|mktexlsr|.
%
% \subsection{Some details for the interested}
%
% \paragraph{Attached source.}
%
% The PDF documentation on CTAN also includes the
% \xfile{.dtx} source file. It can be extracted by
% AcrobatReader 6 or higher. Another option is \textsf{pdftk},
% e.g. unpack the file into the current directory:
% \begin{quote}
%   \verb|pdftk iflang.pdf unpack_files output .|
% \end{quote}
%
% \paragraph{Unpacking with \LaTeX.}
% The \xfile{.dtx} chooses its action depending on the format:
% \begin{description}
% \item[\plainTeX:] Run \docstrip\ and extract the files.
% \item[\LaTeX:] Generate the documentation.
% \end{description}
% If you insist on using \LaTeX\ for \docstrip\ (really,
% \docstrip\ does not need \LaTeX), then inform the autodetect routine
% about your intention:
% \begin{quote}
%   \verb|latex \let\install=y\input{iflang.dtx}|
% \end{quote}
% Do not forget to quote the argument according to the demands
% of your shell.
%
% \paragraph{Generating the documentation.}
% You can use both the \xfile{.dtx} or the \xfile{.drv} to generate
% the documentation. The process can be configured by the
% configuration file \xfile{ltxdoc.cfg}. For instance, put this
% line into this file, if you want to have A4 as paper format:
% \begin{quote}
%   \verb|\PassOptionsToClass{a4paper}{article}|
% \end{quote}
% An example follows how to generate the
% documentation with pdf\LaTeX:
% \begin{quote}
%\begin{verbatim}
%pdflatex iflang.dtx
%makeindex -s gind.ist iflang.idx
%pdflatex iflang.dtx
%makeindex -s gind.ist iflang.idx
%pdflatex iflang.dtx
%\end{verbatim}
% \end{quote}
%
% \section{Catalogue}
%
% The following XML file can be used as source for the
% \href{http://mirror.ctan.org/help/Catalogue/catalogue.html}{\TeX\ Catalogue}.
% The elements \texttt{caption} and \texttt{description} are imported
% from the original XML file from the Catalogue.
% The name of the XML file in the Catalogue is \xfile{iflang.xml}.
%    \begin{macrocode}
%<*catalogue>
<?xml version='1.0' encoding='us-ascii'?>
<!DOCTYPE entry SYSTEM 'catalogue.dtd'>
<entry datestamp='$Date$' modifier='$Author$' id='iflang'>
  <name>iflang</name>
  <caption>Expandable checks for the current language.</caption>
  <authorref id='auth:oberdiek'/>
  <copyright owner='Heiko Oberdiek' year='2007'/>
  <license type='lppl1.3'/>
  <version number='1.6'/>
  <description>
    This package provides expandable checks for the current language
    based on macro <tt>\languagename</tt> or hyphenation patterns.
    <p/>
    The package is part of the <xref refid='oberdiek'>oberdiek</xref> bundle.
  </description>
  <documentation details='Package documentation'
      href='ctan:/macros/latex/contrib/oberdiek/iflang.pdf'/>
  <ctan file='true' path='/macros/latex/contrib/oberdiek/iflang.dtx'/>
  <miktex location='oberdiek'/>
  <texlive location='oberdiek'/>
  <install path='/macros/latex/contrib/oberdiek/oberdiek.tds.zip'/>
</entry>
%</catalogue>
%    \end{macrocode}
%
% \section{Acknowledgement}
% I wish to thank:
% \begin{description}
% \item[Markus Kohm] Useful hints for version 1.2.
% \end{description}
%
% \begin{History}
%   \begin{Version}{2007/04/10 v1.0}
%   \item
%     First public version.
%   \end{Version}
%   \begin{Version}{2007/04/11 v1.1}
%   \item
%     Line ends sanitized.
%   \end{Version}
%   \begin{Version}{2007/04/12 v1.2}
%   \item
%     Initialization of \cs{languagename} in case of \xfile{etex.src}.
%   \item
%     Some sanity tests added.
%   \item
%     Documentation improved.
%   \end{Version}
%   \begin{Version}{2007/04/26 v1.3}
%   \item
%     Use of package \xpackage{infwarerr}.
%   \end{Version}
%   \begin{Version}{2007/09/09 v1.4}
%   \item
%     Bug fix: \cs{IfLang@StrEqual} $\rightarrow$ \cs{IfLangStrEqual}
%     (Gabriele Balducci).
%   \item
%     Catcode section rewritten.
%   \end{Version}
%   \begin{Version}{2007/11/11 v1.5}
%   \item
%     Use of package \xpackage{pdftexcmds} for \LuaTeX\ support.
%   \end{Version}
%   \begin{Version}{2016/05/16 v1.6}
%   \item
%     Documentation updates.
%   \end{Version}
% \end{History}
%
% \PrintIndex
%
% \Finale
\endinput

%        (quote the arguments according to the demands of your shell)
%
% Documentation:
%    (a) If iflang.drv is present:
%           latex iflang.drv
%    (b) Without iflang.drv:
%           latex iflang.dtx; ...
%    The class ltxdoc loads the configuration file ltxdoc.cfg
%    if available. Here you can specify further options, e.g.
%    use A4 as paper format:
%       \PassOptionsToClass{a4paper}{article}
%
%    Programm calls to get the documentation (example):
%       pdflatex iflang.dtx
%       makeindex -s gind.ist iflang.idx
%       pdflatex iflang.dtx
%       makeindex -s gind.ist iflang.idx
%       pdflatex iflang.dtx
%
% Installation:
%    TDS:tex/generic/oberdiek/iflang.sty
%    TDS:doc/latex/oberdiek/iflang.pdf
%    TDS:doc/latex/oberdiek/test/iflang-test1.tex
%    TDS:doc/latex/oberdiek/test/iflang-test2.tex
%    TDS:doc/latex/oberdiek/test/iflang-test3.tex
%    TDS:doc/latex/oberdiek/test/iflang-test4.tex
%    TDS:doc/latex/oberdiek/test/iflang-test5.tex
%    TDS:source/latex/oberdiek/iflang.dtx
%
%<*ignore>
\begingroup
  \catcode123=1 %
  \catcode125=2 %
  \def\x{LaTeX2e}%
\expandafter\endgroup
\ifcase 0\ifx\install y1\fi\expandafter
         \ifx\csname processbatchFile\endcsname\relax\else1\fi
         \ifx\fmtname\x\else 1\fi\relax
\else\csname fi\endcsname
%</ignore>
%<*install>
\input docstrip.tex
\Msg{************************************************************************}
\Msg{* Installation}
\Msg{* Package: iflang 2016/05/16 v1.6 Checks for the current language (HO)}
\Msg{************************************************************************}

\keepsilent
\askforoverwritefalse

\let\MetaPrefix\relax
\preamble

This is a generated file.

Project: iflang
Version: 2016/05/16 v1.6

Copyright (C) 2007 by
   Heiko Oberdiek <heiko.oberdiek at googlemail.com>

This work may be distributed and/or modified under the
conditions of the LaTeX Project Public License, either
version 1.3c of this license or (at your option) any later
version. This version of this license is in
   http://www.latex-project.org/lppl/lppl-1-3c.txt
and the latest version of this license is in
   http://www.latex-project.org/lppl.txt
and version 1.3 or later is part of all distributions of
LaTeX version 2005/12/01 or later.

This work has the LPPL maintenance status "maintained".

This Current Maintainer of this work is Heiko Oberdiek.

The Base Interpreter refers to any `TeX-Format',
because some files are installed in TDS:tex/generic//.

This work consists of the main source file iflang.dtx
and the derived files
   iflang.sty, iflang.pdf, iflang.ins, iflang.drv, iflang-test1.tex,
   iflang-test2.tex, iflang-test3.tex, iflang-test4.tex,
   iflang-test5.tex.

\endpreamble
\let\MetaPrefix\DoubleperCent

\generate{%
  \file{iflang.ins}{\from{iflang.dtx}{install}}%
  \file{iflang.drv}{\from{iflang.dtx}{driver}}%
  \usedir{tex/generic/oberdiek}%
  \file{iflang.sty}{\from{iflang.dtx}{package}}%
  \usedir{doc/latex/oberdiek/test}%
  \file{iflang-test1.tex}{\from{iflang.dtx}{test1}}%
  \file{iflang-test2.tex}{\from{iflang.dtx}{test2}}%
  \file{iflang-test3.tex}{\from{iflang.dtx}{test3}}%
  \file{iflang-test4.tex}{\from{iflang.dtx}{test4}}%
  \file{iflang-test5.tex}{\from{iflang.dtx}{test5}}%
  \nopreamble
  \nopostamble
  \usedir{source/latex/oberdiek/catalogue}%
  \file{iflang.xml}{\from{iflang.dtx}{catalogue}}%
}

\catcode32=13\relax% active space
\let =\space%
\Msg{************************************************************************}
\Msg{*}
\Msg{* To finish the installation you have to move the following}
\Msg{* file into a directory searched by TeX:}
\Msg{*}
\Msg{*     iflang.sty}
\Msg{*}
\Msg{* To produce the documentation run the file `iflang.drv'}
\Msg{* through LaTeX.}
\Msg{*}
\Msg{* Happy TeXing!}
\Msg{*}
\Msg{************************************************************************}

\endbatchfile
%</install>
%<*ignore>
\fi
%</ignore>
%<*driver>
\NeedsTeXFormat{LaTeX2e}
\ProvidesFile{iflang.drv}%
  [2016/05/16 v1.6 Checks for the current language (HO)]%
\documentclass{ltxdoc}
\usepackage{holtxdoc}[2011/11/22]
\begin{document}
  \DocInput{iflang.dtx}%
\end{document}
%</driver>
% \fi
%
%
% \CharacterTable
%  {Upper-case    \A\B\C\D\E\F\G\H\I\J\K\L\M\N\O\P\Q\R\S\T\U\V\W\X\Y\Z
%   Lower-case    \a\b\c\d\e\f\g\h\i\j\k\l\m\n\o\p\q\r\s\t\u\v\w\x\y\z
%   Digits        \0\1\2\3\4\5\6\7\8\9
%   Exclamation   \!     Double quote  \"     Hash (number) \#
%   Dollar        \$     Percent       \%     Ampersand     \&
%   Acute accent  \'     Left paren    \(     Right paren   \)
%   Asterisk      \*     Plus          \+     Comma         \,
%   Minus         \-     Point         \.     Solidus       \/
%   Colon         \:     Semicolon     \;     Less than     \<
%   Equals        \=     Greater than  \>     Question mark \?
%   Commercial at \@     Left bracket  \[     Backslash     \\
%   Right bracket \]     Circumflex    \^     Underscore    \_
%   Grave accent  \`     Left brace    \{     Vertical bar  \|
%   Right brace   \}     Tilde         \~}
%
% \GetFileInfo{iflang.drv}
%
% \title{The \xpackage{iflang} package}
% \date{2016/05/16 v1.6}
% \author{Heiko Oberdiek\thanks
% {Please report any issues at https://github.com/ho-tex/oberdiek/issues}\\
% \xemail{heiko.oberdiek at googlemail.com}}
%
% \maketitle
%
% \begin{abstract}
% This package provides expandible checks for the current language
% based on macro \cs{languagename} or hyphenation patterns.
% \end{abstract}
%
% \tableofcontents
%
% \section{Documentation}
%
% Package \xpackage{babel} defines \cs{iflanguagename}. As first
% argument it takes a language name and executes the second or
% third argument depending on the current language. This
% language test is based on hypenation patterns. However, it
% is possible that different languages or dialects share the
% same patterns. In such cases \cs{iflanguagename} fails.
%
% However, package \xpackage{babel} and some other packages
% such as \xpackage{german} or \xpackage{ngerman} store
% the language name in the macro \cs{languagename} if
% \cs{selectlanguage} is called.
%
% \begin{declcs}{IfLanguageName} \M{lang} \M{then} \M{else}
% \end{declcs}
% Makro \cs{IfLanguageName} compares language \meta{lang} with
% the current setting of macro \cs{languagename}. If both
% contains the same name then the \meta{then} part is called,
% otherwise the \meta{else} part.
%
% The macro is expandable. Thus it can be safely used inside \cs{edef}
% or \cs{csname}. If case of errors like an undefined \cs{languagename}
% the \meta{else} part is executed.
%
% Note: Macro \cs{IfLanguageName} relies on the fact, that
% \cs{languagename} is set correctly:
% \begin{description}
% \item[Package \xpackage{babel}:]
%    \mbox{}\\
%    Full support of \cs{languagename} in its language switching commands.
% \item[Format based on \xpackage{babel} (\xfile{language.dat}):]
%    \mbox{}\\
%    If package \xpackage{babel} is not used (or not yet loaded),
%    then \xpackage{babel}'s
%    \xfile{hyphen.cfg} has set \cs{languagename} to the last
%    language in \xfile{language.dat}, but \cs{language} (current
%    patterns) is zero and points to the first language.
%    Thus the value of \cs{languagename} is basically garbage.
%    Package \xpackage{iflang} warns if \cs{languagename} and
%    \cs{language} do not fit.
%    This can be fixed by loading package \xpackage{babel} previously.
% \item[Format based on \eTeX's \xfile{etex.src} (\xfile{language.def}):]
%    \mbox{}\\
%    Unhappily it does not support
%    \cs{languagename}. Thus this package hooks into \cs{uselanguage}
%    to get \cs{languagename} defined and updated there.
%    At package loading time the changed \cs{uselanguage} has not
%    been called yet. Thus package \xpackage{iflang} tries
%    |USenglish|. This is the definite default language of \xfile{etex.src}.
%    If the current patterns suit this default language, an
%    undefined \cs{languagename} is set to this language. Otherwise
%    a \cs{languagename} remains undefined and a warning is given.
% \end{description}
%
% \begin{declcs}{IfLanguagePatterns} \M{lang} \M{then} \M{else}
% \end{declcs}
% This macro behaves similar to \cs{IfLanguageName}. But the
% language test is based on the current pattern in force (\cs{language}).
% Also this macro is expandable, in case of errors the \meta{else} part
% is called.
%
% The following naming convention for the pattern are supported:
% \begin{description}
% \item[\xpackage{babel}/\xfile{language.dat}]:
%    \cs{l@}\meta{language}
% \item[\xpackage{etex.src}/\xfile{language.def}]:
%    \cs{lang@}\meta{language}
% \end{description}
% Package \xpackage{iflang} looks for \cs{uselanguage} (defined in
% \xfile{etex.src}) to find out the naming convention in use.
%
% \StopEventually{
% }
%
% \section{Implementation}
%
%    \begin{macrocode}
%<*package>
%    \end{macrocode}
%
% \subsection{Reload check and package identification}
%    Reload check, especially if the package is not used with \LaTeX.
%    \begin{macrocode}
\begingroup\catcode61\catcode48\catcode32=10\relax%
  \catcode13=5 % ^^M
  \endlinechar=13 %
  \catcode35=6 % #
  \catcode39=12 % '
  \catcode44=12 % ,
  \catcode45=12 % -
  \catcode46=12 % .
  \catcode58=12 % :
  \catcode64=11 % @
  \catcode123=1 % {
  \catcode125=2 % }
  \expandafter\let\expandafter\x\csname ver@iflang.sty\endcsname
  \ifx\x\relax % plain-TeX, first loading
  \else
    \def\empty{}%
    \ifx\x\empty % LaTeX, first loading,
      % variable is initialized, but \ProvidesPackage not yet seen
    \else
      \expandafter\ifx\csname PackageInfo\endcsname\relax
        \def\x#1#2{%
          \immediate\write-1{Package #1 Info: #2.}%
        }%
      \else
        \def\x#1#2{\PackageInfo{#1}{#2, stopped}}%
      \fi
      \x{iflang}{The package is already loaded}%
      \aftergroup\endinput
    \fi
  \fi
\endgroup%
%    \end{macrocode}
%    Package identification:
%    \begin{macrocode}
\begingroup\catcode61\catcode48\catcode32=10\relax%
  \catcode13=5 % ^^M
  \endlinechar=13 %
  \catcode35=6 % #
  \catcode39=12 % '
  \catcode40=12 % (
  \catcode41=12 % )
  \catcode44=12 % ,
  \catcode45=12 % -
  \catcode46=12 % .
  \catcode47=12 % /
  \catcode58=12 % :
  \catcode64=11 % @
  \catcode91=12 % [
  \catcode93=12 % ]
  \catcode123=1 % {
  \catcode125=2 % }
  \expandafter\ifx\csname ProvidesPackage\endcsname\relax
    \def\x#1#2#3[#4]{\endgroup
      \immediate\write-1{Package: #3 #4}%
      \xdef#1{#4}%
    }%
  \else
    \def\x#1#2[#3]{\endgroup
      #2[{#3}]%
      \ifx#1\@undefined
        \xdef#1{#3}%
      \fi
      \ifx#1\relax
        \xdef#1{#3}%
      \fi
    }%
  \fi
\expandafter\x\csname ver@iflang.sty\endcsname
\ProvidesPackage{iflang}%
  [2016/05/16 v1.6 Checks for the current language (HO)]%
%    \end{macrocode}
%
%    \begin{macrocode}
\begingroup\catcode61\catcode48\catcode32=10\relax%
  \catcode13=5 % ^^M
  \endlinechar=13 %
  \catcode123=1 % {
  \catcode125=2 % }
  \catcode64=11 % @
  \def\x{\endgroup
    \expandafter\edef\csname IfLang@AtEnd\endcsname{%
      \endlinechar=\the\endlinechar\relax
      \catcode13=\the\catcode13\relax
      \catcode32=\the\catcode32\relax
      \catcode35=\the\catcode35\relax
      \catcode61=\the\catcode61\relax
      \catcode64=\the\catcode64\relax
      \catcode123=\the\catcode123\relax
      \catcode125=\the\catcode125\relax
    }%
  }%
\x\catcode61\catcode48\catcode32=10\relax%
\catcode13=5 % ^^M
\endlinechar=13 %
\catcode35=6 % #
\catcode64=11 % @
\catcode123=1 % {
\catcode125=2 % }
\def\TMP@EnsureCode#1#2{%
  \edef\IfLang@AtEnd{%
    \IfLang@AtEnd
    \catcode#1=\the\catcode#1\relax
  }%
  \catcode#1=#2\relax
}
\TMP@EnsureCode{39}{12}% '
\TMP@EnsureCode{40}{12}% (
\TMP@EnsureCode{41}{12}% )
\TMP@EnsureCode{44}{12}% ,
\TMP@EnsureCode{46}{12}% .
\TMP@EnsureCode{47}{12}% /
\TMP@EnsureCode{58}{12}% :
\TMP@EnsureCode{91}{12}% [
\TMP@EnsureCode{93}{12}% ]
\edef\IfLang@AtEnd{\IfLang@AtEnd\noexpand\endinput}
%    \end{macrocode}
%
% \subsection{Tools}
%
% \subsubsection{Provide some basic macros of \LaTeX}
%
%    \begin{macro}{\@firstoftwo}
%    \begin{macrocode}
\expandafter\ifx\csname @firstoftwo\endcsname\relax
  \long\def\@firstoftwo#1#2{#1}%
\fi
%    \end{macrocode}
%    \end{macro}
%    \begin{macro}{\@secondoftwo}
%    \begin{macrocode}
\expandafter\ifx\csname @secondoftwo\endcsname\relax
  \long\def\@secondoftwo#1#2{#2}%
\fi
%    \end{macrocode}
%    \end{macro}
%
% \subsubsection{Expandible existence check for macros}
%
%    \begin{macro}{\IfLang@IfDefined}
%    \begin{macrocode}
\begingroup\expandafter\expandafter\expandafter\endgroup
\expandafter\ifx\csname ifcsname\endcsname\relax
  \expandafter\@firstoftwo
\else
  \expandafter\@secondoftwo
\fi
{%
  \def\IfLang@IfDefined#1{%
    \expandafter\ifx\csname#1\endcsname\relax
      \expandafter\@secondoftwo
    \else
      \expandafter\@firstoftwo
    \fi
  }%
}{%
  \def\IfLang@IfDefined#1{%
    \ifnum\ifcsname#1\endcsname
            \expandafter\ifx\csname#1\endcsname\relax
              1%
            \else
              0%
            \fi
          \else
            1%
          \fi
          =0 %
      \expandafter\@firstoftwo
    \else
      \expandafter\@secondoftwo
    \fi
  }%
}
%    \end{macrocode}
%    \end{macro}
%
% \subsubsection{Macros for messages}
%
%    \begin{macrocode}
\begingroup\expandafter\expandafter\expandafter\endgroup
\expandafter\ifx\csname RequirePackage\endcsname\relax
  \input infwarerr.sty\relax
  \input pdftexcmds.sty\relax
\else
  \RequirePackage{infwarerr}[2007/09/09]%
  \RequirePackage{pdftexcmds}[2016/05/16]%
\fi
%    \end{macrocode}
%
% \subsubsection{Support for \xfile{etex.src}}
%
%    \begin{macro}{\IfLang@prefix}
%    \begin{macrocode}
\begingroup\expandafter\expandafter\expandafter\endgroup
\expandafter\ifx\csname uselanguage\endcsname\relax
  \@PackageInfoNoLine{iflang}{%
    Naming convention for patterns: babel%
  }%
  \def\IfLang@prefix{l@}%
\else
  \@PackageInfoNoLine{iflang}{%
    Naming convention for patterns: etex.src%
  }%
  \def\IfLang@prefix{lang@}%
  \let\IfLang@OrgUseLanguage\uselanguage
  \def\uselanguage#1{%
    \edef\languagename{#1}%
    \IfLang@OrgUseLanguage{#1}%
  }%
%    \end{macrocode}
%    \end{macro}
%    The first \cs{uselanguage} that is executed as last
%    line in \xfile{language.def} cannot patched this way.
%    However, \xfile{language.def} is very strict. It forces
%    the first added and used language to be |USenglish|.
%    Thus, if \cs{languagename} is not defined, we can
%    quite safely assume |USenglish|. As additional
%    safety precaution the actual used patterns are checked.
%    \begin{macrocode}
  \begingroup\expandafter\expandafter\expandafter\endgroup
  \expandafter\ifx\csname languagename\endcsname\relax
    \begingroup\expandafter\expandafter\expandafter\endgroup
    \expandafter\ifx\csname lang@USenglish\endcsname\relax
      \@PackageWarningNoLine{iflang}{%
        \string\lang@USenglish\space is missing%
      }%
    \else
      \ifnum\lang@USenglish=\language
        \def\languagename{USenglish}%
      \else
        \@PackageWarningNoLine{iflang}{%
          \string\languagename\space is not set,\MessageBreak
          current language is unknown%
        }%
      \fi
    \fi
  \fi
\fi
\begingroup\expandafter\expandafter\expandafter\endgroup
\expandafter\ifx\csname languagename\endcsname\relax
  \@PackageInfoNoLine{iflang}{%
    \string\languagename\space is not set%
  }%
\fi
%    \end{macrocode}
%
% \subsection{\cs{IfLanguagePatterns}}
%
%    \begin{macro}{\IfLanguagePatterns}
%    \begin{macrocode}
\def\IfLanguagePatterns#1{%
  \ifnum\IfLang@IfDefined{\IfLang@prefix#1}{%
          \ifnum\csname\IfLang@prefix#1\endcsname=\language
            0%
          \else
            1%
          \fi
        }{1}=0 %
    \expandafter\@firstoftwo
  \else
    \expandafter\@secondoftwo
  \fi
}
%    \end{macrocode}
%    \end{macro}
%
% \subsection{\cs{IfLanguageName}}
%
%    \begin{macrocode}
\begingroup\expandafter\expandafter\expandafter\endgroup
\expandafter\ifx\csname pdf@strcmp\endcsname\relax
  \expandafter\@firstoftwo
\else
  \expandafter\@secondoftwo
\fi
{%
%    \end{macrocode}
%    We do not have \cs{pdf@strcmp} (and \cs{pdfstrcmp}).
%    Thus we must define
%    our own expandable string comparison. The following
%    implementation is based on a \TeX\ pearl from David Kastrup, presented
%    at the conference Bacho\TeX~2005:
%    \url{http://www-stary.gust.org.pl/pearls/2005/david-kastrup/bachotex2005-david-kastrup-pearl1.pdf}
%
%    The orignal code allows macros inside the second string.
%    Because also \cs{languagename} might consists of further
%    macros, we need a variant that allows macros in the first
%    string, too.
%    \begin{macrocode}
  \def\IfLang@StrNil{\relax}%
  \def\IfLang@StrEqual#1{%
    \number\IfLang@StrEqualStart{}{}#1\IfLang@StrNil
  }%
  \def\IfLang@StrEqualStart#1#2#3{%
    \ifx#3\IfLang@StrNil
      \IfLang@StrEqualStop
    \fi
    \ifcat\noexpand#3\relax
      \IfLang@StrExpand{#1}{#2}#3%
    \fi
    \IfLang@StrEqualStart{\if#3#1}{#2\fi}%
  }%
  \def\IfLang@StrEqualStop\fi#1\IfLang@StrEqualStart#2#3#4{%
    \fi
    #2#4\relax'#313 %
  }%
  \def\IfLang@StrExpand#1#2#3\fi\IfLang@StrEqualStart#4#5{%
    \fi
    \IfLang@@StrExpand{#1}{#2}#3%
  }%
  \def\IfLang@@StrExpand#1#2#3\IfLang@StrNil{%
    \expandafter\IfLang@@@StrExpand#3\IfLang@StrNil{#1}{#2}%
  }%
  \def\IfLang@@@StrExpand#1\IfLang@StrNil#2#3{%
    \IfLang@StrEqualStart{#2}{#3}#1\IfLang@StrNil
  }%
%    \end{macrocode}
%    \begin{macro}{\IfLanguageName}
%    \begin{macrocode}
  \def\IfLanguageName#1{%
    \ifnum\IfLang@IfDefined{languagename}{%
            \if\expandafter\IfLang@StrEqual\expandafter%
                           {\languagename}{#1}%
              0%
            \else
              1%
            \fi
          }{1}=0 %
      \expandafter\@firstoftwo
    \else
      \expandafter\@secondoftwo
    \fi
  }%
%    \end{macrocode}
%    \end{macro}
%    \begin{macrocode}
}{%
%    \end{macrocode}
%    \begin{macro}{\IfLanguageName}
%    \begin{macrocode}
  \def\IfLanguageName#1{%
    \ifnum\IfLang@IfDefined{languagename}{%
            \pdf@strcmp{#1}{\languagename}%
          }{1}=0 %
      \expandafter\@firstoftwo
    \else
      \expandafter\@secondoftwo
    \fi
  }%
%    \end{macrocode}
%    \end{macro}
%    \begin{macrocode}
}
%    \end{macrocode}
%
% \subsection{Check plausibility of \cs{languagename}}
%
%    \begin{macrocode}
\begingroup\expandafter\expandafter\expandafter\endgroup
\expandafter\ifx\csname languagename\endcsname\relax
\else
  \IfLanguagePatterns{\languagename}{}{%
    \@PackageWarningNoLine{iflang}{%
      Mismatch between \string\language\space
      (patterns)\MessageBreak
      and setting of \string\languagename
    }%
  }%
\fi
%    \end{macrocode}
%
%    \begin{macrocode}
\IfLang@AtEnd%
%</package>
%    \end{macrocode}
%
% \section{Test}
%
% \subsection{Catcode checks for loading}
%
%    \begin{macrocode}
%<*test1>
%    \end{macrocode}
%    \begin{macrocode}
\catcode`\{=1 %
\catcode`\}=2 %
\catcode`\#=6 %
\catcode`\@=11 %
\expandafter\ifx\csname count@\endcsname\relax
  \countdef\count@=255 %
\fi
\expandafter\ifx\csname @gobble\endcsname\relax
  \long\def\@gobble#1{}%
\fi
\expandafter\ifx\csname @firstofone\endcsname\relax
  \long\def\@firstofone#1{#1}%
\fi
\expandafter\ifx\csname loop\endcsname\relax
  \expandafter\@firstofone
\else
  \expandafter\@gobble
\fi
{%
  \def\loop#1\repeat{%
    \def\body{#1}%
    \iterate
  }%
  \def\iterate{%
    \body
      \let\next\iterate
    \else
      \let\next\relax
    \fi
    \next
  }%
  \let\repeat=\fi
}%
\def\RestoreCatcodes{}
\count@=0 %
\loop
  \edef\RestoreCatcodes{%
    \RestoreCatcodes
    \catcode\the\count@=\the\catcode\count@\relax
  }%
\ifnum\count@<255 %
  \advance\count@ 1 %
\repeat

\def\RangeCatcodeInvalid#1#2{%
  \count@=#1\relax
  \loop
    \catcode\count@=15 %
  \ifnum\count@<#2\relax
    \advance\count@ 1 %
  \repeat
}
\def\RangeCatcodeCheck#1#2#3{%
  \count@=#1\relax
  \loop
    \ifnum#3=\catcode\count@
    \else
      \errmessage{%
        Character \the\count@\space
        with wrong catcode \the\catcode\count@\space
        instead of \number#3%
      }%
    \fi
  \ifnum\count@<#2\relax
    \advance\count@ 1 %
  \repeat
}
\def\space{ }
\expandafter\ifx\csname LoadCommand\endcsname\relax
  \def\LoadCommand{\input iflang.sty\relax}%
\fi
\def\Test{%
  \RangeCatcodeInvalid{0}{47}%
  \RangeCatcodeInvalid{58}{64}%
  \RangeCatcodeInvalid{91}{96}%
  \RangeCatcodeInvalid{123}{255}%
  \catcode`\@=12 %
  \catcode`\\=0 %
  \catcode`\%=14 %
  \LoadCommand
  \RangeCatcodeCheck{0}{36}{15}%
  \RangeCatcodeCheck{37}{37}{14}%
  \RangeCatcodeCheck{38}{47}{15}%
  \RangeCatcodeCheck{48}{57}{12}%
  \RangeCatcodeCheck{58}{63}{15}%
  \RangeCatcodeCheck{64}{64}{12}%
  \RangeCatcodeCheck{65}{90}{11}%
  \RangeCatcodeCheck{91}{91}{15}%
  \RangeCatcodeCheck{92}{92}{0}%
  \RangeCatcodeCheck{93}{96}{15}%
  \RangeCatcodeCheck{97}{122}{11}%
  \RangeCatcodeCheck{123}{255}{15}%
  \RestoreCatcodes
}
\Test
\csname @@end\endcsname
\end
%    \end{macrocode}
%    \begin{macrocode}
%</test1>
%    \end{macrocode}
%
% \subsection{Test with \LaTeX}
%
%    \begin{macrocode}
%<*test2|test3>
%    \end{macrocode}
%    \begin{macrocode}
\NeedsTeXFormat{LaTeX2e}
%<test3>\let\pdfstrcmp\relax
\nofiles
\documentclass{minimal}
\usepackage{qstest}
\IncludeTests{*}
\LogTests{log}{*}{*}
\usepackage[english,naustrian,ngerman]{babel}
\usepackage{iflang}
\begin{document}
\begin{qstest}{IfLanguagePatterns}{language, pattern}
  \def\test#1#2{%
    \Expect*{\IfLanguagePatterns{#1}{true}{false}}{#2}%
  }%
  \test{ngerman}{true}%
  \test{naustrian}{true}%
  \test{english}{false}%
  \test{foobar}{false}%
\end{qstest}
\begin{qstest}{IfLanguageName}{language, name}
  \def\test#1#2{%
    \Expect*{\IfLanguageName{#1}{true}{false}}{#2}%
  }%
  \test{ngerman}{true}%
  \test{naustrian}{false}%
  \selectlanguage{naustrian}%
  \test{ngerman}{false}%
  \test{naustrian}{true}%
  \test{foobar}{false}%
  %
  \def\languagename{naustrian}%
  \test{naustrian}{true}%
  \test{ngerman}{false}%
  %
  \edef\languagename{\string naustrian}%
  \test{naustrian}{true}%
  \test{ngerman}{false}%
  %
  \def\languagename{naustrian}%
  \makeatletter
  \@onelevel@sanitize\languagename
  \test{naustrian}{true}%
  \test{ngerman}{false}%
  %
  \def\languagename{naustrian}%
  \def\xaustrian{naustrian}%
  \def\xgerman{ngerman}%
  \test{\xaustrian}{true}%
  \test{\xgerman}{false}%
  %
  \def\languagename{\xaustrian}%
  \test{naustrian}{true}%
  \test{ngerman}{false}%
  \test{\xaustrian}{true}%
  \test{\xgerman}{false}%
  \test{\languagename}{true}%
  \test{\languagename\space}{false}%
  %
  \def\languagename{\empty\xaustrian\empty}%
  \test{naustrian}{true}%
  \test{ngerman}{false}%
  \test{\empty\xaustrian\empty}{true}%
  \test{\empty\xgerman\empty}{false}%
\end{qstest}
\begin{qstest}{IfDefined}{defined}
  \makeatletter
  \let\foobar\relax
  \Expect*{\IfLang@IfDefined{foobar}{true}{false}}{false}%
  \Expect*{\ifx\foobar\relax true\else false\fi}{true}%
  \let\foobar\UNDEFINED
  \Expect*{\IfLang@IfDefined{foobar}{true}{false}}{false}%
  \Expect*{\ifx\foobar\relax true\else false\fi}{false}%
  \Expect*{\ifx\foobar\UNDEFINED true\else false\fi}{true}%
\end{qstest}
\end{document}
%    \end{macrocode}
%    \begin{macrocode}
%</test2|test3>
%    \end{macrocode}
%
% \subsection{Test with \plainTeX\ and \eTeX}
%
%    \begin{macrocode}
%<*test4>
%    \end{macrocode}
%    \begin{macrocode}
%% Format `etex' based on `language.def'
\input iflang.sty
\catcode64=12

\def\TestGeneric#1#2#3{%
  \begingroup
    \edef\x{#1{#2}{true}{false}}%
    \edef\y{#3}%
    \ifx\x\y
    \else
      \errmessage{Failed test: \string#1{#2} <> #3}%
    \fi
  \endgroup
}
\def\TestPatterns{\TestGeneric\IfLanguagePatterns}
\def\TestName{\TestGeneric\IfLanguageName}

\TestPatterns{USenglish}{true}
\TestPatterns{ngerman}{false}

\TestName{USenglish}{true}
\TestName{ngerman}{false}

\uselanguage{ngerman}

\TestPatterns{USenglish}{false}
\TestPatterns{ngerman}{true}

\TestName{USenglish}{false}
\TestName{ngerman}{true}

\csname @@end\endcsname
\end
%    \end{macrocode}
%    \begin{macrocode}
%</test4>
%    \end{macrocode}
%
% \subsection{Test with \plainTeX\ and without \eTeX/\pdfTeX}
%
%    \begin{macrocode}
%<*test5>
%    \end{macrocode}
%    \begin{macrocode}
%% Format `tex' (vanilla plain-TeX)
\let\ifcsname\UNDEFINED
\let\pdfstrcmp\UNDEFINED
\input iflang.sty
\catcode64=11

\def\TestDefined#1{%
  \IfLang@IfDefined{foobar}{}{}%
  \ifx\foobar#1%
  \else
    \errmessage{Failed test: \string\foobar <> \string#1}%
  \fi
}
\let\foobar\relax
\TestDefined\relax
\let\foobar\UNDEFINED
\TestDefined\relax

\def\strip@prefix#1>{}
\def\@onelevel@sanitize#1{%
  \edef#1{\expandafter\strip@prefix\meaning#1}%
}
\def\TestCompare#1#2#3{%
  \begingroup
    \edef\x{%
      \if\IfLang@StrEqual{#1}{#2}%
        true%
      \else
        false%
      \fi
    }%
    \def\expect{#3}%
    \ifx\x\expect
    \else
      \def\a{#1}%
      \@onelevel@sanitize\a
      \def\b{#2}%
      \@onelevel@sanitize\b
      \errmessage{Failed test: `\a'='\b' <> \expect}%
    \fi
  \endgroup
}
\TestCompare{junk}{junk}{true}
\TestCompare{}{}{true}
\TestCompare{a}{b}{false}
\TestCompare{aa}{bb}{false}
\def\a{ax}
\def\b{bx}
\def\c{\a\b}
\def\d{\c\b}
\def\exch#1#2{#2#1}
\def\gobble#1{}
\TestCompare{\gobble a}{}{true}
\TestCompare{}{\gobble a}{true}
\TestCompare{a\exch xyb}{ayxb}{true}
\TestCompare{\c}{\c}{true}
\TestCompare{\d}{\c\b}{true}

\csname @@end\endcsname
\end
%    \end{macrocode}
%    \begin{macrocode}
%</test5>
%    \end{macrocode}
%
% \section{Installation}
%
% \subsection{Download}
%
% \paragraph{Package.} This package is available on
% CTAN\footnote{\url{http://ctan.org/pkg/iflang}}:
% \begin{description}
% \item[\CTAN{macros/latex/contrib/oberdiek/iflang.dtx}] The source file.
% \item[\CTAN{macros/latex/contrib/oberdiek/iflang.pdf}] Documentation.
% \end{description}
%
%
% \paragraph{Bundle.} All the packages of the bundle `oberdiek'
% are also available in a TDS compliant ZIP archive. There
% the packages are already unpacked and the documentation files
% are generated. The files and directories obey the TDS standard.
% \begin{description}
% \item[\CTAN{install/macros/latex/contrib/oberdiek.tds.zip}]
% \end{description}
% \emph{TDS} refers to the standard ``A Directory Structure
% for \TeX\ Files'' (\CTAN{tds/tds.pdf}). Directories
% with \xfile{texmf} in their name are usually organized this way.
%
% \subsection{Bundle installation}
%
% \paragraph{Unpacking.} Unpack the \xfile{oberdiek.tds.zip} in the
% TDS tree (also known as \xfile{texmf} tree) of your choice.
% Example (linux):
% \begin{quote}
%   |unzip oberdiek.tds.zip -d ~/texmf|
% \end{quote}
%
% \paragraph{Script installation.}
% Check the directory \xfile{TDS:scripts/oberdiek/} for
% scripts that need further installation steps.
% Package \xpackage{attachfile2} comes with the Perl script
% \xfile{pdfatfi.pl} that should be installed in such a way
% that it can be called as \texttt{pdfatfi}.
% Example (linux):
% \begin{quote}
%   |chmod +x scripts/oberdiek/pdfatfi.pl|\\
%   |cp scripts/oberdiek/pdfatfi.pl /usr/local/bin/|
% \end{quote}
%
% \subsection{Package installation}
%
% \paragraph{Unpacking.} The \xfile{.dtx} file is a self-extracting
% \docstrip\ archive. The files are extracted by running the
% \xfile{.dtx} through \plainTeX:
% \begin{quote}
%   \verb|tex iflang.dtx|
% \end{quote}
%
% \paragraph{TDS.} Now the different files must be moved into
% the different directories in your installation TDS tree
% (also known as \xfile{texmf} tree):
% \begin{quote}
% \def\t{^^A
% \begin{tabular}{@{}>{\ttfamily}l@{ $\rightarrow$ }>{\ttfamily}l@{}}
%   iflang.sty & tex/generic/oberdiek/iflang.sty\\
%   iflang.pdf & doc/latex/oberdiek/iflang.pdf\\
%   test/iflang-test1.tex & doc/latex/oberdiek/test/iflang-test1.tex\\
%   test/iflang-test2.tex & doc/latex/oberdiek/test/iflang-test2.tex\\
%   test/iflang-test3.tex & doc/latex/oberdiek/test/iflang-test3.tex\\
%   test/iflang-test4.tex & doc/latex/oberdiek/test/iflang-test4.tex\\
%   test/iflang-test5.tex & doc/latex/oberdiek/test/iflang-test5.tex\\
%   iflang.dtx & source/latex/oberdiek/iflang.dtx\\
% \end{tabular}^^A
% }^^A
% \sbox0{\t}^^A
% \ifdim\wd0>\linewidth
%   \begingroup
%     \advance\linewidth by\leftmargin
%     \advance\linewidth by\rightmargin
%   \edef\x{\endgroup
%     \def\noexpand\lw{\the\linewidth}^^A
%   }\x
%   \def\lwbox{^^A
%     \leavevmode
%     \hbox to \linewidth{^^A
%       \kern-\leftmargin\relax
%       \hss
%       \usebox0
%       \hss
%       \kern-\rightmargin\relax
%     }^^A
%   }^^A
%   \ifdim\wd0>\lw
%     \sbox0{\small\t}^^A
%     \ifdim\wd0>\linewidth
%       \ifdim\wd0>\lw
%         \sbox0{\footnotesize\t}^^A
%         \ifdim\wd0>\linewidth
%           \ifdim\wd0>\lw
%             \sbox0{\scriptsize\t}^^A
%             \ifdim\wd0>\linewidth
%               \ifdim\wd0>\lw
%                 \sbox0{\tiny\t}^^A
%                 \ifdim\wd0>\linewidth
%                   \lwbox
%                 \else
%                   \usebox0
%                 \fi
%               \else
%                 \lwbox
%               \fi
%             \else
%               \usebox0
%             \fi
%           \else
%             \lwbox
%           \fi
%         \else
%           \usebox0
%         \fi
%       \else
%         \lwbox
%       \fi
%     \else
%       \usebox0
%     \fi
%   \else
%     \lwbox
%   \fi
% \else
%   \usebox0
% \fi
% \end{quote}
% If you have a \xfile{docstrip.cfg} that configures and enables \docstrip's
% TDS installing feature, then some files can already be in the right
% place, see the documentation of \docstrip.
%
% \subsection{Refresh file name databases}
%
% If your \TeX~distribution
% (\teTeX, \mikTeX, \dots) relies on file name databases, you must refresh
% these. For example, \teTeX\ users run \verb|texhash| or
% \verb|mktexlsr|.
%
% \subsection{Some details for the interested}
%
% \paragraph{Attached source.}
%
% The PDF documentation on CTAN also includes the
% \xfile{.dtx} source file. It can be extracted by
% AcrobatReader 6 or higher. Another option is \textsf{pdftk},
% e.g. unpack the file into the current directory:
% \begin{quote}
%   \verb|pdftk iflang.pdf unpack_files output .|
% \end{quote}
%
% \paragraph{Unpacking with \LaTeX.}
% The \xfile{.dtx} chooses its action depending on the format:
% \begin{description}
% \item[\plainTeX:] Run \docstrip\ and extract the files.
% \item[\LaTeX:] Generate the documentation.
% \end{description}
% If you insist on using \LaTeX\ for \docstrip\ (really,
% \docstrip\ does not need \LaTeX), then inform the autodetect routine
% about your intention:
% \begin{quote}
%   \verb|latex \let\install=y% \iffalse meta-comment
%
% File: iflang.dtx
% Version: 2016/05/16 v1.6
% Info: Checks for the current language
%
% Copyright (C) 2007 by
%    Heiko Oberdiek <heiko.oberdiek at googlemail.com>
%    2016
%    https://github.com/ho-tex/oberdiek/issues
%
% This work may be distributed and/or modified under the
% conditions of the LaTeX Project Public License, either
% version 1.3c of this license or (at your option) any later
% version. This version of this license is in
%    http://www.latex-project.org/lppl/lppl-1-3c.txt
% and the latest version of this license is in
%    http://www.latex-project.org/lppl.txt
% and version 1.3 or later is part of all distributions of
% LaTeX version 2005/12/01 or later.
%
% This work has the LPPL maintenance status "maintained".
%
% This Current Maintainer of this work is Heiko Oberdiek.
%
% The Base Interpreter refers to any `TeX-Format',
% because some files are installed in TDS:tex/generic//.
%
% This work consists of the main source file iflang.dtx
% and the derived files
%    iflang.sty, iflang.pdf, iflang.ins, iflang.drv, iflang-test1.tex,
%    iflang-test2.tex, iflang-test3.tex, iflang-test4.tex,
%    iflang-test5.tex.
%
% Distribution:
%    CTAN:macros/latex/contrib/oberdiek/iflang.dtx
%    CTAN:macros/latex/contrib/oberdiek/iflang.pdf
%
% Unpacking:
%    (a) If iflang.ins is present:
%           tex iflang.ins
%    (b) Without iflang.ins:
%           tex iflang.dtx
%    (c) If you insist on using LaTeX
%           latex \let\install=y\input{iflang.dtx}
%        (quote the arguments according to the demands of your shell)
%
% Documentation:
%    (a) If iflang.drv is present:
%           latex iflang.drv
%    (b) Without iflang.drv:
%           latex iflang.dtx; ...
%    The class ltxdoc loads the configuration file ltxdoc.cfg
%    if available. Here you can specify further options, e.g.
%    use A4 as paper format:
%       \PassOptionsToClass{a4paper}{article}
%
%    Programm calls to get the documentation (example):
%       pdflatex iflang.dtx
%       makeindex -s gind.ist iflang.idx
%       pdflatex iflang.dtx
%       makeindex -s gind.ist iflang.idx
%       pdflatex iflang.dtx
%
% Installation:
%    TDS:tex/generic/oberdiek/iflang.sty
%    TDS:doc/latex/oberdiek/iflang.pdf
%    TDS:doc/latex/oberdiek/test/iflang-test1.tex
%    TDS:doc/latex/oberdiek/test/iflang-test2.tex
%    TDS:doc/latex/oberdiek/test/iflang-test3.tex
%    TDS:doc/latex/oberdiek/test/iflang-test4.tex
%    TDS:doc/latex/oberdiek/test/iflang-test5.tex
%    TDS:source/latex/oberdiek/iflang.dtx
%
%<*ignore>
\begingroup
  \catcode123=1 %
  \catcode125=2 %
  \def\x{LaTeX2e}%
\expandafter\endgroup
\ifcase 0\ifx\install y1\fi\expandafter
         \ifx\csname processbatchFile\endcsname\relax\else1\fi
         \ifx\fmtname\x\else 1\fi\relax
\else\csname fi\endcsname
%</ignore>
%<*install>
\input docstrip.tex
\Msg{************************************************************************}
\Msg{* Installation}
\Msg{* Package: iflang 2016/05/16 v1.6 Checks for the current language (HO)}
\Msg{************************************************************************}

\keepsilent
\askforoverwritefalse

\let\MetaPrefix\relax
\preamble

This is a generated file.

Project: iflang
Version: 2016/05/16 v1.6

Copyright (C) 2007 by
   Heiko Oberdiek <heiko.oberdiek at googlemail.com>

This work may be distributed and/or modified under the
conditions of the LaTeX Project Public License, either
version 1.3c of this license or (at your option) any later
version. This version of this license is in
   http://www.latex-project.org/lppl/lppl-1-3c.txt
and the latest version of this license is in
   http://www.latex-project.org/lppl.txt
and version 1.3 or later is part of all distributions of
LaTeX version 2005/12/01 or later.

This work has the LPPL maintenance status "maintained".

This Current Maintainer of this work is Heiko Oberdiek.

The Base Interpreter refers to any `TeX-Format',
because some files are installed in TDS:tex/generic//.

This work consists of the main source file iflang.dtx
and the derived files
   iflang.sty, iflang.pdf, iflang.ins, iflang.drv, iflang-test1.tex,
   iflang-test2.tex, iflang-test3.tex, iflang-test4.tex,
   iflang-test5.tex.

\endpreamble
\let\MetaPrefix\DoubleperCent

\generate{%
  \file{iflang.ins}{\from{iflang.dtx}{install}}%
  \file{iflang.drv}{\from{iflang.dtx}{driver}}%
  \usedir{tex/generic/oberdiek}%
  \file{iflang.sty}{\from{iflang.dtx}{package}}%
  \usedir{doc/latex/oberdiek/test}%
  \file{iflang-test1.tex}{\from{iflang.dtx}{test1}}%
  \file{iflang-test2.tex}{\from{iflang.dtx}{test2}}%
  \file{iflang-test3.tex}{\from{iflang.dtx}{test3}}%
  \file{iflang-test4.tex}{\from{iflang.dtx}{test4}}%
  \file{iflang-test5.tex}{\from{iflang.dtx}{test5}}%
  \nopreamble
  \nopostamble
  \usedir{source/latex/oberdiek/catalogue}%
  \file{iflang.xml}{\from{iflang.dtx}{catalogue}}%
}

\catcode32=13\relax% active space
\let =\space%
\Msg{************************************************************************}
\Msg{*}
\Msg{* To finish the installation you have to move the following}
\Msg{* file into a directory searched by TeX:}
\Msg{*}
\Msg{*     iflang.sty}
\Msg{*}
\Msg{* To produce the documentation run the file `iflang.drv'}
\Msg{* through LaTeX.}
\Msg{*}
\Msg{* Happy TeXing!}
\Msg{*}
\Msg{************************************************************************}

\endbatchfile
%</install>
%<*ignore>
\fi
%</ignore>
%<*driver>
\NeedsTeXFormat{LaTeX2e}
\ProvidesFile{iflang.drv}%
  [2016/05/16 v1.6 Checks for the current language (HO)]%
\documentclass{ltxdoc}
\usepackage{holtxdoc}[2011/11/22]
\begin{document}
  \DocInput{iflang.dtx}%
\end{document}
%</driver>
% \fi
%
%
% \CharacterTable
%  {Upper-case    \A\B\C\D\E\F\G\H\I\J\K\L\M\N\O\P\Q\R\S\T\U\V\W\X\Y\Z
%   Lower-case    \a\b\c\d\e\f\g\h\i\j\k\l\m\n\o\p\q\r\s\t\u\v\w\x\y\z
%   Digits        \0\1\2\3\4\5\6\7\8\9
%   Exclamation   \!     Double quote  \"     Hash (number) \#
%   Dollar        \$     Percent       \%     Ampersand     \&
%   Acute accent  \'     Left paren    \(     Right paren   \)
%   Asterisk      \*     Plus          \+     Comma         \,
%   Minus         \-     Point         \.     Solidus       \/
%   Colon         \:     Semicolon     \;     Less than     \<
%   Equals        \=     Greater than  \>     Question mark \?
%   Commercial at \@     Left bracket  \[     Backslash     \\
%   Right bracket \]     Circumflex    \^     Underscore    \_
%   Grave accent  \`     Left brace    \{     Vertical bar  \|
%   Right brace   \}     Tilde         \~}
%
% \GetFileInfo{iflang.drv}
%
% \title{The \xpackage{iflang} package}
% \date{2016/05/16 v1.6}
% \author{Heiko Oberdiek\thanks
% {Please report any issues at https://github.com/ho-tex/oberdiek/issues}\\
% \xemail{heiko.oberdiek at googlemail.com}}
%
% \maketitle
%
% \begin{abstract}
% This package provides expandible checks for the current language
% based on macro \cs{languagename} or hyphenation patterns.
% \end{abstract}
%
% \tableofcontents
%
% \section{Documentation}
%
% Package \xpackage{babel} defines \cs{iflanguagename}. As first
% argument it takes a language name and executes the second or
% third argument depending on the current language. This
% language test is based on hypenation patterns. However, it
% is possible that different languages or dialects share the
% same patterns. In such cases \cs{iflanguagename} fails.
%
% However, package \xpackage{babel} and some other packages
% such as \xpackage{german} or \xpackage{ngerman} store
% the language name in the macro \cs{languagename} if
% \cs{selectlanguage} is called.
%
% \begin{declcs}{IfLanguageName} \M{lang} \M{then} \M{else}
% \end{declcs}
% Makro \cs{IfLanguageName} compares language \meta{lang} with
% the current setting of macro \cs{languagename}. If both
% contains the same name then the \meta{then} part is called,
% otherwise the \meta{else} part.
%
% The macro is expandable. Thus it can be safely used inside \cs{edef}
% or \cs{csname}. If case of errors like an undefined \cs{languagename}
% the \meta{else} part is executed.
%
% Note: Macro \cs{IfLanguageName} relies on the fact, that
% \cs{languagename} is set correctly:
% \begin{description}
% \item[Package \xpackage{babel}:]
%    \mbox{}\\
%    Full support of \cs{languagename} in its language switching commands.
% \item[Format based on \xpackage{babel} (\xfile{language.dat}):]
%    \mbox{}\\
%    If package \xpackage{babel} is not used (or not yet loaded),
%    then \xpackage{babel}'s
%    \xfile{hyphen.cfg} has set \cs{languagename} to the last
%    language in \xfile{language.dat}, but \cs{language} (current
%    patterns) is zero and points to the first language.
%    Thus the value of \cs{languagename} is basically garbage.
%    Package \xpackage{iflang} warns if \cs{languagename} and
%    \cs{language} do not fit.
%    This can be fixed by loading package \xpackage{babel} previously.
% \item[Format based on \eTeX's \xfile{etex.src} (\xfile{language.def}):]
%    \mbox{}\\
%    Unhappily it does not support
%    \cs{languagename}. Thus this package hooks into \cs{uselanguage}
%    to get \cs{languagename} defined and updated there.
%    At package loading time the changed \cs{uselanguage} has not
%    been called yet. Thus package \xpackage{iflang} tries
%    |USenglish|. This is the definite default language of \xfile{etex.src}.
%    If the current patterns suit this default language, an
%    undefined \cs{languagename} is set to this language. Otherwise
%    a \cs{languagename} remains undefined and a warning is given.
% \end{description}
%
% \begin{declcs}{IfLanguagePatterns} \M{lang} \M{then} \M{else}
% \end{declcs}
% This macro behaves similar to \cs{IfLanguageName}. But the
% language test is based on the current pattern in force (\cs{language}).
% Also this macro is expandable, in case of errors the \meta{else} part
% is called.
%
% The following naming convention for the pattern are supported:
% \begin{description}
% \item[\xpackage{babel}/\xfile{language.dat}]:
%    \cs{l@}\meta{language}
% \item[\xpackage{etex.src}/\xfile{language.def}]:
%    \cs{lang@}\meta{language}
% \end{description}
% Package \xpackage{iflang} looks for \cs{uselanguage} (defined in
% \xfile{etex.src}) to find out the naming convention in use.
%
% \StopEventually{
% }
%
% \section{Implementation}
%
%    \begin{macrocode}
%<*package>
%    \end{macrocode}
%
% \subsection{Reload check and package identification}
%    Reload check, especially if the package is not used with \LaTeX.
%    \begin{macrocode}
\begingroup\catcode61\catcode48\catcode32=10\relax%
  \catcode13=5 % ^^M
  \endlinechar=13 %
  \catcode35=6 % #
  \catcode39=12 % '
  \catcode44=12 % ,
  \catcode45=12 % -
  \catcode46=12 % .
  \catcode58=12 % :
  \catcode64=11 % @
  \catcode123=1 % {
  \catcode125=2 % }
  \expandafter\let\expandafter\x\csname ver@iflang.sty\endcsname
  \ifx\x\relax % plain-TeX, first loading
  \else
    \def\empty{}%
    \ifx\x\empty % LaTeX, first loading,
      % variable is initialized, but \ProvidesPackage not yet seen
    \else
      \expandafter\ifx\csname PackageInfo\endcsname\relax
        \def\x#1#2{%
          \immediate\write-1{Package #1 Info: #2.}%
        }%
      \else
        \def\x#1#2{\PackageInfo{#1}{#2, stopped}}%
      \fi
      \x{iflang}{The package is already loaded}%
      \aftergroup\endinput
    \fi
  \fi
\endgroup%
%    \end{macrocode}
%    Package identification:
%    \begin{macrocode}
\begingroup\catcode61\catcode48\catcode32=10\relax%
  \catcode13=5 % ^^M
  \endlinechar=13 %
  \catcode35=6 % #
  \catcode39=12 % '
  \catcode40=12 % (
  \catcode41=12 % )
  \catcode44=12 % ,
  \catcode45=12 % -
  \catcode46=12 % .
  \catcode47=12 % /
  \catcode58=12 % :
  \catcode64=11 % @
  \catcode91=12 % [
  \catcode93=12 % ]
  \catcode123=1 % {
  \catcode125=2 % }
  \expandafter\ifx\csname ProvidesPackage\endcsname\relax
    \def\x#1#2#3[#4]{\endgroup
      \immediate\write-1{Package: #3 #4}%
      \xdef#1{#4}%
    }%
  \else
    \def\x#1#2[#3]{\endgroup
      #2[{#3}]%
      \ifx#1\@undefined
        \xdef#1{#3}%
      \fi
      \ifx#1\relax
        \xdef#1{#3}%
      \fi
    }%
  \fi
\expandafter\x\csname ver@iflang.sty\endcsname
\ProvidesPackage{iflang}%
  [2016/05/16 v1.6 Checks for the current language (HO)]%
%    \end{macrocode}
%
%    \begin{macrocode}
\begingroup\catcode61\catcode48\catcode32=10\relax%
  \catcode13=5 % ^^M
  \endlinechar=13 %
  \catcode123=1 % {
  \catcode125=2 % }
  \catcode64=11 % @
  \def\x{\endgroup
    \expandafter\edef\csname IfLang@AtEnd\endcsname{%
      \endlinechar=\the\endlinechar\relax
      \catcode13=\the\catcode13\relax
      \catcode32=\the\catcode32\relax
      \catcode35=\the\catcode35\relax
      \catcode61=\the\catcode61\relax
      \catcode64=\the\catcode64\relax
      \catcode123=\the\catcode123\relax
      \catcode125=\the\catcode125\relax
    }%
  }%
\x\catcode61\catcode48\catcode32=10\relax%
\catcode13=5 % ^^M
\endlinechar=13 %
\catcode35=6 % #
\catcode64=11 % @
\catcode123=1 % {
\catcode125=2 % }
\def\TMP@EnsureCode#1#2{%
  \edef\IfLang@AtEnd{%
    \IfLang@AtEnd
    \catcode#1=\the\catcode#1\relax
  }%
  \catcode#1=#2\relax
}
\TMP@EnsureCode{39}{12}% '
\TMP@EnsureCode{40}{12}% (
\TMP@EnsureCode{41}{12}% )
\TMP@EnsureCode{44}{12}% ,
\TMP@EnsureCode{46}{12}% .
\TMP@EnsureCode{47}{12}% /
\TMP@EnsureCode{58}{12}% :
\TMP@EnsureCode{91}{12}% [
\TMP@EnsureCode{93}{12}% ]
\edef\IfLang@AtEnd{\IfLang@AtEnd\noexpand\endinput}
%    \end{macrocode}
%
% \subsection{Tools}
%
% \subsubsection{Provide some basic macros of \LaTeX}
%
%    \begin{macro}{\@firstoftwo}
%    \begin{macrocode}
\expandafter\ifx\csname @firstoftwo\endcsname\relax
  \long\def\@firstoftwo#1#2{#1}%
\fi
%    \end{macrocode}
%    \end{macro}
%    \begin{macro}{\@secondoftwo}
%    \begin{macrocode}
\expandafter\ifx\csname @secondoftwo\endcsname\relax
  \long\def\@secondoftwo#1#2{#2}%
\fi
%    \end{macrocode}
%    \end{macro}
%
% \subsubsection{Expandible existence check for macros}
%
%    \begin{macro}{\IfLang@IfDefined}
%    \begin{macrocode}
\begingroup\expandafter\expandafter\expandafter\endgroup
\expandafter\ifx\csname ifcsname\endcsname\relax
  \expandafter\@firstoftwo
\else
  \expandafter\@secondoftwo
\fi
{%
  \def\IfLang@IfDefined#1{%
    \expandafter\ifx\csname#1\endcsname\relax
      \expandafter\@secondoftwo
    \else
      \expandafter\@firstoftwo
    \fi
  }%
}{%
  \def\IfLang@IfDefined#1{%
    \ifnum\ifcsname#1\endcsname
            \expandafter\ifx\csname#1\endcsname\relax
              1%
            \else
              0%
            \fi
          \else
            1%
          \fi
          =0 %
      \expandafter\@firstoftwo
    \else
      \expandafter\@secondoftwo
    \fi
  }%
}
%    \end{macrocode}
%    \end{macro}
%
% \subsubsection{Macros for messages}
%
%    \begin{macrocode}
\begingroup\expandafter\expandafter\expandafter\endgroup
\expandafter\ifx\csname RequirePackage\endcsname\relax
  \input infwarerr.sty\relax
  \input pdftexcmds.sty\relax
\else
  \RequirePackage{infwarerr}[2007/09/09]%
  \RequirePackage{pdftexcmds}[2016/05/16]%
\fi
%    \end{macrocode}
%
% \subsubsection{Support for \xfile{etex.src}}
%
%    \begin{macro}{\IfLang@prefix}
%    \begin{macrocode}
\begingroup\expandafter\expandafter\expandafter\endgroup
\expandafter\ifx\csname uselanguage\endcsname\relax
  \@PackageInfoNoLine{iflang}{%
    Naming convention for patterns: babel%
  }%
  \def\IfLang@prefix{l@}%
\else
  \@PackageInfoNoLine{iflang}{%
    Naming convention for patterns: etex.src%
  }%
  \def\IfLang@prefix{lang@}%
  \let\IfLang@OrgUseLanguage\uselanguage
  \def\uselanguage#1{%
    \edef\languagename{#1}%
    \IfLang@OrgUseLanguage{#1}%
  }%
%    \end{macrocode}
%    \end{macro}
%    The first \cs{uselanguage} that is executed as last
%    line in \xfile{language.def} cannot patched this way.
%    However, \xfile{language.def} is very strict. It forces
%    the first added and used language to be |USenglish|.
%    Thus, if \cs{languagename} is not defined, we can
%    quite safely assume |USenglish|. As additional
%    safety precaution the actual used patterns are checked.
%    \begin{macrocode}
  \begingroup\expandafter\expandafter\expandafter\endgroup
  \expandafter\ifx\csname languagename\endcsname\relax
    \begingroup\expandafter\expandafter\expandafter\endgroup
    \expandafter\ifx\csname lang@USenglish\endcsname\relax
      \@PackageWarningNoLine{iflang}{%
        \string\lang@USenglish\space is missing%
      }%
    \else
      \ifnum\lang@USenglish=\language
        \def\languagename{USenglish}%
      \else
        \@PackageWarningNoLine{iflang}{%
          \string\languagename\space is not set,\MessageBreak
          current language is unknown%
        }%
      \fi
    \fi
  \fi
\fi
\begingroup\expandafter\expandafter\expandafter\endgroup
\expandafter\ifx\csname languagename\endcsname\relax
  \@PackageInfoNoLine{iflang}{%
    \string\languagename\space is not set%
  }%
\fi
%    \end{macrocode}
%
% \subsection{\cs{IfLanguagePatterns}}
%
%    \begin{macro}{\IfLanguagePatterns}
%    \begin{macrocode}
\def\IfLanguagePatterns#1{%
  \ifnum\IfLang@IfDefined{\IfLang@prefix#1}{%
          \ifnum\csname\IfLang@prefix#1\endcsname=\language
            0%
          \else
            1%
          \fi
        }{1}=0 %
    \expandafter\@firstoftwo
  \else
    \expandafter\@secondoftwo
  \fi
}
%    \end{macrocode}
%    \end{macro}
%
% \subsection{\cs{IfLanguageName}}
%
%    \begin{macrocode}
\begingroup\expandafter\expandafter\expandafter\endgroup
\expandafter\ifx\csname pdf@strcmp\endcsname\relax
  \expandafter\@firstoftwo
\else
  \expandafter\@secondoftwo
\fi
{%
%    \end{macrocode}
%    We do not have \cs{pdf@strcmp} (and \cs{pdfstrcmp}).
%    Thus we must define
%    our own expandable string comparison. The following
%    implementation is based on a \TeX\ pearl from David Kastrup, presented
%    at the conference Bacho\TeX~2005:
%    \url{http://www-stary.gust.org.pl/pearls/2005/david-kastrup/bachotex2005-david-kastrup-pearl1.pdf}
%
%    The orignal code allows macros inside the second string.
%    Because also \cs{languagename} might consists of further
%    macros, we need a variant that allows macros in the first
%    string, too.
%    \begin{macrocode}
  \def\IfLang@StrNil{\relax}%
  \def\IfLang@StrEqual#1{%
    \number\IfLang@StrEqualStart{}{}#1\IfLang@StrNil
  }%
  \def\IfLang@StrEqualStart#1#2#3{%
    \ifx#3\IfLang@StrNil
      \IfLang@StrEqualStop
    \fi
    \ifcat\noexpand#3\relax
      \IfLang@StrExpand{#1}{#2}#3%
    \fi
    \IfLang@StrEqualStart{\if#3#1}{#2\fi}%
  }%
  \def\IfLang@StrEqualStop\fi#1\IfLang@StrEqualStart#2#3#4{%
    \fi
    #2#4\relax'#313 %
  }%
  \def\IfLang@StrExpand#1#2#3\fi\IfLang@StrEqualStart#4#5{%
    \fi
    \IfLang@@StrExpand{#1}{#2}#3%
  }%
  \def\IfLang@@StrExpand#1#2#3\IfLang@StrNil{%
    \expandafter\IfLang@@@StrExpand#3\IfLang@StrNil{#1}{#2}%
  }%
  \def\IfLang@@@StrExpand#1\IfLang@StrNil#2#3{%
    \IfLang@StrEqualStart{#2}{#3}#1\IfLang@StrNil
  }%
%    \end{macrocode}
%    \begin{macro}{\IfLanguageName}
%    \begin{macrocode}
  \def\IfLanguageName#1{%
    \ifnum\IfLang@IfDefined{languagename}{%
            \if\expandafter\IfLang@StrEqual\expandafter%
                           {\languagename}{#1}%
              0%
            \else
              1%
            \fi
          }{1}=0 %
      \expandafter\@firstoftwo
    \else
      \expandafter\@secondoftwo
    \fi
  }%
%    \end{macrocode}
%    \end{macro}
%    \begin{macrocode}
}{%
%    \end{macrocode}
%    \begin{macro}{\IfLanguageName}
%    \begin{macrocode}
  \def\IfLanguageName#1{%
    \ifnum\IfLang@IfDefined{languagename}{%
            \pdf@strcmp{#1}{\languagename}%
          }{1}=0 %
      \expandafter\@firstoftwo
    \else
      \expandafter\@secondoftwo
    \fi
  }%
%    \end{macrocode}
%    \end{macro}
%    \begin{macrocode}
}
%    \end{macrocode}
%
% \subsection{Check plausibility of \cs{languagename}}
%
%    \begin{macrocode}
\begingroup\expandafter\expandafter\expandafter\endgroup
\expandafter\ifx\csname languagename\endcsname\relax
\else
  \IfLanguagePatterns{\languagename}{}{%
    \@PackageWarningNoLine{iflang}{%
      Mismatch between \string\language\space
      (patterns)\MessageBreak
      and setting of \string\languagename
    }%
  }%
\fi
%    \end{macrocode}
%
%    \begin{macrocode}
\IfLang@AtEnd%
%</package>
%    \end{macrocode}
%
% \section{Test}
%
% \subsection{Catcode checks for loading}
%
%    \begin{macrocode}
%<*test1>
%    \end{macrocode}
%    \begin{macrocode}
\catcode`\{=1 %
\catcode`\}=2 %
\catcode`\#=6 %
\catcode`\@=11 %
\expandafter\ifx\csname count@\endcsname\relax
  \countdef\count@=255 %
\fi
\expandafter\ifx\csname @gobble\endcsname\relax
  \long\def\@gobble#1{}%
\fi
\expandafter\ifx\csname @firstofone\endcsname\relax
  \long\def\@firstofone#1{#1}%
\fi
\expandafter\ifx\csname loop\endcsname\relax
  \expandafter\@firstofone
\else
  \expandafter\@gobble
\fi
{%
  \def\loop#1\repeat{%
    \def\body{#1}%
    \iterate
  }%
  \def\iterate{%
    \body
      \let\next\iterate
    \else
      \let\next\relax
    \fi
    \next
  }%
  \let\repeat=\fi
}%
\def\RestoreCatcodes{}
\count@=0 %
\loop
  \edef\RestoreCatcodes{%
    \RestoreCatcodes
    \catcode\the\count@=\the\catcode\count@\relax
  }%
\ifnum\count@<255 %
  \advance\count@ 1 %
\repeat

\def\RangeCatcodeInvalid#1#2{%
  \count@=#1\relax
  \loop
    \catcode\count@=15 %
  \ifnum\count@<#2\relax
    \advance\count@ 1 %
  \repeat
}
\def\RangeCatcodeCheck#1#2#3{%
  \count@=#1\relax
  \loop
    \ifnum#3=\catcode\count@
    \else
      \errmessage{%
        Character \the\count@\space
        with wrong catcode \the\catcode\count@\space
        instead of \number#3%
      }%
    \fi
  \ifnum\count@<#2\relax
    \advance\count@ 1 %
  \repeat
}
\def\space{ }
\expandafter\ifx\csname LoadCommand\endcsname\relax
  \def\LoadCommand{\input iflang.sty\relax}%
\fi
\def\Test{%
  \RangeCatcodeInvalid{0}{47}%
  \RangeCatcodeInvalid{58}{64}%
  \RangeCatcodeInvalid{91}{96}%
  \RangeCatcodeInvalid{123}{255}%
  \catcode`\@=12 %
  \catcode`\\=0 %
  \catcode`\%=14 %
  \LoadCommand
  \RangeCatcodeCheck{0}{36}{15}%
  \RangeCatcodeCheck{37}{37}{14}%
  \RangeCatcodeCheck{38}{47}{15}%
  \RangeCatcodeCheck{48}{57}{12}%
  \RangeCatcodeCheck{58}{63}{15}%
  \RangeCatcodeCheck{64}{64}{12}%
  \RangeCatcodeCheck{65}{90}{11}%
  \RangeCatcodeCheck{91}{91}{15}%
  \RangeCatcodeCheck{92}{92}{0}%
  \RangeCatcodeCheck{93}{96}{15}%
  \RangeCatcodeCheck{97}{122}{11}%
  \RangeCatcodeCheck{123}{255}{15}%
  \RestoreCatcodes
}
\Test
\csname @@end\endcsname
\end
%    \end{macrocode}
%    \begin{macrocode}
%</test1>
%    \end{macrocode}
%
% \subsection{Test with \LaTeX}
%
%    \begin{macrocode}
%<*test2|test3>
%    \end{macrocode}
%    \begin{macrocode}
\NeedsTeXFormat{LaTeX2e}
%<test3>\let\pdfstrcmp\relax
\nofiles
\documentclass{minimal}
\usepackage{qstest}
\IncludeTests{*}
\LogTests{log}{*}{*}
\usepackage[english,naustrian,ngerman]{babel}
\usepackage{iflang}
\begin{document}
\begin{qstest}{IfLanguagePatterns}{language, pattern}
  \def\test#1#2{%
    \Expect*{\IfLanguagePatterns{#1}{true}{false}}{#2}%
  }%
  \test{ngerman}{true}%
  \test{naustrian}{true}%
  \test{english}{false}%
  \test{foobar}{false}%
\end{qstest}
\begin{qstest}{IfLanguageName}{language, name}
  \def\test#1#2{%
    \Expect*{\IfLanguageName{#1}{true}{false}}{#2}%
  }%
  \test{ngerman}{true}%
  \test{naustrian}{false}%
  \selectlanguage{naustrian}%
  \test{ngerman}{false}%
  \test{naustrian}{true}%
  \test{foobar}{false}%
  %
  \def\languagename{naustrian}%
  \test{naustrian}{true}%
  \test{ngerman}{false}%
  %
  \edef\languagename{\string naustrian}%
  \test{naustrian}{true}%
  \test{ngerman}{false}%
  %
  \def\languagename{naustrian}%
  \makeatletter
  \@onelevel@sanitize\languagename
  \test{naustrian}{true}%
  \test{ngerman}{false}%
  %
  \def\languagename{naustrian}%
  \def\xaustrian{naustrian}%
  \def\xgerman{ngerman}%
  \test{\xaustrian}{true}%
  \test{\xgerman}{false}%
  %
  \def\languagename{\xaustrian}%
  \test{naustrian}{true}%
  \test{ngerman}{false}%
  \test{\xaustrian}{true}%
  \test{\xgerman}{false}%
  \test{\languagename}{true}%
  \test{\languagename\space}{false}%
  %
  \def\languagename{\empty\xaustrian\empty}%
  \test{naustrian}{true}%
  \test{ngerman}{false}%
  \test{\empty\xaustrian\empty}{true}%
  \test{\empty\xgerman\empty}{false}%
\end{qstest}
\begin{qstest}{IfDefined}{defined}
  \makeatletter
  \let\foobar\relax
  \Expect*{\IfLang@IfDefined{foobar}{true}{false}}{false}%
  \Expect*{\ifx\foobar\relax true\else false\fi}{true}%
  \let\foobar\UNDEFINED
  \Expect*{\IfLang@IfDefined{foobar}{true}{false}}{false}%
  \Expect*{\ifx\foobar\relax true\else false\fi}{false}%
  \Expect*{\ifx\foobar\UNDEFINED true\else false\fi}{true}%
\end{qstest}
\end{document}
%    \end{macrocode}
%    \begin{macrocode}
%</test2|test3>
%    \end{macrocode}
%
% \subsection{Test with \plainTeX\ and \eTeX}
%
%    \begin{macrocode}
%<*test4>
%    \end{macrocode}
%    \begin{macrocode}
%% Format `etex' based on `language.def'
\input iflang.sty
\catcode64=12

\def\TestGeneric#1#2#3{%
  \begingroup
    \edef\x{#1{#2}{true}{false}}%
    \edef\y{#3}%
    \ifx\x\y
    \else
      \errmessage{Failed test: \string#1{#2} <> #3}%
    \fi
  \endgroup
}
\def\TestPatterns{\TestGeneric\IfLanguagePatterns}
\def\TestName{\TestGeneric\IfLanguageName}

\TestPatterns{USenglish}{true}
\TestPatterns{ngerman}{false}

\TestName{USenglish}{true}
\TestName{ngerman}{false}

\uselanguage{ngerman}

\TestPatterns{USenglish}{false}
\TestPatterns{ngerman}{true}

\TestName{USenglish}{false}
\TestName{ngerman}{true}

\csname @@end\endcsname
\end
%    \end{macrocode}
%    \begin{macrocode}
%</test4>
%    \end{macrocode}
%
% \subsection{Test with \plainTeX\ and without \eTeX/\pdfTeX}
%
%    \begin{macrocode}
%<*test5>
%    \end{macrocode}
%    \begin{macrocode}
%% Format `tex' (vanilla plain-TeX)
\let\ifcsname\UNDEFINED
\let\pdfstrcmp\UNDEFINED
\input iflang.sty
\catcode64=11

\def\TestDefined#1{%
  \IfLang@IfDefined{foobar}{}{}%
  \ifx\foobar#1%
  \else
    \errmessage{Failed test: \string\foobar <> \string#1}%
  \fi
}
\let\foobar\relax
\TestDefined\relax
\let\foobar\UNDEFINED
\TestDefined\relax

\def\strip@prefix#1>{}
\def\@onelevel@sanitize#1{%
  \edef#1{\expandafter\strip@prefix\meaning#1}%
}
\def\TestCompare#1#2#3{%
  \begingroup
    \edef\x{%
      \if\IfLang@StrEqual{#1}{#2}%
        true%
      \else
        false%
      \fi
    }%
    \def\expect{#3}%
    \ifx\x\expect
    \else
      \def\a{#1}%
      \@onelevel@sanitize\a
      \def\b{#2}%
      \@onelevel@sanitize\b
      \errmessage{Failed test: `\a'='\b' <> \expect}%
    \fi
  \endgroup
}
\TestCompare{junk}{junk}{true}
\TestCompare{}{}{true}
\TestCompare{a}{b}{false}
\TestCompare{aa}{bb}{false}
\def\a{ax}
\def\b{bx}
\def\c{\a\b}
\def\d{\c\b}
\def\exch#1#2{#2#1}
\def\gobble#1{}
\TestCompare{\gobble a}{}{true}
\TestCompare{}{\gobble a}{true}
\TestCompare{a\exch xyb}{ayxb}{true}
\TestCompare{\c}{\c}{true}
\TestCompare{\d}{\c\b}{true}

\csname @@end\endcsname
\end
%    \end{macrocode}
%    \begin{macrocode}
%</test5>
%    \end{macrocode}
%
% \section{Installation}
%
% \subsection{Download}
%
% \paragraph{Package.} This package is available on
% CTAN\footnote{\url{http://ctan.org/pkg/iflang}}:
% \begin{description}
% \item[\CTAN{macros/latex/contrib/oberdiek/iflang.dtx}] The source file.
% \item[\CTAN{macros/latex/contrib/oberdiek/iflang.pdf}] Documentation.
% \end{description}
%
%
% \paragraph{Bundle.} All the packages of the bundle `oberdiek'
% are also available in a TDS compliant ZIP archive. There
% the packages are already unpacked and the documentation files
% are generated. The files and directories obey the TDS standard.
% \begin{description}
% \item[\CTAN{install/macros/latex/contrib/oberdiek.tds.zip}]
% \end{description}
% \emph{TDS} refers to the standard ``A Directory Structure
% for \TeX\ Files'' (\CTAN{tds/tds.pdf}). Directories
% with \xfile{texmf} in their name are usually organized this way.
%
% \subsection{Bundle installation}
%
% \paragraph{Unpacking.} Unpack the \xfile{oberdiek.tds.zip} in the
% TDS tree (also known as \xfile{texmf} tree) of your choice.
% Example (linux):
% \begin{quote}
%   |unzip oberdiek.tds.zip -d ~/texmf|
% \end{quote}
%
% \paragraph{Script installation.}
% Check the directory \xfile{TDS:scripts/oberdiek/} for
% scripts that need further installation steps.
% Package \xpackage{attachfile2} comes with the Perl script
% \xfile{pdfatfi.pl} that should be installed in such a way
% that it can be called as \texttt{pdfatfi}.
% Example (linux):
% \begin{quote}
%   |chmod +x scripts/oberdiek/pdfatfi.pl|\\
%   |cp scripts/oberdiek/pdfatfi.pl /usr/local/bin/|
% \end{quote}
%
% \subsection{Package installation}
%
% \paragraph{Unpacking.} The \xfile{.dtx} file is a self-extracting
% \docstrip\ archive. The files are extracted by running the
% \xfile{.dtx} through \plainTeX:
% \begin{quote}
%   \verb|tex iflang.dtx|
% \end{quote}
%
% \paragraph{TDS.} Now the different files must be moved into
% the different directories in your installation TDS tree
% (also known as \xfile{texmf} tree):
% \begin{quote}
% \def\t{^^A
% \begin{tabular}{@{}>{\ttfamily}l@{ $\rightarrow$ }>{\ttfamily}l@{}}
%   iflang.sty & tex/generic/oberdiek/iflang.sty\\
%   iflang.pdf & doc/latex/oberdiek/iflang.pdf\\
%   test/iflang-test1.tex & doc/latex/oberdiek/test/iflang-test1.tex\\
%   test/iflang-test2.tex & doc/latex/oberdiek/test/iflang-test2.tex\\
%   test/iflang-test3.tex & doc/latex/oberdiek/test/iflang-test3.tex\\
%   test/iflang-test4.tex & doc/latex/oberdiek/test/iflang-test4.tex\\
%   test/iflang-test5.tex & doc/latex/oberdiek/test/iflang-test5.tex\\
%   iflang.dtx & source/latex/oberdiek/iflang.dtx\\
% \end{tabular}^^A
% }^^A
% \sbox0{\t}^^A
% \ifdim\wd0>\linewidth
%   \begingroup
%     \advance\linewidth by\leftmargin
%     \advance\linewidth by\rightmargin
%   \edef\x{\endgroup
%     \def\noexpand\lw{\the\linewidth}^^A
%   }\x
%   \def\lwbox{^^A
%     \leavevmode
%     \hbox to \linewidth{^^A
%       \kern-\leftmargin\relax
%       \hss
%       \usebox0
%       \hss
%       \kern-\rightmargin\relax
%     }^^A
%   }^^A
%   \ifdim\wd0>\lw
%     \sbox0{\small\t}^^A
%     \ifdim\wd0>\linewidth
%       \ifdim\wd0>\lw
%         \sbox0{\footnotesize\t}^^A
%         \ifdim\wd0>\linewidth
%           \ifdim\wd0>\lw
%             \sbox0{\scriptsize\t}^^A
%             \ifdim\wd0>\linewidth
%               \ifdim\wd0>\lw
%                 \sbox0{\tiny\t}^^A
%                 \ifdim\wd0>\linewidth
%                   \lwbox
%                 \else
%                   \usebox0
%                 \fi
%               \else
%                 \lwbox
%               \fi
%             \else
%               \usebox0
%             \fi
%           \else
%             \lwbox
%           \fi
%         \else
%           \usebox0
%         \fi
%       \else
%         \lwbox
%       \fi
%     \else
%       \usebox0
%     \fi
%   \else
%     \lwbox
%   \fi
% \else
%   \usebox0
% \fi
% \end{quote}
% If you have a \xfile{docstrip.cfg} that configures and enables \docstrip's
% TDS installing feature, then some files can already be in the right
% place, see the documentation of \docstrip.
%
% \subsection{Refresh file name databases}
%
% If your \TeX~distribution
% (\teTeX, \mikTeX, \dots) relies on file name databases, you must refresh
% these. For example, \teTeX\ users run \verb|texhash| or
% \verb|mktexlsr|.
%
% \subsection{Some details for the interested}
%
% \paragraph{Attached source.}
%
% The PDF documentation on CTAN also includes the
% \xfile{.dtx} source file. It can be extracted by
% AcrobatReader 6 or higher. Another option is \textsf{pdftk},
% e.g. unpack the file into the current directory:
% \begin{quote}
%   \verb|pdftk iflang.pdf unpack_files output .|
% \end{quote}
%
% \paragraph{Unpacking with \LaTeX.}
% The \xfile{.dtx} chooses its action depending on the format:
% \begin{description}
% \item[\plainTeX:] Run \docstrip\ and extract the files.
% \item[\LaTeX:] Generate the documentation.
% \end{description}
% If you insist on using \LaTeX\ for \docstrip\ (really,
% \docstrip\ does not need \LaTeX), then inform the autodetect routine
% about your intention:
% \begin{quote}
%   \verb|latex \let\install=y\input{iflang.dtx}|
% \end{quote}
% Do not forget to quote the argument according to the demands
% of your shell.
%
% \paragraph{Generating the documentation.}
% You can use both the \xfile{.dtx} or the \xfile{.drv} to generate
% the documentation. The process can be configured by the
% configuration file \xfile{ltxdoc.cfg}. For instance, put this
% line into this file, if you want to have A4 as paper format:
% \begin{quote}
%   \verb|\PassOptionsToClass{a4paper}{article}|
% \end{quote}
% An example follows how to generate the
% documentation with pdf\LaTeX:
% \begin{quote}
%\begin{verbatim}
%pdflatex iflang.dtx
%makeindex -s gind.ist iflang.idx
%pdflatex iflang.dtx
%makeindex -s gind.ist iflang.idx
%pdflatex iflang.dtx
%\end{verbatim}
% \end{quote}
%
% \section{Catalogue}
%
% The following XML file can be used as source for the
% \href{http://mirror.ctan.org/help/Catalogue/catalogue.html}{\TeX\ Catalogue}.
% The elements \texttt{caption} and \texttt{description} are imported
% from the original XML file from the Catalogue.
% The name of the XML file in the Catalogue is \xfile{iflang.xml}.
%    \begin{macrocode}
%<*catalogue>
<?xml version='1.0' encoding='us-ascii'?>
<!DOCTYPE entry SYSTEM 'catalogue.dtd'>
<entry datestamp='$Date$' modifier='$Author$' id='iflang'>
  <name>iflang</name>
  <caption>Expandable checks for the current language.</caption>
  <authorref id='auth:oberdiek'/>
  <copyright owner='Heiko Oberdiek' year='2007'/>
  <license type='lppl1.3'/>
  <version number='1.6'/>
  <description>
    This package provides expandable checks for the current language
    based on macro <tt>\languagename</tt> or hyphenation patterns.
    <p/>
    The package is part of the <xref refid='oberdiek'>oberdiek</xref> bundle.
  </description>
  <documentation details='Package documentation'
      href='ctan:/macros/latex/contrib/oberdiek/iflang.pdf'/>
  <ctan file='true' path='/macros/latex/contrib/oberdiek/iflang.dtx'/>
  <miktex location='oberdiek'/>
  <texlive location='oberdiek'/>
  <install path='/macros/latex/contrib/oberdiek/oberdiek.tds.zip'/>
</entry>
%</catalogue>
%    \end{macrocode}
%
% \section{Acknowledgement}
% I wish to thank:
% \begin{description}
% \item[Markus Kohm] Useful hints for version 1.2.
% \end{description}
%
% \begin{History}
%   \begin{Version}{2007/04/10 v1.0}
%   \item
%     First public version.
%   \end{Version}
%   \begin{Version}{2007/04/11 v1.1}
%   \item
%     Line ends sanitized.
%   \end{Version}
%   \begin{Version}{2007/04/12 v1.2}
%   \item
%     Initialization of \cs{languagename} in case of \xfile{etex.src}.
%   \item
%     Some sanity tests added.
%   \item
%     Documentation improved.
%   \end{Version}
%   \begin{Version}{2007/04/26 v1.3}
%   \item
%     Use of package \xpackage{infwarerr}.
%   \end{Version}
%   \begin{Version}{2007/09/09 v1.4}
%   \item
%     Bug fix: \cs{IfLang@StrEqual} $\rightarrow$ \cs{IfLangStrEqual}
%     (Gabriele Balducci).
%   \item
%     Catcode section rewritten.
%   \end{Version}
%   \begin{Version}{2007/11/11 v1.5}
%   \item
%     Use of package \xpackage{pdftexcmds} for \LuaTeX\ support.
%   \end{Version}
%   \begin{Version}{2016/05/16 v1.6}
%   \item
%     Documentation updates.
%   \end{Version}
% \end{History}
%
% \PrintIndex
%
% \Finale
\endinput
|
% \end{quote}
% Do not forget to quote the argument according to the demands
% of your shell.
%
% \paragraph{Generating the documentation.}
% You can use both the \xfile{.dtx} or the \xfile{.drv} to generate
% the documentation. The process can be configured by the
% configuration file \xfile{ltxdoc.cfg}. For instance, put this
% line into this file, if you want to have A4 as paper format:
% \begin{quote}
%   \verb|\PassOptionsToClass{a4paper}{article}|
% \end{quote}
% An example follows how to generate the
% documentation with pdf\LaTeX:
% \begin{quote}
%\begin{verbatim}
%pdflatex iflang.dtx
%makeindex -s gind.ist iflang.idx
%pdflatex iflang.dtx
%makeindex -s gind.ist iflang.idx
%pdflatex iflang.dtx
%\end{verbatim}
% \end{quote}
%
% \section{Catalogue}
%
% The following XML file can be used as source for the
% \href{http://mirror.ctan.org/help/Catalogue/catalogue.html}{\TeX\ Catalogue}.
% The elements \texttt{caption} and \texttt{description} are imported
% from the original XML file from the Catalogue.
% The name of the XML file in the Catalogue is \xfile{iflang.xml}.
%    \begin{macrocode}
%<*catalogue>
<?xml version='1.0' encoding='us-ascii'?>
<!DOCTYPE entry SYSTEM 'catalogue.dtd'>
<entry datestamp='$Date$' modifier='$Author$' id='iflang'>
  <name>iflang</name>
  <caption>Expandable checks for the current language.</caption>
  <authorref id='auth:oberdiek'/>
  <copyright owner='Heiko Oberdiek' year='2007'/>
  <license type='lppl1.3'/>
  <version number='1.6'/>
  <description>
    This package provides expandable checks for the current language
    based on macro <tt>\languagename</tt> or hyphenation patterns.
    <p/>
    The package is part of the <xref refid='oberdiek'>oberdiek</xref> bundle.
  </description>
  <documentation details='Package documentation'
      href='ctan:/macros/latex/contrib/oberdiek/iflang.pdf'/>
  <ctan file='true' path='/macros/latex/contrib/oberdiek/iflang.dtx'/>
  <miktex location='oberdiek'/>
  <texlive location='oberdiek'/>
  <install path='/macros/latex/contrib/oberdiek/oberdiek.tds.zip'/>
</entry>
%</catalogue>
%    \end{macrocode}
%
% \section{Acknowledgement}
% I wish to thank:
% \begin{description}
% \item[Markus Kohm] Useful hints for version 1.2.
% \end{description}
%
% \begin{History}
%   \begin{Version}{2007/04/10 v1.0}
%   \item
%     First public version.
%   \end{Version}
%   \begin{Version}{2007/04/11 v1.1}
%   \item
%     Line ends sanitized.
%   \end{Version}
%   \begin{Version}{2007/04/12 v1.2}
%   \item
%     Initialization of \cs{languagename} in case of \xfile{etex.src}.
%   \item
%     Some sanity tests added.
%   \item
%     Documentation improved.
%   \end{Version}
%   \begin{Version}{2007/04/26 v1.3}
%   \item
%     Use of package \xpackage{infwarerr}.
%   \end{Version}
%   \begin{Version}{2007/09/09 v1.4}
%   \item
%     Bug fix: \cs{IfLang@StrEqual} $\rightarrow$ \cs{IfLangStrEqual}
%     (Gabriele Balducci).
%   \item
%     Catcode section rewritten.
%   \end{Version}
%   \begin{Version}{2007/11/11 v1.5}
%   \item
%     Use of package \xpackage{pdftexcmds} for \LuaTeX\ support.
%   \end{Version}
%   \begin{Version}{2016/05/16 v1.6}
%   \item
%     Documentation updates.
%   \end{Version}
% \end{History}
%
% \PrintIndex
%
% \Finale
\endinput
|
% \end{quote}
% Do not forget to quote the argument according to the demands
% of your shell.
%
% \paragraph{Generating the documentation.}
% You can use both the \xfile{.dtx} or the \xfile{.drv} to generate
% the documentation. The process can be configured by the
% configuration file \xfile{ltxdoc.cfg}. For instance, put this
% line into this file, if you want to have A4 as paper format:
% \begin{quote}
%   \verb|\PassOptionsToClass{a4paper}{article}|
% \end{quote}
% An example follows how to generate the
% documentation with pdf\LaTeX:
% \begin{quote}
%\begin{verbatim}
%pdflatex iflang.dtx
%makeindex -s gind.ist iflang.idx
%pdflatex iflang.dtx
%makeindex -s gind.ist iflang.idx
%pdflatex iflang.dtx
%\end{verbatim}
% \end{quote}
%
% \section{Catalogue}
%
% The following XML file can be used as source for the
% \href{http://mirror.ctan.org/help/Catalogue/catalogue.html}{\TeX\ Catalogue}.
% The elements \texttt{caption} and \texttt{description} are imported
% from the original XML file from the Catalogue.
% The name of the XML file in the Catalogue is \xfile{iflang.xml}.
%    \begin{macrocode}
%<*catalogue>
<?xml version='1.0' encoding='us-ascii'?>
<!DOCTYPE entry SYSTEM 'catalogue.dtd'>
<entry datestamp='$Date$' modifier='$Author$' id='iflang'>
  <name>iflang</name>
  <caption>Expandable checks for the current language.</caption>
  <authorref id='auth:oberdiek'/>
  <copyright owner='Heiko Oberdiek' year='2007'/>
  <license type='lppl1.3'/>
  <version number='1.6'/>
  <description>
    This package provides expandable checks for the current language
    based on macro <tt>\languagename</tt> or hyphenation patterns.
    <p/>
    The package is part of the <xref refid='oberdiek'>oberdiek</xref> bundle.
  </description>
  <documentation details='Package documentation'
      href='ctan:/macros/latex/contrib/oberdiek/iflang.pdf'/>
  <ctan file='true' path='/macros/latex/contrib/oberdiek/iflang.dtx'/>
  <miktex location='oberdiek'/>
  <texlive location='oberdiek'/>
  <install path='/macros/latex/contrib/oberdiek/oberdiek.tds.zip'/>
</entry>
%</catalogue>
%    \end{macrocode}
%
% \section{Acknowledgement}
% I wish to thank:
% \begin{description}
% \item[Markus Kohm] Useful hints for version 1.2.
% \end{description}
%
% \begin{History}
%   \begin{Version}{2007/04/10 v1.0}
%   \item
%     First public version.
%   \end{Version}
%   \begin{Version}{2007/04/11 v1.1}
%   \item
%     Line ends sanitized.
%   \end{Version}
%   \begin{Version}{2007/04/12 v1.2}
%   \item
%     Initialization of \cs{languagename} in case of \xfile{etex.src}.
%   \item
%     Some sanity tests added.
%   \item
%     Documentation improved.
%   \end{Version}
%   \begin{Version}{2007/04/26 v1.3}
%   \item
%     Use of package \xpackage{infwarerr}.
%   \end{Version}
%   \begin{Version}{2007/09/09 v1.4}
%   \item
%     Bug fix: \cs{IfLang@StrEqual} $\rightarrow$ \cs{IfLangStrEqual}
%     (Gabriele Balducci).
%   \item
%     Catcode section rewritten.
%   \end{Version}
%   \begin{Version}{2007/11/11 v1.5}
%   \item
%     Use of package \xpackage{pdftexcmds} for \LuaTeX\ support.
%   \end{Version}
%   \begin{Version}{2016/05/16 v1.6}
%   \item
%     Documentation updates.
%   \end{Version}
% \end{History}
%
% \PrintIndex
%
% \Finale
\endinput

%        (quote the arguments according to the demands of your shell)
%
% Documentation:
%    (a) If iflang.drv is present:
%           latex iflang.drv
%    (b) Without iflang.drv:
%           latex iflang.dtx; ...
%    The class ltxdoc loads the configuration file ltxdoc.cfg
%    if available. Here you can specify further options, e.g.
%    use A4 as paper format:
%       \PassOptionsToClass{a4paper}{article}
%
%    Programm calls to get the documentation (example):
%       pdflatex iflang.dtx
%       makeindex -s gind.ist iflang.idx
%       pdflatex iflang.dtx
%       makeindex -s gind.ist iflang.idx
%       pdflatex iflang.dtx
%
% Installation:
%    TDS:tex/generic/oberdiek/iflang.sty
%    TDS:doc/latex/oberdiek/iflang.pdf
%    TDS:doc/latex/oberdiek/test/iflang-test1.tex
%    TDS:doc/latex/oberdiek/test/iflang-test2.tex
%    TDS:doc/latex/oberdiek/test/iflang-test3.tex
%    TDS:doc/latex/oberdiek/test/iflang-test4.tex
%    TDS:doc/latex/oberdiek/test/iflang-test5.tex
%    TDS:source/latex/oberdiek/iflang.dtx
%
%<*ignore>
\begingroup
  \catcode123=1 %
  \catcode125=2 %
  \def\x{LaTeX2e}%
\expandafter\endgroup
\ifcase 0\ifx\install y1\fi\expandafter
         \ifx\csname processbatchFile\endcsname\relax\else1\fi
         \ifx\fmtname\x\else 1\fi\relax
\else\csname fi\endcsname
%</ignore>
%<*install>
\input docstrip.tex
\Msg{************************************************************************}
\Msg{* Installation}
\Msg{* Package: iflang 2016/05/16 v1.6 Checks for the current language (HO)}
\Msg{************************************************************************}

\keepsilent
\askforoverwritefalse

\let\MetaPrefix\relax
\preamble

This is a generated file.

Project: iflang
Version: 2016/05/16 v1.6

Copyright (C) 2007 by
   Heiko Oberdiek <heiko.oberdiek at googlemail.com>

This work may be distributed and/or modified under the
conditions of the LaTeX Project Public License, either
version 1.3c of this license or (at your option) any later
version. This version of this license is in
   http://www.latex-project.org/lppl/lppl-1-3c.txt
and the latest version of this license is in
   http://www.latex-project.org/lppl.txt
and version 1.3 or later is part of all distributions of
LaTeX version 2005/12/01 or later.

This work has the LPPL maintenance status "maintained".

This Current Maintainer of this work is Heiko Oberdiek.

The Base Interpreter refers to any `TeX-Format',
because some files are installed in TDS:tex/generic//.

This work consists of the main source file iflang.dtx
and the derived files
   iflang.sty, iflang.pdf, iflang.ins, iflang.drv, iflang-test1.tex,
   iflang-test2.tex, iflang-test3.tex, iflang-test4.tex,
   iflang-test5.tex.

\endpreamble
\let\MetaPrefix\DoubleperCent

\generate{%
  \file{iflang.ins}{\from{iflang.dtx}{install}}%
  \file{iflang.drv}{\from{iflang.dtx}{driver}}%
  \usedir{tex/generic/oberdiek}%
  \file{iflang.sty}{\from{iflang.dtx}{package}}%
  \usedir{doc/latex/oberdiek/test}%
  \file{iflang-test1.tex}{\from{iflang.dtx}{test1}}%
  \file{iflang-test2.tex}{\from{iflang.dtx}{test2}}%
  \file{iflang-test3.tex}{\from{iflang.dtx}{test3}}%
  \file{iflang-test4.tex}{\from{iflang.dtx}{test4}}%
  \file{iflang-test5.tex}{\from{iflang.dtx}{test5}}%
  \nopreamble
  \nopostamble
  \usedir{source/latex/oberdiek/catalogue}%
  \file{iflang.xml}{\from{iflang.dtx}{catalogue}}%
}

\catcode32=13\relax% active space
\let =\space%
\Msg{************************************************************************}
\Msg{*}
\Msg{* To finish the installation you have to move the following}
\Msg{* file into a directory searched by TeX:}
\Msg{*}
\Msg{*     iflang.sty}
\Msg{*}
\Msg{* To produce the documentation run the file `iflang.drv'}
\Msg{* through LaTeX.}
\Msg{*}
\Msg{* Happy TeXing!}
\Msg{*}
\Msg{************************************************************************}

\endbatchfile
%</install>
%<*ignore>
\fi
%</ignore>
%<*driver>
\NeedsTeXFormat{LaTeX2e}
\ProvidesFile{iflang.drv}%
  [2016/05/16 v1.6 Checks for the current language (HO)]%
\documentclass{ltxdoc}
\usepackage{holtxdoc}[2011/11/22]
\begin{document}
  \DocInput{iflang.dtx}%
\end{document}
%</driver>
% \fi
%
%
% \CharacterTable
%  {Upper-case    \A\B\C\D\E\F\G\H\I\J\K\L\M\N\O\P\Q\R\S\T\U\V\W\X\Y\Z
%   Lower-case    \a\b\c\d\e\f\g\h\i\j\k\l\m\n\o\p\q\r\s\t\u\v\w\x\y\z
%   Digits        \0\1\2\3\4\5\6\7\8\9
%   Exclamation   \!     Double quote  \"     Hash (number) \#
%   Dollar        \$     Percent       \%     Ampersand     \&
%   Acute accent  \'     Left paren    \(     Right paren   \)
%   Asterisk      \*     Plus          \+     Comma         \,
%   Minus         \-     Point         \.     Solidus       \/
%   Colon         \:     Semicolon     \;     Less than     \<
%   Equals        \=     Greater than  \>     Question mark \?
%   Commercial at \@     Left bracket  \[     Backslash     \\
%   Right bracket \]     Circumflex    \^     Underscore    \_
%   Grave accent  \`     Left brace    \{     Vertical bar  \|
%   Right brace   \}     Tilde         \~}
%
% \GetFileInfo{iflang.drv}
%
% \title{The \xpackage{iflang} package}
% \date{2016/05/16 v1.6}
% \author{Heiko Oberdiek\thanks
% {Please report any issues at https://github.com/ho-tex/oberdiek/issues}\\
% \xemail{heiko.oberdiek at googlemail.com}}
%
% \maketitle
%
% \begin{abstract}
% This package provides expandible checks for the current language
% based on macro \cs{languagename} or hyphenation patterns.
% \end{abstract}
%
% \tableofcontents
%
% \section{Documentation}
%
% Package \xpackage{babel} defines \cs{iflanguagename}. As first
% argument it takes a language name and executes the second or
% third argument depending on the current language. This
% language test is based on hypenation patterns. However, it
% is possible that different languages or dialects share the
% same patterns. In such cases \cs{iflanguagename} fails.
%
% However, package \xpackage{babel} and some other packages
% such as \xpackage{german} or \xpackage{ngerman} store
% the language name in the macro \cs{languagename} if
% \cs{selectlanguage} is called.
%
% \begin{declcs}{IfLanguageName} \M{lang} \M{then} \M{else}
% \end{declcs}
% Makro \cs{IfLanguageName} compares language \meta{lang} with
% the current setting of macro \cs{languagename}. If both
% contains the same name then the \meta{then} part is called,
% otherwise the \meta{else} part.
%
% The macro is expandable. Thus it can be safely used inside \cs{edef}
% or \cs{csname}. If case of errors like an undefined \cs{languagename}
% the \meta{else} part is executed.
%
% Note: Macro \cs{IfLanguageName} relies on the fact, that
% \cs{languagename} is set correctly:
% \begin{description}
% \item[Package \xpackage{babel}:]
%    \mbox{}\\
%    Full support of \cs{languagename} in its language switching commands.
% \item[Format based on \xpackage{babel} (\xfile{language.dat}):]
%    \mbox{}\\
%    If package \xpackage{babel} is not used (or not yet loaded),
%    then \xpackage{babel}'s
%    \xfile{hyphen.cfg} has set \cs{languagename} to the last
%    language in \xfile{language.dat}, but \cs{language} (current
%    patterns) is zero and points to the first language.
%    Thus the value of \cs{languagename} is basically garbage.
%    Package \xpackage{iflang} warns if \cs{languagename} and
%    \cs{language} do not fit.
%    This can be fixed by loading package \xpackage{babel} previously.
% \item[Format based on \eTeX's \xfile{etex.src} (\xfile{language.def}):]
%    \mbox{}\\
%    Unhappily it does not support
%    \cs{languagename}. Thus this package hooks into \cs{uselanguage}
%    to get \cs{languagename} defined and updated there.
%    At package loading time the changed \cs{uselanguage} has not
%    been called yet. Thus package \xpackage{iflang} tries
%    |USenglish|. This is the definite default language of \xfile{etex.src}.
%    If the current patterns suit this default language, an
%    undefined \cs{languagename} is set to this language. Otherwise
%    a \cs{languagename} remains undefined and a warning is given.
% \end{description}
%
% \begin{declcs}{IfLanguagePatterns} \M{lang} \M{then} \M{else}
% \end{declcs}
% This macro behaves similar to \cs{IfLanguageName}. But the
% language test is based on the current pattern in force (\cs{language}).
% Also this macro is expandable, in case of errors the \meta{else} part
% is called.
%
% The following naming convention for the pattern are supported:
% \begin{description}
% \item[\xpackage{babel}/\xfile{language.dat}]:
%    \cs{l@}\meta{language}
% \item[\xpackage{etex.src}/\xfile{language.def}]:
%    \cs{lang@}\meta{language}
% \end{description}
% Package \xpackage{iflang} looks for \cs{uselanguage} (defined in
% \xfile{etex.src}) to find out the naming convention in use.
%
% \StopEventually{
% }
%
% \section{Implementation}
%
%    \begin{macrocode}
%<*package>
%    \end{macrocode}
%
% \subsection{Reload check and package identification}
%    Reload check, especially if the package is not used with \LaTeX.
%    \begin{macrocode}
\begingroup\catcode61\catcode48\catcode32=10\relax%
  \catcode13=5 % ^^M
  \endlinechar=13 %
  \catcode35=6 % #
  \catcode39=12 % '
  \catcode44=12 % ,
  \catcode45=12 % -
  \catcode46=12 % .
  \catcode58=12 % :
  \catcode64=11 % @
  \catcode123=1 % {
  \catcode125=2 % }
  \expandafter\let\expandafter\x\csname ver@iflang.sty\endcsname
  \ifx\x\relax % plain-TeX, first loading
  \else
    \def\empty{}%
    \ifx\x\empty % LaTeX, first loading,
      % variable is initialized, but \ProvidesPackage not yet seen
    \else
      \expandafter\ifx\csname PackageInfo\endcsname\relax
        \def\x#1#2{%
          \immediate\write-1{Package #1 Info: #2.}%
        }%
      \else
        \def\x#1#2{\PackageInfo{#1}{#2, stopped}}%
      \fi
      \x{iflang}{The package is already loaded}%
      \aftergroup\endinput
    \fi
  \fi
\endgroup%
%    \end{macrocode}
%    Package identification:
%    \begin{macrocode}
\begingroup\catcode61\catcode48\catcode32=10\relax%
  \catcode13=5 % ^^M
  \endlinechar=13 %
  \catcode35=6 % #
  \catcode39=12 % '
  \catcode40=12 % (
  \catcode41=12 % )
  \catcode44=12 % ,
  \catcode45=12 % -
  \catcode46=12 % .
  \catcode47=12 % /
  \catcode58=12 % :
  \catcode64=11 % @
  \catcode91=12 % [
  \catcode93=12 % ]
  \catcode123=1 % {
  \catcode125=2 % }
  \expandafter\ifx\csname ProvidesPackage\endcsname\relax
    \def\x#1#2#3[#4]{\endgroup
      \immediate\write-1{Package: #3 #4}%
      \xdef#1{#4}%
    }%
  \else
    \def\x#1#2[#3]{\endgroup
      #2[{#3}]%
      \ifx#1\@undefined
        \xdef#1{#3}%
      \fi
      \ifx#1\relax
        \xdef#1{#3}%
      \fi
    }%
  \fi
\expandafter\x\csname ver@iflang.sty\endcsname
\ProvidesPackage{iflang}%
  [2016/05/16 v1.6 Checks for the current language (HO)]%
%    \end{macrocode}
%
%    \begin{macrocode}
\begingroup\catcode61\catcode48\catcode32=10\relax%
  \catcode13=5 % ^^M
  \endlinechar=13 %
  \catcode123=1 % {
  \catcode125=2 % }
  \catcode64=11 % @
  \def\x{\endgroup
    \expandafter\edef\csname IfLang@AtEnd\endcsname{%
      \endlinechar=\the\endlinechar\relax
      \catcode13=\the\catcode13\relax
      \catcode32=\the\catcode32\relax
      \catcode35=\the\catcode35\relax
      \catcode61=\the\catcode61\relax
      \catcode64=\the\catcode64\relax
      \catcode123=\the\catcode123\relax
      \catcode125=\the\catcode125\relax
    }%
  }%
\x\catcode61\catcode48\catcode32=10\relax%
\catcode13=5 % ^^M
\endlinechar=13 %
\catcode35=6 % #
\catcode64=11 % @
\catcode123=1 % {
\catcode125=2 % }
\def\TMP@EnsureCode#1#2{%
  \edef\IfLang@AtEnd{%
    \IfLang@AtEnd
    \catcode#1=\the\catcode#1\relax
  }%
  \catcode#1=#2\relax
}
\TMP@EnsureCode{39}{12}% '
\TMP@EnsureCode{40}{12}% (
\TMP@EnsureCode{41}{12}% )
\TMP@EnsureCode{44}{12}% ,
\TMP@EnsureCode{46}{12}% .
\TMP@EnsureCode{47}{12}% /
\TMP@EnsureCode{58}{12}% :
\TMP@EnsureCode{91}{12}% [
\TMP@EnsureCode{93}{12}% ]
\edef\IfLang@AtEnd{\IfLang@AtEnd\noexpand\endinput}
%    \end{macrocode}
%
% \subsection{Tools}
%
% \subsubsection{Provide some basic macros of \LaTeX}
%
%    \begin{macro}{\@firstoftwo}
%    \begin{macrocode}
\expandafter\ifx\csname @firstoftwo\endcsname\relax
  \long\def\@firstoftwo#1#2{#1}%
\fi
%    \end{macrocode}
%    \end{macro}
%    \begin{macro}{\@secondoftwo}
%    \begin{macrocode}
\expandafter\ifx\csname @secondoftwo\endcsname\relax
  \long\def\@secondoftwo#1#2{#2}%
\fi
%    \end{macrocode}
%    \end{macro}
%
% \subsubsection{Expandible existence check for macros}
%
%    \begin{macro}{\IfLang@IfDefined}
%    \begin{macrocode}
\begingroup\expandafter\expandafter\expandafter\endgroup
\expandafter\ifx\csname ifcsname\endcsname\relax
  \expandafter\@firstoftwo
\else
  \expandafter\@secondoftwo
\fi
{%
  \def\IfLang@IfDefined#1{%
    \expandafter\ifx\csname#1\endcsname\relax
      \expandafter\@secondoftwo
    \else
      \expandafter\@firstoftwo
    \fi
  }%
}{%
  \def\IfLang@IfDefined#1{%
    \ifnum\ifcsname#1\endcsname
            \expandafter\ifx\csname#1\endcsname\relax
              1%
            \else
              0%
            \fi
          \else
            1%
          \fi
          =0 %
      \expandafter\@firstoftwo
    \else
      \expandafter\@secondoftwo
    \fi
  }%
}
%    \end{macrocode}
%    \end{macro}
%
% \subsubsection{Macros for messages}
%
%    \begin{macrocode}
\begingroup\expandafter\expandafter\expandafter\endgroup
\expandafter\ifx\csname RequirePackage\endcsname\relax
  \input infwarerr.sty\relax
  \input pdftexcmds.sty\relax
\else
  \RequirePackage{infwarerr}[2007/09/09]%
  \RequirePackage{pdftexcmds}[2016/05/16]%
\fi
%    \end{macrocode}
%
% \subsubsection{Support for \xfile{etex.src}}
%
%    \begin{macro}{\IfLang@prefix}
%    \begin{macrocode}
\begingroup\expandafter\expandafter\expandafter\endgroup
\expandafter\ifx\csname uselanguage\endcsname\relax
  \@PackageInfoNoLine{iflang}{%
    Naming convention for patterns: babel%
  }%
  \def\IfLang@prefix{l@}%
\else
  \@PackageInfoNoLine{iflang}{%
    Naming convention for patterns: etex.src%
  }%
  \def\IfLang@prefix{lang@}%
  \let\IfLang@OrgUseLanguage\uselanguage
  \def\uselanguage#1{%
    \edef\languagename{#1}%
    \IfLang@OrgUseLanguage{#1}%
  }%
%    \end{macrocode}
%    \end{macro}
%    The first \cs{uselanguage} that is executed as last
%    line in \xfile{language.def} cannot patched this way.
%    However, \xfile{language.def} is very strict. It forces
%    the first added and used language to be |USenglish|.
%    Thus, if \cs{languagename} is not defined, we can
%    quite safely assume |USenglish|. As additional
%    safety precaution the actual used patterns are checked.
%    \begin{macrocode}
  \begingroup\expandafter\expandafter\expandafter\endgroup
  \expandafter\ifx\csname languagename\endcsname\relax
    \begingroup\expandafter\expandafter\expandafter\endgroup
    \expandafter\ifx\csname lang@USenglish\endcsname\relax
      \@PackageWarningNoLine{iflang}{%
        \string\lang@USenglish\space is missing%
      }%
    \else
      \ifnum\lang@USenglish=\language
        \def\languagename{USenglish}%
      \else
        \@PackageWarningNoLine{iflang}{%
          \string\languagename\space is not set,\MessageBreak
          current language is unknown%
        }%
      \fi
    \fi
  \fi
\fi
\begingroup\expandafter\expandafter\expandafter\endgroup
\expandafter\ifx\csname languagename\endcsname\relax
  \@PackageInfoNoLine{iflang}{%
    \string\languagename\space is not set%
  }%
\fi
%    \end{macrocode}
%
% \subsection{\cs{IfLanguagePatterns}}
%
%    \begin{macro}{\IfLanguagePatterns}
%    \begin{macrocode}
\def\IfLanguagePatterns#1{%
  \ifnum\IfLang@IfDefined{\IfLang@prefix#1}{%
          \ifnum\csname\IfLang@prefix#1\endcsname=\language
            0%
          \else
            1%
          \fi
        }{1}=0 %
    \expandafter\@firstoftwo
  \else
    \expandafter\@secondoftwo
  \fi
}
%    \end{macrocode}
%    \end{macro}
%
% \subsection{\cs{IfLanguageName}}
%
%    \begin{macrocode}
\begingroup\expandafter\expandafter\expandafter\endgroup
\expandafter\ifx\csname pdf@strcmp\endcsname\relax
  \expandafter\@firstoftwo
\else
  \expandafter\@secondoftwo
\fi
{%
%    \end{macrocode}
%    We do not have \cs{pdf@strcmp} (and \cs{pdfstrcmp}).
%    Thus we must define
%    our own expandable string comparison. The following
%    implementation is based on a \TeX\ pearl from David Kastrup, presented
%    at the conference Bacho\TeX~2005:
%    \url{http://www-stary.gust.org.pl/pearls/2005/david-kastrup/bachotex2005-david-kastrup-pearl1.pdf}
%
%    The orignal code allows macros inside the second string.
%    Because also \cs{languagename} might consists of further
%    macros, we need a variant that allows macros in the first
%    string, too.
%    \begin{macrocode}
  \def\IfLang@StrNil{\relax}%
  \def\IfLang@StrEqual#1{%
    \number\IfLang@StrEqualStart{}{}#1\IfLang@StrNil
  }%
  \def\IfLang@StrEqualStart#1#2#3{%
    \ifx#3\IfLang@StrNil
      \IfLang@StrEqualStop
    \fi
    \ifcat\noexpand#3\relax
      \IfLang@StrExpand{#1}{#2}#3%
    \fi
    \IfLang@StrEqualStart{\if#3#1}{#2\fi}%
  }%
  \def\IfLang@StrEqualStop\fi#1\IfLang@StrEqualStart#2#3#4{%
    \fi
    #2#4\relax'#313 %
  }%
  \def\IfLang@StrExpand#1#2#3\fi\IfLang@StrEqualStart#4#5{%
    \fi
    \IfLang@@StrExpand{#1}{#2}#3%
  }%
  \def\IfLang@@StrExpand#1#2#3\IfLang@StrNil{%
    \expandafter\IfLang@@@StrExpand#3\IfLang@StrNil{#1}{#2}%
  }%
  \def\IfLang@@@StrExpand#1\IfLang@StrNil#2#3{%
    \IfLang@StrEqualStart{#2}{#3}#1\IfLang@StrNil
  }%
%    \end{macrocode}
%    \begin{macro}{\IfLanguageName}
%    \begin{macrocode}
  \def\IfLanguageName#1{%
    \ifnum\IfLang@IfDefined{languagename}{%
            \if\expandafter\IfLang@StrEqual\expandafter%
                           {\languagename}{#1}%
              0%
            \else
              1%
            \fi
          }{1}=0 %
      \expandafter\@firstoftwo
    \else
      \expandafter\@secondoftwo
    \fi
  }%
%    \end{macrocode}
%    \end{macro}
%    \begin{macrocode}
}{%
%    \end{macrocode}
%    \begin{macro}{\IfLanguageName}
%    \begin{macrocode}
  \def\IfLanguageName#1{%
    \ifnum\IfLang@IfDefined{languagename}{%
            \pdf@strcmp{#1}{\languagename}%
          }{1}=0 %
      \expandafter\@firstoftwo
    \else
      \expandafter\@secondoftwo
    \fi
  }%
%    \end{macrocode}
%    \end{macro}
%    \begin{macrocode}
}
%    \end{macrocode}
%
% \subsection{Check plausibility of \cs{languagename}}
%
%    \begin{macrocode}
\begingroup\expandafter\expandafter\expandafter\endgroup
\expandafter\ifx\csname languagename\endcsname\relax
\else
  \IfLanguagePatterns{\languagename}{}{%
    \@PackageWarningNoLine{iflang}{%
      Mismatch between \string\language\space
      (patterns)\MessageBreak
      and setting of \string\languagename
    }%
  }%
\fi
%    \end{macrocode}
%
%    \begin{macrocode}
\IfLang@AtEnd%
%</package>
%    \end{macrocode}
%
% \section{Test}
%
% \subsection{Catcode checks for loading}
%
%    \begin{macrocode}
%<*test1>
%    \end{macrocode}
%    \begin{macrocode}
\catcode`\{=1 %
\catcode`\}=2 %
\catcode`\#=6 %
\catcode`\@=11 %
\expandafter\ifx\csname count@\endcsname\relax
  \countdef\count@=255 %
\fi
\expandafter\ifx\csname @gobble\endcsname\relax
  \long\def\@gobble#1{}%
\fi
\expandafter\ifx\csname @firstofone\endcsname\relax
  \long\def\@firstofone#1{#1}%
\fi
\expandafter\ifx\csname loop\endcsname\relax
  \expandafter\@firstofone
\else
  \expandafter\@gobble
\fi
{%
  \def\loop#1\repeat{%
    \def\body{#1}%
    \iterate
  }%
  \def\iterate{%
    \body
      \let\next\iterate
    \else
      \let\next\relax
    \fi
    \next
  }%
  \let\repeat=\fi
}%
\def\RestoreCatcodes{}
\count@=0 %
\loop
  \edef\RestoreCatcodes{%
    \RestoreCatcodes
    \catcode\the\count@=\the\catcode\count@\relax
  }%
\ifnum\count@<255 %
  \advance\count@ 1 %
\repeat

\def\RangeCatcodeInvalid#1#2{%
  \count@=#1\relax
  \loop
    \catcode\count@=15 %
  \ifnum\count@<#2\relax
    \advance\count@ 1 %
  \repeat
}
\def\RangeCatcodeCheck#1#2#3{%
  \count@=#1\relax
  \loop
    \ifnum#3=\catcode\count@
    \else
      \errmessage{%
        Character \the\count@\space
        with wrong catcode \the\catcode\count@\space
        instead of \number#3%
      }%
    \fi
  \ifnum\count@<#2\relax
    \advance\count@ 1 %
  \repeat
}
\def\space{ }
\expandafter\ifx\csname LoadCommand\endcsname\relax
  \def\LoadCommand{\input iflang.sty\relax}%
\fi
\def\Test{%
  \RangeCatcodeInvalid{0}{47}%
  \RangeCatcodeInvalid{58}{64}%
  \RangeCatcodeInvalid{91}{96}%
  \RangeCatcodeInvalid{123}{255}%
  \catcode`\@=12 %
  \catcode`\\=0 %
  \catcode`\%=14 %
  \LoadCommand
  \RangeCatcodeCheck{0}{36}{15}%
  \RangeCatcodeCheck{37}{37}{14}%
  \RangeCatcodeCheck{38}{47}{15}%
  \RangeCatcodeCheck{48}{57}{12}%
  \RangeCatcodeCheck{58}{63}{15}%
  \RangeCatcodeCheck{64}{64}{12}%
  \RangeCatcodeCheck{65}{90}{11}%
  \RangeCatcodeCheck{91}{91}{15}%
  \RangeCatcodeCheck{92}{92}{0}%
  \RangeCatcodeCheck{93}{96}{15}%
  \RangeCatcodeCheck{97}{122}{11}%
  \RangeCatcodeCheck{123}{255}{15}%
  \RestoreCatcodes
}
\Test
\csname @@end\endcsname
\end
%    \end{macrocode}
%    \begin{macrocode}
%</test1>
%    \end{macrocode}
%
% \subsection{Test with \LaTeX}
%
%    \begin{macrocode}
%<*test2|test3>
%    \end{macrocode}
%    \begin{macrocode}
\NeedsTeXFormat{LaTeX2e}
%<test3>\let\pdfstrcmp\relax
\nofiles
\documentclass{minimal}
\usepackage{qstest}
\IncludeTests{*}
\LogTests{log}{*}{*}
\usepackage[english,naustrian,ngerman]{babel}
\usepackage{iflang}
\begin{document}
\begin{qstest}{IfLanguagePatterns}{language, pattern}
  \def\test#1#2{%
    \Expect*{\IfLanguagePatterns{#1}{true}{false}}{#2}%
  }%
  \test{ngerman}{true}%
  \test{naustrian}{true}%
  \test{english}{false}%
  \test{foobar}{false}%
\end{qstest}
\begin{qstest}{IfLanguageName}{language, name}
  \def\test#1#2{%
    \Expect*{\IfLanguageName{#1}{true}{false}}{#2}%
  }%
  \test{ngerman}{true}%
  \test{naustrian}{false}%
  \selectlanguage{naustrian}%
  \test{ngerman}{false}%
  \test{naustrian}{true}%
  \test{foobar}{false}%
  %
  \def\languagename{naustrian}%
  \test{naustrian}{true}%
  \test{ngerman}{false}%
  %
  \edef\languagename{\string naustrian}%
  \test{naustrian}{true}%
  \test{ngerman}{false}%
  %
  \def\languagename{naustrian}%
  \makeatletter
  \@onelevel@sanitize\languagename
  \test{naustrian}{true}%
  \test{ngerman}{false}%
  %
  \def\languagename{naustrian}%
  \def\xaustrian{naustrian}%
  \def\xgerman{ngerman}%
  \test{\xaustrian}{true}%
  \test{\xgerman}{false}%
  %
  \def\languagename{\xaustrian}%
  \test{naustrian}{true}%
  \test{ngerman}{false}%
  \test{\xaustrian}{true}%
  \test{\xgerman}{false}%
  \test{\languagename}{true}%
  \test{\languagename\space}{false}%
  %
  \def\languagename{\empty\xaustrian\empty}%
  \test{naustrian}{true}%
  \test{ngerman}{false}%
  \test{\empty\xaustrian\empty}{true}%
  \test{\empty\xgerman\empty}{false}%
\end{qstest}
\begin{qstest}{IfDefined}{defined}
  \makeatletter
  \let\foobar\relax
  \Expect*{\IfLang@IfDefined{foobar}{true}{false}}{false}%
  \Expect*{\ifx\foobar\relax true\else false\fi}{true}%
  \let\foobar\UNDEFINED
  \Expect*{\IfLang@IfDefined{foobar}{true}{false}}{false}%
  \Expect*{\ifx\foobar\relax true\else false\fi}{false}%
  \Expect*{\ifx\foobar\UNDEFINED true\else false\fi}{true}%
\end{qstest}
\end{document}
%    \end{macrocode}
%    \begin{macrocode}
%</test2|test3>
%    \end{macrocode}
%
% \subsection{Test with \plainTeX\ and \eTeX}
%
%    \begin{macrocode}
%<*test4>
%    \end{macrocode}
%    \begin{macrocode}
%% Format `etex' based on `language.def'
\input iflang.sty
\catcode64=12

\def\TestGeneric#1#2#3{%
  \begingroup
    \edef\x{#1{#2}{true}{false}}%
    \edef\y{#3}%
    \ifx\x\y
    \else
      \errmessage{Failed test: \string#1{#2} <> #3}%
    \fi
  \endgroup
}
\def\TestPatterns{\TestGeneric\IfLanguagePatterns}
\def\TestName{\TestGeneric\IfLanguageName}

\TestPatterns{USenglish}{true}
\TestPatterns{ngerman}{false}

\TestName{USenglish}{true}
\TestName{ngerman}{false}

\uselanguage{ngerman}

\TestPatterns{USenglish}{false}
\TestPatterns{ngerman}{true}

\TestName{USenglish}{false}
\TestName{ngerman}{true}

\csname @@end\endcsname
\end
%    \end{macrocode}
%    \begin{macrocode}
%</test4>
%    \end{macrocode}
%
% \subsection{Test with \plainTeX\ and without \eTeX/\pdfTeX}
%
%    \begin{macrocode}
%<*test5>
%    \end{macrocode}
%    \begin{macrocode}
%% Format `tex' (vanilla plain-TeX)
\let\ifcsname\UNDEFINED
\let\pdfstrcmp\UNDEFINED
\input iflang.sty
\catcode64=11

\def\TestDefined#1{%
  \IfLang@IfDefined{foobar}{}{}%
  \ifx\foobar#1%
  \else
    \errmessage{Failed test: \string\foobar <> \string#1}%
  \fi
}
\let\foobar\relax
\TestDefined\relax
\let\foobar\UNDEFINED
\TestDefined\relax

\def\strip@prefix#1>{}
\def\@onelevel@sanitize#1{%
  \edef#1{\expandafter\strip@prefix\meaning#1}%
}
\def\TestCompare#1#2#3{%
  \begingroup
    \edef\x{%
      \if\IfLang@StrEqual{#1}{#2}%
        true%
      \else
        false%
      \fi
    }%
    \def\expect{#3}%
    \ifx\x\expect
    \else
      \def\a{#1}%
      \@onelevel@sanitize\a
      \def\b{#2}%
      \@onelevel@sanitize\b
      \errmessage{Failed test: `\a'='\b' <> \expect}%
    \fi
  \endgroup
}
\TestCompare{junk}{junk}{true}
\TestCompare{}{}{true}
\TestCompare{a}{b}{false}
\TestCompare{aa}{bb}{false}
\def\a{ax}
\def\b{bx}
\def\c{\a\b}
\def\d{\c\b}
\def\exch#1#2{#2#1}
\def\gobble#1{}
\TestCompare{\gobble a}{}{true}
\TestCompare{}{\gobble a}{true}
\TestCompare{a\exch xyb}{ayxb}{true}
\TestCompare{\c}{\c}{true}
\TestCompare{\d}{\c\b}{true}

\csname @@end\endcsname
\end
%    \end{macrocode}
%    \begin{macrocode}
%</test5>
%    \end{macrocode}
%
% \section{Installation}
%
% \subsection{Download}
%
% \paragraph{Package.} This package is available on
% CTAN\footnote{\url{http://ctan.org/pkg/iflang}}:
% \begin{description}
% \item[\CTAN{macros/latex/contrib/oberdiek/iflang.dtx}] The source file.
% \item[\CTAN{macros/latex/contrib/oberdiek/iflang.pdf}] Documentation.
% \end{description}
%
%
% \paragraph{Bundle.} All the packages of the bundle `oberdiek'
% are also available in a TDS compliant ZIP archive. There
% the packages are already unpacked and the documentation files
% are generated. The files and directories obey the TDS standard.
% \begin{description}
% \item[\CTAN{install/macros/latex/contrib/oberdiek.tds.zip}]
% \end{description}
% \emph{TDS} refers to the standard ``A Directory Structure
% for \TeX\ Files'' (\CTAN{tds/tds.pdf}). Directories
% with \xfile{texmf} in their name are usually organized this way.
%
% \subsection{Bundle installation}
%
% \paragraph{Unpacking.} Unpack the \xfile{oberdiek.tds.zip} in the
% TDS tree (also known as \xfile{texmf} tree) of your choice.
% Example (linux):
% \begin{quote}
%   |unzip oberdiek.tds.zip -d ~/texmf|
% \end{quote}
%
% \paragraph{Script installation.}
% Check the directory \xfile{TDS:scripts/oberdiek/} for
% scripts that need further installation steps.
% Package \xpackage{attachfile2} comes with the Perl script
% \xfile{pdfatfi.pl} that should be installed in such a way
% that it can be called as \texttt{pdfatfi}.
% Example (linux):
% \begin{quote}
%   |chmod +x scripts/oberdiek/pdfatfi.pl|\\
%   |cp scripts/oberdiek/pdfatfi.pl /usr/local/bin/|
% \end{quote}
%
% \subsection{Package installation}
%
% \paragraph{Unpacking.} The \xfile{.dtx} file is a self-extracting
% \docstrip\ archive. The files are extracted by running the
% \xfile{.dtx} through \plainTeX:
% \begin{quote}
%   \verb|tex iflang.dtx|
% \end{quote}
%
% \paragraph{TDS.} Now the different files must be moved into
% the different directories in your installation TDS tree
% (also known as \xfile{texmf} tree):
% \begin{quote}
% \def\t{^^A
% \begin{tabular}{@{}>{\ttfamily}l@{ $\rightarrow$ }>{\ttfamily}l@{}}
%   iflang.sty & tex/generic/oberdiek/iflang.sty\\
%   iflang.pdf & doc/latex/oberdiek/iflang.pdf\\
%   test/iflang-test1.tex & doc/latex/oberdiek/test/iflang-test1.tex\\
%   test/iflang-test2.tex & doc/latex/oberdiek/test/iflang-test2.tex\\
%   test/iflang-test3.tex & doc/latex/oberdiek/test/iflang-test3.tex\\
%   test/iflang-test4.tex & doc/latex/oberdiek/test/iflang-test4.tex\\
%   test/iflang-test5.tex & doc/latex/oberdiek/test/iflang-test5.tex\\
%   iflang.dtx & source/latex/oberdiek/iflang.dtx\\
% \end{tabular}^^A
% }^^A
% \sbox0{\t}^^A
% \ifdim\wd0>\linewidth
%   \begingroup
%     \advance\linewidth by\leftmargin
%     \advance\linewidth by\rightmargin
%   \edef\x{\endgroup
%     \def\noexpand\lw{\the\linewidth}^^A
%   }\x
%   \def\lwbox{^^A
%     \leavevmode
%     \hbox to \linewidth{^^A
%       \kern-\leftmargin\relax
%       \hss
%       \usebox0
%       \hss
%       \kern-\rightmargin\relax
%     }^^A
%   }^^A
%   \ifdim\wd0>\lw
%     \sbox0{\small\t}^^A
%     \ifdim\wd0>\linewidth
%       \ifdim\wd0>\lw
%         \sbox0{\footnotesize\t}^^A
%         \ifdim\wd0>\linewidth
%           \ifdim\wd0>\lw
%             \sbox0{\scriptsize\t}^^A
%             \ifdim\wd0>\linewidth
%               \ifdim\wd0>\lw
%                 \sbox0{\tiny\t}^^A
%                 \ifdim\wd0>\linewidth
%                   \lwbox
%                 \else
%                   \usebox0
%                 \fi
%               \else
%                 \lwbox
%               \fi
%             \else
%               \usebox0
%             \fi
%           \else
%             \lwbox
%           \fi
%         \else
%           \usebox0
%         \fi
%       \else
%         \lwbox
%       \fi
%     \else
%       \usebox0
%     \fi
%   \else
%     \lwbox
%   \fi
% \else
%   \usebox0
% \fi
% \end{quote}
% If you have a \xfile{docstrip.cfg} that configures and enables \docstrip's
% TDS installing feature, then some files can already be in the right
% place, see the documentation of \docstrip.
%
% \subsection{Refresh file name databases}
%
% If your \TeX~distribution
% (\teTeX, \mikTeX, \dots) relies on file name databases, you must refresh
% these. For example, \teTeX\ users run \verb|texhash| or
% \verb|mktexlsr|.
%
% \subsection{Some details for the interested}
%
% \paragraph{Attached source.}
%
% The PDF documentation on CTAN also includes the
% \xfile{.dtx} source file. It can be extracted by
% AcrobatReader 6 or higher. Another option is \textsf{pdftk},
% e.g. unpack the file into the current directory:
% \begin{quote}
%   \verb|pdftk iflang.pdf unpack_files output .|
% \end{quote}
%
% \paragraph{Unpacking with \LaTeX.}
% The \xfile{.dtx} chooses its action depending on the format:
% \begin{description}
% \item[\plainTeX:] Run \docstrip\ and extract the files.
% \item[\LaTeX:] Generate the documentation.
% \end{description}
% If you insist on using \LaTeX\ for \docstrip\ (really,
% \docstrip\ does not need \LaTeX), then inform the autodetect routine
% about your intention:
% \begin{quote}
%   \verb|latex \let\install=y% \iffalse meta-comment
%
% File: iflang.dtx
% Version: 2016/05/16 v1.6
% Info: Checks for the current language
%
% Copyright (C) 2007 by
%    Heiko Oberdiek <heiko.oberdiek at googlemail.com>
%    2016
%    https://github.com/ho-tex/oberdiek/issues
%
% This work may be distributed and/or modified under the
% conditions of the LaTeX Project Public License, either
% version 1.3c of this license or (at your option) any later
% version. This version of this license is in
%    http://www.latex-project.org/lppl/lppl-1-3c.txt
% and the latest version of this license is in
%    http://www.latex-project.org/lppl.txt
% and version 1.3 or later is part of all distributions of
% LaTeX version 2005/12/01 or later.
%
% This work has the LPPL maintenance status "maintained".
%
% This Current Maintainer of this work is Heiko Oberdiek.
%
% The Base Interpreter refers to any `TeX-Format',
% because some files are installed in TDS:tex/generic//.
%
% This work consists of the main source file iflang.dtx
% and the derived files
%    iflang.sty, iflang.pdf, iflang.ins, iflang.drv, iflang-test1.tex,
%    iflang-test2.tex, iflang-test3.tex, iflang-test4.tex,
%    iflang-test5.tex.
%
% Distribution:
%    CTAN:macros/latex/contrib/oberdiek/iflang.dtx
%    CTAN:macros/latex/contrib/oberdiek/iflang.pdf
%
% Unpacking:
%    (a) If iflang.ins is present:
%           tex iflang.ins
%    (b) Without iflang.ins:
%           tex iflang.dtx
%    (c) If you insist on using LaTeX
%           latex \let\install=y% \iffalse meta-comment
%
% File: iflang.dtx
% Version: 2016/05/16 v1.6
% Info: Checks for the current language
%
% Copyright (C) 2007 by
%    Heiko Oberdiek <heiko.oberdiek at googlemail.com>
%    2016
%    https://github.com/ho-tex/oberdiek/issues
%
% This work may be distributed and/or modified under the
% conditions of the LaTeX Project Public License, either
% version 1.3c of this license or (at your option) any later
% version. This version of this license is in
%    http://www.latex-project.org/lppl/lppl-1-3c.txt
% and the latest version of this license is in
%    http://www.latex-project.org/lppl.txt
% and version 1.3 or later is part of all distributions of
% LaTeX version 2005/12/01 or later.
%
% This work has the LPPL maintenance status "maintained".
%
% This Current Maintainer of this work is Heiko Oberdiek.
%
% The Base Interpreter refers to any `TeX-Format',
% because some files are installed in TDS:tex/generic//.
%
% This work consists of the main source file iflang.dtx
% and the derived files
%    iflang.sty, iflang.pdf, iflang.ins, iflang.drv, iflang-test1.tex,
%    iflang-test2.tex, iflang-test3.tex, iflang-test4.tex,
%    iflang-test5.tex.
%
% Distribution:
%    CTAN:macros/latex/contrib/oberdiek/iflang.dtx
%    CTAN:macros/latex/contrib/oberdiek/iflang.pdf
%
% Unpacking:
%    (a) If iflang.ins is present:
%           tex iflang.ins
%    (b) Without iflang.ins:
%           tex iflang.dtx
%    (c) If you insist on using LaTeX
%           latex \let\install=y% \iffalse meta-comment
%
% File: iflang.dtx
% Version: 2016/05/16 v1.6
% Info: Checks for the current language
%
% Copyright (C) 2007 by
%    Heiko Oberdiek <heiko.oberdiek at googlemail.com>
%    2016
%    https://github.com/ho-tex/oberdiek/issues
%
% This work may be distributed and/or modified under the
% conditions of the LaTeX Project Public License, either
% version 1.3c of this license or (at your option) any later
% version. This version of this license is in
%    http://www.latex-project.org/lppl/lppl-1-3c.txt
% and the latest version of this license is in
%    http://www.latex-project.org/lppl.txt
% and version 1.3 or later is part of all distributions of
% LaTeX version 2005/12/01 or later.
%
% This work has the LPPL maintenance status "maintained".
%
% This Current Maintainer of this work is Heiko Oberdiek.
%
% The Base Interpreter refers to any `TeX-Format',
% because some files are installed in TDS:tex/generic//.
%
% This work consists of the main source file iflang.dtx
% and the derived files
%    iflang.sty, iflang.pdf, iflang.ins, iflang.drv, iflang-test1.tex,
%    iflang-test2.tex, iflang-test3.tex, iflang-test4.tex,
%    iflang-test5.tex.
%
% Distribution:
%    CTAN:macros/latex/contrib/oberdiek/iflang.dtx
%    CTAN:macros/latex/contrib/oberdiek/iflang.pdf
%
% Unpacking:
%    (a) If iflang.ins is present:
%           tex iflang.ins
%    (b) Without iflang.ins:
%           tex iflang.dtx
%    (c) If you insist on using LaTeX
%           latex \let\install=y\input{iflang.dtx}
%        (quote the arguments according to the demands of your shell)
%
% Documentation:
%    (a) If iflang.drv is present:
%           latex iflang.drv
%    (b) Without iflang.drv:
%           latex iflang.dtx; ...
%    The class ltxdoc loads the configuration file ltxdoc.cfg
%    if available. Here you can specify further options, e.g.
%    use A4 as paper format:
%       \PassOptionsToClass{a4paper}{article}
%
%    Programm calls to get the documentation (example):
%       pdflatex iflang.dtx
%       makeindex -s gind.ist iflang.idx
%       pdflatex iflang.dtx
%       makeindex -s gind.ist iflang.idx
%       pdflatex iflang.dtx
%
% Installation:
%    TDS:tex/generic/oberdiek/iflang.sty
%    TDS:doc/latex/oberdiek/iflang.pdf
%    TDS:doc/latex/oberdiek/test/iflang-test1.tex
%    TDS:doc/latex/oberdiek/test/iflang-test2.tex
%    TDS:doc/latex/oberdiek/test/iflang-test3.tex
%    TDS:doc/latex/oberdiek/test/iflang-test4.tex
%    TDS:doc/latex/oberdiek/test/iflang-test5.tex
%    TDS:source/latex/oberdiek/iflang.dtx
%
%<*ignore>
\begingroup
  \catcode123=1 %
  \catcode125=2 %
  \def\x{LaTeX2e}%
\expandafter\endgroup
\ifcase 0\ifx\install y1\fi\expandafter
         \ifx\csname processbatchFile\endcsname\relax\else1\fi
         \ifx\fmtname\x\else 1\fi\relax
\else\csname fi\endcsname
%</ignore>
%<*install>
\input docstrip.tex
\Msg{************************************************************************}
\Msg{* Installation}
\Msg{* Package: iflang 2016/05/16 v1.6 Checks for the current language (HO)}
\Msg{************************************************************************}

\keepsilent
\askforoverwritefalse

\let\MetaPrefix\relax
\preamble

This is a generated file.

Project: iflang
Version: 2016/05/16 v1.6

Copyright (C) 2007 by
   Heiko Oberdiek <heiko.oberdiek at googlemail.com>

This work may be distributed and/or modified under the
conditions of the LaTeX Project Public License, either
version 1.3c of this license or (at your option) any later
version. This version of this license is in
   http://www.latex-project.org/lppl/lppl-1-3c.txt
and the latest version of this license is in
   http://www.latex-project.org/lppl.txt
and version 1.3 or later is part of all distributions of
LaTeX version 2005/12/01 or later.

This work has the LPPL maintenance status "maintained".

This Current Maintainer of this work is Heiko Oberdiek.

The Base Interpreter refers to any `TeX-Format',
because some files are installed in TDS:tex/generic//.

This work consists of the main source file iflang.dtx
and the derived files
   iflang.sty, iflang.pdf, iflang.ins, iflang.drv, iflang-test1.tex,
   iflang-test2.tex, iflang-test3.tex, iflang-test4.tex,
   iflang-test5.tex.

\endpreamble
\let\MetaPrefix\DoubleperCent

\generate{%
  \file{iflang.ins}{\from{iflang.dtx}{install}}%
  \file{iflang.drv}{\from{iflang.dtx}{driver}}%
  \usedir{tex/generic/oberdiek}%
  \file{iflang.sty}{\from{iflang.dtx}{package}}%
  \usedir{doc/latex/oberdiek/test}%
  \file{iflang-test1.tex}{\from{iflang.dtx}{test1}}%
  \file{iflang-test2.tex}{\from{iflang.dtx}{test2}}%
  \file{iflang-test3.tex}{\from{iflang.dtx}{test3}}%
  \file{iflang-test4.tex}{\from{iflang.dtx}{test4}}%
  \file{iflang-test5.tex}{\from{iflang.dtx}{test5}}%
  \nopreamble
  \nopostamble
  \usedir{source/latex/oberdiek/catalogue}%
  \file{iflang.xml}{\from{iflang.dtx}{catalogue}}%
}

\catcode32=13\relax% active space
\let =\space%
\Msg{************************************************************************}
\Msg{*}
\Msg{* To finish the installation you have to move the following}
\Msg{* file into a directory searched by TeX:}
\Msg{*}
\Msg{*     iflang.sty}
\Msg{*}
\Msg{* To produce the documentation run the file `iflang.drv'}
\Msg{* through LaTeX.}
\Msg{*}
\Msg{* Happy TeXing!}
\Msg{*}
\Msg{************************************************************************}

\endbatchfile
%</install>
%<*ignore>
\fi
%</ignore>
%<*driver>
\NeedsTeXFormat{LaTeX2e}
\ProvidesFile{iflang.drv}%
  [2016/05/16 v1.6 Checks for the current language (HO)]%
\documentclass{ltxdoc}
\usepackage{holtxdoc}[2011/11/22]
\begin{document}
  \DocInput{iflang.dtx}%
\end{document}
%</driver>
% \fi
%
%
% \CharacterTable
%  {Upper-case    \A\B\C\D\E\F\G\H\I\J\K\L\M\N\O\P\Q\R\S\T\U\V\W\X\Y\Z
%   Lower-case    \a\b\c\d\e\f\g\h\i\j\k\l\m\n\o\p\q\r\s\t\u\v\w\x\y\z
%   Digits        \0\1\2\3\4\5\6\7\8\9
%   Exclamation   \!     Double quote  \"     Hash (number) \#
%   Dollar        \$     Percent       \%     Ampersand     \&
%   Acute accent  \'     Left paren    \(     Right paren   \)
%   Asterisk      \*     Plus          \+     Comma         \,
%   Minus         \-     Point         \.     Solidus       \/
%   Colon         \:     Semicolon     \;     Less than     \<
%   Equals        \=     Greater than  \>     Question mark \?
%   Commercial at \@     Left bracket  \[     Backslash     \\
%   Right bracket \]     Circumflex    \^     Underscore    \_
%   Grave accent  \`     Left brace    \{     Vertical bar  \|
%   Right brace   \}     Tilde         \~}
%
% \GetFileInfo{iflang.drv}
%
% \title{The \xpackage{iflang} package}
% \date{2016/05/16 v1.6}
% \author{Heiko Oberdiek\thanks
% {Please report any issues at https://github.com/ho-tex/oberdiek/issues}\\
% \xemail{heiko.oberdiek at googlemail.com}}
%
% \maketitle
%
% \begin{abstract}
% This package provides expandible checks for the current language
% based on macro \cs{languagename} or hyphenation patterns.
% \end{abstract}
%
% \tableofcontents
%
% \section{Documentation}
%
% Package \xpackage{babel} defines \cs{iflanguagename}. As first
% argument it takes a language name and executes the second or
% third argument depending on the current language. This
% language test is based on hypenation patterns. However, it
% is possible that different languages or dialects share the
% same patterns. In such cases \cs{iflanguagename} fails.
%
% However, package \xpackage{babel} and some other packages
% such as \xpackage{german} or \xpackage{ngerman} store
% the language name in the macro \cs{languagename} if
% \cs{selectlanguage} is called.
%
% \begin{declcs}{IfLanguageName} \M{lang} \M{then} \M{else}
% \end{declcs}
% Makro \cs{IfLanguageName} compares language \meta{lang} with
% the current setting of macro \cs{languagename}. If both
% contains the same name then the \meta{then} part is called,
% otherwise the \meta{else} part.
%
% The macro is expandable. Thus it can be safely used inside \cs{edef}
% or \cs{csname}. If case of errors like an undefined \cs{languagename}
% the \meta{else} part is executed.
%
% Note: Macro \cs{IfLanguageName} relies on the fact, that
% \cs{languagename} is set correctly:
% \begin{description}
% \item[Package \xpackage{babel}:]
%    \mbox{}\\
%    Full support of \cs{languagename} in its language switching commands.
% \item[Format based on \xpackage{babel} (\xfile{language.dat}):]
%    \mbox{}\\
%    If package \xpackage{babel} is not used (or not yet loaded),
%    then \xpackage{babel}'s
%    \xfile{hyphen.cfg} has set \cs{languagename} to the last
%    language in \xfile{language.dat}, but \cs{language} (current
%    patterns) is zero and points to the first language.
%    Thus the value of \cs{languagename} is basically garbage.
%    Package \xpackage{iflang} warns if \cs{languagename} and
%    \cs{language} do not fit.
%    This can be fixed by loading package \xpackage{babel} previously.
% \item[Format based on \eTeX's \xfile{etex.src} (\xfile{language.def}):]
%    \mbox{}\\
%    Unhappily it does not support
%    \cs{languagename}. Thus this package hooks into \cs{uselanguage}
%    to get \cs{languagename} defined and updated there.
%    At package loading time the changed \cs{uselanguage} has not
%    been called yet. Thus package \xpackage{iflang} tries
%    |USenglish|. This is the definite default language of \xfile{etex.src}.
%    If the current patterns suit this default language, an
%    undefined \cs{languagename} is set to this language. Otherwise
%    a \cs{languagename} remains undefined and a warning is given.
% \end{description}
%
% \begin{declcs}{IfLanguagePatterns} \M{lang} \M{then} \M{else}
% \end{declcs}
% This macro behaves similar to \cs{IfLanguageName}. But the
% language test is based on the current pattern in force (\cs{language}).
% Also this macro is expandable, in case of errors the \meta{else} part
% is called.
%
% The following naming convention for the pattern are supported:
% \begin{description}
% \item[\xpackage{babel}/\xfile{language.dat}]:
%    \cs{l@}\meta{language}
% \item[\xpackage{etex.src}/\xfile{language.def}]:
%    \cs{lang@}\meta{language}
% \end{description}
% Package \xpackage{iflang} looks for \cs{uselanguage} (defined in
% \xfile{etex.src}) to find out the naming convention in use.
%
% \StopEventually{
% }
%
% \section{Implementation}
%
%    \begin{macrocode}
%<*package>
%    \end{macrocode}
%
% \subsection{Reload check and package identification}
%    Reload check, especially if the package is not used with \LaTeX.
%    \begin{macrocode}
\begingroup\catcode61\catcode48\catcode32=10\relax%
  \catcode13=5 % ^^M
  \endlinechar=13 %
  \catcode35=6 % #
  \catcode39=12 % '
  \catcode44=12 % ,
  \catcode45=12 % -
  \catcode46=12 % .
  \catcode58=12 % :
  \catcode64=11 % @
  \catcode123=1 % {
  \catcode125=2 % }
  \expandafter\let\expandafter\x\csname ver@iflang.sty\endcsname
  \ifx\x\relax % plain-TeX, first loading
  \else
    \def\empty{}%
    \ifx\x\empty % LaTeX, first loading,
      % variable is initialized, but \ProvidesPackage not yet seen
    \else
      \expandafter\ifx\csname PackageInfo\endcsname\relax
        \def\x#1#2{%
          \immediate\write-1{Package #1 Info: #2.}%
        }%
      \else
        \def\x#1#2{\PackageInfo{#1}{#2, stopped}}%
      \fi
      \x{iflang}{The package is already loaded}%
      \aftergroup\endinput
    \fi
  \fi
\endgroup%
%    \end{macrocode}
%    Package identification:
%    \begin{macrocode}
\begingroup\catcode61\catcode48\catcode32=10\relax%
  \catcode13=5 % ^^M
  \endlinechar=13 %
  \catcode35=6 % #
  \catcode39=12 % '
  \catcode40=12 % (
  \catcode41=12 % )
  \catcode44=12 % ,
  \catcode45=12 % -
  \catcode46=12 % .
  \catcode47=12 % /
  \catcode58=12 % :
  \catcode64=11 % @
  \catcode91=12 % [
  \catcode93=12 % ]
  \catcode123=1 % {
  \catcode125=2 % }
  \expandafter\ifx\csname ProvidesPackage\endcsname\relax
    \def\x#1#2#3[#4]{\endgroup
      \immediate\write-1{Package: #3 #4}%
      \xdef#1{#4}%
    }%
  \else
    \def\x#1#2[#3]{\endgroup
      #2[{#3}]%
      \ifx#1\@undefined
        \xdef#1{#3}%
      \fi
      \ifx#1\relax
        \xdef#1{#3}%
      \fi
    }%
  \fi
\expandafter\x\csname ver@iflang.sty\endcsname
\ProvidesPackage{iflang}%
  [2016/05/16 v1.6 Checks for the current language (HO)]%
%    \end{macrocode}
%
%    \begin{macrocode}
\begingroup\catcode61\catcode48\catcode32=10\relax%
  \catcode13=5 % ^^M
  \endlinechar=13 %
  \catcode123=1 % {
  \catcode125=2 % }
  \catcode64=11 % @
  \def\x{\endgroup
    \expandafter\edef\csname IfLang@AtEnd\endcsname{%
      \endlinechar=\the\endlinechar\relax
      \catcode13=\the\catcode13\relax
      \catcode32=\the\catcode32\relax
      \catcode35=\the\catcode35\relax
      \catcode61=\the\catcode61\relax
      \catcode64=\the\catcode64\relax
      \catcode123=\the\catcode123\relax
      \catcode125=\the\catcode125\relax
    }%
  }%
\x\catcode61\catcode48\catcode32=10\relax%
\catcode13=5 % ^^M
\endlinechar=13 %
\catcode35=6 % #
\catcode64=11 % @
\catcode123=1 % {
\catcode125=2 % }
\def\TMP@EnsureCode#1#2{%
  \edef\IfLang@AtEnd{%
    \IfLang@AtEnd
    \catcode#1=\the\catcode#1\relax
  }%
  \catcode#1=#2\relax
}
\TMP@EnsureCode{39}{12}% '
\TMP@EnsureCode{40}{12}% (
\TMP@EnsureCode{41}{12}% )
\TMP@EnsureCode{44}{12}% ,
\TMP@EnsureCode{46}{12}% .
\TMP@EnsureCode{47}{12}% /
\TMP@EnsureCode{58}{12}% :
\TMP@EnsureCode{91}{12}% [
\TMP@EnsureCode{93}{12}% ]
\edef\IfLang@AtEnd{\IfLang@AtEnd\noexpand\endinput}
%    \end{macrocode}
%
% \subsection{Tools}
%
% \subsubsection{Provide some basic macros of \LaTeX}
%
%    \begin{macro}{\@firstoftwo}
%    \begin{macrocode}
\expandafter\ifx\csname @firstoftwo\endcsname\relax
  \long\def\@firstoftwo#1#2{#1}%
\fi
%    \end{macrocode}
%    \end{macro}
%    \begin{macro}{\@secondoftwo}
%    \begin{macrocode}
\expandafter\ifx\csname @secondoftwo\endcsname\relax
  \long\def\@secondoftwo#1#2{#2}%
\fi
%    \end{macrocode}
%    \end{macro}
%
% \subsubsection{Expandible existence check for macros}
%
%    \begin{macro}{\IfLang@IfDefined}
%    \begin{macrocode}
\begingroup\expandafter\expandafter\expandafter\endgroup
\expandafter\ifx\csname ifcsname\endcsname\relax
  \expandafter\@firstoftwo
\else
  \expandafter\@secondoftwo
\fi
{%
  \def\IfLang@IfDefined#1{%
    \expandafter\ifx\csname#1\endcsname\relax
      \expandafter\@secondoftwo
    \else
      \expandafter\@firstoftwo
    \fi
  }%
}{%
  \def\IfLang@IfDefined#1{%
    \ifnum\ifcsname#1\endcsname
            \expandafter\ifx\csname#1\endcsname\relax
              1%
            \else
              0%
            \fi
          \else
            1%
          \fi
          =0 %
      \expandafter\@firstoftwo
    \else
      \expandafter\@secondoftwo
    \fi
  }%
}
%    \end{macrocode}
%    \end{macro}
%
% \subsubsection{Macros for messages}
%
%    \begin{macrocode}
\begingroup\expandafter\expandafter\expandafter\endgroup
\expandafter\ifx\csname RequirePackage\endcsname\relax
  \input infwarerr.sty\relax
  \input pdftexcmds.sty\relax
\else
  \RequirePackage{infwarerr}[2007/09/09]%
  \RequirePackage{pdftexcmds}[2016/05/16]%
\fi
%    \end{macrocode}
%
% \subsubsection{Support for \xfile{etex.src}}
%
%    \begin{macro}{\IfLang@prefix}
%    \begin{macrocode}
\begingroup\expandafter\expandafter\expandafter\endgroup
\expandafter\ifx\csname uselanguage\endcsname\relax
  \@PackageInfoNoLine{iflang}{%
    Naming convention for patterns: babel%
  }%
  \def\IfLang@prefix{l@}%
\else
  \@PackageInfoNoLine{iflang}{%
    Naming convention for patterns: etex.src%
  }%
  \def\IfLang@prefix{lang@}%
  \let\IfLang@OrgUseLanguage\uselanguage
  \def\uselanguage#1{%
    \edef\languagename{#1}%
    \IfLang@OrgUseLanguage{#1}%
  }%
%    \end{macrocode}
%    \end{macro}
%    The first \cs{uselanguage} that is executed as last
%    line in \xfile{language.def} cannot patched this way.
%    However, \xfile{language.def} is very strict. It forces
%    the first added and used language to be |USenglish|.
%    Thus, if \cs{languagename} is not defined, we can
%    quite safely assume |USenglish|. As additional
%    safety precaution the actual used patterns are checked.
%    \begin{macrocode}
  \begingroup\expandafter\expandafter\expandafter\endgroup
  \expandafter\ifx\csname languagename\endcsname\relax
    \begingroup\expandafter\expandafter\expandafter\endgroup
    \expandafter\ifx\csname lang@USenglish\endcsname\relax
      \@PackageWarningNoLine{iflang}{%
        \string\lang@USenglish\space is missing%
      }%
    \else
      \ifnum\lang@USenglish=\language
        \def\languagename{USenglish}%
      \else
        \@PackageWarningNoLine{iflang}{%
          \string\languagename\space is not set,\MessageBreak
          current language is unknown%
        }%
      \fi
    \fi
  \fi
\fi
\begingroup\expandafter\expandafter\expandafter\endgroup
\expandafter\ifx\csname languagename\endcsname\relax
  \@PackageInfoNoLine{iflang}{%
    \string\languagename\space is not set%
  }%
\fi
%    \end{macrocode}
%
% \subsection{\cs{IfLanguagePatterns}}
%
%    \begin{macro}{\IfLanguagePatterns}
%    \begin{macrocode}
\def\IfLanguagePatterns#1{%
  \ifnum\IfLang@IfDefined{\IfLang@prefix#1}{%
          \ifnum\csname\IfLang@prefix#1\endcsname=\language
            0%
          \else
            1%
          \fi
        }{1}=0 %
    \expandafter\@firstoftwo
  \else
    \expandafter\@secondoftwo
  \fi
}
%    \end{macrocode}
%    \end{macro}
%
% \subsection{\cs{IfLanguageName}}
%
%    \begin{macrocode}
\begingroup\expandafter\expandafter\expandafter\endgroup
\expandafter\ifx\csname pdf@strcmp\endcsname\relax
  \expandafter\@firstoftwo
\else
  \expandafter\@secondoftwo
\fi
{%
%    \end{macrocode}
%    We do not have \cs{pdf@strcmp} (and \cs{pdfstrcmp}).
%    Thus we must define
%    our own expandable string comparison. The following
%    implementation is based on a \TeX\ pearl from David Kastrup, presented
%    at the conference Bacho\TeX~2005:
%    \url{http://www-stary.gust.org.pl/pearls/2005/david-kastrup/bachotex2005-david-kastrup-pearl1.pdf}
%
%    The orignal code allows macros inside the second string.
%    Because also \cs{languagename} might consists of further
%    macros, we need a variant that allows macros in the first
%    string, too.
%    \begin{macrocode}
  \def\IfLang@StrNil{\relax}%
  \def\IfLang@StrEqual#1{%
    \number\IfLang@StrEqualStart{}{}#1\IfLang@StrNil
  }%
  \def\IfLang@StrEqualStart#1#2#3{%
    \ifx#3\IfLang@StrNil
      \IfLang@StrEqualStop
    \fi
    \ifcat\noexpand#3\relax
      \IfLang@StrExpand{#1}{#2}#3%
    \fi
    \IfLang@StrEqualStart{\if#3#1}{#2\fi}%
  }%
  \def\IfLang@StrEqualStop\fi#1\IfLang@StrEqualStart#2#3#4{%
    \fi
    #2#4\relax'#313 %
  }%
  \def\IfLang@StrExpand#1#2#3\fi\IfLang@StrEqualStart#4#5{%
    \fi
    \IfLang@@StrExpand{#1}{#2}#3%
  }%
  \def\IfLang@@StrExpand#1#2#3\IfLang@StrNil{%
    \expandafter\IfLang@@@StrExpand#3\IfLang@StrNil{#1}{#2}%
  }%
  \def\IfLang@@@StrExpand#1\IfLang@StrNil#2#3{%
    \IfLang@StrEqualStart{#2}{#3}#1\IfLang@StrNil
  }%
%    \end{macrocode}
%    \begin{macro}{\IfLanguageName}
%    \begin{macrocode}
  \def\IfLanguageName#1{%
    \ifnum\IfLang@IfDefined{languagename}{%
            \if\expandafter\IfLang@StrEqual\expandafter%
                           {\languagename}{#1}%
              0%
            \else
              1%
            \fi
          }{1}=0 %
      \expandafter\@firstoftwo
    \else
      \expandafter\@secondoftwo
    \fi
  }%
%    \end{macrocode}
%    \end{macro}
%    \begin{macrocode}
}{%
%    \end{macrocode}
%    \begin{macro}{\IfLanguageName}
%    \begin{macrocode}
  \def\IfLanguageName#1{%
    \ifnum\IfLang@IfDefined{languagename}{%
            \pdf@strcmp{#1}{\languagename}%
          }{1}=0 %
      \expandafter\@firstoftwo
    \else
      \expandafter\@secondoftwo
    \fi
  }%
%    \end{macrocode}
%    \end{macro}
%    \begin{macrocode}
}
%    \end{macrocode}
%
% \subsection{Check plausibility of \cs{languagename}}
%
%    \begin{macrocode}
\begingroup\expandafter\expandafter\expandafter\endgroup
\expandafter\ifx\csname languagename\endcsname\relax
\else
  \IfLanguagePatterns{\languagename}{}{%
    \@PackageWarningNoLine{iflang}{%
      Mismatch between \string\language\space
      (patterns)\MessageBreak
      and setting of \string\languagename
    }%
  }%
\fi
%    \end{macrocode}
%
%    \begin{macrocode}
\IfLang@AtEnd%
%</package>
%    \end{macrocode}
%
% \section{Test}
%
% \subsection{Catcode checks for loading}
%
%    \begin{macrocode}
%<*test1>
%    \end{macrocode}
%    \begin{macrocode}
\catcode`\{=1 %
\catcode`\}=2 %
\catcode`\#=6 %
\catcode`\@=11 %
\expandafter\ifx\csname count@\endcsname\relax
  \countdef\count@=255 %
\fi
\expandafter\ifx\csname @gobble\endcsname\relax
  \long\def\@gobble#1{}%
\fi
\expandafter\ifx\csname @firstofone\endcsname\relax
  \long\def\@firstofone#1{#1}%
\fi
\expandafter\ifx\csname loop\endcsname\relax
  \expandafter\@firstofone
\else
  \expandafter\@gobble
\fi
{%
  \def\loop#1\repeat{%
    \def\body{#1}%
    \iterate
  }%
  \def\iterate{%
    \body
      \let\next\iterate
    \else
      \let\next\relax
    \fi
    \next
  }%
  \let\repeat=\fi
}%
\def\RestoreCatcodes{}
\count@=0 %
\loop
  \edef\RestoreCatcodes{%
    \RestoreCatcodes
    \catcode\the\count@=\the\catcode\count@\relax
  }%
\ifnum\count@<255 %
  \advance\count@ 1 %
\repeat

\def\RangeCatcodeInvalid#1#2{%
  \count@=#1\relax
  \loop
    \catcode\count@=15 %
  \ifnum\count@<#2\relax
    \advance\count@ 1 %
  \repeat
}
\def\RangeCatcodeCheck#1#2#3{%
  \count@=#1\relax
  \loop
    \ifnum#3=\catcode\count@
    \else
      \errmessage{%
        Character \the\count@\space
        with wrong catcode \the\catcode\count@\space
        instead of \number#3%
      }%
    \fi
  \ifnum\count@<#2\relax
    \advance\count@ 1 %
  \repeat
}
\def\space{ }
\expandafter\ifx\csname LoadCommand\endcsname\relax
  \def\LoadCommand{\input iflang.sty\relax}%
\fi
\def\Test{%
  \RangeCatcodeInvalid{0}{47}%
  \RangeCatcodeInvalid{58}{64}%
  \RangeCatcodeInvalid{91}{96}%
  \RangeCatcodeInvalid{123}{255}%
  \catcode`\@=12 %
  \catcode`\\=0 %
  \catcode`\%=14 %
  \LoadCommand
  \RangeCatcodeCheck{0}{36}{15}%
  \RangeCatcodeCheck{37}{37}{14}%
  \RangeCatcodeCheck{38}{47}{15}%
  \RangeCatcodeCheck{48}{57}{12}%
  \RangeCatcodeCheck{58}{63}{15}%
  \RangeCatcodeCheck{64}{64}{12}%
  \RangeCatcodeCheck{65}{90}{11}%
  \RangeCatcodeCheck{91}{91}{15}%
  \RangeCatcodeCheck{92}{92}{0}%
  \RangeCatcodeCheck{93}{96}{15}%
  \RangeCatcodeCheck{97}{122}{11}%
  \RangeCatcodeCheck{123}{255}{15}%
  \RestoreCatcodes
}
\Test
\csname @@end\endcsname
\end
%    \end{macrocode}
%    \begin{macrocode}
%</test1>
%    \end{macrocode}
%
% \subsection{Test with \LaTeX}
%
%    \begin{macrocode}
%<*test2|test3>
%    \end{macrocode}
%    \begin{macrocode}
\NeedsTeXFormat{LaTeX2e}
%<test3>\let\pdfstrcmp\relax
\nofiles
\documentclass{minimal}
\usepackage{qstest}
\IncludeTests{*}
\LogTests{log}{*}{*}
\usepackage[english,naustrian,ngerman]{babel}
\usepackage{iflang}
\begin{document}
\begin{qstest}{IfLanguagePatterns}{language, pattern}
  \def\test#1#2{%
    \Expect*{\IfLanguagePatterns{#1}{true}{false}}{#2}%
  }%
  \test{ngerman}{true}%
  \test{naustrian}{true}%
  \test{english}{false}%
  \test{foobar}{false}%
\end{qstest}
\begin{qstest}{IfLanguageName}{language, name}
  \def\test#1#2{%
    \Expect*{\IfLanguageName{#1}{true}{false}}{#2}%
  }%
  \test{ngerman}{true}%
  \test{naustrian}{false}%
  \selectlanguage{naustrian}%
  \test{ngerman}{false}%
  \test{naustrian}{true}%
  \test{foobar}{false}%
  %
  \def\languagename{naustrian}%
  \test{naustrian}{true}%
  \test{ngerman}{false}%
  %
  \edef\languagename{\string naustrian}%
  \test{naustrian}{true}%
  \test{ngerman}{false}%
  %
  \def\languagename{naustrian}%
  \makeatletter
  \@onelevel@sanitize\languagename
  \test{naustrian}{true}%
  \test{ngerman}{false}%
  %
  \def\languagename{naustrian}%
  \def\xaustrian{naustrian}%
  \def\xgerman{ngerman}%
  \test{\xaustrian}{true}%
  \test{\xgerman}{false}%
  %
  \def\languagename{\xaustrian}%
  \test{naustrian}{true}%
  \test{ngerman}{false}%
  \test{\xaustrian}{true}%
  \test{\xgerman}{false}%
  \test{\languagename}{true}%
  \test{\languagename\space}{false}%
  %
  \def\languagename{\empty\xaustrian\empty}%
  \test{naustrian}{true}%
  \test{ngerman}{false}%
  \test{\empty\xaustrian\empty}{true}%
  \test{\empty\xgerman\empty}{false}%
\end{qstest}
\begin{qstest}{IfDefined}{defined}
  \makeatletter
  \let\foobar\relax
  \Expect*{\IfLang@IfDefined{foobar}{true}{false}}{false}%
  \Expect*{\ifx\foobar\relax true\else false\fi}{true}%
  \let\foobar\UNDEFINED
  \Expect*{\IfLang@IfDefined{foobar}{true}{false}}{false}%
  \Expect*{\ifx\foobar\relax true\else false\fi}{false}%
  \Expect*{\ifx\foobar\UNDEFINED true\else false\fi}{true}%
\end{qstest}
\end{document}
%    \end{macrocode}
%    \begin{macrocode}
%</test2|test3>
%    \end{macrocode}
%
% \subsection{Test with \plainTeX\ and \eTeX}
%
%    \begin{macrocode}
%<*test4>
%    \end{macrocode}
%    \begin{macrocode}
%% Format `etex' based on `language.def'
\input iflang.sty
\catcode64=12

\def\TestGeneric#1#2#3{%
  \begingroup
    \edef\x{#1{#2}{true}{false}}%
    \edef\y{#3}%
    \ifx\x\y
    \else
      \errmessage{Failed test: \string#1{#2} <> #3}%
    \fi
  \endgroup
}
\def\TestPatterns{\TestGeneric\IfLanguagePatterns}
\def\TestName{\TestGeneric\IfLanguageName}

\TestPatterns{USenglish}{true}
\TestPatterns{ngerman}{false}

\TestName{USenglish}{true}
\TestName{ngerman}{false}

\uselanguage{ngerman}

\TestPatterns{USenglish}{false}
\TestPatterns{ngerman}{true}

\TestName{USenglish}{false}
\TestName{ngerman}{true}

\csname @@end\endcsname
\end
%    \end{macrocode}
%    \begin{macrocode}
%</test4>
%    \end{macrocode}
%
% \subsection{Test with \plainTeX\ and without \eTeX/\pdfTeX}
%
%    \begin{macrocode}
%<*test5>
%    \end{macrocode}
%    \begin{macrocode}
%% Format `tex' (vanilla plain-TeX)
\let\ifcsname\UNDEFINED
\let\pdfstrcmp\UNDEFINED
\input iflang.sty
\catcode64=11

\def\TestDefined#1{%
  \IfLang@IfDefined{foobar}{}{}%
  \ifx\foobar#1%
  \else
    \errmessage{Failed test: \string\foobar <> \string#1}%
  \fi
}
\let\foobar\relax
\TestDefined\relax
\let\foobar\UNDEFINED
\TestDefined\relax

\def\strip@prefix#1>{}
\def\@onelevel@sanitize#1{%
  \edef#1{\expandafter\strip@prefix\meaning#1}%
}
\def\TestCompare#1#2#3{%
  \begingroup
    \edef\x{%
      \if\IfLang@StrEqual{#1}{#2}%
        true%
      \else
        false%
      \fi
    }%
    \def\expect{#3}%
    \ifx\x\expect
    \else
      \def\a{#1}%
      \@onelevel@sanitize\a
      \def\b{#2}%
      \@onelevel@sanitize\b
      \errmessage{Failed test: `\a'='\b' <> \expect}%
    \fi
  \endgroup
}
\TestCompare{junk}{junk}{true}
\TestCompare{}{}{true}
\TestCompare{a}{b}{false}
\TestCompare{aa}{bb}{false}
\def\a{ax}
\def\b{bx}
\def\c{\a\b}
\def\d{\c\b}
\def\exch#1#2{#2#1}
\def\gobble#1{}
\TestCompare{\gobble a}{}{true}
\TestCompare{}{\gobble a}{true}
\TestCompare{a\exch xyb}{ayxb}{true}
\TestCompare{\c}{\c}{true}
\TestCompare{\d}{\c\b}{true}

\csname @@end\endcsname
\end
%    \end{macrocode}
%    \begin{macrocode}
%</test5>
%    \end{macrocode}
%
% \section{Installation}
%
% \subsection{Download}
%
% \paragraph{Package.} This package is available on
% CTAN\footnote{\url{http://ctan.org/pkg/iflang}}:
% \begin{description}
% \item[\CTAN{macros/latex/contrib/oberdiek/iflang.dtx}] The source file.
% \item[\CTAN{macros/latex/contrib/oberdiek/iflang.pdf}] Documentation.
% \end{description}
%
%
% \paragraph{Bundle.} All the packages of the bundle `oberdiek'
% are also available in a TDS compliant ZIP archive. There
% the packages are already unpacked and the documentation files
% are generated. The files and directories obey the TDS standard.
% \begin{description}
% \item[\CTAN{install/macros/latex/contrib/oberdiek.tds.zip}]
% \end{description}
% \emph{TDS} refers to the standard ``A Directory Structure
% for \TeX\ Files'' (\CTAN{tds/tds.pdf}). Directories
% with \xfile{texmf} in their name are usually organized this way.
%
% \subsection{Bundle installation}
%
% \paragraph{Unpacking.} Unpack the \xfile{oberdiek.tds.zip} in the
% TDS tree (also known as \xfile{texmf} tree) of your choice.
% Example (linux):
% \begin{quote}
%   |unzip oberdiek.tds.zip -d ~/texmf|
% \end{quote}
%
% \paragraph{Script installation.}
% Check the directory \xfile{TDS:scripts/oberdiek/} for
% scripts that need further installation steps.
% Package \xpackage{attachfile2} comes with the Perl script
% \xfile{pdfatfi.pl} that should be installed in such a way
% that it can be called as \texttt{pdfatfi}.
% Example (linux):
% \begin{quote}
%   |chmod +x scripts/oberdiek/pdfatfi.pl|\\
%   |cp scripts/oberdiek/pdfatfi.pl /usr/local/bin/|
% \end{quote}
%
% \subsection{Package installation}
%
% \paragraph{Unpacking.} The \xfile{.dtx} file is a self-extracting
% \docstrip\ archive. The files are extracted by running the
% \xfile{.dtx} through \plainTeX:
% \begin{quote}
%   \verb|tex iflang.dtx|
% \end{quote}
%
% \paragraph{TDS.} Now the different files must be moved into
% the different directories in your installation TDS tree
% (also known as \xfile{texmf} tree):
% \begin{quote}
% \def\t{^^A
% \begin{tabular}{@{}>{\ttfamily}l@{ $\rightarrow$ }>{\ttfamily}l@{}}
%   iflang.sty & tex/generic/oberdiek/iflang.sty\\
%   iflang.pdf & doc/latex/oberdiek/iflang.pdf\\
%   test/iflang-test1.tex & doc/latex/oberdiek/test/iflang-test1.tex\\
%   test/iflang-test2.tex & doc/latex/oberdiek/test/iflang-test2.tex\\
%   test/iflang-test3.tex & doc/latex/oberdiek/test/iflang-test3.tex\\
%   test/iflang-test4.tex & doc/latex/oberdiek/test/iflang-test4.tex\\
%   test/iflang-test5.tex & doc/latex/oberdiek/test/iflang-test5.tex\\
%   iflang.dtx & source/latex/oberdiek/iflang.dtx\\
% \end{tabular}^^A
% }^^A
% \sbox0{\t}^^A
% \ifdim\wd0>\linewidth
%   \begingroup
%     \advance\linewidth by\leftmargin
%     \advance\linewidth by\rightmargin
%   \edef\x{\endgroup
%     \def\noexpand\lw{\the\linewidth}^^A
%   }\x
%   \def\lwbox{^^A
%     \leavevmode
%     \hbox to \linewidth{^^A
%       \kern-\leftmargin\relax
%       \hss
%       \usebox0
%       \hss
%       \kern-\rightmargin\relax
%     }^^A
%   }^^A
%   \ifdim\wd0>\lw
%     \sbox0{\small\t}^^A
%     \ifdim\wd0>\linewidth
%       \ifdim\wd0>\lw
%         \sbox0{\footnotesize\t}^^A
%         \ifdim\wd0>\linewidth
%           \ifdim\wd0>\lw
%             \sbox0{\scriptsize\t}^^A
%             \ifdim\wd0>\linewidth
%               \ifdim\wd0>\lw
%                 \sbox0{\tiny\t}^^A
%                 \ifdim\wd0>\linewidth
%                   \lwbox
%                 \else
%                   \usebox0
%                 \fi
%               \else
%                 \lwbox
%               \fi
%             \else
%               \usebox0
%             \fi
%           \else
%             \lwbox
%           \fi
%         \else
%           \usebox0
%         \fi
%       \else
%         \lwbox
%       \fi
%     \else
%       \usebox0
%     \fi
%   \else
%     \lwbox
%   \fi
% \else
%   \usebox0
% \fi
% \end{quote}
% If you have a \xfile{docstrip.cfg} that configures and enables \docstrip's
% TDS installing feature, then some files can already be in the right
% place, see the documentation of \docstrip.
%
% \subsection{Refresh file name databases}
%
% If your \TeX~distribution
% (\teTeX, \mikTeX, \dots) relies on file name databases, you must refresh
% these. For example, \teTeX\ users run \verb|texhash| or
% \verb|mktexlsr|.
%
% \subsection{Some details for the interested}
%
% \paragraph{Attached source.}
%
% The PDF documentation on CTAN also includes the
% \xfile{.dtx} source file. It can be extracted by
% AcrobatReader 6 or higher. Another option is \textsf{pdftk},
% e.g. unpack the file into the current directory:
% \begin{quote}
%   \verb|pdftk iflang.pdf unpack_files output .|
% \end{quote}
%
% \paragraph{Unpacking with \LaTeX.}
% The \xfile{.dtx} chooses its action depending on the format:
% \begin{description}
% \item[\plainTeX:] Run \docstrip\ and extract the files.
% \item[\LaTeX:] Generate the documentation.
% \end{description}
% If you insist on using \LaTeX\ for \docstrip\ (really,
% \docstrip\ does not need \LaTeX), then inform the autodetect routine
% about your intention:
% \begin{quote}
%   \verb|latex \let\install=y\input{iflang.dtx}|
% \end{quote}
% Do not forget to quote the argument according to the demands
% of your shell.
%
% \paragraph{Generating the documentation.}
% You can use both the \xfile{.dtx} or the \xfile{.drv} to generate
% the documentation. The process can be configured by the
% configuration file \xfile{ltxdoc.cfg}. For instance, put this
% line into this file, if you want to have A4 as paper format:
% \begin{quote}
%   \verb|\PassOptionsToClass{a4paper}{article}|
% \end{quote}
% An example follows how to generate the
% documentation with pdf\LaTeX:
% \begin{quote}
%\begin{verbatim}
%pdflatex iflang.dtx
%makeindex -s gind.ist iflang.idx
%pdflatex iflang.dtx
%makeindex -s gind.ist iflang.idx
%pdflatex iflang.dtx
%\end{verbatim}
% \end{quote}
%
% \section{Catalogue}
%
% The following XML file can be used as source for the
% \href{http://mirror.ctan.org/help/Catalogue/catalogue.html}{\TeX\ Catalogue}.
% The elements \texttt{caption} and \texttt{description} are imported
% from the original XML file from the Catalogue.
% The name of the XML file in the Catalogue is \xfile{iflang.xml}.
%    \begin{macrocode}
%<*catalogue>
<?xml version='1.0' encoding='us-ascii'?>
<!DOCTYPE entry SYSTEM 'catalogue.dtd'>
<entry datestamp='$Date$' modifier='$Author$' id='iflang'>
  <name>iflang</name>
  <caption>Expandable checks for the current language.</caption>
  <authorref id='auth:oberdiek'/>
  <copyright owner='Heiko Oberdiek' year='2007'/>
  <license type='lppl1.3'/>
  <version number='1.6'/>
  <description>
    This package provides expandable checks for the current language
    based on macro <tt>\languagename</tt> or hyphenation patterns.
    <p/>
    The package is part of the <xref refid='oberdiek'>oberdiek</xref> bundle.
  </description>
  <documentation details='Package documentation'
      href='ctan:/macros/latex/contrib/oberdiek/iflang.pdf'/>
  <ctan file='true' path='/macros/latex/contrib/oberdiek/iflang.dtx'/>
  <miktex location='oberdiek'/>
  <texlive location='oberdiek'/>
  <install path='/macros/latex/contrib/oberdiek/oberdiek.tds.zip'/>
</entry>
%</catalogue>
%    \end{macrocode}
%
% \section{Acknowledgement}
% I wish to thank:
% \begin{description}
% \item[Markus Kohm] Useful hints for version 1.2.
% \end{description}
%
% \begin{History}
%   \begin{Version}{2007/04/10 v1.0}
%   \item
%     First public version.
%   \end{Version}
%   \begin{Version}{2007/04/11 v1.1}
%   \item
%     Line ends sanitized.
%   \end{Version}
%   \begin{Version}{2007/04/12 v1.2}
%   \item
%     Initialization of \cs{languagename} in case of \xfile{etex.src}.
%   \item
%     Some sanity tests added.
%   \item
%     Documentation improved.
%   \end{Version}
%   \begin{Version}{2007/04/26 v1.3}
%   \item
%     Use of package \xpackage{infwarerr}.
%   \end{Version}
%   \begin{Version}{2007/09/09 v1.4}
%   \item
%     Bug fix: \cs{IfLang@StrEqual} $\rightarrow$ \cs{IfLangStrEqual}
%     (Gabriele Balducci).
%   \item
%     Catcode section rewritten.
%   \end{Version}
%   \begin{Version}{2007/11/11 v1.5}
%   \item
%     Use of package \xpackage{pdftexcmds} for \LuaTeX\ support.
%   \end{Version}
%   \begin{Version}{2016/05/16 v1.6}
%   \item
%     Documentation updates.
%   \end{Version}
% \end{History}
%
% \PrintIndex
%
% \Finale
\endinput

%        (quote the arguments according to the demands of your shell)
%
% Documentation:
%    (a) If iflang.drv is present:
%           latex iflang.drv
%    (b) Without iflang.drv:
%           latex iflang.dtx; ...
%    The class ltxdoc loads the configuration file ltxdoc.cfg
%    if available. Here you can specify further options, e.g.
%    use A4 as paper format:
%       \PassOptionsToClass{a4paper}{article}
%
%    Programm calls to get the documentation (example):
%       pdflatex iflang.dtx
%       makeindex -s gind.ist iflang.idx
%       pdflatex iflang.dtx
%       makeindex -s gind.ist iflang.idx
%       pdflatex iflang.dtx
%
% Installation:
%    TDS:tex/generic/oberdiek/iflang.sty
%    TDS:doc/latex/oberdiek/iflang.pdf
%    TDS:doc/latex/oberdiek/test/iflang-test1.tex
%    TDS:doc/latex/oberdiek/test/iflang-test2.tex
%    TDS:doc/latex/oberdiek/test/iflang-test3.tex
%    TDS:doc/latex/oberdiek/test/iflang-test4.tex
%    TDS:doc/latex/oberdiek/test/iflang-test5.tex
%    TDS:source/latex/oberdiek/iflang.dtx
%
%<*ignore>
\begingroup
  \catcode123=1 %
  \catcode125=2 %
  \def\x{LaTeX2e}%
\expandafter\endgroup
\ifcase 0\ifx\install y1\fi\expandafter
         \ifx\csname processbatchFile\endcsname\relax\else1\fi
         \ifx\fmtname\x\else 1\fi\relax
\else\csname fi\endcsname
%</ignore>
%<*install>
\input docstrip.tex
\Msg{************************************************************************}
\Msg{* Installation}
\Msg{* Package: iflang 2016/05/16 v1.6 Checks for the current language (HO)}
\Msg{************************************************************************}

\keepsilent
\askforoverwritefalse

\let\MetaPrefix\relax
\preamble

This is a generated file.

Project: iflang
Version: 2016/05/16 v1.6

Copyright (C) 2007 by
   Heiko Oberdiek <heiko.oberdiek at googlemail.com>

This work may be distributed and/or modified under the
conditions of the LaTeX Project Public License, either
version 1.3c of this license or (at your option) any later
version. This version of this license is in
   http://www.latex-project.org/lppl/lppl-1-3c.txt
and the latest version of this license is in
   http://www.latex-project.org/lppl.txt
and version 1.3 or later is part of all distributions of
LaTeX version 2005/12/01 or later.

This work has the LPPL maintenance status "maintained".

This Current Maintainer of this work is Heiko Oberdiek.

The Base Interpreter refers to any `TeX-Format',
because some files are installed in TDS:tex/generic//.

This work consists of the main source file iflang.dtx
and the derived files
   iflang.sty, iflang.pdf, iflang.ins, iflang.drv, iflang-test1.tex,
   iflang-test2.tex, iflang-test3.tex, iflang-test4.tex,
   iflang-test5.tex.

\endpreamble
\let\MetaPrefix\DoubleperCent

\generate{%
  \file{iflang.ins}{\from{iflang.dtx}{install}}%
  \file{iflang.drv}{\from{iflang.dtx}{driver}}%
  \usedir{tex/generic/oberdiek}%
  \file{iflang.sty}{\from{iflang.dtx}{package}}%
  \usedir{doc/latex/oberdiek/test}%
  \file{iflang-test1.tex}{\from{iflang.dtx}{test1}}%
  \file{iflang-test2.tex}{\from{iflang.dtx}{test2}}%
  \file{iflang-test3.tex}{\from{iflang.dtx}{test3}}%
  \file{iflang-test4.tex}{\from{iflang.dtx}{test4}}%
  \file{iflang-test5.tex}{\from{iflang.dtx}{test5}}%
  \nopreamble
  \nopostamble
  \usedir{source/latex/oberdiek/catalogue}%
  \file{iflang.xml}{\from{iflang.dtx}{catalogue}}%
}

\catcode32=13\relax% active space
\let =\space%
\Msg{************************************************************************}
\Msg{*}
\Msg{* To finish the installation you have to move the following}
\Msg{* file into a directory searched by TeX:}
\Msg{*}
\Msg{*     iflang.sty}
\Msg{*}
\Msg{* To produce the documentation run the file `iflang.drv'}
\Msg{* through LaTeX.}
\Msg{*}
\Msg{* Happy TeXing!}
\Msg{*}
\Msg{************************************************************************}

\endbatchfile
%</install>
%<*ignore>
\fi
%</ignore>
%<*driver>
\NeedsTeXFormat{LaTeX2e}
\ProvidesFile{iflang.drv}%
  [2016/05/16 v1.6 Checks for the current language (HO)]%
\documentclass{ltxdoc}
\usepackage{holtxdoc}[2011/11/22]
\begin{document}
  \DocInput{iflang.dtx}%
\end{document}
%</driver>
% \fi
%
%
% \CharacterTable
%  {Upper-case    \A\B\C\D\E\F\G\H\I\J\K\L\M\N\O\P\Q\R\S\T\U\V\W\X\Y\Z
%   Lower-case    \a\b\c\d\e\f\g\h\i\j\k\l\m\n\o\p\q\r\s\t\u\v\w\x\y\z
%   Digits        \0\1\2\3\4\5\6\7\8\9
%   Exclamation   \!     Double quote  \"     Hash (number) \#
%   Dollar        \$     Percent       \%     Ampersand     \&
%   Acute accent  \'     Left paren    \(     Right paren   \)
%   Asterisk      \*     Plus          \+     Comma         \,
%   Minus         \-     Point         \.     Solidus       \/
%   Colon         \:     Semicolon     \;     Less than     \<
%   Equals        \=     Greater than  \>     Question mark \?
%   Commercial at \@     Left bracket  \[     Backslash     \\
%   Right bracket \]     Circumflex    \^     Underscore    \_
%   Grave accent  \`     Left brace    \{     Vertical bar  \|
%   Right brace   \}     Tilde         \~}
%
% \GetFileInfo{iflang.drv}
%
% \title{The \xpackage{iflang} package}
% \date{2016/05/16 v1.6}
% \author{Heiko Oberdiek\thanks
% {Please report any issues at https://github.com/ho-tex/oberdiek/issues}\\
% \xemail{heiko.oberdiek at googlemail.com}}
%
% \maketitle
%
% \begin{abstract}
% This package provides expandible checks for the current language
% based on macro \cs{languagename} or hyphenation patterns.
% \end{abstract}
%
% \tableofcontents
%
% \section{Documentation}
%
% Package \xpackage{babel} defines \cs{iflanguagename}. As first
% argument it takes a language name and executes the second or
% third argument depending on the current language. This
% language test is based on hypenation patterns. However, it
% is possible that different languages or dialects share the
% same patterns. In such cases \cs{iflanguagename} fails.
%
% However, package \xpackage{babel} and some other packages
% such as \xpackage{german} or \xpackage{ngerman} store
% the language name in the macro \cs{languagename} if
% \cs{selectlanguage} is called.
%
% \begin{declcs}{IfLanguageName} \M{lang} \M{then} \M{else}
% \end{declcs}
% Makro \cs{IfLanguageName} compares language \meta{lang} with
% the current setting of macro \cs{languagename}. If both
% contains the same name then the \meta{then} part is called,
% otherwise the \meta{else} part.
%
% The macro is expandable. Thus it can be safely used inside \cs{edef}
% or \cs{csname}. If case of errors like an undefined \cs{languagename}
% the \meta{else} part is executed.
%
% Note: Macro \cs{IfLanguageName} relies on the fact, that
% \cs{languagename} is set correctly:
% \begin{description}
% \item[Package \xpackage{babel}:]
%    \mbox{}\\
%    Full support of \cs{languagename} in its language switching commands.
% \item[Format based on \xpackage{babel} (\xfile{language.dat}):]
%    \mbox{}\\
%    If package \xpackage{babel} is not used (or not yet loaded),
%    then \xpackage{babel}'s
%    \xfile{hyphen.cfg} has set \cs{languagename} to the last
%    language in \xfile{language.dat}, but \cs{language} (current
%    patterns) is zero and points to the first language.
%    Thus the value of \cs{languagename} is basically garbage.
%    Package \xpackage{iflang} warns if \cs{languagename} and
%    \cs{language} do not fit.
%    This can be fixed by loading package \xpackage{babel} previously.
% \item[Format based on \eTeX's \xfile{etex.src} (\xfile{language.def}):]
%    \mbox{}\\
%    Unhappily it does not support
%    \cs{languagename}. Thus this package hooks into \cs{uselanguage}
%    to get \cs{languagename} defined and updated there.
%    At package loading time the changed \cs{uselanguage} has not
%    been called yet. Thus package \xpackage{iflang} tries
%    |USenglish|. This is the definite default language of \xfile{etex.src}.
%    If the current patterns suit this default language, an
%    undefined \cs{languagename} is set to this language. Otherwise
%    a \cs{languagename} remains undefined and a warning is given.
% \end{description}
%
% \begin{declcs}{IfLanguagePatterns} \M{lang} \M{then} \M{else}
% \end{declcs}
% This macro behaves similar to \cs{IfLanguageName}. But the
% language test is based on the current pattern in force (\cs{language}).
% Also this macro is expandable, in case of errors the \meta{else} part
% is called.
%
% The following naming convention for the pattern are supported:
% \begin{description}
% \item[\xpackage{babel}/\xfile{language.dat}]:
%    \cs{l@}\meta{language}
% \item[\xpackage{etex.src}/\xfile{language.def}]:
%    \cs{lang@}\meta{language}
% \end{description}
% Package \xpackage{iflang} looks for \cs{uselanguage} (defined in
% \xfile{etex.src}) to find out the naming convention in use.
%
% \StopEventually{
% }
%
% \section{Implementation}
%
%    \begin{macrocode}
%<*package>
%    \end{macrocode}
%
% \subsection{Reload check and package identification}
%    Reload check, especially if the package is not used with \LaTeX.
%    \begin{macrocode}
\begingroup\catcode61\catcode48\catcode32=10\relax%
  \catcode13=5 % ^^M
  \endlinechar=13 %
  \catcode35=6 % #
  \catcode39=12 % '
  \catcode44=12 % ,
  \catcode45=12 % -
  \catcode46=12 % .
  \catcode58=12 % :
  \catcode64=11 % @
  \catcode123=1 % {
  \catcode125=2 % }
  \expandafter\let\expandafter\x\csname ver@iflang.sty\endcsname
  \ifx\x\relax % plain-TeX, first loading
  \else
    \def\empty{}%
    \ifx\x\empty % LaTeX, first loading,
      % variable is initialized, but \ProvidesPackage not yet seen
    \else
      \expandafter\ifx\csname PackageInfo\endcsname\relax
        \def\x#1#2{%
          \immediate\write-1{Package #1 Info: #2.}%
        }%
      \else
        \def\x#1#2{\PackageInfo{#1}{#2, stopped}}%
      \fi
      \x{iflang}{The package is already loaded}%
      \aftergroup\endinput
    \fi
  \fi
\endgroup%
%    \end{macrocode}
%    Package identification:
%    \begin{macrocode}
\begingroup\catcode61\catcode48\catcode32=10\relax%
  \catcode13=5 % ^^M
  \endlinechar=13 %
  \catcode35=6 % #
  \catcode39=12 % '
  \catcode40=12 % (
  \catcode41=12 % )
  \catcode44=12 % ,
  \catcode45=12 % -
  \catcode46=12 % .
  \catcode47=12 % /
  \catcode58=12 % :
  \catcode64=11 % @
  \catcode91=12 % [
  \catcode93=12 % ]
  \catcode123=1 % {
  \catcode125=2 % }
  \expandafter\ifx\csname ProvidesPackage\endcsname\relax
    \def\x#1#2#3[#4]{\endgroup
      \immediate\write-1{Package: #3 #4}%
      \xdef#1{#4}%
    }%
  \else
    \def\x#1#2[#3]{\endgroup
      #2[{#3}]%
      \ifx#1\@undefined
        \xdef#1{#3}%
      \fi
      \ifx#1\relax
        \xdef#1{#3}%
      \fi
    }%
  \fi
\expandafter\x\csname ver@iflang.sty\endcsname
\ProvidesPackage{iflang}%
  [2016/05/16 v1.6 Checks for the current language (HO)]%
%    \end{macrocode}
%
%    \begin{macrocode}
\begingroup\catcode61\catcode48\catcode32=10\relax%
  \catcode13=5 % ^^M
  \endlinechar=13 %
  \catcode123=1 % {
  \catcode125=2 % }
  \catcode64=11 % @
  \def\x{\endgroup
    \expandafter\edef\csname IfLang@AtEnd\endcsname{%
      \endlinechar=\the\endlinechar\relax
      \catcode13=\the\catcode13\relax
      \catcode32=\the\catcode32\relax
      \catcode35=\the\catcode35\relax
      \catcode61=\the\catcode61\relax
      \catcode64=\the\catcode64\relax
      \catcode123=\the\catcode123\relax
      \catcode125=\the\catcode125\relax
    }%
  }%
\x\catcode61\catcode48\catcode32=10\relax%
\catcode13=5 % ^^M
\endlinechar=13 %
\catcode35=6 % #
\catcode64=11 % @
\catcode123=1 % {
\catcode125=2 % }
\def\TMP@EnsureCode#1#2{%
  \edef\IfLang@AtEnd{%
    \IfLang@AtEnd
    \catcode#1=\the\catcode#1\relax
  }%
  \catcode#1=#2\relax
}
\TMP@EnsureCode{39}{12}% '
\TMP@EnsureCode{40}{12}% (
\TMP@EnsureCode{41}{12}% )
\TMP@EnsureCode{44}{12}% ,
\TMP@EnsureCode{46}{12}% .
\TMP@EnsureCode{47}{12}% /
\TMP@EnsureCode{58}{12}% :
\TMP@EnsureCode{91}{12}% [
\TMP@EnsureCode{93}{12}% ]
\edef\IfLang@AtEnd{\IfLang@AtEnd\noexpand\endinput}
%    \end{macrocode}
%
% \subsection{Tools}
%
% \subsubsection{Provide some basic macros of \LaTeX}
%
%    \begin{macro}{\@firstoftwo}
%    \begin{macrocode}
\expandafter\ifx\csname @firstoftwo\endcsname\relax
  \long\def\@firstoftwo#1#2{#1}%
\fi
%    \end{macrocode}
%    \end{macro}
%    \begin{macro}{\@secondoftwo}
%    \begin{macrocode}
\expandafter\ifx\csname @secondoftwo\endcsname\relax
  \long\def\@secondoftwo#1#2{#2}%
\fi
%    \end{macrocode}
%    \end{macro}
%
% \subsubsection{Expandible existence check for macros}
%
%    \begin{macro}{\IfLang@IfDefined}
%    \begin{macrocode}
\begingroup\expandafter\expandafter\expandafter\endgroup
\expandafter\ifx\csname ifcsname\endcsname\relax
  \expandafter\@firstoftwo
\else
  \expandafter\@secondoftwo
\fi
{%
  \def\IfLang@IfDefined#1{%
    \expandafter\ifx\csname#1\endcsname\relax
      \expandafter\@secondoftwo
    \else
      \expandafter\@firstoftwo
    \fi
  }%
}{%
  \def\IfLang@IfDefined#1{%
    \ifnum\ifcsname#1\endcsname
            \expandafter\ifx\csname#1\endcsname\relax
              1%
            \else
              0%
            \fi
          \else
            1%
          \fi
          =0 %
      \expandafter\@firstoftwo
    \else
      \expandafter\@secondoftwo
    \fi
  }%
}
%    \end{macrocode}
%    \end{macro}
%
% \subsubsection{Macros for messages}
%
%    \begin{macrocode}
\begingroup\expandafter\expandafter\expandafter\endgroup
\expandafter\ifx\csname RequirePackage\endcsname\relax
  \input infwarerr.sty\relax
  \input pdftexcmds.sty\relax
\else
  \RequirePackage{infwarerr}[2007/09/09]%
  \RequirePackage{pdftexcmds}[2016/05/16]%
\fi
%    \end{macrocode}
%
% \subsubsection{Support for \xfile{etex.src}}
%
%    \begin{macro}{\IfLang@prefix}
%    \begin{macrocode}
\begingroup\expandafter\expandafter\expandafter\endgroup
\expandafter\ifx\csname uselanguage\endcsname\relax
  \@PackageInfoNoLine{iflang}{%
    Naming convention for patterns: babel%
  }%
  \def\IfLang@prefix{l@}%
\else
  \@PackageInfoNoLine{iflang}{%
    Naming convention for patterns: etex.src%
  }%
  \def\IfLang@prefix{lang@}%
  \let\IfLang@OrgUseLanguage\uselanguage
  \def\uselanguage#1{%
    \edef\languagename{#1}%
    \IfLang@OrgUseLanguage{#1}%
  }%
%    \end{macrocode}
%    \end{macro}
%    The first \cs{uselanguage} that is executed as last
%    line in \xfile{language.def} cannot patched this way.
%    However, \xfile{language.def} is very strict. It forces
%    the first added and used language to be |USenglish|.
%    Thus, if \cs{languagename} is not defined, we can
%    quite safely assume |USenglish|. As additional
%    safety precaution the actual used patterns are checked.
%    \begin{macrocode}
  \begingroup\expandafter\expandafter\expandafter\endgroup
  \expandafter\ifx\csname languagename\endcsname\relax
    \begingroup\expandafter\expandafter\expandafter\endgroup
    \expandafter\ifx\csname lang@USenglish\endcsname\relax
      \@PackageWarningNoLine{iflang}{%
        \string\lang@USenglish\space is missing%
      }%
    \else
      \ifnum\lang@USenglish=\language
        \def\languagename{USenglish}%
      \else
        \@PackageWarningNoLine{iflang}{%
          \string\languagename\space is not set,\MessageBreak
          current language is unknown%
        }%
      \fi
    \fi
  \fi
\fi
\begingroup\expandafter\expandafter\expandafter\endgroup
\expandafter\ifx\csname languagename\endcsname\relax
  \@PackageInfoNoLine{iflang}{%
    \string\languagename\space is not set%
  }%
\fi
%    \end{macrocode}
%
% \subsection{\cs{IfLanguagePatterns}}
%
%    \begin{macro}{\IfLanguagePatterns}
%    \begin{macrocode}
\def\IfLanguagePatterns#1{%
  \ifnum\IfLang@IfDefined{\IfLang@prefix#1}{%
          \ifnum\csname\IfLang@prefix#1\endcsname=\language
            0%
          \else
            1%
          \fi
        }{1}=0 %
    \expandafter\@firstoftwo
  \else
    \expandafter\@secondoftwo
  \fi
}
%    \end{macrocode}
%    \end{macro}
%
% \subsection{\cs{IfLanguageName}}
%
%    \begin{macrocode}
\begingroup\expandafter\expandafter\expandafter\endgroup
\expandafter\ifx\csname pdf@strcmp\endcsname\relax
  \expandafter\@firstoftwo
\else
  \expandafter\@secondoftwo
\fi
{%
%    \end{macrocode}
%    We do not have \cs{pdf@strcmp} (and \cs{pdfstrcmp}).
%    Thus we must define
%    our own expandable string comparison. The following
%    implementation is based on a \TeX\ pearl from David Kastrup, presented
%    at the conference Bacho\TeX~2005:
%    \url{http://www-stary.gust.org.pl/pearls/2005/david-kastrup/bachotex2005-david-kastrup-pearl1.pdf}
%
%    The orignal code allows macros inside the second string.
%    Because also \cs{languagename} might consists of further
%    macros, we need a variant that allows macros in the first
%    string, too.
%    \begin{macrocode}
  \def\IfLang@StrNil{\relax}%
  \def\IfLang@StrEqual#1{%
    \number\IfLang@StrEqualStart{}{}#1\IfLang@StrNil
  }%
  \def\IfLang@StrEqualStart#1#2#3{%
    \ifx#3\IfLang@StrNil
      \IfLang@StrEqualStop
    \fi
    \ifcat\noexpand#3\relax
      \IfLang@StrExpand{#1}{#2}#3%
    \fi
    \IfLang@StrEqualStart{\if#3#1}{#2\fi}%
  }%
  \def\IfLang@StrEqualStop\fi#1\IfLang@StrEqualStart#2#3#4{%
    \fi
    #2#4\relax'#313 %
  }%
  \def\IfLang@StrExpand#1#2#3\fi\IfLang@StrEqualStart#4#5{%
    \fi
    \IfLang@@StrExpand{#1}{#2}#3%
  }%
  \def\IfLang@@StrExpand#1#2#3\IfLang@StrNil{%
    \expandafter\IfLang@@@StrExpand#3\IfLang@StrNil{#1}{#2}%
  }%
  \def\IfLang@@@StrExpand#1\IfLang@StrNil#2#3{%
    \IfLang@StrEqualStart{#2}{#3}#1\IfLang@StrNil
  }%
%    \end{macrocode}
%    \begin{macro}{\IfLanguageName}
%    \begin{macrocode}
  \def\IfLanguageName#1{%
    \ifnum\IfLang@IfDefined{languagename}{%
            \if\expandafter\IfLang@StrEqual\expandafter%
                           {\languagename}{#1}%
              0%
            \else
              1%
            \fi
          }{1}=0 %
      \expandafter\@firstoftwo
    \else
      \expandafter\@secondoftwo
    \fi
  }%
%    \end{macrocode}
%    \end{macro}
%    \begin{macrocode}
}{%
%    \end{macrocode}
%    \begin{macro}{\IfLanguageName}
%    \begin{macrocode}
  \def\IfLanguageName#1{%
    \ifnum\IfLang@IfDefined{languagename}{%
            \pdf@strcmp{#1}{\languagename}%
          }{1}=0 %
      \expandafter\@firstoftwo
    \else
      \expandafter\@secondoftwo
    \fi
  }%
%    \end{macrocode}
%    \end{macro}
%    \begin{macrocode}
}
%    \end{macrocode}
%
% \subsection{Check plausibility of \cs{languagename}}
%
%    \begin{macrocode}
\begingroup\expandafter\expandafter\expandafter\endgroup
\expandafter\ifx\csname languagename\endcsname\relax
\else
  \IfLanguagePatterns{\languagename}{}{%
    \@PackageWarningNoLine{iflang}{%
      Mismatch between \string\language\space
      (patterns)\MessageBreak
      and setting of \string\languagename
    }%
  }%
\fi
%    \end{macrocode}
%
%    \begin{macrocode}
\IfLang@AtEnd%
%</package>
%    \end{macrocode}
%
% \section{Test}
%
% \subsection{Catcode checks for loading}
%
%    \begin{macrocode}
%<*test1>
%    \end{macrocode}
%    \begin{macrocode}
\catcode`\{=1 %
\catcode`\}=2 %
\catcode`\#=6 %
\catcode`\@=11 %
\expandafter\ifx\csname count@\endcsname\relax
  \countdef\count@=255 %
\fi
\expandafter\ifx\csname @gobble\endcsname\relax
  \long\def\@gobble#1{}%
\fi
\expandafter\ifx\csname @firstofone\endcsname\relax
  \long\def\@firstofone#1{#1}%
\fi
\expandafter\ifx\csname loop\endcsname\relax
  \expandafter\@firstofone
\else
  \expandafter\@gobble
\fi
{%
  \def\loop#1\repeat{%
    \def\body{#1}%
    \iterate
  }%
  \def\iterate{%
    \body
      \let\next\iterate
    \else
      \let\next\relax
    \fi
    \next
  }%
  \let\repeat=\fi
}%
\def\RestoreCatcodes{}
\count@=0 %
\loop
  \edef\RestoreCatcodes{%
    \RestoreCatcodes
    \catcode\the\count@=\the\catcode\count@\relax
  }%
\ifnum\count@<255 %
  \advance\count@ 1 %
\repeat

\def\RangeCatcodeInvalid#1#2{%
  \count@=#1\relax
  \loop
    \catcode\count@=15 %
  \ifnum\count@<#2\relax
    \advance\count@ 1 %
  \repeat
}
\def\RangeCatcodeCheck#1#2#3{%
  \count@=#1\relax
  \loop
    \ifnum#3=\catcode\count@
    \else
      \errmessage{%
        Character \the\count@\space
        with wrong catcode \the\catcode\count@\space
        instead of \number#3%
      }%
    \fi
  \ifnum\count@<#2\relax
    \advance\count@ 1 %
  \repeat
}
\def\space{ }
\expandafter\ifx\csname LoadCommand\endcsname\relax
  \def\LoadCommand{\input iflang.sty\relax}%
\fi
\def\Test{%
  \RangeCatcodeInvalid{0}{47}%
  \RangeCatcodeInvalid{58}{64}%
  \RangeCatcodeInvalid{91}{96}%
  \RangeCatcodeInvalid{123}{255}%
  \catcode`\@=12 %
  \catcode`\\=0 %
  \catcode`\%=14 %
  \LoadCommand
  \RangeCatcodeCheck{0}{36}{15}%
  \RangeCatcodeCheck{37}{37}{14}%
  \RangeCatcodeCheck{38}{47}{15}%
  \RangeCatcodeCheck{48}{57}{12}%
  \RangeCatcodeCheck{58}{63}{15}%
  \RangeCatcodeCheck{64}{64}{12}%
  \RangeCatcodeCheck{65}{90}{11}%
  \RangeCatcodeCheck{91}{91}{15}%
  \RangeCatcodeCheck{92}{92}{0}%
  \RangeCatcodeCheck{93}{96}{15}%
  \RangeCatcodeCheck{97}{122}{11}%
  \RangeCatcodeCheck{123}{255}{15}%
  \RestoreCatcodes
}
\Test
\csname @@end\endcsname
\end
%    \end{macrocode}
%    \begin{macrocode}
%</test1>
%    \end{macrocode}
%
% \subsection{Test with \LaTeX}
%
%    \begin{macrocode}
%<*test2|test3>
%    \end{macrocode}
%    \begin{macrocode}
\NeedsTeXFormat{LaTeX2e}
%<test3>\let\pdfstrcmp\relax
\nofiles
\documentclass{minimal}
\usepackage{qstest}
\IncludeTests{*}
\LogTests{log}{*}{*}
\usepackage[english,naustrian,ngerman]{babel}
\usepackage{iflang}
\begin{document}
\begin{qstest}{IfLanguagePatterns}{language, pattern}
  \def\test#1#2{%
    \Expect*{\IfLanguagePatterns{#1}{true}{false}}{#2}%
  }%
  \test{ngerman}{true}%
  \test{naustrian}{true}%
  \test{english}{false}%
  \test{foobar}{false}%
\end{qstest}
\begin{qstest}{IfLanguageName}{language, name}
  \def\test#1#2{%
    \Expect*{\IfLanguageName{#1}{true}{false}}{#2}%
  }%
  \test{ngerman}{true}%
  \test{naustrian}{false}%
  \selectlanguage{naustrian}%
  \test{ngerman}{false}%
  \test{naustrian}{true}%
  \test{foobar}{false}%
  %
  \def\languagename{naustrian}%
  \test{naustrian}{true}%
  \test{ngerman}{false}%
  %
  \edef\languagename{\string naustrian}%
  \test{naustrian}{true}%
  \test{ngerman}{false}%
  %
  \def\languagename{naustrian}%
  \makeatletter
  \@onelevel@sanitize\languagename
  \test{naustrian}{true}%
  \test{ngerman}{false}%
  %
  \def\languagename{naustrian}%
  \def\xaustrian{naustrian}%
  \def\xgerman{ngerman}%
  \test{\xaustrian}{true}%
  \test{\xgerman}{false}%
  %
  \def\languagename{\xaustrian}%
  \test{naustrian}{true}%
  \test{ngerman}{false}%
  \test{\xaustrian}{true}%
  \test{\xgerman}{false}%
  \test{\languagename}{true}%
  \test{\languagename\space}{false}%
  %
  \def\languagename{\empty\xaustrian\empty}%
  \test{naustrian}{true}%
  \test{ngerman}{false}%
  \test{\empty\xaustrian\empty}{true}%
  \test{\empty\xgerman\empty}{false}%
\end{qstest}
\begin{qstest}{IfDefined}{defined}
  \makeatletter
  \let\foobar\relax
  \Expect*{\IfLang@IfDefined{foobar}{true}{false}}{false}%
  \Expect*{\ifx\foobar\relax true\else false\fi}{true}%
  \let\foobar\UNDEFINED
  \Expect*{\IfLang@IfDefined{foobar}{true}{false}}{false}%
  \Expect*{\ifx\foobar\relax true\else false\fi}{false}%
  \Expect*{\ifx\foobar\UNDEFINED true\else false\fi}{true}%
\end{qstest}
\end{document}
%    \end{macrocode}
%    \begin{macrocode}
%</test2|test3>
%    \end{macrocode}
%
% \subsection{Test with \plainTeX\ and \eTeX}
%
%    \begin{macrocode}
%<*test4>
%    \end{macrocode}
%    \begin{macrocode}
%% Format `etex' based on `language.def'
\input iflang.sty
\catcode64=12

\def\TestGeneric#1#2#3{%
  \begingroup
    \edef\x{#1{#2}{true}{false}}%
    \edef\y{#3}%
    \ifx\x\y
    \else
      \errmessage{Failed test: \string#1{#2} <> #3}%
    \fi
  \endgroup
}
\def\TestPatterns{\TestGeneric\IfLanguagePatterns}
\def\TestName{\TestGeneric\IfLanguageName}

\TestPatterns{USenglish}{true}
\TestPatterns{ngerman}{false}

\TestName{USenglish}{true}
\TestName{ngerman}{false}

\uselanguage{ngerman}

\TestPatterns{USenglish}{false}
\TestPatterns{ngerman}{true}

\TestName{USenglish}{false}
\TestName{ngerman}{true}

\csname @@end\endcsname
\end
%    \end{macrocode}
%    \begin{macrocode}
%</test4>
%    \end{macrocode}
%
% \subsection{Test with \plainTeX\ and without \eTeX/\pdfTeX}
%
%    \begin{macrocode}
%<*test5>
%    \end{macrocode}
%    \begin{macrocode}
%% Format `tex' (vanilla plain-TeX)
\let\ifcsname\UNDEFINED
\let\pdfstrcmp\UNDEFINED
\input iflang.sty
\catcode64=11

\def\TestDefined#1{%
  \IfLang@IfDefined{foobar}{}{}%
  \ifx\foobar#1%
  \else
    \errmessage{Failed test: \string\foobar <> \string#1}%
  \fi
}
\let\foobar\relax
\TestDefined\relax
\let\foobar\UNDEFINED
\TestDefined\relax

\def\strip@prefix#1>{}
\def\@onelevel@sanitize#1{%
  \edef#1{\expandafter\strip@prefix\meaning#1}%
}
\def\TestCompare#1#2#3{%
  \begingroup
    \edef\x{%
      \if\IfLang@StrEqual{#1}{#2}%
        true%
      \else
        false%
      \fi
    }%
    \def\expect{#3}%
    \ifx\x\expect
    \else
      \def\a{#1}%
      \@onelevel@sanitize\a
      \def\b{#2}%
      \@onelevel@sanitize\b
      \errmessage{Failed test: `\a'='\b' <> \expect}%
    \fi
  \endgroup
}
\TestCompare{junk}{junk}{true}
\TestCompare{}{}{true}
\TestCompare{a}{b}{false}
\TestCompare{aa}{bb}{false}
\def\a{ax}
\def\b{bx}
\def\c{\a\b}
\def\d{\c\b}
\def\exch#1#2{#2#1}
\def\gobble#1{}
\TestCompare{\gobble a}{}{true}
\TestCompare{}{\gobble a}{true}
\TestCompare{a\exch xyb}{ayxb}{true}
\TestCompare{\c}{\c}{true}
\TestCompare{\d}{\c\b}{true}

\csname @@end\endcsname
\end
%    \end{macrocode}
%    \begin{macrocode}
%</test5>
%    \end{macrocode}
%
% \section{Installation}
%
% \subsection{Download}
%
% \paragraph{Package.} This package is available on
% CTAN\footnote{\url{http://ctan.org/pkg/iflang}}:
% \begin{description}
% \item[\CTAN{macros/latex/contrib/oberdiek/iflang.dtx}] The source file.
% \item[\CTAN{macros/latex/contrib/oberdiek/iflang.pdf}] Documentation.
% \end{description}
%
%
% \paragraph{Bundle.} All the packages of the bundle `oberdiek'
% are also available in a TDS compliant ZIP archive. There
% the packages are already unpacked and the documentation files
% are generated. The files and directories obey the TDS standard.
% \begin{description}
% \item[\CTAN{install/macros/latex/contrib/oberdiek.tds.zip}]
% \end{description}
% \emph{TDS} refers to the standard ``A Directory Structure
% for \TeX\ Files'' (\CTAN{tds/tds.pdf}). Directories
% with \xfile{texmf} in their name are usually organized this way.
%
% \subsection{Bundle installation}
%
% \paragraph{Unpacking.} Unpack the \xfile{oberdiek.tds.zip} in the
% TDS tree (also known as \xfile{texmf} tree) of your choice.
% Example (linux):
% \begin{quote}
%   |unzip oberdiek.tds.zip -d ~/texmf|
% \end{quote}
%
% \paragraph{Script installation.}
% Check the directory \xfile{TDS:scripts/oberdiek/} for
% scripts that need further installation steps.
% Package \xpackage{attachfile2} comes with the Perl script
% \xfile{pdfatfi.pl} that should be installed in such a way
% that it can be called as \texttt{pdfatfi}.
% Example (linux):
% \begin{quote}
%   |chmod +x scripts/oberdiek/pdfatfi.pl|\\
%   |cp scripts/oberdiek/pdfatfi.pl /usr/local/bin/|
% \end{quote}
%
% \subsection{Package installation}
%
% \paragraph{Unpacking.} The \xfile{.dtx} file is a self-extracting
% \docstrip\ archive. The files are extracted by running the
% \xfile{.dtx} through \plainTeX:
% \begin{quote}
%   \verb|tex iflang.dtx|
% \end{quote}
%
% \paragraph{TDS.} Now the different files must be moved into
% the different directories in your installation TDS tree
% (also known as \xfile{texmf} tree):
% \begin{quote}
% \def\t{^^A
% \begin{tabular}{@{}>{\ttfamily}l@{ $\rightarrow$ }>{\ttfamily}l@{}}
%   iflang.sty & tex/generic/oberdiek/iflang.sty\\
%   iflang.pdf & doc/latex/oberdiek/iflang.pdf\\
%   test/iflang-test1.tex & doc/latex/oberdiek/test/iflang-test1.tex\\
%   test/iflang-test2.tex & doc/latex/oberdiek/test/iflang-test2.tex\\
%   test/iflang-test3.tex & doc/latex/oberdiek/test/iflang-test3.tex\\
%   test/iflang-test4.tex & doc/latex/oberdiek/test/iflang-test4.tex\\
%   test/iflang-test5.tex & doc/latex/oberdiek/test/iflang-test5.tex\\
%   iflang.dtx & source/latex/oberdiek/iflang.dtx\\
% \end{tabular}^^A
% }^^A
% \sbox0{\t}^^A
% \ifdim\wd0>\linewidth
%   \begingroup
%     \advance\linewidth by\leftmargin
%     \advance\linewidth by\rightmargin
%   \edef\x{\endgroup
%     \def\noexpand\lw{\the\linewidth}^^A
%   }\x
%   \def\lwbox{^^A
%     \leavevmode
%     \hbox to \linewidth{^^A
%       \kern-\leftmargin\relax
%       \hss
%       \usebox0
%       \hss
%       \kern-\rightmargin\relax
%     }^^A
%   }^^A
%   \ifdim\wd0>\lw
%     \sbox0{\small\t}^^A
%     \ifdim\wd0>\linewidth
%       \ifdim\wd0>\lw
%         \sbox0{\footnotesize\t}^^A
%         \ifdim\wd0>\linewidth
%           \ifdim\wd0>\lw
%             \sbox0{\scriptsize\t}^^A
%             \ifdim\wd0>\linewidth
%               \ifdim\wd0>\lw
%                 \sbox0{\tiny\t}^^A
%                 \ifdim\wd0>\linewidth
%                   \lwbox
%                 \else
%                   \usebox0
%                 \fi
%               \else
%                 \lwbox
%               \fi
%             \else
%               \usebox0
%             \fi
%           \else
%             \lwbox
%           \fi
%         \else
%           \usebox0
%         \fi
%       \else
%         \lwbox
%       \fi
%     \else
%       \usebox0
%     \fi
%   \else
%     \lwbox
%   \fi
% \else
%   \usebox0
% \fi
% \end{quote}
% If you have a \xfile{docstrip.cfg} that configures and enables \docstrip's
% TDS installing feature, then some files can already be in the right
% place, see the documentation of \docstrip.
%
% \subsection{Refresh file name databases}
%
% If your \TeX~distribution
% (\teTeX, \mikTeX, \dots) relies on file name databases, you must refresh
% these. For example, \teTeX\ users run \verb|texhash| or
% \verb|mktexlsr|.
%
% \subsection{Some details for the interested}
%
% \paragraph{Attached source.}
%
% The PDF documentation on CTAN also includes the
% \xfile{.dtx} source file. It can be extracted by
% AcrobatReader 6 or higher. Another option is \textsf{pdftk},
% e.g. unpack the file into the current directory:
% \begin{quote}
%   \verb|pdftk iflang.pdf unpack_files output .|
% \end{quote}
%
% \paragraph{Unpacking with \LaTeX.}
% The \xfile{.dtx} chooses its action depending on the format:
% \begin{description}
% \item[\plainTeX:] Run \docstrip\ and extract the files.
% \item[\LaTeX:] Generate the documentation.
% \end{description}
% If you insist on using \LaTeX\ for \docstrip\ (really,
% \docstrip\ does not need \LaTeX), then inform the autodetect routine
% about your intention:
% \begin{quote}
%   \verb|latex \let\install=y% \iffalse meta-comment
%
% File: iflang.dtx
% Version: 2016/05/16 v1.6
% Info: Checks for the current language
%
% Copyright (C) 2007 by
%    Heiko Oberdiek <heiko.oberdiek at googlemail.com>
%    2016
%    https://github.com/ho-tex/oberdiek/issues
%
% This work may be distributed and/or modified under the
% conditions of the LaTeX Project Public License, either
% version 1.3c of this license or (at your option) any later
% version. This version of this license is in
%    http://www.latex-project.org/lppl/lppl-1-3c.txt
% and the latest version of this license is in
%    http://www.latex-project.org/lppl.txt
% and version 1.3 or later is part of all distributions of
% LaTeX version 2005/12/01 or later.
%
% This work has the LPPL maintenance status "maintained".
%
% This Current Maintainer of this work is Heiko Oberdiek.
%
% The Base Interpreter refers to any `TeX-Format',
% because some files are installed in TDS:tex/generic//.
%
% This work consists of the main source file iflang.dtx
% and the derived files
%    iflang.sty, iflang.pdf, iflang.ins, iflang.drv, iflang-test1.tex,
%    iflang-test2.tex, iflang-test3.tex, iflang-test4.tex,
%    iflang-test5.tex.
%
% Distribution:
%    CTAN:macros/latex/contrib/oberdiek/iflang.dtx
%    CTAN:macros/latex/contrib/oberdiek/iflang.pdf
%
% Unpacking:
%    (a) If iflang.ins is present:
%           tex iflang.ins
%    (b) Without iflang.ins:
%           tex iflang.dtx
%    (c) If you insist on using LaTeX
%           latex \let\install=y\input{iflang.dtx}
%        (quote the arguments according to the demands of your shell)
%
% Documentation:
%    (a) If iflang.drv is present:
%           latex iflang.drv
%    (b) Without iflang.drv:
%           latex iflang.dtx; ...
%    The class ltxdoc loads the configuration file ltxdoc.cfg
%    if available. Here you can specify further options, e.g.
%    use A4 as paper format:
%       \PassOptionsToClass{a4paper}{article}
%
%    Programm calls to get the documentation (example):
%       pdflatex iflang.dtx
%       makeindex -s gind.ist iflang.idx
%       pdflatex iflang.dtx
%       makeindex -s gind.ist iflang.idx
%       pdflatex iflang.dtx
%
% Installation:
%    TDS:tex/generic/oberdiek/iflang.sty
%    TDS:doc/latex/oberdiek/iflang.pdf
%    TDS:doc/latex/oberdiek/test/iflang-test1.tex
%    TDS:doc/latex/oberdiek/test/iflang-test2.tex
%    TDS:doc/latex/oberdiek/test/iflang-test3.tex
%    TDS:doc/latex/oberdiek/test/iflang-test4.tex
%    TDS:doc/latex/oberdiek/test/iflang-test5.tex
%    TDS:source/latex/oberdiek/iflang.dtx
%
%<*ignore>
\begingroup
  \catcode123=1 %
  \catcode125=2 %
  \def\x{LaTeX2e}%
\expandafter\endgroup
\ifcase 0\ifx\install y1\fi\expandafter
         \ifx\csname processbatchFile\endcsname\relax\else1\fi
         \ifx\fmtname\x\else 1\fi\relax
\else\csname fi\endcsname
%</ignore>
%<*install>
\input docstrip.tex
\Msg{************************************************************************}
\Msg{* Installation}
\Msg{* Package: iflang 2016/05/16 v1.6 Checks for the current language (HO)}
\Msg{************************************************************************}

\keepsilent
\askforoverwritefalse

\let\MetaPrefix\relax
\preamble

This is a generated file.

Project: iflang
Version: 2016/05/16 v1.6

Copyright (C) 2007 by
   Heiko Oberdiek <heiko.oberdiek at googlemail.com>

This work may be distributed and/or modified under the
conditions of the LaTeX Project Public License, either
version 1.3c of this license or (at your option) any later
version. This version of this license is in
   http://www.latex-project.org/lppl/lppl-1-3c.txt
and the latest version of this license is in
   http://www.latex-project.org/lppl.txt
and version 1.3 or later is part of all distributions of
LaTeX version 2005/12/01 or later.

This work has the LPPL maintenance status "maintained".

This Current Maintainer of this work is Heiko Oberdiek.

The Base Interpreter refers to any `TeX-Format',
because some files are installed in TDS:tex/generic//.

This work consists of the main source file iflang.dtx
and the derived files
   iflang.sty, iflang.pdf, iflang.ins, iflang.drv, iflang-test1.tex,
   iflang-test2.tex, iflang-test3.tex, iflang-test4.tex,
   iflang-test5.tex.

\endpreamble
\let\MetaPrefix\DoubleperCent

\generate{%
  \file{iflang.ins}{\from{iflang.dtx}{install}}%
  \file{iflang.drv}{\from{iflang.dtx}{driver}}%
  \usedir{tex/generic/oberdiek}%
  \file{iflang.sty}{\from{iflang.dtx}{package}}%
  \usedir{doc/latex/oberdiek/test}%
  \file{iflang-test1.tex}{\from{iflang.dtx}{test1}}%
  \file{iflang-test2.tex}{\from{iflang.dtx}{test2}}%
  \file{iflang-test3.tex}{\from{iflang.dtx}{test3}}%
  \file{iflang-test4.tex}{\from{iflang.dtx}{test4}}%
  \file{iflang-test5.tex}{\from{iflang.dtx}{test5}}%
  \nopreamble
  \nopostamble
  \usedir{source/latex/oberdiek/catalogue}%
  \file{iflang.xml}{\from{iflang.dtx}{catalogue}}%
}

\catcode32=13\relax% active space
\let =\space%
\Msg{************************************************************************}
\Msg{*}
\Msg{* To finish the installation you have to move the following}
\Msg{* file into a directory searched by TeX:}
\Msg{*}
\Msg{*     iflang.sty}
\Msg{*}
\Msg{* To produce the documentation run the file `iflang.drv'}
\Msg{* through LaTeX.}
\Msg{*}
\Msg{* Happy TeXing!}
\Msg{*}
\Msg{************************************************************************}

\endbatchfile
%</install>
%<*ignore>
\fi
%</ignore>
%<*driver>
\NeedsTeXFormat{LaTeX2e}
\ProvidesFile{iflang.drv}%
  [2016/05/16 v1.6 Checks for the current language (HO)]%
\documentclass{ltxdoc}
\usepackage{holtxdoc}[2011/11/22]
\begin{document}
  \DocInput{iflang.dtx}%
\end{document}
%</driver>
% \fi
%
%
% \CharacterTable
%  {Upper-case    \A\B\C\D\E\F\G\H\I\J\K\L\M\N\O\P\Q\R\S\T\U\V\W\X\Y\Z
%   Lower-case    \a\b\c\d\e\f\g\h\i\j\k\l\m\n\o\p\q\r\s\t\u\v\w\x\y\z
%   Digits        \0\1\2\3\4\5\6\7\8\9
%   Exclamation   \!     Double quote  \"     Hash (number) \#
%   Dollar        \$     Percent       \%     Ampersand     \&
%   Acute accent  \'     Left paren    \(     Right paren   \)
%   Asterisk      \*     Plus          \+     Comma         \,
%   Minus         \-     Point         \.     Solidus       \/
%   Colon         \:     Semicolon     \;     Less than     \<
%   Equals        \=     Greater than  \>     Question mark \?
%   Commercial at \@     Left bracket  \[     Backslash     \\
%   Right bracket \]     Circumflex    \^     Underscore    \_
%   Grave accent  \`     Left brace    \{     Vertical bar  \|
%   Right brace   \}     Tilde         \~}
%
% \GetFileInfo{iflang.drv}
%
% \title{The \xpackage{iflang} package}
% \date{2016/05/16 v1.6}
% \author{Heiko Oberdiek\thanks
% {Please report any issues at https://github.com/ho-tex/oberdiek/issues}\\
% \xemail{heiko.oberdiek at googlemail.com}}
%
% \maketitle
%
% \begin{abstract}
% This package provides expandible checks for the current language
% based on macro \cs{languagename} or hyphenation patterns.
% \end{abstract}
%
% \tableofcontents
%
% \section{Documentation}
%
% Package \xpackage{babel} defines \cs{iflanguagename}. As first
% argument it takes a language name and executes the second or
% third argument depending on the current language. This
% language test is based on hypenation patterns. However, it
% is possible that different languages or dialects share the
% same patterns. In such cases \cs{iflanguagename} fails.
%
% However, package \xpackage{babel} and some other packages
% such as \xpackage{german} or \xpackage{ngerman} store
% the language name in the macro \cs{languagename} if
% \cs{selectlanguage} is called.
%
% \begin{declcs}{IfLanguageName} \M{lang} \M{then} \M{else}
% \end{declcs}
% Makro \cs{IfLanguageName} compares language \meta{lang} with
% the current setting of macro \cs{languagename}. If both
% contains the same name then the \meta{then} part is called,
% otherwise the \meta{else} part.
%
% The macro is expandable. Thus it can be safely used inside \cs{edef}
% or \cs{csname}. If case of errors like an undefined \cs{languagename}
% the \meta{else} part is executed.
%
% Note: Macro \cs{IfLanguageName} relies on the fact, that
% \cs{languagename} is set correctly:
% \begin{description}
% \item[Package \xpackage{babel}:]
%    \mbox{}\\
%    Full support of \cs{languagename} in its language switching commands.
% \item[Format based on \xpackage{babel} (\xfile{language.dat}):]
%    \mbox{}\\
%    If package \xpackage{babel} is not used (or not yet loaded),
%    then \xpackage{babel}'s
%    \xfile{hyphen.cfg} has set \cs{languagename} to the last
%    language in \xfile{language.dat}, but \cs{language} (current
%    patterns) is zero and points to the first language.
%    Thus the value of \cs{languagename} is basically garbage.
%    Package \xpackage{iflang} warns if \cs{languagename} and
%    \cs{language} do not fit.
%    This can be fixed by loading package \xpackage{babel} previously.
% \item[Format based on \eTeX's \xfile{etex.src} (\xfile{language.def}):]
%    \mbox{}\\
%    Unhappily it does not support
%    \cs{languagename}. Thus this package hooks into \cs{uselanguage}
%    to get \cs{languagename} defined and updated there.
%    At package loading time the changed \cs{uselanguage} has not
%    been called yet. Thus package \xpackage{iflang} tries
%    |USenglish|. This is the definite default language of \xfile{etex.src}.
%    If the current patterns suit this default language, an
%    undefined \cs{languagename} is set to this language. Otherwise
%    a \cs{languagename} remains undefined and a warning is given.
% \end{description}
%
% \begin{declcs}{IfLanguagePatterns} \M{lang} \M{then} \M{else}
% \end{declcs}
% This macro behaves similar to \cs{IfLanguageName}. But the
% language test is based on the current pattern in force (\cs{language}).
% Also this macro is expandable, in case of errors the \meta{else} part
% is called.
%
% The following naming convention for the pattern are supported:
% \begin{description}
% \item[\xpackage{babel}/\xfile{language.dat}]:
%    \cs{l@}\meta{language}
% \item[\xpackage{etex.src}/\xfile{language.def}]:
%    \cs{lang@}\meta{language}
% \end{description}
% Package \xpackage{iflang} looks for \cs{uselanguage} (defined in
% \xfile{etex.src}) to find out the naming convention in use.
%
% \StopEventually{
% }
%
% \section{Implementation}
%
%    \begin{macrocode}
%<*package>
%    \end{macrocode}
%
% \subsection{Reload check and package identification}
%    Reload check, especially if the package is not used with \LaTeX.
%    \begin{macrocode}
\begingroup\catcode61\catcode48\catcode32=10\relax%
  \catcode13=5 % ^^M
  \endlinechar=13 %
  \catcode35=6 % #
  \catcode39=12 % '
  \catcode44=12 % ,
  \catcode45=12 % -
  \catcode46=12 % .
  \catcode58=12 % :
  \catcode64=11 % @
  \catcode123=1 % {
  \catcode125=2 % }
  \expandafter\let\expandafter\x\csname ver@iflang.sty\endcsname
  \ifx\x\relax % plain-TeX, first loading
  \else
    \def\empty{}%
    \ifx\x\empty % LaTeX, first loading,
      % variable is initialized, but \ProvidesPackage not yet seen
    \else
      \expandafter\ifx\csname PackageInfo\endcsname\relax
        \def\x#1#2{%
          \immediate\write-1{Package #1 Info: #2.}%
        }%
      \else
        \def\x#1#2{\PackageInfo{#1}{#2, stopped}}%
      \fi
      \x{iflang}{The package is already loaded}%
      \aftergroup\endinput
    \fi
  \fi
\endgroup%
%    \end{macrocode}
%    Package identification:
%    \begin{macrocode}
\begingroup\catcode61\catcode48\catcode32=10\relax%
  \catcode13=5 % ^^M
  \endlinechar=13 %
  \catcode35=6 % #
  \catcode39=12 % '
  \catcode40=12 % (
  \catcode41=12 % )
  \catcode44=12 % ,
  \catcode45=12 % -
  \catcode46=12 % .
  \catcode47=12 % /
  \catcode58=12 % :
  \catcode64=11 % @
  \catcode91=12 % [
  \catcode93=12 % ]
  \catcode123=1 % {
  \catcode125=2 % }
  \expandafter\ifx\csname ProvidesPackage\endcsname\relax
    \def\x#1#2#3[#4]{\endgroup
      \immediate\write-1{Package: #3 #4}%
      \xdef#1{#4}%
    }%
  \else
    \def\x#1#2[#3]{\endgroup
      #2[{#3}]%
      \ifx#1\@undefined
        \xdef#1{#3}%
      \fi
      \ifx#1\relax
        \xdef#1{#3}%
      \fi
    }%
  \fi
\expandafter\x\csname ver@iflang.sty\endcsname
\ProvidesPackage{iflang}%
  [2016/05/16 v1.6 Checks for the current language (HO)]%
%    \end{macrocode}
%
%    \begin{macrocode}
\begingroup\catcode61\catcode48\catcode32=10\relax%
  \catcode13=5 % ^^M
  \endlinechar=13 %
  \catcode123=1 % {
  \catcode125=2 % }
  \catcode64=11 % @
  \def\x{\endgroup
    \expandafter\edef\csname IfLang@AtEnd\endcsname{%
      \endlinechar=\the\endlinechar\relax
      \catcode13=\the\catcode13\relax
      \catcode32=\the\catcode32\relax
      \catcode35=\the\catcode35\relax
      \catcode61=\the\catcode61\relax
      \catcode64=\the\catcode64\relax
      \catcode123=\the\catcode123\relax
      \catcode125=\the\catcode125\relax
    }%
  }%
\x\catcode61\catcode48\catcode32=10\relax%
\catcode13=5 % ^^M
\endlinechar=13 %
\catcode35=6 % #
\catcode64=11 % @
\catcode123=1 % {
\catcode125=2 % }
\def\TMP@EnsureCode#1#2{%
  \edef\IfLang@AtEnd{%
    \IfLang@AtEnd
    \catcode#1=\the\catcode#1\relax
  }%
  \catcode#1=#2\relax
}
\TMP@EnsureCode{39}{12}% '
\TMP@EnsureCode{40}{12}% (
\TMP@EnsureCode{41}{12}% )
\TMP@EnsureCode{44}{12}% ,
\TMP@EnsureCode{46}{12}% .
\TMP@EnsureCode{47}{12}% /
\TMP@EnsureCode{58}{12}% :
\TMP@EnsureCode{91}{12}% [
\TMP@EnsureCode{93}{12}% ]
\edef\IfLang@AtEnd{\IfLang@AtEnd\noexpand\endinput}
%    \end{macrocode}
%
% \subsection{Tools}
%
% \subsubsection{Provide some basic macros of \LaTeX}
%
%    \begin{macro}{\@firstoftwo}
%    \begin{macrocode}
\expandafter\ifx\csname @firstoftwo\endcsname\relax
  \long\def\@firstoftwo#1#2{#1}%
\fi
%    \end{macrocode}
%    \end{macro}
%    \begin{macro}{\@secondoftwo}
%    \begin{macrocode}
\expandafter\ifx\csname @secondoftwo\endcsname\relax
  \long\def\@secondoftwo#1#2{#2}%
\fi
%    \end{macrocode}
%    \end{macro}
%
% \subsubsection{Expandible existence check for macros}
%
%    \begin{macro}{\IfLang@IfDefined}
%    \begin{macrocode}
\begingroup\expandafter\expandafter\expandafter\endgroup
\expandafter\ifx\csname ifcsname\endcsname\relax
  \expandafter\@firstoftwo
\else
  \expandafter\@secondoftwo
\fi
{%
  \def\IfLang@IfDefined#1{%
    \expandafter\ifx\csname#1\endcsname\relax
      \expandafter\@secondoftwo
    \else
      \expandafter\@firstoftwo
    \fi
  }%
}{%
  \def\IfLang@IfDefined#1{%
    \ifnum\ifcsname#1\endcsname
            \expandafter\ifx\csname#1\endcsname\relax
              1%
            \else
              0%
            \fi
          \else
            1%
          \fi
          =0 %
      \expandafter\@firstoftwo
    \else
      \expandafter\@secondoftwo
    \fi
  }%
}
%    \end{macrocode}
%    \end{macro}
%
% \subsubsection{Macros for messages}
%
%    \begin{macrocode}
\begingroup\expandafter\expandafter\expandafter\endgroup
\expandafter\ifx\csname RequirePackage\endcsname\relax
  \input infwarerr.sty\relax
  \input pdftexcmds.sty\relax
\else
  \RequirePackage{infwarerr}[2007/09/09]%
  \RequirePackage{pdftexcmds}[2016/05/16]%
\fi
%    \end{macrocode}
%
% \subsubsection{Support for \xfile{etex.src}}
%
%    \begin{macro}{\IfLang@prefix}
%    \begin{macrocode}
\begingroup\expandafter\expandafter\expandafter\endgroup
\expandafter\ifx\csname uselanguage\endcsname\relax
  \@PackageInfoNoLine{iflang}{%
    Naming convention for patterns: babel%
  }%
  \def\IfLang@prefix{l@}%
\else
  \@PackageInfoNoLine{iflang}{%
    Naming convention for patterns: etex.src%
  }%
  \def\IfLang@prefix{lang@}%
  \let\IfLang@OrgUseLanguage\uselanguage
  \def\uselanguage#1{%
    \edef\languagename{#1}%
    \IfLang@OrgUseLanguage{#1}%
  }%
%    \end{macrocode}
%    \end{macro}
%    The first \cs{uselanguage} that is executed as last
%    line in \xfile{language.def} cannot patched this way.
%    However, \xfile{language.def} is very strict. It forces
%    the first added and used language to be |USenglish|.
%    Thus, if \cs{languagename} is not defined, we can
%    quite safely assume |USenglish|. As additional
%    safety precaution the actual used patterns are checked.
%    \begin{macrocode}
  \begingroup\expandafter\expandafter\expandafter\endgroup
  \expandafter\ifx\csname languagename\endcsname\relax
    \begingroup\expandafter\expandafter\expandafter\endgroup
    \expandafter\ifx\csname lang@USenglish\endcsname\relax
      \@PackageWarningNoLine{iflang}{%
        \string\lang@USenglish\space is missing%
      }%
    \else
      \ifnum\lang@USenglish=\language
        \def\languagename{USenglish}%
      \else
        \@PackageWarningNoLine{iflang}{%
          \string\languagename\space is not set,\MessageBreak
          current language is unknown%
        }%
      \fi
    \fi
  \fi
\fi
\begingroup\expandafter\expandafter\expandafter\endgroup
\expandafter\ifx\csname languagename\endcsname\relax
  \@PackageInfoNoLine{iflang}{%
    \string\languagename\space is not set%
  }%
\fi
%    \end{macrocode}
%
% \subsection{\cs{IfLanguagePatterns}}
%
%    \begin{macro}{\IfLanguagePatterns}
%    \begin{macrocode}
\def\IfLanguagePatterns#1{%
  \ifnum\IfLang@IfDefined{\IfLang@prefix#1}{%
          \ifnum\csname\IfLang@prefix#1\endcsname=\language
            0%
          \else
            1%
          \fi
        }{1}=0 %
    \expandafter\@firstoftwo
  \else
    \expandafter\@secondoftwo
  \fi
}
%    \end{macrocode}
%    \end{macro}
%
% \subsection{\cs{IfLanguageName}}
%
%    \begin{macrocode}
\begingroup\expandafter\expandafter\expandafter\endgroup
\expandafter\ifx\csname pdf@strcmp\endcsname\relax
  \expandafter\@firstoftwo
\else
  \expandafter\@secondoftwo
\fi
{%
%    \end{macrocode}
%    We do not have \cs{pdf@strcmp} (and \cs{pdfstrcmp}).
%    Thus we must define
%    our own expandable string comparison. The following
%    implementation is based on a \TeX\ pearl from David Kastrup, presented
%    at the conference Bacho\TeX~2005:
%    \url{http://www-stary.gust.org.pl/pearls/2005/david-kastrup/bachotex2005-david-kastrup-pearl1.pdf}
%
%    The orignal code allows macros inside the second string.
%    Because also \cs{languagename} might consists of further
%    macros, we need a variant that allows macros in the first
%    string, too.
%    \begin{macrocode}
  \def\IfLang@StrNil{\relax}%
  \def\IfLang@StrEqual#1{%
    \number\IfLang@StrEqualStart{}{}#1\IfLang@StrNil
  }%
  \def\IfLang@StrEqualStart#1#2#3{%
    \ifx#3\IfLang@StrNil
      \IfLang@StrEqualStop
    \fi
    \ifcat\noexpand#3\relax
      \IfLang@StrExpand{#1}{#2}#3%
    \fi
    \IfLang@StrEqualStart{\if#3#1}{#2\fi}%
  }%
  \def\IfLang@StrEqualStop\fi#1\IfLang@StrEqualStart#2#3#4{%
    \fi
    #2#4\relax'#313 %
  }%
  \def\IfLang@StrExpand#1#2#3\fi\IfLang@StrEqualStart#4#5{%
    \fi
    \IfLang@@StrExpand{#1}{#2}#3%
  }%
  \def\IfLang@@StrExpand#1#2#3\IfLang@StrNil{%
    \expandafter\IfLang@@@StrExpand#3\IfLang@StrNil{#1}{#2}%
  }%
  \def\IfLang@@@StrExpand#1\IfLang@StrNil#2#3{%
    \IfLang@StrEqualStart{#2}{#3}#1\IfLang@StrNil
  }%
%    \end{macrocode}
%    \begin{macro}{\IfLanguageName}
%    \begin{macrocode}
  \def\IfLanguageName#1{%
    \ifnum\IfLang@IfDefined{languagename}{%
            \if\expandafter\IfLang@StrEqual\expandafter%
                           {\languagename}{#1}%
              0%
            \else
              1%
            \fi
          }{1}=0 %
      \expandafter\@firstoftwo
    \else
      \expandafter\@secondoftwo
    \fi
  }%
%    \end{macrocode}
%    \end{macro}
%    \begin{macrocode}
}{%
%    \end{macrocode}
%    \begin{macro}{\IfLanguageName}
%    \begin{macrocode}
  \def\IfLanguageName#1{%
    \ifnum\IfLang@IfDefined{languagename}{%
            \pdf@strcmp{#1}{\languagename}%
          }{1}=0 %
      \expandafter\@firstoftwo
    \else
      \expandafter\@secondoftwo
    \fi
  }%
%    \end{macrocode}
%    \end{macro}
%    \begin{macrocode}
}
%    \end{macrocode}
%
% \subsection{Check plausibility of \cs{languagename}}
%
%    \begin{macrocode}
\begingroup\expandafter\expandafter\expandafter\endgroup
\expandafter\ifx\csname languagename\endcsname\relax
\else
  \IfLanguagePatterns{\languagename}{}{%
    \@PackageWarningNoLine{iflang}{%
      Mismatch between \string\language\space
      (patterns)\MessageBreak
      and setting of \string\languagename
    }%
  }%
\fi
%    \end{macrocode}
%
%    \begin{macrocode}
\IfLang@AtEnd%
%</package>
%    \end{macrocode}
%
% \section{Test}
%
% \subsection{Catcode checks for loading}
%
%    \begin{macrocode}
%<*test1>
%    \end{macrocode}
%    \begin{macrocode}
\catcode`\{=1 %
\catcode`\}=2 %
\catcode`\#=6 %
\catcode`\@=11 %
\expandafter\ifx\csname count@\endcsname\relax
  \countdef\count@=255 %
\fi
\expandafter\ifx\csname @gobble\endcsname\relax
  \long\def\@gobble#1{}%
\fi
\expandafter\ifx\csname @firstofone\endcsname\relax
  \long\def\@firstofone#1{#1}%
\fi
\expandafter\ifx\csname loop\endcsname\relax
  \expandafter\@firstofone
\else
  \expandafter\@gobble
\fi
{%
  \def\loop#1\repeat{%
    \def\body{#1}%
    \iterate
  }%
  \def\iterate{%
    \body
      \let\next\iterate
    \else
      \let\next\relax
    \fi
    \next
  }%
  \let\repeat=\fi
}%
\def\RestoreCatcodes{}
\count@=0 %
\loop
  \edef\RestoreCatcodes{%
    \RestoreCatcodes
    \catcode\the\count@=\the\catcode\count@\relax
  }%
\ifnum\count@<255 %
  \advance\count@ 1 %
\repeat

\def\RangeCatcodeInvalid#1#2{%
  \count@=#1\relax
  \loop
    \catcode\count@=15 %
  \ifnum\count@<#2\relax
    \advance\count@ 1 %
  \repeat
}
\def\RangeCatcodeCheck#1#2#3{%
  \count@=#1\relax
  \loop
    \ifnum#3=\catcode\count@
    \else
      \errmessage{%
        Character \the\count@\space
        with wrong catcode \the\catcode\count@\space
        instead of \number#3%
      }%
    \fi
  \ifnum\count@<#2\relax
    \advance\count@ 1 %
  \repeat
}
\def\space{ }
\expandafter\ifx\csname LoadCommand\endcsname\relax
  \def\LoadCommand{\input iflang.sty\relax}%
\fi
\def\Test{%
  \RangeCatcodeInvalid{0}{47}%
  \RangeCatcodeInvalid{58}{64}%
  \RangeCatcodeInvalid{91}{96}%
  \RangeCatcodeInvalid{123}{255}%
  \catcode`\@=12 %
  \catcode`\\=0 %
  \catcode`\%=14 %
  \LoadCommand
  \RangeCatcodeCheck{0}{36}{15}%
  \RangeCatcodeCheck{37}{37}{14}%
  \RangeCatcodeCheck{38}{47}{15}%
  \RangeCatcodeCheck{48}{57}{12}%
  \RangeCatcodeCheck{58}{63}{15}%
  \RangeCatcodeCheck{64}{64}{12}%
  \RangeCatcodeCheck{65}{90}{11}%
  \RangeCatcodeCheck{91}{91}{15}%
  \RangeCatcodeCheck{92}{92}{0}%
  \RangeCatcodeCheck{93}{96}{15}%
  \RangeCatcodeCheck{97}{122}{11}%
  \RangeCatcodeCheck{123}{255}{15}%
  \RestoreCatcodes
}
\Test
\csname @@end\endcsname
\end
%    \end{macrocode}
%    \begin{macrocode}
%</test1>
%    \end{macrocode}
%
% \subsection{Test with \LaTeX}
%
%    \begin{macrocode}
%<*test2|test3>
%    \end{macrocode}
%    \begin{macrocode}
\NeedsTeXFormat{LaTeX2e}
%<test3>\let\pdfstrcmp\relax
\nofiles
\documentclass{minimal}
\usepackage{qstest}
\IncludeTests{*}
\LogTests{log}{*}{*}
\usepackage[english,naustrian,ngerman]{babel}
\usepackage{iflang}
\begin{document}
\begin{qstest}{IfLanguagePatterns}{language, pattern}
  \def\test#1#2{%
    \Expect*{\IfLanguagePatterns{#1}{true}{false}}{#2}%
  }%
  \test{ngerman}{true}%
  \test{naustrian}{true}%
  \test{english}{false}%
  \test{foobar}{false}%
\end{qstest}
\begin{qstest}{IfLanguageName}{language, name}
  \def\test#1#2{%
    \Expect*{\IfLanguageName{#1}{true}{false}}{#2}%
  }%
  \test{ngerman}{true}%
  \test{naustrian}{false}%
  \selectlanguage{naustrian}%
  \test{ngerman}{false}%
  \test{naustrian}{true}%
  \test{foobar}{false}%
  %
  \def\languagename{naustrian}%
  \test{naustrian}{true}%
  \test{ngerman}{false}%
  %
  \edef\languagename{\string naustrian}%
  \test{naustrian}{true}%
  \test{ngerman}{false}%
  %
  \def\languagename{naustrian}%
  \makeatletter
  \@onelevel@sanitize\languagename
  \test{naustrian}{true}%
  \test{ngerman}{false}%
  %
  \def\languagename{naustrian}%
  \def\xaustrian{naustrian}%
  \def\xgerman{ngerman}%
  \test{\xaustrian}{true}%
  \test{\xgerman}{false}%
  %
  \def\languagename{\xaustrian}%
  \test{naustrian}{true}%
  \test{ngerman}{false}%
  \test{\xaustrian}{true}%
  \test{\xgerman}{false}%
  \test{\languagename}{true}%
  \test{\languagename\space}{false}%
  %
  \def\languagename{\empty\xaustrian\empty}%
  \test{naustrian}{true}%
  \test{ngerman}{false}%
  \test{\empty\xaustrian\empty}{true}%
  \test{\empty\xgerman\empty}{false}%
\end{qstest}
\begin{qstest}{IfDefined}{defined}
  \makeatletter
  \let\foobar\relax
  \Expect*{\IfLang@IfDefined{foobar}{true}{false}}{false}%
  \Expect*{\ifx\foobar\relax true\else false\fi}{true}%
  \let\foobar\UNDEFINED
  \Expect*{\IfLang@IfDefined{foobar}{true}{false}}{false}%
  \Expect*{\ifx\foobar\relax true\else false\fi}{false}%
  \Expect*{\ifx\foobar\UNDEFINED true\else false\fi}{true}%
\end{qstest}
\end{document}
%    \end{macrocode}
%    \begin{macrocode}
%</test2|test3>
%    \end{macrocode}
%
% \subsection{Test with \plainTeX\ and \eTeX}
%
%    \begin{macrocode}
%<*test4>
%    \end{macrocode}
%    \begin{macrocode}
%% Format `etex' based on `language.def'
\input iflang.sty
\catcode64=12

\def\TestGeneric#1#2#3{%
  \begingroup
    \edef\x{#1{#2}{true}{false}}%
    \edef\y{#3}%
    \ifx\x\y
    \else
      \errmessage{Failed test: \string#1{#2} <> #3}%
    \fi
  \endgroup
}
\def\TestPatterns{\TestGeneric\IfLanguagePatterns}
\def\TestName{\TestGeneric\IfLanguageName}

\TestPatterns{USenglish}{true}
\TestPatterns{ngerman}{false}

\TestName{USenglish}{true}
\TestName{ngerman}{false}

\uselanguage{ngerman}

\TestPatterns{USenglish}{false}
\TestPatterns{ngerman}{true}

\TestName{USenglish}{false}
\TestName{ngerman}{true}

\csname @@end\endcsname
\end
%    \end{macrocode}
%    \begin{macrocode}
%</test4>
%    \end{macrocode}
%
% \subsection{Test with \plainTeX\ and without \eTeX/\pdfTeX}
%
%    \begin{macrocode}
%<*test5>
%    \end{macrocode}
%    \begin{macrocode}
%% Format `tex' (vanilla plain-TeX)
\let\ifcsname\UNDEFINED
\let\pdfstrcmp\UNDEFINED
\input iflang.sty
\catcode64=11

\def\TestDefined#1{%
  \IfLang@IfDefined{foobar}{}{}%
  \ifx\foobar#1%
  \else
    \errmessage{Failed test: \string\foobar <> \string#1}%
  \fi
}
\let\foobar\relax
\TestDefined\relax
\let\foobar\UNDEFINED
\TestDefined\relax

\def\strip@prefix#1>{}
\def\@onelevel@sanitize#1{%
  \edef#1{\expandafter\strip@prefix\meaning#1}%
}
\def\TestCompare#1#2#3{%
  \begingroup
    \edef\x{%
      \if\IfLang@StrEqual{#1}{#2}%
        true%
      \else
        false%
      \fi
    }%
    \def\expect{#3}%
    \ifx\x\expect
    \else
      \def\a{#1}%
      \@onelevel@sanitize\a
      \def\b{#2}%
      \@onelevel@sanitize\b
      \errmessage{Failed test: `\a'='\b' <> \expect}%
    \fi
  \endgroup
}
\TestCompare{junk}{junk}{true}
\TestCompare{}{}{true}
\TestCompare{a}{b}{false}
\TestCompare{aa}{bb}{false}
\def\a{ax}
\def\b{bx}
\def\c{\a\b}
\def\d{\c\b}
\def\exch#1#2{#2#1}
\def\gobble#1{}
\TestCompare{\gobble a}{}{true}
\TestCompare{}{\gobble a}{true}
\TestCompare{a\exch xyb}{ayxb}{true}
\TestCompare{\c}{\c}{true}
\TestCompare{\d}{\c\b}{true}

\csname @@end\endcsname
\end
%    \end{macrocode}
%    \begin{macrocode}
%</test5>
%    \end{macrocode}
%
% \section{Installation}
%
% \subsection{Download}
%
% \paragraph{Package.} This package is available on
% CTAN\footnote{\url{http://ctan.org/pkg/iflang}}:
% \begin{description}
% \item[\CTAN{macros/latex/contrib/oberdiek/iflang.dtx}] The source file.
% \item[\CTAN{macros/latex/contrib/oberdiek/iflang.pdf}] Documentation.
% \end{description}
%
%
% \paragraph{Bundle.} All the packages of the bundle `oberdiek'
% are also available in a TDS compliant ZIP archive. There
% the packages are already unpacked and the documentation files
% are generated. The files and directories obey the TDS standard.
% \begin{description}
% \item[\CTAN{install/macros/latex/contrib/oberdiek.tds.zip}]
% \end{description}
% \emph{TDS} refers to the standard ``A Directory Structure
% for \TeX\ Files'' (\CTAN{tds/tds.pdf}). Directories
% with \xfile{texmf} in their name are usually organized this way.
%
% \subsection{Bundle installation}
%
% \paragraph{Unpacking.} Unpack the \xfile{oberdiek.tds.zip} in the
% TDS tree (also known as \xfile{texmf} tree) of your choice.
% Example (linux):
% \begin{quote}
%   |unzip oberdiek.tds.zip -d ~/texmf|
% \end{quote}
%
% \paragraph{Script installation.}
% Check the directory \xfile{TDS:scripts/oberdiek/} for
% scripts that need further installation steps.
% Package \xpackage{attachfile2} comes with the Perl script
% \xfile{pdfatfi.pl} that should be installed in such a way
% that it can be called as \texttt{pdfatfi}.
% Example (linux):
% \begin{quote}
%   |chmod +x scripts/oberdiek/pdfatfi.pl|\\
%   |cp scripts/oberdiek/pdfatfi.pl /usr/local/bin/|
% \end{quote}
%
% \subsection{Package installation}
%
% \paragraph{Unpacking.} The \xfile{.dtx} file is a self-extracting
% \docstrip\ archive. The files are extracted by running the
% \xfile{.dtx} through \plainTeX:
% \begin{quote}
%   \verb|tex iflang.dtx|
% \end{quote}
%
% \paragraph{TDS.} Now the different files must be moved into
% the different directories in your installation TDS tree
% (also known as \xfile{texmf} tree):
% \begin{quote}
% \def\t{^^A
% \begin{tabular}{@{}>{\ttfamily}l@{ $\rightarrow$ }>{\ttfamily}l@{}}
%   iflang.sty & tex/generic/oberdiek/iflang.sty\\
%   iflang.pdf & doc/latex/oberdiek/iflang.pdf\\
%   test/iflang-test1.tex & doc/latex/oberdiek/test/iflang-test1.tex\\
%   test/iflang-test2.tex & doc/latex/oberdiek/test/iflang-test2.tex\\
%   test/iflang-test3.tex & doc/latex/oberdiek/test/iflang-test3.tex\\
%   test/iflang-test4.tex & doc/latex/oberdiek/test/iflang-test4.tex\\
%   test/iflang-test5.tex & doc/latex/oberdiek/test/iflang-test5.tex\\
%   iflang.dtx & source/latex/oberdiek/iflang.dtx\\
% \end{tabular}^^A
% }^^A
% \sbox0{\t}^^A
% \ifdim\wd0>\linewidth
%   \begingroup
%     \advance\linewidth by\leftmargin
%     \advance\linewidth by\rightmargin
%   \edef\x{\endgroup
%     \def\noexpand\lw{\the\linewidth}^^A
%   }\x
%   \def\lwbox{^^A
%     \leavevmode
%     \hbox to \linewidth{^^A
%       \kern-\leftmargin\relax
%       \hss
%       \usebox0
%       \hss
%       \kern-\rightmargin\relax
%     }^^A
%   }^^A
%   \ifdim\wd0>\lw
%     \sbox0{\small\t}^^A
%     \ifdim\wd0>\linewidth
%       \ifdim\wd0>\lw
%         \sbox0{\footnotesize\t}^^A
%         \ifdim\wd0>\linewidth
%           \ifdim\wd0>\lw
%             \sbox0{\scriptsize\t}^^A
%             \ifdim\wd0>\linewidth
%               \ifdim\wd0>\lw
%                 \sbox0{\tiny\t}^^A
%                 \ifdim\wd0>\linewidth
%                   \lwbox
%                 \else
%                   \usebox0
%                 \fi
%               \else
%                 \lwbox
%               \fi
%             \else
%               \usebox0
%             \fi
%           \else
%             \lwbox
%           \fi
%         \else
%           \usebox0
%         \fi
%       \else
%         \lwbox
%       \fi
%     \else
%       \usebox0
%     \fi
%   \else
%     \lwbox
%   \fi
% \else
%   \usebox0
% \fi
% \end{quote}
% If you have a \xfile{docstrip.cfg} that configures and enables \docstrip's
% TDS installing feature, then some files can already be in the right
% place, see the documentation of \docstrip.
%
% \subsection{Refresh file name databases}
%
% If your \TeX~distribution
% (\teTeX, \mikTeX, \dots) relies on file name databases, you must refresh
% these. For example, \teTeX\ users run \verb|texhash| or
% \verb|mktexlsr|.
%
% \subsection{Some details for the interested}
%
% \paragraph{Attached source.}
%
% The PDF documentation on CTAN also includes the
% \xfile{.dtx} source file. It can be extracted by
% AcrobatReader 6 or higher. Another option is \textsf{pdftk},
% e.g. unpack the file into the current directory:
% \begin{quote}
%   \verb|pdftk iflang.pdf unpack_files output .|
% \end{quote}
%
% \paragraph{Unpacking with \LaTeX.}
% The \xfile{.dtx} chooses its action depending on the format:
% \begin{description}
% \item[\plainTeX:] Run \docstrip\ and extract the files.
% \item[\LaTeX:] Generate the documentation.
% \end{description}
% If you insist on using \LaTeX\ for \docstrip\ (really,
% \docstrip\ does not need \LaTeX), then inform the autodetect routine
% about your intention:
% \begin{quote}
%   \verb|latex \let\install=y\input{iflang.dtx}|
% \end{quote}
% Do not forget to quote the argument according to the demands
% of your shell.
%
% \paragraph{Generating the documentation.}
% You can use both the \xfile{.dtx} or the \xfile{.drv} to generate
% the documentation. The process can be configured by the
% configuration file \xfile{ltxdoc.cfg}. For instance, put this
% line into this file, if you want to have A4 as paper format:
% \begin{quote}
%   \verb|\PassOptionsToClass{a4paper}{article}|
% \end{quote}
% An example follows how to generate the
% documentation with pdf\LaTeX:
% \begin{quote}
%\begin{verbatim}
%pdflatex iflang.dtx
%makeindex -s gind.ist iflang.idx
%pdflatex iflang.dtx
%makeindex -s gind.ist iflang.idx
%pdflatex iflang.dtx
%\end{verbatim}
% \end{quote}
%
% \section{Catalogue}
%
% The following XML file can be used as source for the
% \href{http://mirror.ctan.org/help/Catalogue/catalogue.html}{\TeX\ Catalogue}.
% The elements \texttt{caption} and \texttt{description} are imported
% from the original XML file from the Catalogue.
% The name of the XML file in the Catalogue is \xfile{iflang.xml}.
%    \begin{macrocode}
%<*catalogue>
<?xml version='1.0' encoding='us-ascii'?>
<!DOCTYPE entry SYSTEM 'catalogue.dtd'>
<entry datestamp='$Date$' modifier='$Author$' id='iflang'>
  <name>iflang</name>
  <caption>Expandable checks for the current language.</caption>
  <authorref id='auth:oberdiek'/>
  <copyright owner='Heiko Oberdiek' year='2007'/>
  <license type='lppl1.3'/>
  <version number='1.6'/>
  <description>
    This package provides expandable checks for the current language
    based on macro <tt>\languagename</tt> or hyphenation patterns.
    <p/>
    The package is part of the <xref refid='oberdiek'>oberdiek</xref> bundle.
  </description>
  <documentation details='Package documentation'
      href='ctan:/macros/latex/contrib/oberdiek/iflang.pdf'/>
  <ctan file='true' path='/macros/latex/contrib/oberdiek/iflang.dtx'/>
  <miktex location='oberdiek'/>
  <texlive location='oberdiek'/>
  <install path='/macros/latex/contrib/oberdiek/oberdiek.tds.zip'/>
</entry>
%</catalogue>
%    \end{macrocode}
%
% \section{Acknowledgement}
% I wish to thank:
% \begin{description}
% \item[Markus Kohm] Useful hints for version 1.2.
% \end{description}
%
% \begin{History}
%   \begin{Version}{2007/04/10 v1.0}
%   \item
%     First public version.
%   \end{Version}
%   \begin{Version}{2007/04/11 v1.1}
%   \item
%     Line ends sanitized.
%   \end{Version}
%   \begin{Version}{2007/04/12 v1.2}
%   \item
%     Initialization of \cs{languagename} in case of \xfile{etex.src}.
%   \item
%     Some sanity tests added.
%   \item
%     Documentation improved.
%   \end{Version}
%   \begin{Version}{2007/04/26 v1.3}
%   \item
%     Use of package \xpackage{infwarerr}.
%   \end{Version}
%   \begin{Version}{2007/09/09 v1.4}
%   \item
%     Bug fix: \cs{IfLang@StrEqual} $\rightarrow$ \cs{IfLangStrEqual}
%     (Gabriele Balducci).
%   \item
%     Catcode section rewritten.
%   \end{Version}
%   \begin{Version}{2007/11/11 v1.5}
%   \item
%     Use of package \xpackage{pdftexcmds} for \LuaTeX\ support.
%   \end{Version}
%   \begin{Version}{2016/05/16 v1.6}
%   \item
%     Documentation updates.
%   \end{Version}
% \end{History}
%
% \PrintIndex
%
% \Finale
\endinput
|
% \end{quote}
% Do not forget to quote the argument according to the demands
% of your shell.
%
% \paragraph{Generating the documentation.}
% You can use both the \xfile{.dtx} or the \xfile{.drv} to generate
% the documentation. The process can be configured by the
% configuration file \xfile{ltxdoc.cfg}. For instance, put this
% line into this file, if you want to have A4 as paper format:
% \begin{quote}
%   \verb|\PassOptionsToClass{a4paper}{article}|
% \end{quote}
% An example follows how to generate the
% documentation with pdf\LaTeX:
% \begin{quote}
%\begin{verbatim}
%pdflatex iflang.dtx
%makeindex -s gind.ist iflang.idx
%pdflatex iflang.dtx
%makeindex -s gind.ist iflang.idx
%pdflatex iflang.dtx
%\end{verbatim}
% \end{quote}
%
% \section{Catalogue}
%
% The following XML file can be used as source for the
% \href{http://mirror.ctan.org/help/Catalogue/catalogue.html}{\TeX\ Catalogue}.
% The elements \texttt{caption} and \texttt{description} are imported
% from the original XML file from the Catalogue.
% The name of the XML file in the Catalogue is \xfile{iflang.xml}.
%    \begin{macrocode}
%<*catalogue>
<?xml version='1.0' encoding='us-ascii'?>
<!DOCTYPE entry SYSTEM 'catalogue.dtd'>
<entry datestamp='$Date$' modifier='$Author$' id='iflang'>
  <name>iflang</name>
  <caption>Expandable checks for the current language.</caption>
  <authorref id='auth:oberdiek'/>
  <copyright owner='Heiko Oberdiek' year='2007'/>
  <license type='lppl1.3'/>
  <version number='1.6'/>
  <description>
    This package provides expandable checks for the current language
    based on macro <tt>\languagename</tt> or hyphenation patterns.
    <p/>
    The package is part of the <xref refid='oberdiek'>oberdiek</xref> bundle.
  </description>
  <documentation details='Package documentation'
      href='ctan:/macros/latex/contrib/oberdiek/iflang.pdf'/>
  <ctan file='true' path='/macros/latex/contrib/oberdiek/iflang.dtx'/>
  <miktex location='oberdiek'/>
  <texlive location='oberdiek'/>
  <install path='/macros/latex/contrib/oberdiek/oberdiek.tds.zip'/>
</entry>
%</catalogue>
%    \end{macrocode}
%
% \section{Acknowledgement}
% I wish to thank:
% \begin{description}
% \item[Markus Kohm] Useful hints for version 1.2.
% \end{description}
%
% \begin{History}
%   \begin{Version}{2007/04/10 v1.0}
%   \item
%     First public version.
%   \end{Version}
%   \begin{Version}{2007/04/11 v1.1}
%   \item
%     Line ends sanitized.
%   \end{Version}
%   \begin{Version}{2007/04/12 v1.2}
%   \item
%     Initialization of \cs{languagename} in case of \xfile{etex.src}.
%   \item
%     Some sanity tests added.
%   \item
%     Documentation improved.
%   \end{Version}
%   \begin{Version}{2007/04/26 v1.3}
%   \item
%     Use of package \xpackage{infwarerr}.
%   \end{Version}
%   \begin{Version}{2007/09/09 v1.4}
%   \item
%     Bug fix: \cs{IfLang@StrEqual} $\rightarrow$ \cs{IfLangStrEqual}
%     (Gabriele Balducci).
%   \item
%     Catcode section rewritten.
%   \end{Version}
%   \begin{Version}{2007/11/11 v1.5}
%   \item
%     Use of package \xpackage{pdftexcmds} for \LuaTeX\ support.
%   \end{Version}
%   \begin{Version}{2016/05/16 v1.6}
%   \item
%     Documentation updates.
%   \end{Version}
% \end{History}
%
% \PrintIndex
%
% \Finale
\endinput

%        (quote the arguments according to the demands of your shell)
%
% Documentation:
%    (a) If iflang.drv is present:
%           latex iflang.drv
%    (b) Without iflang.drv:
%           latex iflang.dtx; ...
%    The class ltxdoc loads the configuration file ltxdoc.cfg
%    if available. Here you can specify further options, e.g.
%    use A4 as paper format:
%       \PassOptionsToClass{a4paper}{article}
%
%    Programm calls to get the documentation (example):
%       pdflatex iflang.dtx
%       makeindex -s gind.ist iflang.idx
%       pdflatex iflang.dtx
%       makeindex -s gind.ist iflang.idx
%       pdflatex iflang.dtx
%
% Installation:
%    TDS:tex/generic/oberdiek/iflang.sty
%    TDS:doc/latex/oberdiek/iflang.pdf
%    TDS:doc/latex/oberdiek/test/iflang-test1.tex
%    TDS:doc/latex/oberdiek/test/iflang-test2.tex
%    TDS:doc/latex/oberdiek/test/iflang-test3.tex
%    TDS:doc/latex/oberdiek/test/iflang-test4.tex
%    TDS:doc/latex/oberdiek/test/iflang-test5.tex
%    TDS:source/latex/oberdiek/iflang.dtx
%
%<*ignore>
\begingroup
  \catcode123=1 %
  \catcode125=2 %
  \def\x{LaTeX2e}%
\expandafter\endgroup
\ifcase 0\ifx\install y1\fi\expandafter
         \ifx\csname processbatchFile\endcsname\relax\else1\fi
         \ifx\fmtname\x\else 1\fi\relax
\else\csname fi\endcsname
%</ignore>
%<*install>
\input docstrip.tex
\Msg{************************************************************************}
\Msg{* Installation}
\Msg{* Package: iflang 2016/05/16 v1.6 Checks for the current language (HO)}
\Msg{************************************************************************}

\keepsilent
\askforoverwritefalse

\let\MetaPrefix\relax
\preamble

This is a generated file.

Project: iflang
Version: 2016/05/16 v1.6

Copyright (C) 2007 by
   Heiko Oberdiek <heiko.oberdiek at googlemail.com>

This work may be distributed and/or modified under the
conditions of the LaTeX Project Public License, either
version 1.3c of this license or (at your option) any later
version. This version of this license is in
   http://www.latex-project.org/lppl/lppl-1-3c.txt
and the latest version of this license is in
   http://www.latex-project.org/lppl.txt
and version 1.3 or later is part of all distributions of
LaTeX version 2005/12/01 or later.

This work has the LPPL maintenance status "maintained".

This Current Maintainer of this work is Heiko Oberdiek.

The Base Interpreter refers to any `TeX-Format',
because some files are installed in TDS:tex/generic//.

This work consists of the main source file iflang.dtx
and the derived files
   iflang.sty, iflang.pdf, iflang.ins, iflang.drv, iflang-test1.tex,
   iflang-test2.tex, iflang-test3.tex, iflang-test4.tex,
   iflang-test5.tex.

\endpreamble
\let\MetaPrefix\DoubleperCent

\generate{%
  \file{iflang.ins}{\from{iflang.dtx}{install}}%
  \file{iflang.drv}{\from{iflang.dtx}{driver}}%
  \usedir{tex/generic/oberdiek}%
  \file{iflang.sty}{\from{iflang.dtx}{package}}%
  \usedir{doc/latex/oberdiek/test}%
  \file{iflang-test1.tex}{\from{iflang.dtx}{test1}}%
  \file{iflang-test2.tex}{\from{iflang.dtx}{test2}}%
  \file{iflang-test3.tex}{\from{iflang.dtx}{test3}}%
  \file{iflang-test4.tex}{\from{iflang.dtx}{test4}}%
  \file{iflang-test5.tex}{\from{iflang.dtx}{test5}}%
  \nopreamble
  \nopostamble
  \usedir{source/latex/oberdiek/catalogue}%
  \file{iflang.xml}{\from{iflang.dtx}{catalogue}}%
}

\catcode32=13\relax% active space
\let =\space%
\Msg{************************************************************************}
\Msg{*}
\Msg{* To finish the installation you have to move the following}
\Msg{* file into a directory searched by TeX:}
\Msg{*}
\Msg{*     iflang.sty}
\Msg{*}
\Msg{* To produce the documentation run the file `iflang.drv'}
\Msg{* through LaTeX.}
\Msg{*}
\Msg{* Happy TeXing!}
\Msg{*}
\Msg{************************************************************************}

\endbatchfile
%</install>
%<*ignore>
\fi
%</ignore>
%<*driver>
\NeedsTeXFormat{LaTeX2e}
\ProvidesFile{iflang.drv}%
  [2016/05/16 v1.6 Checks for the current language (HO)]%
\documentclass{ltxdoc}
\usepackage{holtxdoc}[2011/11/22]
\begin{document}
  \DocInput{iflang.dtx}%
\end{document}
%</driver>
% \fi
%
%
% \CharacterTable
%  {Upper-case    \A\B\C\D\E\F\G\H\I\J\K\L\M\N\O\P\Q\R\S\T\U\V\W\X\Y\Z
%   Lower-case    \a\b\c\d\e\f\g\h\i\j\k\l\m\n\o\p\q\r\s\t\u\v\w\x\y\z
%   Digits        \0\1\2\3\4\5\6\7\8\9
%   Exclamation   \!     Double quote  \"     Hash (number) \#
%   Dollar        \$     Percent       \%     Ampersand     \&
%   Acute accent  \'     Left paren    \(     Right paren   \)
%   Asterisk      \*     Plus          \+     Comma         \,
%   Minus         \-     Point         \.     Solidus       \/
%   Colon         \:     Semicolon     \;     Less than     \<
%   Equals        \=     Greater than  \>     Question mark \?
%   Commercial at \@     Left bracket  \[     Backslash     \\
%   Right bracket \]     Circumflex    \^     Underscore    \_
%   Grave accent  \`     Left brace    \{     Vertical bar  \|
%   Right brace   \}     Tilde         \~}
%
% \GetFileInfo{iflang.drv}
%
% \title{The \xpackage{iflang} package}
% \date{2016/05/16 v1.6}
% \author{Heiko Oberdiek\thanks
% {Please report any issues at https://github.com/ho-tex/oberdiek/issues}\\
% \xemail{heiko.oberdiek at googlemail.com}}
%
% \maketitle
%
% \begin{abstract}
% This package provides expandible checks for the current language
% based on macro \cs{languagename} or hyphenation patterns.
% \end{abstract}
%
% \tableofcontents
%
% \section{Documentation}
%
% Package \xpackage{babel} defines \cs{iflanguagename}. As first
% argument it takes a language name and executes the second or
% third argument depending on the current language. This
% language test is based on hypenation patterns. However, it
% is possible that different languages or dialects share the
% same patterns. In such cases \cs{iflanguagename} fails.
%
% However, package \xpackage{babel} and some other packages
% such as \xpackage{german} or \xpackage{ngerman} store
% the language name in the macro \cs{languagename} if
% \cs{selectlanguage} is called.
%
% \begin{declcs}{IfLanguageName} \M{lang} \M{then} \M{else}
% \end{declcs}
% Makro \cs{IfLanguageName} compares language \meta{lang} with
% the current setting of macro \cs{languagename}. If both
% contains the same name then the \meta{then} part is called,
% otherwise the \meta{else} part.
%
% The macro is expandable. Thus it can be safely used inside \cs{edef}
% or \cs{csname}. If case of errors like an undefined \cs{languagename}
% the \meta{else} part is executed.
%
% Note: Macro \cs{IfLanguageName} relies on the fact, that
% \cs{languagename} is set correctly:
% \begin{description}
% \item[Package \xpackage{babel}:]
%    \mbox{}\\
%    Full support of \cs{languagename} in its language switching commands.
% \item[Format based on \xpackage{babel} (\xfile{language.dat}):]
%    \mbox{}\\
%    If package \xpackage{babel} is not used (or not yet loaded),
%    then \xpackage{babel}'s
%    \xfile{hyphen.cfg} has set \cs{languagename} to the last
%    language in \xfile{language.dat}, but \cs{language} (current
%    patterns) is zero and points to the first language.
%    Thus the value of \cs{languagename} is basically garbage.
%    Package \xpackage{iflang} warns if \cs{languagename} and
%    \cs{language} do not fit.
%    This can be fixed by loading package \xpackage{babel} previously.
% \item[Format based on \eTeX's \xfile{etex.src} (\xfile{language.def}):]
%    \mbox{}\\
%    Unhappily it does not support
%    \cs{languagename}. Thus this package hooks into \cs{uselanguage}
%    to get \cs{languagename} defined and updated there.
%    At package loading time the changed \cs{uselanguage} has not
%    been called yet. Thus package \xpackage{iflang} tries
%    |USenglish|. This is the definite default language of \xfile{etex.src}.
%    If the current patterns suit this default language, an
%    undefined \cs{languagename} is set to this language. Otherwise
%    a \cs{languagename} remains undefined and a warning is given.
% \end{description}
%
% \begin{declcs}{IfLanguagePatterns} \M{lang} \M{then} \M{else}
% \end{declcs}
% This macro behaves similar to \cs{IfLanguageName}. But the
% language test is based on the current pattern in force (\cs{language}).
% Also this macro is expandable, in case of errors the \meta{else} part
% is called.
%
% The following naming convention for the pattern are supported:
% \begin{description}
% \item[\xpackage{babel}/\xfile{language.dat}]:
%    \cs{l@}\meta{language}
% \item[\xpackage{etex.src}/\xfile{language.def}]:
%    \cs{lang@}\meta{language}
% \end{description}
% Package \xpackage{iflang} looks for \cs{uselanguage} (defined in
% \xfile{etex.src}) to find out the naming convention in use.
%
% \StopEventually{
% }
%
% \section{Implementation}
%
%    \begin{macrocode}
%<*package>
%    \end{macrocode}
%
% \subsection{Reload check and package identification}
%    Reload check, especially if the package is not used with \LaTeX.
%    \begin{macrocode}
\begingroup\catcode61\catcode48\catcode32=10\relax%
  \catcode13=5 % ^^M
  \endlinechar=13 %
  \catcode35=6 % #
  \catcode39=12 % '
  \catcode44=12 % ,
  \catcode45=12 % -
  \catcode46=12 % .
  \catcode58=12 % :
  \catcode64=11 % @
  \catcode123=1 % {
  \catcode125=2 % }
  \expandafter\let\expandafter\x\csname ver@iflang.sty\endcsname
  \ifx\x\relax % plain-TeX, first loading
  \else
    \def\empty{}%
    \ifx\x\empty % LaTeX, first loading,
      % variable is initialized, but \ProvidesPackage not yet seen
    \else
      \expandafter\ifx\csname PackageInfo\endcsname\relax
        \def\x#1#2{%
          \immediate\write-1{Package #1 Info: #2.}%
        }%
      \else
        \def\x#1#2{\PackageInfo{#1}{#2, stopped}}%
      \fi
      \x{iflang}{The package is already loaded}%
      \aftergroup\endinput
    \fi
  \fi
\endgroup%
%    \end{macrocode}
%    Package identification:
%    \begin{macrocode}
\begingroup\catcode61\catcode48\catcode32=10\relax%
  \catcode13=5 % ^^M
  \endlinechar=13 %
  \catcode35=6 % #
  \catcode39=12 % '
  \catcode40=12 % (
  \catcode41=12 % )
  \catcode44=12 % ,
  \catcode45=12 % -
  \catcode46=12 % .
  \catcode47=12 % /
  \catcode58=12 % :
  \catcode64=11 % @
  \catcode91=12 % [
  \catcode93=12 % ]
  \catcode123=1 % {
  \catcode125=2 % }
  \expandafter\ifx\csname ProvidesPackage\endcsname\relax
    \def\x#1#2#3[#4]{\endgroup
      \immediate\write-1{Package: #3 #4}%
      \xdef#1{#4}%
    }%
  \else
    \def\x#1#2[#3]{\endgroup
      #2[{#3}]%
      \ifx#1\@undefined
        \xdef#1{#3}%
      \fi
      \ifx#1\relax
        \xdef#1{#3}%
      \fi
    }%
  \fi
\expandafter\x\csname ver@iflang.sty\endcsname
\ProvidesPackage{iflang}%
  [2016/05/16 v1.6 Checks for the current language (HO)]%
%    \end{macrocode}
%
%    \begin{macrocode}
\begingroup\catcode61\catcode48\catcode32=10\relax%
  \catcode13=5 % ^^M
  \endlinechar=13 %
  \catcode123=1 % {
  \catcode125=2 % }
  \catcode64=11 % @
  \def\x{\endgroup
    \expandafter\edef\csname IfLang@AtEnd\endcsname{%
      \endlinechar=\the\endlinechar\relax
      \catcode13=\the\catcode13\relax
      \catcode32=\the\catcode32\relax
      \catcode35=\the\catcode35\relax
      \catcode61=\the\catcode61\relax
      \catcode64=\the\catcode64\relax
      \catcode123=\the\catcode123\relax
      \catcode125=\the\catcode125\relax
    }%
  }%
\x\catcode61\catcode48\catcode32=10\relax%
\catcode13=5 % ^^M
\endlinechar=13 %
\catcode35=6 % #
\catcode64=11 % @
\catcode123=1 % {
\catcode125=2 % }
\def\TMP@EnsureCode#1#2{%
  \edef\IfLang@AtEnd{%
    \IfLang@AtEnd
    \catcode#1=\the\catcode#1\relax
  }%
  \catcode#1=#2\relax
}
\TMP@EnsureCode{39}{12}% '
\TMP@EnsureCode{40}{12}% (
\TMP@EnsureCode{41}{12}% )
\TMP@EnsureCode{44}{12}% ,
\TMP@EnsureCode{46}{12}% .
\TMP@EnsureCode{47}{12}% /
\TMP@EnsureCode{58}{12}% :
\TMP@EnsureCode{91}{12}% [
\TMP@EnsureCode{93}{12}% ]
\edef\IfLang@AtEnd{\IfLang@AtEnd\noexpand\endinput}
%    \end{macrocode}
%
% \subsection{Tools}
%
% \subsubsection{Provide some basic macros of \LaTeX}
%
%    \begin{macro}{\@firstoftwo}
%    \begin{macrocode}
\expandafter\ifx\csname @firstoftwo\endcsname\relax
  \long\def\@firstoftwo#1#2{#1}%
\fi
%    \end{macrocode}
%    \end{macro}
%    \begin{macro}{\@secondoftwo}
%    \begin{macrocode}
\expandafter\ifx\csname @secondoftwo\endcsname\relax
  \long\def\@secondoftwo#1#2{#2}%
\fi
%    \end{macrocode}
%    \end{macro}
%
% \subsubsection{Expandible existence check for macros}
%
%    \begin{macro}{\IfLang@IfDefined}
%    \begin{macrocode}
\begingroup\expandafter\expandafter\expandafter\endgroup
\expandafter\ifx\csname ifcsname\endcsname\relax
  \expandafter\@firstoftwo
\else
  \expandafter\@secondoftwo
\fi
{%
  \def\IfLang@IfDefined#1{%
    \expandafter\ifx\csname#1\endcsname\relax
      \expandafter\@secondoftwo
    \else
      \expandafter\@firstoftwo
    \fi
  }%
}{%
  \def\IfLang@IfDefined#1{%
    \ifnum\ifcsname#1\endcsname
            \expandafter\ifx\csname#1\endcsname\relax
              1%
            \else
              0%
            \fi
          \else
            1%
          \fi
          =0 %
      \expandafter\@firstoftwo
    \else
      \expandafter\@secondoftwo
    \fi
  }%
}
%    \end{macrocode}
%    \end{macro}
%
% \subsubsection{Macros for messages}
%
%    \begin{macrocode}
\begingroup\expandafter\expandafter\expandafter\endgroup
\expandafter\ifx\csname RequirePackage\endcsname\relax
  \input infwarerr.sty\relax
  \input pdftexcmds.sty\relax
\else
  \RequirePackage{infwarerr}[2007/09/09]%
  \RequirePackage{pdftexcmds}[2016/05/16]%
\fi
%    \end{macrocode}
%
% \subsubsection{Support for \xfile{etex.src}}
%
%    \begin{macro}{\IfLang@prefix}
%    \begin{macrocode}
\begingroup\expandafter\expandafter\expandafter\endgroup
\expandafter\ifx\csname uselanguage\endcsname\relax
  \@PackageInfoNoLine{iflang}{%
    Naming convention for patterns: babel%
  }%
  \def\IfLang@prefix{l@}%
\else
  \@PackageInfoNoLine{iflang}{%
    Naming convention for patterns: etex.src%
  }%
  \def\IfLang@prefix{lang@}%
  \let\IfLang@OrgUseLanguage\uselanguage
  \def\uselanguage#1{%
    \edef\languagename{#1}%
    \IfLang@OrgUseLanguage{#1}%
  }%
%    \end{macrocode}
%    \end{macro}
%    The first \cs{uselanguage} that is executed as last
%    line in \xfile{language.def} cannot patched this way.
%    However, \xfile{language.def} is very strict. It forces
%    the first added and used language to be |USenglish|.
%    Thus, if \cs{languagename} is not defined, we can
%    quite safely assume |USenglish|. As additional
%    safety precaution the actual used patterns are checked.
%    \begin{macrocode}
  \begingroup\expandafter\expandafter\expandafter\endgroup
  \expandafter\ifx\csname languagename\endcsname\relax
    \begingroup\expandafter\expandafter\expandafter\endgroup
    \expandafter\ifx\csname lang@USenglish\endcsname\relax
      \@PackageWarningNoLine{iflang}{%
        \string\lang@USenglish\space is missing%
      }%
    \else
      \ifnum\lang@USenglish=\language
        \def\languagename{USenglish}%
      \else
        \@PackageWarningNoLine{iflang}{%
          \string\languagename\space is not set,\MessageBreak
          current language is unknown%
        }%
      \fi
    \fi
  \fi
\fi
\begingroup\expandafter\expandafter\expandafter\endgroup
\expandafter\ifx\csname languagename\endcsname\relax
  \@PackageInfoNoLine{iflang}{%
    \string\languagename\space is not set%
  }%
\fi
%    \end{macrocode}
%
% \subsection{\cs{IfLanguagePatterns}}
%
%    \begin{macro}{\IfLanguagePatterns}
%    \begin{macrocode}
\def\IfLanguagePatterns#1{%
  \ifnum\IfLang@IfDefined{\IfLang@prefix#1}{%
          \ifnum\csname\IfLang@prefix#1\endcsname=\language
            0%
          \else
            1%
          \fi
        }{1}=0 %
    \expandafter\@firstoftwo
  \else
    \expandafter\@secondoftwo
  \fi
}
%    \end{macrocode}
%    \end{macro}
%
% \subsection{\cs{IfLanguageName}}
%
%    \begin{macrocode}
\begingroup\expandafter\expandafter\expandafter\endgroup
\expandafter\ifx\csname pdf@strcmp\endcsname\relax
  \expandafter\@firstoftwo
\else
  \expandafter\@secondoftwo
\fi
{%
%    \end{macrocode}
%    We do not have \cs{pdf@strcmp} (and \cs{pdfstrcmp}).
%    Thus we must define
%    our own expandable string comparison. The following
%    implementation is based on a \TeX\ pearl from David Kastrup, presented
%    at the conference Bacho\TeX~2005:
%    \url{http://www-stary.gust.org.pl/pearls/2005/david-kastrup/bachotex2005-david-kastrup-pearl1.pdf}
%
%    The orignal code allows macros inside the second string.
%    Because also \cs{languagename} might consists of further
%    macros, we need a variant that allows macros in the first
%    string, too.
%    \begin{macrocode}
  \def\IfLang@StrNil{\relax}%
  \def\IfLang@StrEqual#1{%
    \number\IfLang@StrEqualStart{}{}#1\IfLang@StrNil
  }%
  \def\IfLang@StrEqualStart#1#2#3{%
    \ifx#3\IfLang@StrNil
      \IfLang@StrEqualStop
    \fi
    \ifcat\noexpand#3\relax
      \IfLang@StrExpand{#1}{#2}#3%
    \fi
    \IfLang@StrEqualStart{\if#3#1}{#2\fi}%
  }%
  \def\IfLang@StrEqualStop\fi#1\IfLang@StrEqualStart#2#3#4{%
    \fi
    #2#4\relax'#313 %
  }%
  \def\IfLang@StrExpand#1#2#3\fi\IfLang@StrEqualStart#4#5{%
    \fi
    \IfLang@@StrExpand{#1}{#2}#3%
  }%
  \def\IfLang@@StrExpand#1#2#3\IfLang@StrNil{%
    \expandafter\IfLang@@@StrExpand#3\IfLang@StrNil{#1}{#2}%
  }%
  \def\IfLang@@@StrExpand#1\IfLang@StrNil#2#3{%
    \IfLang@StrEqualStart{#2}{#3}#1\IfLang@StrNil
  }%
%    \end{macrocode}
%    \begin{macro}{\IfLanguageName}
%    \begin{macrocode}
  \def\IfLanguageName#1{%
    \ifnum\IfLang@IfDefined{languagename}{%
            \if\expandafter\IfLang@StrEqual\expandafter%
                           {\languagename}{#1}%
              0%
            \else
              1%
            \fi
          }{1}=0 %
      \expandafter\@firstoftwo
    \else
      \expandafter\@secondoftwo
    \fi
  }%
%    \end{macrocode}
%    \end{macro}
%    \begin{macrocode}
}{%
%    \end{macrocode}
%    \begin{macro}{\IfLanguageName}
%    \begin{macrocode}
  \def\IfLanguageName#1{%
    \ifnum\IfLang@IfDefined{languagename}{%
            \pdf@strcmp{#1}{\languagename}%
          }{1}=0 %
      \expandafter\@firstoftwo
    \else
      \expandafter\@secondoftwo
    \fi
  }%
%    \end{macrocode}
%    \end{macro}
%    \begin{macrocode}
}
%    \end{macrocode}
%
% \subsection{Check plausibility of \cs{languagename}}
%
%    \begin{macrocode}
\begingroup\expandafter\expandafter\expandafter\endgroup
\expandafter\ifx\csname languagename\endcsname\relax
\else
  \IfLanguagePatterns{\languagename}{}{%
    \@PackageWarningNoLine{iflang}{%
      Mismatch between \string\language\space
      (patterns)\MessageBreak
      and setting of \string\languagename
    }%
  }%
\fi
%    \end{macrocode}
%
%    \begin{macrocode}
\IfLang@AtEnd%
%</package>
%    \end{macrocode}
%
% \section{Test}
%
% \subsection{Catcode checks for loading}
%
%    \begin{macrocode}
%<*test1>
%    \end{macrocode}
%    \begin{macrocode}
\catcode`\{=1 %
\catcode`\}=2 %
\catcode`\#=6 %
\catcode`\@=11 %
\expandafter\ifx\csname count@\endcsname\relax
  \countdef\count@=255 %
\fi
\expandafter\ifx\csname @gobble\endcsname\relax
  \long\def\@gobble#1{}%
\fi
\expandafter\ifx\csname @firstofone\endcsname\relax
  \long\def\@firstofone#1{#1}%
\fi
\expandafter\ifx\csname loop\endcsname\relax
  \expandafter\@firstofone
\else
  \expandafter\@gobble
\fi
{%
  \def\loop#1\repeat{%
    \def\body{#1}%
    \iterate
  }%
  \def\iterate{%
    \body
      \let\next\iterate
    \else
      \let\next\relax
    \fi
    \next
  }%
  \let\repeat=\fi
}%
\def\RestoreCatcodes{}
\count@=0 %
\loop
  \edef\RestoreCatcodes{%
    \RestoreCatcodes
    \catcode\the\count@=\the\catcode\count@\relax
  }%
\ifnum\count@<255 %
  \advance\count@ 1 %
\repeat

\def\RangeCatcodeInvalid#1#2{%
  \count@=#1\relax
  \loop
    \catcode\count@=15 %
  \ifnum\count@<#2\relax
    \advance\count@ 1 %
  \repeat
}
\def\RangeCatcodeCheck#1#2#3{%
  \count@=#1\relax
  \loop
    \ifnum#3=\catcode\count@
    \else
      \errmessage{%
        Character \the\count@\space
        with wrong catcode \the\catcode\count@\space
        instead of \number#3%
      }%
    \fi
  \ifnum\count@<#2\relax
    \advance\count@ 1 %
  \repeat
}
\def\space{ }
\expandafter\ifx\csname LoadCommand\endcsname\relax
  \def\LoadCommand{\input iflang.sty\relax}%
\fi
\def\Test{%
  \RangeCatcodeInvalid{0}{47}%
  \RangeCatcodeInvalid{58}{64}%
  \RangeCatcodeInvalid{91}{96}%
  \RangeCatcodeInvalid{123}{255}%
  \catcode`\@=12 %
  \catcode`\\=0 %
  \catcode`\%=14 %
  \LoadCommand
  \RangeCatcodeCheck{0}{36}{15}%
  \RangeCatcodeCheck{37}{37}{14}%
  \RangeCatcodeCheck{38}{47}{15}%
  \RangeCatcodeCheck{48}{57}{12}%
  \RangeCatcodeCheck{58}{63}{15}%
  \RangeCatcodeCheck{64}{64}{12}%
  \RangeCatcodeCheck{65}{90}{11}%
  \RangeCatcodeCheck{91}{91}{15}%
  \RangeCatcodeCheck{92}{92}{0}%
  \RangeCatcodeCheck{93}{96}{15}%
  \RangeCatcodeCheck{97}{122}{11}%
  \RangeCatcodeCheck{123}{255}{15}%
  \RestoreCatcodes
}
\Test
\csname @@end\endcsname
\end
%    \end{macrocode}
%    \begin{macrocode}
%</test1>
%    \end{macrocode}
%
% \subsection{Test with \LaTeX}
%
%    \begin{macrocode}
%<*test2|test3>
%    \end{macrocode}
%    \begin{macrocode}
\NeedsTeXFormat{LaTeX2e}
%<test3>\let\pdfstrcmp\relax
\nofiles
\documentclass{minimal}
\usepackage{qstest}
\IncludeTests{*}
\LogTests{log}{*}{*}
\usepackage[english,naustrian,ngerman]{babel}
\usepackage{iflang}
\begin{document}
\begin{qstest}{IfLanguagePatterns}{language, pattern}
  \def\test#1#2{%
    \Expect*{\IfLanguagePatterns{#1}{true}{false}}{#2}%
  }%
  \test{ngerman}{true}%
  \test{naustrian}{true}%
  \test{english}{false}%
  \test{foobar}{false}%
\end{qstest}
\begin{qstest}{IfLanguageName}{language, name}
  \def\test#1#2{%
    \Expect*{\IfLanguageName{#1}{true}{false}}{#2}%
  }%
  \test{ngerman}{true}%
  \test{naustrian}{false}%
  \selectlanguage{naustrian}%
  \test{ngerman}{false}%
  \test{naustrian}{true}%
  \test{foobar}{false}%
  %
  \def\languagename{naustrian}%
  \test{naustrian}{true}%
  \test{ngerman}{false}%
  %
  \edef\languagename{\string naustrian}%
  \test{naustrian}{true}%
  \test{ngerman}{false}%
  %
  \def\languagename{naustrian}%
  \makeatletter
  \@onelevel@sanitize\languagename
  \test{naustrian}{true}%
  \test{ngerman}{false}%
  %
  \def\languagename{naustrian}%
  \def\xaustrian{naustrian}%
  \def\xgerman{ngerman}%
  \test{\xaustrian}{true}%
  \test{\xgerman}{false}%
  %
  \def\languagename{\xaustrian}%
  \test{naustrian}{true}%
  \test{ngerman}{false}%
  \test{\xaustrian}{true}%
  \test{\xgerman}{false}%
  \test{\languagename}{true}%
  \test{\languagename\space}{false}%
  %
  \def\languagename{\empty\xaustrian\empty}%
  \test{naustrian}{true}%
  \test{ngerman}{false}%
  \test{\empty\xaustrian\empty}{true}%
  \test{\empty\xgerman\empty}{false}%
\end{qstest}
\begin{qstest}{IfDefined}{defined}
  \makeatletter
  \let\foobar\relax
  \Expect*{\IfLang@IfDefined{foobar}{true}{false}}{false}%
  \Expect*{\ifx\foobar\relax true\else false\fi}{true}%
  \let\foobar\UNDEFINED
  \Expect*{\IfLang@IfDefined{foobar}{true}{false}}{false}%
  \Expect*{\ifx\foobar\relax true\else false\fi}{false}%
  \Expect*{\ifx\foobar\UNDEFINED true\else false\fi}{true}%
\end{qstest}
\end{document}
%    \end{macrocode}
%    \begin{macrocode}
%</test2|test3>
%    \end{macrocode}
%
% \subsection{Test with \plainTeX\ and \eTeX}
%
%    \begin{macrocode}
%<*test4>
%    \end{macrocode}
%    \begin{macrocode}
%% Format `etex' based on `language.def'
\input iflang.sty
\catcode64=12

\def\TestGeneric#1#2#3{%
  \begingroup
    \edef\x{#1{#2}{true}{false}}%
    \edef\y{#3}%
    \ifx\x\y
    \else
      \errmessage{Failed test: \string#1{#2} <> #3}%
    \fi
  \endgroup
}
\def\TestPatterns{\TestGeneric\IfLanguagePatterns}
\def\TestName{\TestGeneric\IfLanguageName}

\TestPatterns{USenglish}{true}
\TestPatterns{ngerman}{false}

\TestName{USenglish}{true}
\TestName{ngerman}{false}

\uselanguage{ngerman}

\TestPatterns{USenglish}{false}
\TestPatterns{ngerman}{true}

\TestName{USenglish}{false}
\TestName{ngerman}{true}

\csname @@end\endcsname
\end
%    \end{macrocode}
%    \begin{macrocode}
%</test4>
%    \end{macrocode}
%
% \subsection{Test with \plainTeX\ and without \eTeX/\pdfTeX}
%
%    \begin{macrocode}
%<*test5>
%    \end{macrocode}
%    \begin{macrocode}
%% Format `tex' (vanilla plain-TeX)
\let\ifcsname\UNDEFINED
\let\pdfstrcmp\UNDEFINED
\input iflang.sty
\catcode64=11

\def\TestDefined#1{%
  \IfLang@IfDefined{foobar}{}{}%
  \ifx\foobar#1%
  \else
    \errmessage{Failed test: \string\foobar <> \string#1}%
  \fi
}
\let\foobar\relax
\TestDefined\relax
\let\foobar\UNDEFINED
\TestDefined\relax

\def\strip@prefix#1>{}
\def\@onelevel@sanitize#1{%
  \edef#1{\expandafter\strip@prefix\meaning#1}%
}
\def\TestCompare#1#2#3{%
  \begingroup
    \edef\x{%
      \if\IfLang@StrEqual{#1}{#2}%
        true%
      \else
        false%
      \fi
    }%
    \def\expect{#3}%
    \ifx\x\expect
    \else
      \def\a{#1}%
      \@onelevel@sanitize\a
      \def\b{#2}%
      \@onelevel@sanitize\b
      \errmessage{Failed test: `\a'='\b' <> \expect}%
    \fi
  \endgroup
}
\TestCompare{junk}{junk}{true}
\TestCompare{}{}{true}
\TestCompare{a}{b}{false}
\TestCompare{aa}{bb}{false}
\def\a{ax}
\def\b{bx}
\def\c{\a\b}
\def\d{\c\b}
\def\exch#1#2{#2#1}
\def\gobble#1{}
\TestCompare{\gobble a}{}{true}
\TestCompare{}{\gobble a}{true}
\TestCompare{a\exch xyb}{ayxb}{true}
\TestCompare{\c}{\c}{true}
\TestCompare{\d}{\c\b}{true}

\csname @@end\endcsname
\end
%    \end{macrocode}
%    \begin{macrocode}
%</test5>
%    \end{macrocode}
%
% \section{Installation}
%
% \subsection{Download}
%
% \paragraph{Package.} This package is available on
% CTAN\footnote{\url{http://ctan.org/pkg/iflang}}:
% \begin{description}
% \item[\CTAN{macros/latex/contrib/oberdiek/iflang.dtx}] The source file.
% \item[\CTAN{macros/latex/contrib/oberdiek/iflang.pdf}] Documentation.
% \end{description}
%
%
% \paragraph{Bundle.} All the packages of the bundle `oberdiek'
% are also available in a TDS compliant ZIP archive. There
% the packages are already unpacked and the documentation files
% are generated. The files and directories obey the TDS standard.
% \begin{description}
% \item[\CTAN{install/macros/latex/contrib/oberdiek.tds.zip}]
% \end{description}
% \emph{TDS} refers to the standard ``A Directory Structure
% for \TeX\ Files'' (\CTAN{tds/tds.pdf}). Directories
% with \xfile{texmf} in their name are usually organized this way.
%
% \subsection{Bundle installation}
%
% \paragraph{Unpacking.} Unpack the \xfile{oberdiek.tds.zip} in the
% TDS tree (also known as \xfile{texmf} tree) of your choice.
% Example (linux):
% \begin{quote}
%   |unzip oberdiek.tds.zip -d ~/texmf|
% \end{quote}
%
% \paragraph{Script installation.}
% Check the directory \xfile{TDS:scripts/oberdiek/} for
% scripts that need further installation steps.
% Package \xpackage{attachfile2} comes with the Perl script
% \xfile{pdfatfi.pl} that should be installed in such a way
% that it can be called as \texttt{pdfatfi}.
% Example (linux):
% \begin{quote}
%   |chmod +x scripts/oberdiek/pdfatfi.pl|\\
%   |cp scripts/oberdiek/pdfatfi.pl /usr/local/bin/|
% \end{quote}
%
% \subsection{Package installation}
%
% \paragraph{Unpacking.} The \xfile{.dtx} file is a self-extracting
% \docstrip\ archive. The files are extracted by running the
% \xfile{.dtx} through \plainTeX:
% \begin{quote}
%   \verb|tex iflang.dtx|
% \end{quote}
%
% \paragraph{TDS.} Now the different files must be moved into
% the different directories in your installation TDS tree
% (also known as \xfile{texmf} tree):
% \begin{quote}
% \def\t{^^A
% \begin{tabular}{@{}>{\ttfamily}l@{ $\rightarrow$ }>{\ttfamily}l@{}}
%   iflang.sty & tex/generic/oberdiek/iflang.sty\\
%   iflang.pdf & doc/latex/oberdiek/iflang.pdf\\
%   test/iflang-test1.tex & doc/latex/oberdiek/test/iflang-test1.tex\\
%   test/iflang-test2.tex & doc/latex/oberdiek/test/iflang-test2.tex\\
%   test/iflang-test3.tex & doc/latex/oberdiek/test/iflang-test3.tex\\
%   test/iflang-test4.tex & doc/latex/oberdiek/test/iflang-test4.tex\\
%   test/iflang-test5.tex & doc/latex/oberdiek/test/iflang-test5.tex\\
%   iflang.dtx & source/latex/oberdiek/iflang.dtx\\
% \end{tabular}^^A
% }^^A
% \sbox0{\t}^^A
% \ifdim\wd0>\linewidth
%   \begingroup
%     \advance\linewidth by\leftmargin
%     \advance\linewidth by\rightmargin
%   \edef\x{\endgroup
%     \def\noexpand\lw{\the\linewidth}^^A
%   }\x
%   \def\lwbox{^^A
%     \leavevmode
%     \hbox to \linewidth{^^A
%       \kern-\leftmargin\relax
%       \hss
%       \usebox0
%       \hss
%       \kern-\rightmargin\relax
%     }^^A
%   }^^A
%   \ifdim\wd0>\lw
%     \sbox0{\small\t}^^A
%     \ifdim\wd0>\linewidth
%       \ifdim\wd0>\lw
%         \sbox0{\footnotesize\t}^^A
%         \ifdim\wd0>\linewidth
%           \ifdim\wd0>\lw
%             \sbox0{\scriptsize\t}^^A
%             \ifdim\wd0>\linewidth
%               \ifdim\wd0>\lw
%                 \sbox0{\tiny\t}^^A
%                 \ifdim\wd0>\linewidth
%                   \lwbox
%                 \else
%                   \usebox0
%                 \fi
%               \else
%                 \lwbox
%               \fi
%             \else
%               \usebox0
%             \fi
%           \else
%             \lwbox
%           \fi
%         \else
%           \usebox0
%         \fi
%       \else
%         \lwbox
%       \fi
%     \else
%       \usebox0
%     \fi
%   \else
%     \lwbox
%   \fi
% \else
%   \usebox0
% \fi
% \end{quote}
% If you have a \xfile{docstrip.cfg} that configures and enables \docstrip's
% TDS installing feature, then some files can already be in the right
% place, see the documentation of \docstrip.
%
% \subsection{Refresh file name databases}
%
% If your \TeX~distribution
% (\teTeX, \mikTeX, \dots) relies on file name databases, you must refresh
% these. For example, \teTeX\ users run \verb|texhash| or
% \verb|mktexlsr|.
%
% \subsection{Some details for the interested}
%
% \paragraph{Attached source.}
%
% The PDF documentation on CTAN also includes the
% \xfile{.dtx} source file. It can be extracted by
% AcrobatReader 6 or higher. Another option is \textsf{pdftk},
% e.g. unpack the file into the current directory:
% \begin{quote}
%   \verb|pdftk iflang.pdf unpack_files output .|
% \end{quote}
%
% \paragraph{Unpacking with \LaTeX.}
% The \xfile{.dtx} chooses its action depending on the format:
% \begin{description}
% \item[\plainTeX:] Run \docstrip\ and extract the files.
% \item[\LaTeX:] Generate the documentation.
% \end{description}
% If you insist on using \LaTeX\ for \docstrip\ (really,
% \docstrip\ does not need \LaTeX), then inform the autodetect routine
% about your intention:
% \begin{quote}
%   \verb|latex \let\install=y% \iffalse meta-comment
%
% File: iflang.dtx
% Version: 2016/05/16 v1.6
% Info: Checks for the current language
%
% Copyright (C) 2007 by
%    Heiko Oberdiek <heiko.oberdiek at googlemail.com>
%    2016
%    https://github.com/ho-tex/oberdiek/issues
%
% This work may be distributed and/or modified under the
% conditions of the LaTeX Project Public License, either
% version 1.3c of this license or (at your option) any later
% version. This version of this license is in
%    http://www.latex-project.org/lppl/lppl-1-3c.txt
% and the latest version of this license is in
%    http://www.latex-project.org/lppl.txt
% and version 1.3 or later is part of all distributions of
% LaTeX version 2005/12/01 or later.
%
% This work has the LPPL maintenance status "maintained".
%
% This Current Maintainer of this work is Heiko Oberdiek.
%
% The Base Interpreter refers to any `TeX-Format',
% because some files are installed in TDS:tex/generic//.
%
% This work consists of the main source file iflang.dtx
% and the derived files
%    iflang.sty, iflang.pdf, iflang.ins, iflang.drv, iflang-test1.tex,
%    iflang-test2.tex, iflang-test3.tex, iflang-test4.tex,
%    iflang-test5.tex.
%
% Distribution:
%    CTAN:macros/latex/contrib/oberdiek/iflang.dtx
%    CTAN:macros/latex/contrib/oberdiek/iflang.pdf
%
% Unpacking:
%    (a) If iflang.ins is present:
%           tex iflang.ins
%    (b) Without iflang.ins:
%           tex iflang.dtx
%    (c) If you insist on using LaTeX
%           latex \let\install=y% \iffalse meta-comment
%
% File: iflang.dtx
% Version: 2016/05/16 v1.6
% Info: Checks for the current language
%
% Copyright (C) 2007 by
%    Heiko Oberdiek <heiko.oberdiek at googlemail.com>
%    2016
%    https://github.com/ho-tex/oberdiek/issues
%
% This work may be distributed and/or modified under the
% conditions of the LaTeX Project Public License, either
% version 1.3c of this license or (at your option) any later
% version. This version of this license is in
%    http://www.latex-project.org/lppl/lppl-1-3c.txt
% and the latest version of this license is in
%    http://www.latex-project.org/lppl.txt
% and version 1.3 or later is part of all distributions of
% LaTeX version 2005/12/01 or later.
%
% This work has the LPPL maintenance status "maintained".
%
% This Current Maintainer of this work is Heiko Oberdiek.
%
% The Base Interpreter refers to any `TeX-Format',
% because some files are installed in TDS:tex/generic//.
%
% This work consists of the main source file iflang.dtx
% and the derived files
%    iflang.sty, iflang.pdf, iflang.ins, iflang.drv, iflang-test1.tex,
%    iflang-test2.tex, iflang-test3.tex, iflang-test4.tex,
%    iflang-test5.tex.
%
% Distribution:
%    CTAN:macros/latex/contrib/oberdiek/iflang.dtx
%    CTAN:macros/latex/contrib/oberdiek/iflang.pdf
%
% Unpacking:
%    (a) If iflang.ins is present:
%           tex iflang.ins
%    (b) Without iflang.ins:
%           tex iflang.dtx
%    (c) If you insist on using LaTeX
%           latex \let\install=y\input{iflang.dtx}
%        (quote the arguments according to the demands of your shell)
%
% Documentation:
%    (a) If iflang.drv is present:
%           latex iflang.drv
%    (b) Without iflang.drv:
%           latex iflang.dtx; ...
%    The class ltxdoc loads the configuration file ltxdoc.cfg
%    if available. Here you can specify further options, e.g.
%    use A4 as paper format:
%       \PassOptionsToClass{a4paper}{article}
%
%    Programm calls to get the documentation (example):
%       pdflatex iflang.dtx
%       makeindex -s gind.ist iflang.idx
%       pdflatex iflang.dtx
%       makeindex -s gind.ist iflang.idx
%       pdflatex iflang.dtx
%
% Installation:
%    TDS:tex/generic/oberdiek/iflang.sty
%    TDS:doc/latex/oberdiek/iflang.pdf
%    TDS:doc/latex/oberdiek/test/iflang-test1.tex
%    TDS:doc/latex/oberdiek/test/iflang-test2.tex
%    TDS:doc/latex/oberdiek/test/iflang-test3.tex
%    TDS:doc/latex/oberdiek/test/iflang-test4.tex
%    TDS:doc/latex/oberdiek/test/iflang-test5.tex
%    TDS:source/latex/oberdiek/iflang.dtx
%
%<*ignore>
\begingroup
  \catcode123=1 %
  \catcode125=2 %
  \def\x{LaTeX2e}%
\expandafter\endgroup
\ifcase 0\ifx\install y1\fi\expandafter
         \ifx\csname processbatchFile\endcsname\relax\else1\fi
         \ifx\fmtname\x\else 1\fi\relax
\else\csname fi\endcsname
%</ignore>
%<*install>
\input docstrip.tex
\Msg{************************************************************************}
\Msg{* Installation}
\Msg{* Package: iflang 2016/05/16 v1.6 Checks for the current language (HO)}
\Msg{************************************************************************}

\keepsilent
\askforoverwritefalse

\let\MetaPrefix\relax
\preamble

This is a generated file.

Project: iflang
Version: 2016/05/16 v1.6

Copyright (C) 2007 by
   Heiko Oberdiek <heiko.oberdiek at googlemail.com>

This work may be distributed and/or modified under the
conditions of the LaTeX Project Public License, either
version 1.3c of this license or (at your option) any later
version. This version of this license is in
   http://www.latex-project.org/lppl/lppl-1-3c.txt
and the latest version of this license is in
   http://www.latex-project.org/lppl.txt
and version 1.3 or later is part of all distributions of
LaTeX version 2005/12/01 or later.

This work has the LPPL maintenance status "maintained".

This Current Maintainer of this work is Heiko Oberdiek.

The Base Interpreter refers to any `TeX-Format',
because some files are installed in TDS:tex/generic//.

This work consists of the main source file iflang.dtx
and the derived files
   iflang.sty, iflang.pdf, iflang.ins, iflang.drv, iflang-test1.tex,
   iflang-test2.tex, iflang-test3.tex, iflang-test4.tex,
   iflang-test5.tex.

\endpreamble
\let\MetaPrefix\DoubleperCent

\generate{%
  \file{iflang.ins}{\from{iflang.dtx}{install}}%
  \file{iflang.drv}{\from{iflang.dtx}{driver}}%
  \usedir{tex/generic/oberdiek}%
  \file{iflang.sty}{\from{iflang.dtx}{package}}%
  \usedir{doc/latex/oberdiek/test}%
  \file{iflang-test1.tex}{\from{iflang.dtx}{test1}}%
  \file{iflang-test2.tex}{\from{iflang.dtx}{test2}}%
  \file{iflang-test3.tex}{\from{iflang.dtx}{test3}}%
  \file{iflang-test4.tex}{\from{iflang.dtx}{test4}}%
  \file{iflang-test5.tex}{\from{iflang.dtx}{test5}}%
  \nopreamble
  \nopostamble
  \usedir{source/latex/oberdiek/catalogue}%
  \file{iflang.xml}{\from{iflang.dtx}{catalogue}}%
}

\catcode32=13\relax% active space
\let =\space%
\Msg{************************************************************************}
\Msg{*}
\Msg{* To finish the installation you have to move the following}
\Msg{* file into a directory searched by TeX:}
\Msg{*}
\Msg{*     iflang.sty}
\Msg{*}
\Msg{* To produce the documentation run the file `iflang.drv'}
\Msg{* through LaTeX.}
\Msg{*}
\Msg{* Happy TeXing!}
\Msg{*}
\Msg{************************************************************************}

\endbatchfile
%</install>
%<*ignore>
\fi
%</ignore>
%<*driver>
\NeedsTeXFormat{LaTeX2e}
\ProvidesFile{iflang.drv}%
  [2016/05/16 v1.6 Checks for the current language (HO)]%
\documentclass{ltxdoc}
\usepackage{holtxdoc}[2011/11/22]
\begin{document}
  \DocInput{iflang.dtx}%
\end{document}
%</driver>
% \fi
%
%
% \CharacterTable
%  {Upper-case    \A\B\C\D\E\F\G\H\I\J\K\L\M\N\O\P\Q\R\S\T\U\V\W\X\Y\Z
%   Lower-case    \a\b\c\d\e\f\g\h\i\j\k\l\m\n\o\p\q\r\s\t\u\v\w\x\y\z
%   Digits        \0\1\2\3\4\5\6\7\8\9
%   Exclamation   \!     Double quote  \"     Hash (number) \#
%   Dollar        \$     Percent       \%     Ampersand     \&
%   Acute accent  \'     Left paren    \(     Right paren   \)
%   Asterisk      \*     Plus          \+     Comma         \,
%   Minus         \-     Point         \.     Solidus       \/
%   Colon         \:     Semicolon     \;     Less than     \<
%   Equals        \=     Greater than  \>     Question mark \?
%   Commercial at \@     Left bracket  \[     Backslash     \\
%   Right bracket \]     Circumflex    \^     Underscore    \_
%   Grave accent  \`     Left brace    \{     Vertical bar  \|
%   Right brace   \}     Tilde         \~}
%
% \GetFileInfo{iflang.drv}
%
% \title{The \xpackage{iflang} package}
% \date{2016/05/16 v1.6}
% \author{Heiko Oberdiek\thanks
% {Please report any issues at https://github.com/ho-tex/oberdiek/issues}\\
% \xemail{heiko.oberdiek at googlemail.com}}
%
% \maketitle
%
% \begin{abstract}
% This package provides expandible checks for the current language
% based on macro \cs{languagename} or hyphenation patterns.
% \end{abstract}
%
% \tableofcontents
%
% \section{Documentation}
%
% Package \xpackage{babel} defines \cs{iflanguagename}. As first
% argument it takes a language name and executes the second or
% third argument depending on the current language. This
% language test is based on hypenation patterns. However, it
% is possible that different languages or dialects share the
% same patterns. In such cases \cs{iflanguagename} fails.
%
% However, package \xpackage{babel} and some other packages
% such as \xpackage{german} or \xpackage{ngerman} store
% the language name in the macro \cs{languagename} if
% \cs{selectlanguage} is called.
%
% \begin{declcs}{IfLanguageName} \M{lang} \M{then} \M{else}
% \end{declcs}
% Makro \cs{IfLanguageName} compares language \meta{lang} with
% the current setting of macro \cs{languagename}. If both
% contains the same name then the \meta{then} part is called,
% otherwise the \meta{else} part.
%
% The macro is expandable. Thus it can be safely used inside \cs{edef}
% or \cs{csname}. If case of errors like an undefined \cs{languagename}
% the \meta{else} part is executed.
%
% Note: Macro \cs{IfLanguageName} relies on the fact, that
% \cs{languagename} is set correctly:
% \begin{description}
% \item[Package \xpackage{babel}:]
%    \mbox{}\\
%    Full support of \cs{languagename} in its language switching commands.
% \item[Format based on \xpackage{babel} (\xfile{language.dat}):]
%    \mbox{}\\
%    If package \xpackage{babel} is not used (or not yet loaded),
%    then \xpackage{babel}'s
%    \xfile{hyphen.cfg} has set \cs{languagename} to the last
%    language in \xfile{language.dat}, but \cs{language} (current
%    patterns) is zero and points to the first language.
%    Thus the value of \cs{languagename} is basically garbage.
%    Package \xpackage{iflang} warns if \cs{languagename} and
%    \cs{language} do not fit.
%    This can be fixed by loading package \xpackage{babel} previously.
% \item[Format based on \eTeX's \xfile{etex.src} (\xfile{language.def}):]
%    \mbox{}\\
%    Unhappily it does not support
%    \cs{languagename}. Thus this package hooks into \cs{uselanguage}
%    to get \cs{languagename} defined and updated there.
%    At package loading time the changed \cs{uselanguage} has not
%    been called yet. Thus package \xpackage{iflang} tries
%    |USenglish|. This is the definite default language of \xfile{etex.src}.
%    If the current patterns suit this default language, an
%    undefined \cs{languagename} is set to this language. Otherwise
%    a \cs{languagename} remains undefined and a warning is given.
% \end{description}
%
% \begin{declcs}{IfLanguagePatterns} \M{lang} \M{then} \M{else}
% \end{declcs}
% This macro behaves similar to \cs{IfLanguageName}. But the
% language test is based on the current pattern in force (\cs{language}).
% Also this macro is expandable, in case of errors the \meta{else} part
% is called.
%
% The following naming convention for the pattern are supported:
% \begin{description}
% \item[\xpackage{babel}/\xfile{language.dat}]:
%    \cs{l@}\meta{language}
% \item[\xpackage{etex.src}/\xfile{language.def}]:
%    \cs{lang@}\meta{language}
% \end{description}
% Package \xpackage{iflang} looks for \cs{uselanguage} (defined in
% \xfile{etex.src}) to find out the naming convention in use.
%
% \StopEventually{
% }
%
% \section{Implementation}
%
%    \begin{macrocode}
%<*package>
%    \end{macrocode}
%
% \subsection{Reload check and package identification}
%    Reload check, especially if the package is not used with \LaTeX.
%    \begin{macrocode}
\begingroup\catcode61\catcode48\catcode32=10\relax%
  \catcode13=5 % ^^M
  \endlinechar=13 %
  \catcode35=6 % #
  \catcode39=12 % '
  \catcode44=12 % ,
  \catcode45=12 % -
  \catcode46=12 % .
  \catcode58=12 % :
  \catcode64=11 % @
  \catcode123=1 % {
  \catcode125=2 % }
  \expandafter\let\expandafter\x\csname ver@iflang.sty\endcsname
  \ifx\x\relax % plain-TeX, first loading
  \else
    \def\empty{}%
    \ifx\x\empty % LaTeX, first loading,
      % variable is initialized, but \ProvidesPackage not yet seen
    \else
      \expandafter\ifx\csname PackageInfo\endcsname\relax
        \def\x#1#2{%
          \immediate\write-1{Package #1 Info: #2.}%
        }%
      \else
        \def\x#1#2{\PackageInfo{#1}{#2, stopped}}%
      \fi
      \x{iflang}{The package is already loaded}%
      \aftergroup\endinput
    \fi
  \fi
\endgroup%
%    \end{macrocode}
%    Package identification:
%    \begin{macrocode}
\begingroup\catcode61\catcode48\catcode32=10\relax%
  \catcode13=5 % ^^M
  \endlinechar=13 %
  \catcode35=6 % #
  \catcode39=12 % '
  \catcode40=12 % (
  \catcode41=12 % )
  \catcode44=12 % ,
  \catcode45=12 % -
  \catcode46=12 % .
  \catcode47=12 % /
  \catcode58=12 % :
  \catcode64=11 % @
  \catcode91=12 % [
  \catcode93=12 % ]
  \catcode123=1 % {
  \catcode125=2 % }
  \expandafter\ifx\csname ProvidesPackage\endcsname\relax
    \def\x#1#2#3[#4]{\endgroup
      \immediate\write-1{Package: #3 #4}%
      \xdef#1{#4}%
    }%
  \else
    \def\x#1#2[#3]{\endgroup
      #2[{#3}]%
      \ifx#1\@undefined
        \xdef#1{#3}%
      \fi
      \ifx#1\relax
        \xdef#1{#3}%
      \fi
    }%
  \fi
\expandafter\x\csname ver@iflang.sty\endcsname
\ProvidesPackage{iflang}%
  [2016/05/16 v1.6 Checks for the current language (HO)]%
%    \end{macrocode}
%
%    \begin{macrocode}
\begingroup\catcode61\catcode48\catcode32=10\relax%
  \catcode13=5 % ^^M
  \endlinechar=13 %
  \catcode123=1 % {
  \catcode125=2 % }
  \catcode64=11 % @
  \def\x{\endgroup
    \expandafter\edef\csname IfLang@AtEnd\endcsname{%
      \endlinechar=\the\endlinechar\relax
      \catcode13=\the\catcode13\relax
      \catcode32=\the\catcode32\relax
      \catcode35=\the\catcode35\relax
      \catcode61=\the\catcode61\relax
      \catcode64=\the\catcode64\relax
      \catcode123=\the\catcode123\relax
      \catcode125=\the\catcode125\relax
    }%
  }%
\x\catcode61\catcode48\catcode32=10\relax%
\catcode13=5 % ^^M
\endlinechar=13 %
\catcode35=6 % #
\catcode64=11 % @
\catcode123=1 % {
\catcode125=2 % }
\def\TMP@EnsureCode#1#2{%
  \edef\IfLang@AtEnd{%
    \IfLang@AtEnd
    \catcode#1=\the\catcode#1\relax
  }%
  \catcode#1=#2\relax
}
\TMP@EnsureCode{39}{12}% '
\TMP@EnsureCode{40}{12}% (
\TMP@EnsureCode{41}{12}% )
\TMP@EnsureCode{44}{12}% ,
\TMP@EnsureCode{46}{12}% .
\TMP@EnsureCode{47}{12}% /
\TMP@EnsureCode{58}{12}% :
\TMP@EnsureCode{91}{12}% [
\TMP@EnsureCode{93}{12}% ]
\edef\IfLang@AtEnd{\IfLang@AtEnd\noexpand\endinput}
%    \end{macrocode}
%
% \subsection{Tools}
%
% \subsubsection{Provide some basic macros of \LaTeX}
%
%    \begin{macro}{\@firstoftwo}
%    \begin{macrocode}
\expandafter\ifx\csname @firstoftwo\endcsname\relax
  \long\def\@firstoftwo#1#2{#1}%
\fi
%    \end{macrocode}
%    \end{macro}
%    \begin{macro}{\@secondoftwo}
%    \begin{macrocode}
\expandafter\ifx\csname @secondoftwo\endcsname\relax
  \long\def\@secondoftwo#1#2{#2}%
\fi
%    \end{macrocode}
%    \end{macro}
%
% \subsubsection{Expandible existence check for macros}
%
%    \begin{macro}{\IfLang@IfDefined}
%    \begin{macrocode}
\begingroup\expandafter\expandafter\expandafter\endgroup
\expandafter\ifx\csname ifcsname\endcsname\relax
  \expandafter\@firstoftwo
\else
  \expandafter\@secondoftwo
\fi
{%
  \def\IfLang@IfDefined#1{%
    \expandafter\ifx\csname#1\endcsname\relax
      \expandafter\@secondoftwo
    \else
      \expandafter\@firstoftwo
    \fi
  }%
}{%
  \def\IfLang@IfDefined#1{%
    \ifnum\ifcsname#1\endcsname
            \expandafter\ifx\csname#1\endcsname\relax
              1%
            \else
              0%
            \fi
          \else
            1%
          \fi
          =0 %
      \expandafter\@firstoftwo
    \else
      \expandafter\@secondoftwo
    \fi
  }%
}
%    \end{macrocode}
%    \end{macro}
%
% \subsubsection{Macros for messages}
%
%    \begin{macrocode}
\begingroup\expandafter\expandafter\expandafter\endgroup
\expandafter\ifx\csname RequirePackage\endcsname\relax
  \input infwarerr.sty\relax
  \input pdftexcmds.sty\relax
\else
  \RequirePackage{infwarerr}[2007/09/09]%
  \RequirePackage{pdftexcmds}[2016/05/16]%
\fi
%    \end{macrocode}
%
% \subsubsection{Support for \xfile{etex.src}}
%
%    \begin{macro}{\IfLang@prefix}
%    \begin{macrocode}
\begingroup\expandafter\expandafter\expandafter\endgroup
\expandafter\ifx\csname uselanguage\endcsname\relax
  \@PackageInfoNoLine{iflang}{%
    Naming convention for patterns: babel%
  }%
  \def\IfLang@prefix{l@}%
\else
  \@PackageInfoNoLine{iflang}{%
    Naming convention for patterns: etex.src%
  }%
  \def\IfLang@prefix{lang@}%
  \let\IfLang@OrgUseLanguage\uselanguage
  \def\uselanguage#1{%
    \edef\languagename{#1}%
    \IfLang@OrgUseLanguage{#1}%
  }%
%    \end{macrocode}
%    \end{macro}
%    The first \cs{uselanguage} that is executed as last
%    line in \xfile{language.def} cannot patched this way.
%    However, \xfile{language.def} is very strict. It forces
%    the first added and used language to be |USenglish|.
%    Thus, if \cs{languagename} is not defined, we can
%    quite safely assume |USenglish|. As additional
%    safety precaution the actual used patterns are checked.
%    \begin{macrocode}
  \begingroup\expandafter\expandafter\expandafter\endgroup
  \expandafter\ifx\csname languagename\endcsname\relax
    \begingroup\expandafter\expandafter\expandafter\endgroup
    \expandafter\ifx\csname lang@USenglish\endcsname\relax
      \@PackageWarningNoLine{iflang}{%
        \string\lang@USenglish\space is missing%
      }%
    \else
      \ifnum\lang@USenglish=\language
        \def\languagename{USenglish}%
      \else
        \@PackageWarningNoLine{iflang}{%
          \string\languagename\space is not set,\MessageBreak
          current language is unknown%
        }%
      \fi
    \fi
  \fi
\fi
\begingroup\expandafter\expandafter\expandafter\endgroup
\expandafter\ifx\csname languagename\endcsname\relax
  \@PackageInfoNoLine{iflang}{%
    \string\languagename\space is not set%
  }%
\fi
%    \end{macrocode}
%
% \subsection{\cs{IfLanguagePatterns}}
%
%    \begin{macro}{\IfLanguagePatterns}
%    \begin{macrocode}
\def\IfLanguagePatterns#1{%
  \ifnum\IfLang@IfDefined{\IfLang@prefix#1}{%
          \ifnum\csname\IfLang@prefix#1\endcsname=\language
            0%
          \else
            1%
          \fi
        }{1}=0 %
    \expandafter\@firstoftwo
  \else
    \expandafter\@secondoftwo
  \fi
}
%    \end{macrocode}
%    \end{macro}
%
% \subsection{\cs{IfLanguageName}}
%
%    \begin{macrocode}
\begingroup\expandafter\expandafter\expandafter\endgroup
\expandafter\ifx\csname pdf@strcmp\endcsname\relax
  \expandafter\@firstoftwo
\else
  \expandafter\@secondoftwo
\fi
{%
%    \end{macrocode}
%    We do not have \cs{pdf@strcmp} (and \cs{pdfstrcmp}).
%    Thus we must define
%    our own expandable string comparison. The following
%    implementation is based on a \TeX\ pearl from David Kastrup, presented
%    at the conference Bacho\TeX~2005:
%    \url{http://www-stary.gust.org.pl/pearls/2005/david-kastrup/bachotex2005-david-kastrup-pearl1.pdf}
%
%    The orignal code allows macros inside the second string.
%    Because also \cs{languagename} might consists of further
%    macros, we need a variant that allows macros in the first
%    string, too.
%    \begin{macrocode}
  \def\IfLang@StrNil{\relax}%
  \def\IfLang@StrEqual#1{%
    \number\IfLang@StrEqualStart{}{}#1\IfLang@StrNil
  }%
  \def\IfLang@StrEqualStart#1#2#3{%
    \ifx#3\IfLang@StrNil
      \IfLang@StrEqualStop
    \fi
    \ifcat\noexpand#3\relax
      \IfLang@StrExpand{#1}{#2}#3%
    \fi
    \IfLang@StrEqualStart{\if#3#1}{#2\fi}%
  }%
  \def\IfLang@StrEqualStop\fi#1\IfLang@StrEqualStart#2#3#4{%
    \fi
    #2#4\relax'#313 %
  }%
  \def\IfLang@StrExpand#1#2#3\fi\IfLang@StrEqualStart#4#5{%
    \fi
    \IfLang@@StrExpand{#1}{#2}#3%
  }%
  \def\IfLang@@StrExpand#1#2#3\IfLang@StrNil{%
    \expandafter\IfLang@@@StrExpand#3\IfLang@StrNil{#1}{#2}%
  }%
  \def\IfLang@@@StrExpand#1\IfLang@StrNil#2#3{%
    \IfLang@StrEqualStart{#2}{#3}#1\IfLang@StrNil
  }%
%    \end{macrocode}
%    \begin{macro}{\IfLanguageName}
%    \begin{macrocode}
  \def\IfLanguageName#1{%
    \ifnum\IfLang@IfDefined{languagename}{%
            \if\expandafter\IfLang@StrEqual\expandafter%
                           {\languagename}{#1}%
              0%
            \else
              1%
            \fi
          }{1}=0 %
      \expandafter\@firstoftwo
    \else
      \expandafter\@secondoftwo
    \fi
  }%
%    \end{macrocode}
%    \end{macro}
%    \begin{macrocode}
}{%
%    \end{macrocode}
%    \begin{macro}{\IfLanguageName}
%    \begin{macrocode}
  \def\IfLanguageName#1{%
    \ifnum\IfLang@IfDefined{languagename}{%
            \pdf@strcmp{#1}{\languagename}%
          }{1}=0 %
      \expandafter\@firstoftwo
    \else
      \expandafter\@secondoftwo
    \fi
  }%
%    \end{macrocode}
%    \end{macro}
%    \begin{macrocode}
}
%    \end{macrocode}
%
% \subsection{Check plausibility of \cs{languagename}}
%
%    \begin{macrocode}
\begingroup\expandafter\expandafter\expandafter\endgroup
\expandafter\ifx\csname languagename\endcsname\relax
\else
  \IfLanguagePatterns{\languagename}{}{%
    \@PackageWarningNoLine{iflang}{%
      Mismatch between \string\language\space
      (patterns)\MessageBreak
      and setting of \string\languagename
    }%
  }%
\fi
%    \end{macrocode}
%
%    \begin{macrocode}
\IfLang@AtEnd%
%</package>
%    \end{macrocode}
%
% \section{Test}
%
% \subsection{Catcode checks for loading}
%
%    \begin{macrocode}
%<*test1>
%    \end{macrocode}
%    \begin{macrocode}
\catcode`\{=1 %
\catcode`\}=2 %
\catcode`\#=6 %
\catcode`\@=11 %
\expandafter\ifx\csname count@\endcsname\relax
  \countdef\count@=255 %
\fi
\expandafter\ifx\csname @gobble\endcsname\relax
  \long\def\@gobble#1{}%
\fi
\expandafter\ifx\csname @firstofone\endcsname\relax
  \long\def\@firstofone#1{#1}%
\fi
\expandafter\ifx\csname loop\endcsname\relax
  \expandafter\@firstofone
\else
  \expandafter\@gobble
\fi
{%
  \def\loop#1\repeat{%
    \def\body{#1}%
    \iterate
  }%
  \def\iterate{%
    \body
      \let\next\iterate
    \else
      \let\next\relax
    \fi
    \next
  }%
  \let\repeat=\fi
}%
\def\RestoreCatcodes{}
\count@=0 %
\loop
  \edef\RestoreCatcodes{%
    \RestoreCatcodes
    \catcode\the\count@=\the\catcode\count@\relax
  }%
\ifnum\count@<255 %
  \advance\count@ 1 %
\repeat

\def\RangeCatcodeInvalid#1#2{%
  \count@=#1\relax
  \loop
    \catcode\count@=15 %
  \ifnum\count@<#2\relax
    \advance\count@ 1 %
  \repeat
}
\def\RangeCatcodeCheck#1#2#3{%
  \count@=#1\relax
  \loop
    \ifnum#3=\catcode\count@
    \else
      \errmessage{%
        Character \the\count@\space
        with wrong catcode \the\catcode\count@\space
        instead of \number#3%
      }%
    \fi
  \ifnum\count@<#2\relax
    \advance\count@ 1 %
  \repeat
}
\def\space{ }
\expandafter\ifx\csname LoadCommand\endcsname\relax
  \def\LoadCommand{\input iflang.sty\relax}%
\fi
\def\Test{%
  \RangeCatcodeInvalid{0}{47}%
  \RangeCatcodeInvalid{58}{64}%
  \RangeCatcodeInvalid{91}{96}%
  \RangeCatcodeInvalid{123}{255}%
  \catcode`\@=12 %
  \catcode`\\=0 %
  \catcode`\%=14 %
  \LoadCommand
  \RangeCatcodeCheck{0}{36}{15}%
  \RangeCatcodeCheck{37}{37}{14}%
  \RangeCatcodeCheck{38}{47}{15}%
  \RangeCatcodeCheck{48}{57}{12}%
  \RangeCatcodeCheck{58}{63}{15}%
  \RangeCatcodeCheck{64}{64}{12}%
  \RangeCatcodeCheck{65}{90}{11}%
  \RangeCatcodeCheck{91}{91}{15}%
  \RangeCatcodeCheck{92}{92}{0}%
  \RangeCatcodeCheck{93}{96}{15}%
  \RangeCatcodeCheck{97}{122}{11}%
  \RangeCatcodeCheck{123}{255}{15}%
  \RestoreCatcodes
}
\Test
\csname @@end\endcsname
\end
%    \end{macrocode}
%    \begin{macrocode}
%</test1>
%    \end{macrocode}
%
% \subsection{Test with \LaTeX}
%
%    \begin{macrocode}
%<*test2|test3>
%    \end{macrocode}
%    \begin{macrocode}
\NeedsTeXFormat{LaTeX2e}
%<test3>\let\pdfstrcmp\relax
\nofiles
\documentclass{minimal}
\usepackage{qstest}
\IncludeTests{*}
\LogTests{log}{*}{*}
\usepackage[english,naustrian,ngerman]{babel}
\usepackage{iflang}
\begin{document}
\begin{qstest}{IfLanguagePatterns}{language, pattern}
  \def\test#1#2{%
    \Expect*{\IfLanguagePatterns{#1}{true}{false}}{#2}%
  }%
  \test{ngerman}{true}%
  \test{naustrian}{true}%
  \test{english}{false}%
  \test{foobar}{false}%
\end{qstest}
\begin{qstest}{IfLanguageName}{language, name}
  \def\test#1#2{%
    \Expect*{\IfLanguageName{#1}{true}{false}}{#2}%
  }%
  \test{ngerman}{true}%
  \test{naustrian}{false}%
  \selectlanguage{naustrian}%
  \test{ngerman}{false}%
  \test{naustrian}{true}%
  \test{foobar}{false}%
  %
  \def\languagename{naustrian}%
  \test{naustrian}{true}%
  \test{ngerman}{false}%
  %
  \edef\languagename{\string naustrian}%
  \test{naustrian}{true}%
  \test{ngerman}{false}%
  %
  \def\languagename{naustrian}%
  \makeatletter
  \@onelevel@sanitize\languagename
  \test{naustrian}{true}%
  \test{ngerman}{false}%
  %
  \def\languagename{naustrian}%
  \def\xaustrian{naustrian}%
  \def\xgerman{ngerman}%
  \test{\xaustrian}{true}%
  \test{\xgerman}{false}%
  %
  \def\languagename{\xaustrian}%
  \test{naustrian}{true}%
  \test{ngerman}{false}%
  \test{\xaustrian}{true}%
  \test{\xgerman}{false}%
  \test{\languagename}{true}%
  \test{\languagename\space}{false}%
  %
  \def\languagename{\empty\xaustrian\empty}%
  \test{naustrian}{true}%
  \test{ngerman}{false}%
  \test{\empty\xaustrian\empty}{true}%
  \test{\empty\xgerman\empty}{false}%
\end{qstest}
\begin{qstest}{IfDefined}{defined}
  \makeatletter
  \let\foobar\relax
  \Expect*{\IfLang@IfDefined{foobar}{true}{false}}{false}%
  \Expect*{\ifx\foobar\relax true\else false\fi}{true}%
  \let\foobar\UNDEFINED
  \Expect*{\IfLang@IfDefined{foobar}{true}{false}}{false}%
  \Expect*{\ifx\foobar\relax true\else false\fi}{false}%
  \Expect*{\ifx\foobar\UNDEFINED true\else false\fi}{true}%
\end{qstest}
\end{document}
%    \end{macrocode}
%    \begin{macrocode}
%</test2|test3>
%    \end{macrocode}
%
% \subsection{Test with \plainTeX\ and \eTeX}
%
%    \begin{macrocode}
%<*test4>
%    \end{macrocode}
%    \begin{macrocode}
%% Format `etex' based on `language.def'
\input iflang.sty
\catcode64=12

\def\TestGeneric#1#2#3{%
  \begingroup
    \edef\x{#1{#2}{true}{false}}%
    \edef\y{#3}%
    \ifx\x\y
    \else
      \errmessage{Failed test: \string#1{#2} <> #3}%
    \fi
  \endgroup
}
\def\TestPatterns{\TestGeneric\IfLanguagePatterns}
\def\TestName{\TestGeneric\IfLanguageName}

\TestPatterns{USenglish}{true}
\TestPatterns{ngerman}{false}

\TestName{USenglish}{true}
\TestName{ngerman}{false}

\uselanguage{ngerman}

\TestPatterns{USenglish}{false}
\TestPatterns{ngerman}{true}

\TestName{USenglish}{false}
\TestName{ngerman}{true}

\csname @@end\endcsname
\end
%    \end{macrocode}
%    \begin{macrocode}
%</test4>
%    \end{macrocode}
%
% \subsection{Test with \plainTeX\ and without \eTeX/\pdfTeX}
%
%    \begin{macrocode}
%<*test5>
%    \end{macrocode}
%    \begin{macrocode}
%% Format `tex' (vanilla plain-TeX)
\let\ifcsname\UNDEFINED
\let\pdfstrcmp\UNDEFINED
\input iflang.sty
\catcode64=11

\def\TestDefined#1{%
  \IfLang@IfDefined{foobar}{}{}%
  \ifx\foobar#1%
  \else
    \errmessage{Failed test: \string\foobar <> \string#1}%
  \fi
}
\let\foobar\relax
\TestDefined\relax
\let\foobar\UNDEFINED
\TestDefined\relax

\def\strip@prefix#1>{}
\def\@onelevel@sanitize#1{%
  \edef#1{\expandafter\strip@prefix\meaning#1}%
}
\def\TestCompare#1#2#3{%
  \begingroup
    \edef\x{%
      \if\IfLang@StrEqual{#1}{#2}%
        true%
      \else
        false%
      \fi
    }%
    \def\expect{#3}%
    \ifx\x\expect
    \else
      \def\a{#1}%
      \@onelevel@sanitize\a
      \def\b{#2}%
      \@onelevel@sanitize\b
      \errmessage{Failed test: `\a'='\b' <> \expect}%
    \fi
  \endgroup
}
\TestCompare{junk}{junk}{true}
\TestCompare{}{}{true}
\TestCompare{a}{b}{false}
\TestCompare{aa}{bb}{false}
\def\a{ax}
\def\b{bx}
\def\c{\a\b}
\def\d{\c\b}
\def\exch#1#2{#2#1}
\def\gobble#1{}
\TestCompare{\gobble a}{}{true}
\TestCompare{}{\gobble a}{true}
\TestCompare{a\exch xyb}{ayxb}{true}
\TestCompare{\c}{\c}{true}
\TestCompare{\d}{\c\b}{true}

\csname @@end\endcsname
\end
%    \end{macrocode}
%    \begin{macrocode}
%</test5>
%    \end{macrocode}
%
% \section{Installation}
%
% \subsection{Download}
%
% \paragraph{Package.} This package is available on
% CTAN\footnote{\url{http://ctan.org/pkg/iflang}}:
% \begin{description}
% \item[\CTAN{macros/latex/contrib/oberdiek/iflang.dtx}] The source file.
% \item[\CTAN{macros/latex/contrib/oberdiek/iflang.pdf}] Documentation.
% \end{description}
%
%
% \paragraph{Bundle.} All the packages of the bundle `oberdiek'
% are also available in a TDS compliant ZIP archive. There
% the packages are already unpacked and the documentation files
% are generated. The files and directories obey the TDS standard.
% \begin{description}
% \item[\CTAN{install/macros/latex/contrib/oberdiek.tds.zip}]
% \end{description}
% \emph{TDS} refers to the standard ``A Directory Structure
% for \TeX\ Files'' (\CTAN{tds/tds.pdf}). Directories
% with \xfile{texmf} in their name are usually organized this way.
%
% \subsection{Bundle installation}
%
% \paragraph{Unpacking.} Unpack the \xfile{oberdiek.tds.zip} in the
% TDS tree (also known as \xfile{texmf} tree) of your choice.
% Example (linux):
% \begin{quote}
%   |unzip oberdiek.tds.zip -d ~/texmf|
% \end{quote}
%
% \paragraph{Script installation.}
% Check the directory \xfile{TDS:scripts/oberdiek/} for
% scripts that need further installation steps.
% Package \xpackage{attachfile2} comes with the Perl script
% \xfile{pdfatfi.pl} that should be installed in such a way
% that it can be called as \texttt{pdfatfi}.
% Example (linux):
% \begin{quote}
%   |chmod +x scripts/oberdiek/pdfatfi.pl|\\
%   |cp scripts/oberdiek/pdfatfi.pl /usr/local/bin/|
% \end{quote}
%
% \subsection{Package installation}
%
% \paragraph{Unpacking.} The \xfile{.dtx} file is a self-extracting
% \docstrip\ archive. The files are extracted by running the
% \xfile{.dtx} through \plainTeX:
% \begin{quote}
%   \verb|tex iflang.dtx|
% \end{quote}
%
% \paragraph{TDS.} Now the different files must be moved into
% the different directories in your installation TDS tree
% (also known as \xfile{texmf} tree):
% \begin{quote}
% \def\t{^^A
% \begin{tabular}{@{}>{\ttfamily}l@{ $\rightarrow$ }>{\ttfamily}l@{}}
%   iflang.sty & tex/generic/oberdiek/iflang.sty\\
%   iflang.pdf & doc/latex/oberdiek/iflang.pdf\\
%   test/iflang-test1.tex & doc/latex/oberdiek/test/iflang-test1.tex\\
%   test/iflang-test2.tex & doc/latex/oberdiek/test/iflang-test2.tex\\
%   test/iflang-test3.tex & doc/latex/oberdiek/test/iflang-test3.tex\\
%   test/iflang-test4.tex & doc/latex/oberdiek/test/iflang-test4.tex\\
%   test/iflang-test5.tex & doc/latex/oberdiek/test/iflang-test5.tex\\
%   iflang.dtx & source/latex/oberdiek/iflang.dtx\\
% \end{tabular}^^A
% }^^A
% \sbox0{\t}^^A
% \ifdim\wd0>\linewidth
%   \begingroup
%     \advance\linewidth by\leftmargin
%     \advance\linewidth by\rightmargin
%   \edef\x{\endgroup
%     \def\noexpand\lw{\the\linewidth}^^A
%   }\x
%   \def\lwbox{^^A
%     \leavevmode
%     \hbox to \linewidth{^^A
%       \kern-\leftmargin\relax
%       \hss
%       \usebox0
%       \hss
%       \kern-\rightmargin\relax
%     }^^A
%   }^^A
%   \ifdim\wd0>\lw
%     \sbox0{\small\t}^^A
%     \ifdim\wd0>\linewidth
%       \ifdim\wd0>\lw
%         \sbox0{\footnotesize\t}^^A
%         \ifdim\wd0>\linewidth
%           \ifdim\wd0>\lw
%             \sbox0{\scriptsize\t}^^A
%             \ifdim\wd0>\linewidth
%               \ifdim\wd0>\lw
%                 \sbox0{\tiny\t}^^A
%                 \ifdim\wd0>\linewidth
%                   \lwbox
%                 \else
%                   \usebox0
%                 \fi
%               \else
%                 \lwbox
%               \fi
%             \else
%               \usebox0
%             \fi
%           \else
%             \lwbox
%           \fi
%         \else
%           \usebox0
%         \fi
%       \else
%         \lwbox
%       \fi
%     \else
%       \usebox0
%     \fi
%   \else
%     \lwbox
%   \fi
% \else
%   \usebox0
% \fi
% \end{quote}
% If you have a \xfile{docstrip.cfg} that configures and enables \docstrip's
% TDS installing feature, then some files can already be in the right
% place, see the documentation of \docstrip.
%
% \subsection{Refresh file name databases}
%
% If your \TeX~distribution
% (\teTeX, \mikTeX, \dots) relies on file name databases, you must refresh
% these. For example, \teTeX\ users run \verb|texhash| or
% \verb|mktexlsr|.
%
% \subsection{Some details for the interested}
%
% \paragraph{Attached source.}
%
% The PDF documentation on CTAN also includes the
% \xfile{.dtx} source file. It can be extracted by
% AcrobatReader 6 or higher. Another option is \textsf{pdftk},
% e.g. unpack the file into the current directory:
% \begin{quote}
%   \verb|pdftk iflang.pdf unpack_files output .|
% \end{quote}
%
% \paragraph{Unpacking with \LaTeX.}
% The \xfile{.dtx} chooses its action depending on the format:
% \begin{description}
% \item[\plainTeX:] Run \docstrip\ and extract the files.
% \item[\LaTeX:] Generate the documentation.
% \end{description}
% If you insist on using \LaTeX\ for \docstrip\ (really,
% \docstrip\ does not need \LaTeX), then inform the autodetect routine
% about your intention:
% \begin{quote}
%   \verb|latex \let\install=y\input{iflang.dtx}|
% \end{quote}
% Do not forget to quote the argument according to the demands
% of your shell.
%
% \paragraph{Generating the documentation.}
% You can use both the \xfile{.dtx} or the \xfile{.drv} to generate
% the documentation. The process can be configured by the
% configuration file \xfile{ltxdoc.cfg}. For instance, put this
% line into this file, if you want to have A4 as paper format:
% \begin{quote}
%   \verb|\PassOptionsToClass{a4paper}{article}|
% \end{quote}
% An example follows how to generate the
% documentation with pdf\LaTeX:
% \begin{quote}
%\begin{verbatim}
%pdflatex iflang.dtx
%makeindex -s gind.ist iflang.idx
%pdflatex iflang.dtx
%makeindex -s gind.ist iflang.idx
%pdflatex iflang.dtx
%\end{verbatim}
% \end{quote}
%
% \section{Catalogue}
%
% The following XML file can be used as source for the
% \href{http://mirror.ctan.org/help/Catalogue/catalogue.html}{\TeX\ Catalogue}.
% The elements \texttt{caption} and \texttt{description} are imported
% from the original XML file from the Catalogue.
% The name of the XML file in the Catalogue is \xfile{iflang.xml}.
%    \begin{macrocode}
%<*catalogue>
<?xml version='1.0' encoding='us-ascii'?>
<!DOCTYPE entry SYSTEM 'catalogue.dtd'>
<entry datestamp='$Date$' modifier='$Author$' id='iflang'>
  <name>iflang</name>
  <caption>Expandable checks for the current language.</caption>
  <authorref id='auth:oberdiek'/>
  <copyright owner='Heiko Oberdiek' year='2007'/>
  <license type='lppl1.3'/>
  <version number='1.6'/>
  <description>
    This package provides expandable checks for the current language
    based on macro <tt>\languagename</tt> or hyphenation patterns.
    <p/>
    The package is part of the <xref refid='oberdiek'>oberdiek</xref> bundle.
  </description>
  <documentation details='Package documentation'
      href='ctan:/macros/latex/contrib/oberdiek/iflang.pdf'/>
  <ctan file='true' path='/macros/latex/contrib/oberdiek/iflang.dtx'/>
  <miktex location='oberdiek'/>
  <texlive location='oberdiek'/>
  <install path='/macros/latex/contrib/oberdiek/oberdiek.tds.zip'/>
</entry>
%</catalogue>
%    \end{macrocode}
%
% \section{Acknowledgement}
% I wish to thank:
% \begin{description}
% \item[Markus Kohm] Useful hints for version 1.2.
% \end{description}
%
% \begin{History}
%   \begin{Version}{2007/04/10 v1.0}
%   \item
%     First public version.
%   \end{Version}
%   \begin{Version}{2007/04/11 v1.1}
%   \item
%     Line ends sanitized.
%   \end{Version}
%   \begin{Version}{2007/04/12 v1.2}
%   \item
%     Initialization of \cs{languagename} in case of \xfile{etex.src}.
%   \item
%     Some sanity tests added.
%   \item
%     Documentation improved.
%   \end{Version}
%   \begin{Version}{2007/04/26 v1.3}
%   \item
%     Use of package \xpackage{infwarerr}.
%   \end{Version}
%   \begin{Version}{2007/09/09 v1.4}
%   \item
%     Bug fix: \cs{IfLang@StrEqual} $\rightarrow$ \cs{IfLangStrEqual}
%     (Gabriele Balducci).
%   \item
%     Catcode section rewritten.
%   \end{Version}
%   \begin{Version}{2007/11/11 v1.5}
%   \item
%     Use of package \xpackage{pdftexcmds} for \LuaTeX\ support.
%   \end{Version}
%   \begin{Version}{2016/05/16 v1.6}
%   \item
%     Documentation updates.
%   \end{Version}
% \end{History}
%
% \PrintIndex
%
% \Finale
\endinput

%        (quote the arguments according to the demands of your shell)
%
% Documentation:
%    (a) If iflang.drv is present:
%           latex iflang.drv
%    (b) Without iflang.drv:
%           latex iflang.dtx; ...
%    The class ltxdoc loads the configuration file ltxdoc.cfg
%    if available. Here you can specify further options, e.g.
%    use A4 as paper format:
%       \PassOptionsToClass{a4paper}{article}
%
%    Programm calls to get the documentation (example):
%       pdflatex iflang.dtx
%       makeindex -s gind.ist iflang.idx
%       pdflatex iflang.dtx
%       makeindex -s gind.ist iflang.idx
%       pdflatex iflang.dtx
%
% Installation:
%    TDS:tex/generic/oberdiek/iflang.sty
%    TDS:doc/latex/oberdiek/iflang.pdf
%    TDS:doc/latex/oberdiek/test/iflang-test1.tex
%    TDS:doc/latex/oberdiek/test/iflang-test2.tex
%    TDS:doc/latex/oberdiek/test/iflang-test3.tex
%    TDS:doc/latex/oberdiek/test/iflang-test4.tex
%    TDS:doc/latex/oberdiek/test/iflang-test5.tex
%    TDS:source/latex/oberdiek/iflang.dtx
%
%<*ignore>
\begingroup
  \catcode123=1 %
  \catcode125=2 %
  \def\x{LaTeX2e}%
\expandafter\endgroup
\ifcase 0\ifx\install y1\fi\expandafter
         \ifx\csname processbatchFile\endcsname\relax\else1\fi
         \ifx\fmtname\x\else 1\fi\relax
\else\csname fi\endcsname
%</ignore>
%<*install>
\input docstrip.tex
\Msg{************************************************************************}
\Msg{* Installation}
\Msg{* Package: iflang 2016/05/16 v1.6 Checks for the current language (HO)}
\Msg{************************************************************************}

\keepsilent
\askforoverwritefalse

\let\MetaPrefix\relax
\preamble

This is a generated file.

Project: iflang
Version: 2016/05/16 v1.6

Copyright (C) 2007 by
   Heiko Oberdiek <heiko.oberdiek at googlemail.com>

This work may be distributed and/or modified under the
conditions of the LaTeX Project Public License, either
version 1.3c of this license or (at your option) any later
version. This version of this license is in
   http://www.latex-project.org/lppl/lppl-1-3c.txt
and the latest version of this license is in
   http://www.latex-project.org/lppl.txt
and version 1.3 or later is part of all distributions of
LaTeX version 2005/12/01 or later.

This work has the LPPL maintenance status "maintained".

This Current Maintainer of this work is Heiko Oberdiek.

The Base Interpreter refers to any `TeX-Format',
because some files are installed in TDS:tex/generic//.

This work consists of the main source file iflang.dtx
and the derived files
   iflang.sty, iflang.pdf, iflang.ins, iflang.drv, iflang-test1.tex,
   iflang-test2.tex, iflang-test3.tex, iflang-test4.tex,
   iflang-test5.tex.

\endpreamble
\let\MetaPrefix\DoubleperCent

\generate{%
  \file{iflang.ins}{\from{iflang.dtx}{install}}%
  \file{iflang.drv}{\from{iflang.dtx}{driver}}%
  \usedir{tex/generic/oberdiek}%
  \file{iflang.sty}{\from{iflang.dtx}{package}}%
  \usedir{doc/latex/oberdiek/test}%
  \file{iflang-test1.tex}{\from{iflang.dtx}{test1}}%
  \file{iflang-test2.tex}{\from{iflang.dtx}{test2}}%
  \file{iflang-test3.tex}{\from{iflang.dtx}{test3}}%
  \file{iflang-test4.tex}{\from{iflang.dtx}{test4}}%
  \file{iflang-test5.tex}{\from{iflang.dtx}{test5}}%
  \nopreamble
  \nopostamble
  \usedir{source/latex/oberdiek/catalogue}%
  \file{iflang.xml}{\from{iflang.dtx}{catalogue}}%
}

\catcode32=13\relax% active space
\let =\space%
\Msg{************************************************************************}
\Msg{*}
\Msg{* To finish the installation you have to move the following}
\Msg{* file into a directory searched by TeX:}
\Msg{*}
\Msg{*     iflang.sty}
\Msg{*}
\Msg{* To produce the documentation run the file `iflang.drv'}
\Msg{* through LaTeX.}
\Msg{*}
\Msg{* Happy TeXing!}
\Msg{*}
\Msg{************************************************************************}

\endbatchfile
%</install>
%<*ignore>
\fi
%</ignore>
%<*driver>
\NeedsTeXFormat{LaTeX2e}
\ProvidesFile{iflang.drv}%
  [2016/05/16 v1.6 Checks for the current language (HO)]%
\documentclass{ltxdoc}
\usepackage{holtxdoc}[2011/11/22]
\begin{document}
  \DocInput{iflang.dtx}%
\end{document}
%</driver>
% \fi
%
%
% \CharacterTable
%  {Upper-case    \A\B\C\D\E\F\G\H\I\J\K\L\M\N\O\P\Q\R\S\T\U\V\W\X\Y\Z
%   Lower-case    \a\b\c\d\e\f\g\h\i\j\k\l\m\n\o\p\q\r\s\t\u\v\w\x\y\z
%   Digits        \0\1\2\3\4\5\6\7\8\9
%   Exclamation   \!     Double quote  \"     Hash (number) \#
%   Dollar        \$     Percent       \%     Ampersand     \&
%   Acute accent  \'     Left paren    \(     Right paren   \)
%   Asterisk      \*     Plus          \+     Comma         \,
%   Minus         \-     Point         \.     Solidus       \/
%   Colon         \:     Semicolon     \;     Less than     \<
%   Equals        \=     Greater than  \>     Question mark \?
%   Commercial at \@     Left bracket  \[     Backslash     \\
%   Right bracket \]     Circumflex    \^     Underscore    \_
%   Grave accent  \`     Left brace    \{     Vertical bar  \|
%   Right brace   \}     Tilde         \~}
%
% \GetFileInfo{iflang.drv}
%
% \title{The \xpackage{iflang} package}
% \date{2016/05/16 v1.6}
% \author{Heiko Oberdiek\thanks
% {Please report any issues at https://github.com/ho-tex/oberdiek/issues}\\
% \xemail{heiko.oberdiek at googlemail.com}}
%
% \maketitle
%
% \begin{abstract}
% This package provides expandible checks for the current language
% based on macro \cs{languagename} or hyphenation patterns.
% \end{abstract}
%
% \tableofcontents
%
% \section{Documentation}
%
% Package \xpackage{babel} defines \cs{iflanguagename}. As first
% argument it takes a language name and executes the second or
% third argument depending on the current language. This
% language test is based on hypenation patterns. However, it
% is possible that different languages or dialects share the
% same patterns. In such cases \cs{iflanguagename} fails.
%
% However, package \xpackage{babel} and some other packages
% such as \xpackage{german} or \xpackage{ngerman} store
% the language name in the macro \cs{languagename} if
% \cs{selectlanguage} is called.
%
% \begin{declcs}{IfLanguageName} \M{lang} \M{then} \M{else}
% \end{declcs}
% Makro \cs{IfLanguageName} compares language \meta{lang} with
% the current setting of macro \cs{languagename}. If both
% contains the same name then the \meta{then} part is called,
% otherwise the \meta{else} part.
%
% The macro is expandable. Thus it can be safely used inside \cs{edef}
% or \cs{csname}. If case of errors like an undefined \cs{languagename}
% the \meta{else} part is executed.
%
% Note: Macro \cs{IfLanguageName} relies on the fact, that
% \cs{languagename} is set correctly:
% \begin{description}
% \item[Package \xpackage{babel}:]
%    \mbox{}\\
%    Full support of \cs{languagename} in its language switching commands.
% \item[Format based on \xpackage{babel} (\xfile{language.dat}):]
%    \mbox{}\\
%    If package \xpackage{babel} is not used (or not yet loaded),
%    then \xpackage{babel}'s
%    \xfile{hyphen.cfg} has set \cs{languagename} to the last
%    language in \xfile{language.dat}, but \cs{language} (current
%    patterns) is zero and points to the first language.
%    Thus the value of \cs{languagename} is basically garbage.
%    Package \xpackage{iflang} warns if \cs{languagename} and
%    \cs{language} do not fit.
%    This can be fixed by loading package \xpackage{babel} previously.
% \item[Format based on \eTeX's \xfile{etex.src} (\xfile{language.def}):]
%    \mbox{}\\
%    Unhappily it does not support
%    \cs{languagename}. Thus this package hooks into \cs{uselanguage}
%    to get \cs{languagename} defined and updated there.
%    At package loading time the changed \cs{uselanguage} has not
%    been called yet. Thus package \xpackage{iflang} tries
%    |USenglish|. This is the definite default language of \xfile{etex.src}.
%    If the current patterns suit this default language, an
%    undefined \cs{languagename} is set to this language. Otherwise
%    a \cs{languagename} remains undefined and a warning is given.
% \end{description}
%
% \begin{declcs}{IfLanguagePatterns} \M{lang} \M{then} \M{else}
% \end{declcs}
% This macro behaves similar to \cs{IfLanguageName}. But the
% language test is based on the current pattern in force (\cs{language}).
% Also this macro is expandable, in case of errors the \meta{else} part
% is called.
%
% The following naming convention for the pattern are supported:
% \begin{description}
% \item[\xpackage{babel}/\xfile{language.dat}]:
%    \cs{l@}\meta{language}
% \item[\xpackage{etex.src}/\xfile{language.def}]:
%    \cs{lang@}\meta{language}
% \end{description}
% Package \xpackage{iflang} looks for \cs{uselanguage} (defined in
% \xfile{etex.src}) to find out the naming convention in use.
%
% \StopEventually{
% }
%
% \section{Implementation}
%
%    \begin{macrocode}
%<*package>
%    \end{macrocode}
%
% \subsection{Reload check and package identification}
%    Reload check, especially if the package is not used with \LaTeX.
%    \begin{macrocode}
\begingroup\catcode61\catcode48\catcode32=10\relax%
  \catcode13=5 % ^^M
  \endlinechar=13 %
  \catcode35=6 % #
  \catcode39=12 % '
  \catcode44=12 % ,
  \catcode45=12 % -
  \catcode46=12 % .
  \catcode58=12 % :
  \catcode64=11 % @
  \catcode123=1 % {
  \catcode125=2 % }
  \expandafter\let\expandafter\x\csname ver@iflang.sty\endcsname
  \ifx\x\relax % plain-TeX, first loading
  \else
    \def\empty{}%
    \ifx\x\empty % LaTeX, first loading,
      % variable is initialized, but \ProvidesPackage not yet seen
    \else
      \expandafter\ifx\csname PackageInfo\endcsname\relax
        \def\x#1#2{%
          \immediate\write-1{Package #1 Info: #2.}%
        }%
      \else
        \def\x#1#2{\PackageInfo{#1}{#2, stopped}}%
      \fi
      \x{iflang}{The package is already loaded}%
      \aftergroup\endinput
    \fi
  \fi
\endgroup%
%    \end{macrocode}
%    Package identification:
%    \begin{macrocode}
\begingroup\catcode61\catcode48\catcode32=10\relax%
  \catcode13=5 % ^^M
  \endlinechar=13 %
  \catcode35=6 % #
  \catcode39=12 % '
  \catcode40=12 % (
  \catcode41=12 % )
  \catcode44=12 % ,
  \catcode45=12 % -
  \catcode46=12 % .
  \catcode47=12 % /
  \catcode58=12 % :
  \catcode64=11 % @
  \catcode91=12 % [
  \catcode93=12 % ]
  \catcode123=1 % {
  \catcode125=2 % }
  \expandafter\ifx\csname ProvidesPackage\endcsname\relax
    \def\x#1#2#3[#4]{\endgroup
      \immediate\write-1{Package: #3 #4}%
      \xdef#1{#4}%
    }%
  \else
    \def\x#1#2[#3]{\endgroup
      #2[{#3}]%
      \ifx#1\@undefined
        \xdef#1{#3}%
      \fi
      \ifx#1\relax
        \xdef#1{#3}%
      \fi
    }%
  \fi
\expandafter\x\csname ver@iflang.sty\endcsname
\ProvidesPackage{iflang}%
  [2016/05/16 v1.6 Checks for the current language (HO)]%
%    \end{macrocode}
%
%    \begin{macrocode}
\begingroup\catcode61\catcode48\catcode32=10\relax%
  \catcode13=5 % ^^M
  \endlinechar=13 %
  \catcode123=1 % {
  \catcode125=2 % }
  \catcode64=11 % @
  \def\x{\endgroup
    \expandafter\edef\csname IfLang@AtEnd\endcsname{%
      \endlinechar=\the\endlinechar\relax
      \catcode13=\the\catcode13\relax
      \catcode32=\the\catcode32\relax
      \catcode35=\the\catcode35\relax
      \catcode61=\the\catcode61\relax
      \catcode64=\the\catcode64\relax
      \catcode123=\the\catcode123\relax
      \catcode125=\the\catcode125\relax
    }%
  }%
\x\catcode61\catcode48\catcode32=10\relax%
\catcode13=5 % ^^M
\endlinechar=13 %
\catcode35=6 % #
\catcode64=11 % @
\catcode123=1 % {
\catcode125=2 % }
\def\TMP@EnsureCode#1#2{%
  \edef\IfLang@AtEnd{%
    \IfLang@AtEnd
    \catcode#1=\the\catcode#1\relax
  }%
  \catcode#1=#2\relax
}
\TMP@EnsureCode{39}{12}% '
\TMP@EnsureCode{40}{12}% (
\TMP@EnsureCode{41}{12}% )
\TMP@EnsureCode{44}{12}% ,
\TMP@EnsureCode{46}{12}% .
\TMP@EnsureCode{47}{12}% /
\TMP@EnsureCode{58}{12}% :
\TMP@EnsureCode{91}{12}% [
\TMP@EnsureCode{93}{12}% ]
\edef\IfLang@AtEnd{\IfLang@AtEnd\noexpand\endinput}
%    \end{macrocode}
%
% \subsection{Tools}
%
% \subsubsection{Provide some basic macros of \LaTeX}
%
%    \begin{macro}{\@firstoftwo}
%    \begin{macrocode}
\expandafter\ifx\csname @firstoftwo\endcsname\relax
  \long\def\@firstoftwo#1#2{#1}%
\fi
%    \end{macrocode}
%    \end{macro}
%    \begin{macro}{\@secondoftwo}
%    \begin{macrocode}
\expandafter\ifx\csname @secondoftwo\endcsname\relax
  \long\def\@secondoftwo#1#2{#2}%
\fi
%    \end{macrocode}
%    \end{macro}
%
% \subsubsection{Expandible existence check for macros}
%
%    \begin{macro}{\IfLang@IfDefined}
%    \begin{macrocode}
\begingroup\expandafter\expandafter\expandafter\endgroup
\expandafter\ifx\csname ifcsname\endcsname\relax
  \expandafter\@firstoftwo
\else
  \expandafter\@secondoftwo
\fi
{%
  \def\IfLang@IfDefined#1{%
    \expandafter\ifx\csname#1\endcsname\relax
      \expandafter\@secondoftwo
    \else
      \expandafter\@firstoftwo
    \fi
  }%
}{%
  \def\IfLang@IfDefined#1{%
    \ifnum\ifcsname#1\endcsname
            \expandafter\ifx\csname#1\endcsname\relax
              1%
            \else
              0%
            \fi
          \else
            1%
          \fi
          =0 %
      \expandafter\@firstoftwo
    \else
      \expandafter\@secondoftwo
    \fi
  }%
}
%    \end{macrocode}
%    \end{macro}
%
% \subsubsection{Macros for messages}
%
%    \begin{macrocode}
\begingroup\expandafter\expandafter\expandafter\endgroup
\expandafter\ifx\csname RequirePackage\endcsname\relax
  \input infwarerr.sty\relax
  \input pdftexcmds.sty\relax
\else
  \RequirePackage{infwarerr}[2007/09/09]%
  \RequirePackage{pdftexcmds}[2016/05/16]%
\fi
%    \end{macrocode}
%
% \subsubsection{Support for \xfile{etex.src}}
%
%    \begin{macro}{\IfLang@prefix}
%    \begin{macrocode}
\begingroup\expandafter\expandafter\expandafter\endgroup
\expandafter\ifx\csname uselanguage\endcsname\relax
  \@PackageInfoNoLine{iflang}{%
    Naming convention for patterns: babel%
  }%
  \def\IfLang@prefix{l@}%
\else
  \@PackageInfoNoLine{iflang}{%
    Naming convention for patterns: etex.src%
  }%
  \def\IfLang@prefix{lang@}%
  \let\IfLang@OrgUseLanguage\uselanguage
  \def\uselanguage#1{%
    \edef\languagename{#1}%
    \IfLang@OrgUseLanguage{#1}%
  }%
%    \end{macrocode}
%    \end{macro}
%    The first \cs{uselanguage} that is executed as last
%    line in \xfile{language.def} cannot patched this way.
%    However, \xfile{language.def} is very strict. It forces
%    the first added and used language to be |USenglish|.
%    Thus, if \cs{languagename} is not defined, we can
%    quite safely assume |USenglish|. As additional
%    safety precaution the actual used patterns are checked.
%    \begin{macrocode}
  \begingroup\expandafter\expandafter\expandafter\endgroup
  \expandafter\ifx\csname languagename\endcsname\relax
    \begingroup\expandafter\expandafter\expandafter\endgroup
    \expandafter\ifx\csname lang@USenglish\endcsname\relax
      \@PackageWarningNoLine{iflang}{%
        \string\lang@USenglish\space is missing%
      }%
    \else
      \ifnum\lang@USenglish=\language
        \def\languagename{USenglish}%
      \else
        \@PackageWarningNoLine{iflang}{%
          \string\languagename\space is not set,\MessageBreak
          current language is unknown%
        }%
      \fi
    \fi
  \fi
\fi
\begingroup\expandafter\expandafter\expandafter\endgroup
\expandafter\ifx\csname languagename\endcsname\relax
  \@PackageInfoNoLine{iflang}{%
    \string\languagename\space is not set%
  }%
\fi
%    \end{macrocode}
%
% \subsection{\cs{IfLanguagePatterns}}
%
%    \begin{macro}{\IfLanguagePatterns}
%    \begin{macrocode}
\def\IfLanguagePatterns#1{%
  \ifnum\IfLang@IfDefined{\IfLang@prefix#1}{%
          \ifnum\csname\IfLang@prefix#1\endcsname=\language
            0%
          \else
            1%
          \fi
        }{1}=0 %
    \expandafter\@firstoftwo
  \else
    \expandafter\@secondoftwo
  \fi
}
%    \end{macrocode}
%    \end{macro}
%
% \subsection{\cs{IfLanguageName}}
%
%    \begin{macrocode}
\begingroup\expandafter\expandafter\expandafter\endgroup
\expandafter\ifx\csname pdf@strcmp\endcsname\relax
  \expandafter\@firstoftwo
\else
  \expandafter\@secondoftwo
\fi
{%
%    \end{macrocode}
%    We do not have \cs{pdf@strcmp} (and \cs{pdfstrcmp}).
%    Thus we must define
%    our own expandable string comparison. The following
%    implementation is based on a \TeX\ pearl from David Kastrup, presented
%    at the conference Bacho\TeX~2005:
%    \url{http://www-stary.gust.org.pl/pearls/2005/david-kastrup/bachotex2005-david-kastrup-pearl1.pdf}
%
%    The orignal code allows macros inside the second string.
%    Because also \cs{languagename} might consists of further
%    macros, we need a variant that allows macros in the first
%    string, too.
%    \begin{macrocode}
  \def\IfLang@StrNil{\relax}%
  \def\IfLang@StrEqual#1{%
    \number\IfLang@StrEqualStart{}{}#1\IfLang@StrNil
  }%
  \def\IfLang@StrEqualStart#1#2#3{%
    \ifx#3\IfLang@StrNil
      \IfLang@StrEqualStop
    \fi
    \ifcat\noexpand#3\relax
      \IfLang@StrExpand{#1}{#2}#3%
    \fi
    \IfLang@StrEqualStart{\if#3#1}{#2\fi}%
  }%
  \def\IfLang@StrEqualStop\fi#1\IfLang@StrEqualStart#2#3#4{%
    \fi
    #2#4\relax'#313 %
  }%
  \def\IfLang@StrExpand#1#2#3\fi\IfLang@StrEqualStart#4#5{%
    \fi
    \IfLang@@StrExpand{#1}{#2}#3%
  }%
  \def\IfLang@@StrExpand#1#2#3\IfLang@StrNil{%
    \expandafter\IfLang@@@StrExpand#3\IfLang@StrNil{#1}{#2}%
  }%
  \def\IfLang@@@StrExpand#1\IfLang@StrNil#2#3{%
    \IfLang@StrEqualStart{#2}{#3}#1\IfLang@StrNil
  }%
%    \end{macrocode}
%    \begin{macro}{\IfLanguageName}
%    \begin{macrocode}
  \def\IfLanguageName#1{%
    \ifnum\IfLang@IfDefined{languagename}{%
            \if\expandafter\IfLang@StrEqual\expandafter%
                           {\languagename}{#1}%
              0%
            \else
              1%
            \fi
          }{1}=0 %
      \expandafter\@firstoftwo
    \else
      \expandafter\@secondoftwo
    \fi
  }%
%    \end{macrocode}
%    \end{macro}
%    \begin{macrocode}
}{%
%    \end{macrocode}
%    \begin{macro}{\IfLanguageName}
%    \begin{macrocode}
  \def\IfLanguageName#1{%
    \ifnum\IfLang@IfDefined{languagename}{%
            \pdf@strcmp{#1}{\languagename}%
          }{1}=0 %
      \expandafter\@firstoftwo
    \else
      \expandafter\@secondoftwo
    \fi
  }%
%    \end{macrocode}
%    \end{macro}
%    \begin{macrocode}
}
%    \end{macrocode}
%
% \subsection{Check plausibility of \cs{languagename}}
%
%    \begin{macrocode}
\begingroup\expandafter\expandafter\expandafter\endgroup
\expandafter\ifx\csname languagename\endcsname\relax
\else
  \IfLanguagePatterns{\languagename}{}{%
    \@PackageWarningNoLine{iflang}{%
      Mismatch between \string\language\space
      (patterns)\MessageBreak
      and setting of \string\languagename
    }%
  }%
\fi
%    \end{macrocode}
%
%    \begin{macrocode}
\IfLang@AtEnd%
%</package>
%    \end{macrocode}
%
% \section{Test}
%
% \subsection{Catcode checks for loading}
%
%    \begin{macrocode}
%<*test1>
%    \end{macrocode}
%    \begin{macrocode}
\catcode`\{=1 %
\catcode`\}=2 %
\catcode`\#=6 %
\catcode`\@=11 %
\expandafter\ifx\csname count@\endcsname\relax
  \countdef\count@=255 %
\fi
\expandafter\ifx\csname @gobble\endcsname\relax
  \long\def\@gobble#1{}%
\fi
\expandafter\ifx\csname @firstofone\endcsname\relax
  \long\def\@firstofone#1{#1}%
\fi
\expandafter\ifx\csname loop\endcsname\relax
  \expandafter\@firstofone
\else
  \expandafter\@gobble
\fi
{%
  \def\loop#1\repeat{%
    \def\body{#1}%
    \iterate
  }%
  \def\iterate{%
    \body
      \let\next\iterate
    \else
      \let\next\relax
    \fi
    \next
  }%
  \let\repeat=\fi
}%
\def\RestoreCatcodes{}
\count@=0 %
\loop
  \edef\RestoreCatcodes{%
    \RestoreCatcodes
    \catcode\the\count@=\the\catcode\count@\relax
  }%
\ifnum\count@<255 %
  \advance\count@ 1 %
\repeat

\def\RangeCatcodeInvalid#1#2{%
  \count@=#1\relax
  \loop
    \catcode\count@=15 %
  \ifnum\count@<#2\relax
    \advance\count@ 1 %
  \repeat
}
\def\RangeCatcodeCheck#1#2#3{%
  \count@=#1\relax
  \loop
    \ifnum#3=\catcode\count@
    \else
      \errmessage{%
        Character \the\count@\space
        with wrong catcode \the\catcode\count@\space
        instead of \number#3%
      }%
    \fi
  \ifnum\count@<#2\relax
    \advance\count@ 1 %
  \repeat
}
\def\space{ }
\expandafter\ifx\csname LoadCommand\endcsname\relax
  \def\LoadCommand{\input iflang.sty\relax}%
\fi
\def\Test{%
  \RangeCatcodeInvalid{0}{47}%
  \RangeCatcodeInvalid{58}{64}%
  \RangeCatcodeInvalid{91}{96}%
  \RangeCatcodeInvalid{123}{255}%
  \catcode`\@=12 %
  \catcode`\\=0 %
  \catcode`\%=14 %
  \LoadCommand
  \RangeCatcodeCheck{0}{36}{15}%
  \RangeCatcodeCheck{37}{37}{14}%
  \RangeCatcodeCheck{38}{47}{15}%
  \RangeCatcodeCheck{48}{57}{12}%
  \RangeCatcodeCheck{58}{63}{15}%
  \RangeCatcodeCheck{64}{64}{12}%
  \RangeCatcodeCheck{65}{90}{11}%
  \RangeCatcodeCheck{91}{91}{15}%
  \RangeCatcodeCheck{92}{92}{0}%
  \RangeCatcodeCheck{93}{96}{15}%
  \RangeCatcodeCheck{97}{122}{11}%
  \RangeCatcodeCheck{123}{255}{15}%
  \RestoreCatcodes
}
\Test
\csname @@end\endcsname
\end
%    \end{macrocode}
%    \begin{macrocode}
%</test1>
%    \end{macrocode}
%
% \subsection{Test with \LaTeX}
%
%    \begin{macrocode}
%<*test2|test3>
%    \end{macrocode}
%    \begin{macrocode}
\NeedsTeXFormat{LaTeX2e}
%<test3>\let\pdfstrcmp\relax
\nofiles
\documentclass{minimal}
\usepackage{qstest}
\IncludeTests{*}
\LogTests{log}{*}{*}
\usepackage[english,naustrian,ngerman]{babel}
\usepackage{iflang}
\begin{document}
\begin{qstest}{IfLanguagePatterns}{language, pattern}
  \def\test#1#2{%
    \Expect*{\IfLanguagePatterns{#1}{true}{false}}{#2}%
  }%
  \test{ngerman}{true}%
  \test{naustrian}{true}%
  \test{english}{false}%
  \test{foobar}{false}%
\end{qstest}
\begin{qstest}{IfLanguageName}{language, name}
  \def\test#1#2{%
    \Expect*{\IfLanguageName{#1}{true}{false}}{#2}%
  }%
  \test{ngerman}{true}%
  \test{naustrian}{false}%
  \selectlanguage{naustrian}%
  \test{ngerman}{false}%
  \test{naustrian}{true}%
  \test{foobar}{false}%
  %
  \def\languagename{naustrian}%
  \test{naustrian}{true}%
  \test{ngerman}{false}%
  %
  \edef\languagename{\string naustrian}%
  \test{naustrian}{true}%
  \test{ngerman}{false}%
  %
  \def\languagename{naustrian}%
  \makeatletter
  \@onelevel@sanitize\languagename
  \test{naustrian}{true}%
  \test{ngerman}{false}%
  %
  \def\languagename{naustrian}%
  \def\xaustrian{naustrian}%
  \def\xgerman{ngerman}%
  \test{\xaustrian}{true}%
  \test{\xgerman}{false}%
  %
  \def\languagename{\xaustrian}%
  \test{naustrian}{true}%
  \test{ngerman}{false}%
  \test{\xaustrian}{true}%
  \test{\xgerman}{false}%
  \test{\languagename}{true}%
  \test{\languagename\space}{false}%
  %
  \def\languagename{\empty\xaustrian\empty}%
  \test{naustrian}{true}%
  \test{ngerman}{false}%
  \test{\empty\xaustrian\empty}{true}%
  \test{\empty\xgerman\empty}{false}%
\end{qstest}
\begin{qstest}{IfDefined}{defined}
  \makeatletter
  \let\foobar\relax
  \Expect*{\IfLang@IfDefined{foobar}{true}{false}}{false}%
  \Expect*{\ifx\foobar\relax true\else false\fi}{true}%
  \let\foobar\UNDEFINED
  \Expect*{\IfLang@IfDefined{foobar}{true}{false}}{false}%
  \Expect*{\ifx\foobar\relax true\else false\fi}{false}%
  \Expect*{\ifx\foobar\UNDEFINED true\else false\fi}{true}%
\end{qstest}
\end{document}
%    \end{macrocode}
%    \begin{macrocode}
%</test2|test3>
%    \end{macrocode}
%
% \subsection{Test with \plainTeX\ and \eTeX}
%
%    \begin{macrocode}
%<*test4>
%    \end{macrocode}
%    \begin{macrocode}
%% Format `etex' based on `language.def'
\input iflang.sty
\catcode64=12

\def\TestGeneric#1#2#3{%
  \begingroup
    \edef\x{#1{#2}{true}{false}}%
    \edef\y{#3}%
    \ifx\x\y
    \else
      \errmessage{Failed test: \string#1{#2} <> #3}%
    \fi
  \endgroup
}
\def\TestPatterns{\TestGeneric\IfLanguagePatterns}
\def\TestName{\TestGeneric\IfLanguageName}

\TestPatterns{USenglish}{true}
\TestPatterns{ngerman}{false}

\TestName{USenglish}{true}
\TestName{ngerman}{false}

\uselanguage{ngerman}

\TestPatterns{USenglish}{false}
\TestPatterns{ngerman}{true}

\TestName{USenglish}{false}
\TestName{ngerman}{true}

\csname @@end\endcsname
\end
%    \end{macrocode}
%    \begin{macrocode}
%</test4>
%    \end{macrocode}
%
% \subsection{Test with \plainTeX\ and without \eTeX/\pdfTeX}
%
%    \begin{macrocode}
%<*test5>
%    \end{macrocode}
%    \begin{macrocode}
%% Format `tex' (vanilla plain-TeX)
\let\ifcsname\UNDEFINED
\let\pdfstrcmp\UNDEFINED
\input iflang.sty
\catcode64=11

\def\TestDefined#1{%
  \IfLang@IfDefined{foobar}{}{}%
  \ifx\foobar#1%
  \else
    \errmessage{Failed test: \string\foobar <> \string#1}%
  \fi
}
\let\foobar\relax
\TestDefined\relax
\let\foobar\UNDEFINED
\TestDefined\relax

\def\strip@prefix#1>{}
\def\@onelevel@sanitize#1{%
  \edef#1{\expandafter\strip@prefix\meaning#1}%
}
\def\TestCompare#1#2#3{%
  \begingroup
    \edef\x{%
      \if\IfLang@StrEqual{#1}{#2}%
        true%
      \else
        false%
      \fi
    }%
    \def\expect{#3}%
    \ifx\x\expect
    \else
      \def\a{#1}%
      \@onelevel@sanitize\a
      \def\b{#2}%
      \@onelevel@sanitize\b
      \errmessage{Failed test: `\a'='\b' <> \expect}%
    \fi
  \endgroup
}
\TestCompare{junk}{junk}{true}
\TestCompare{}{}{true}
\TestCompare{a}{b}{false}
\TestCompare{aa}{bb}{false}
\def\a{ax}
\def\b{bx}
\def\c{\a\b}
\def\d{\c\b}
\def\exch#1#2{#2#1}
\def\gobble#1{}
\TestCompare{\gobble a}{}{true}
\TestCompare{}{\gobble a}{true}
\TestCompare{a\exch xyb}{ayxb}{true}
\TestCompare{\c}{\c}{true}
\TestCompare{\d}{\c\b}{true}

\csname @@end\endcsname
\end
%    \end{macrocode}
%    \begin{macrocode}
%</test5>
%    \end{macrocode}
%
% \section{Installation}
%
% \subsection{Download}
%
% \paragraph{Package.} This package is available on
% CTAN\footnote{\url{http://ctan.org/pkg/iflang}}:
% \begin{description}
% \item[\CTAN{macros/latex/contrib/oberdiek/iflang.dtx}] The source file.
% \item[\CTAN{macros/latex/contrib/oberdiek/iflang.pdf}] Documentation.
% \end{description}
%
%
% \paragraph{Bundle.} All the packages of the bundle `oberdiek'
% are also available in a TDS compliant ZIP archive. There
% the packages are already unpacked and the documentation files
% are generated. The files and directories obey the TDS standard.
% \begin{description}
% \item[\CTAN{install/macros/latex/contrib/oberdiek.tds.zip}]
% \end{description}
% \emph{TDS} refers to the standard ``A Directory Structure
% for \TeX\ Files'' (\CTAN{tds/tds.pdf}). Directories
% with \xfile{texmf} in their name are usually organized this way.
%
% \subsection{Bundle installation}
%
% \paragraph{Unpacking.} Unpack the \xfile{oberdiek.tds.zip} in the
% TDS tree (also known as \xfile{texmf} tree) of your choice.
% Example (linux):
% \begin{quote}
%   |unzip oberdiek.tds.zip -d ~/texmf|
% \end{quote}
%
% \paragraph{Script installation.}
% Check the directory \xfile{TDS:scripts/oberdiek/} for
% scripts that need further installation steps.
% Package \xpackage{attachfile2} comes with the Perl script
% \xfile{pdfatfi.pl} that should be installed in such a way
% that it can be called as \texttt{pdfatfi}.
% Example (linux):
% \begin{quote}
%   |chmod +x scripts/oberdiek/pdfatfi.pl|\\
%   |cp scripts/oberdiek/pdfatfi.pl /usr/local/bin/|
% \end{quote}
%
% \subsection{Package installation}
%
% \paragraph{Unpacking.} The \xfile{.dtx} file is a self-extracting
% \docstrip\ archive. The files are extracted by running the
% \xfile{.dtx} through \plainTeX:
% \begin{quote}
%   \verb|tex iflang.dtx|
% \end{quote}
%
% \paragraph{TDS.} Now the different files must be moved into
% the different directories in your installation TDS tree
% (also known as \xfile{texmf} tree):
% \begin{quote}
% \def\t{^^A
% \begin{tabular}{@{}>{\ttfamily}l@{ $\rightarrow$ }>{\ttfamily}l@{}}
%   iflang.sty & tex/generic/oberdiek/iflang.sty\\
%   iflang.pdf & doc/latex/oberdiek/iflang.pdf\\
%   test/iflang-test1.tex & doc/latex/oberdiek/test/iflang-test1.tex\\
%   test/iflang-test2.tex & doc/latex/oberdiek/test/iflang-test2.tex\\
%   test/iflang-test3.tex & doc/latex/oberdiek/test/iflang-test3.tex\\
%   test/iflang-test4.tex & doc/latex/oberdiek/test/iflang-test4.tex\\
%   test/iflang-test5.tex & doc/latex/oberdiek/test/iflang-test5.tex\\
%   iflang.dtx & source/latex/oberdiek/iflang.dtx\\
% \end{tabular}^^A
% }^^A
% \sbox0{\t}^^A
% \ifdim\wd0>\linewidth
%   \begingroup
%     \advance\linewidth by\leftmargin
%     \advance\linewidth by\rightmargin
%   \edef\x{\endgroup
%     \def\noexpand\lw{\the\linewidth}^^A
%   }\x
%   \def\lwbox{^^A
%     \leavevmode
%     \hbox to \linewidth{^^A
%       \kern-\leftmargin\relax
%       \hss
%       \usebox0
%       \hss
%       \kern-\rightmargin\relax
%     }^^A
%   }^^A
%   \ifdim\wd0>\lw
%     \sbox0{\small\t}^^A
%     \ifdim\wd0>\linewidth
%       \ifdim\wd0>\lw
%         \sbox0{\footnotesize\t}^^A
%         \ifdim\wd0>\linewidth
%           \ifdim\wd0>\lw
%             \sbox0{\scriptsize\t}^^A
%             \ifdim\wd0>\linewidth
%               \ifdim\wd0>\lw
%                 \sbox0{\tiny\t}^^A
%                 \ifdim\wd0>\linewidth
%                   \lwbox
%                 \else
%                   \usebox0
%                 \fi
%               \else
%                 \lwbox
%               \fi
%             \else
%               \usebox0
%             \fi
%           \else
%             \lwbox
%           \fi
%         \else
%           \usebox0
%         \fi
%       \else
%         \lwbox
%       \fi
%     \else
%       \usebox0
%     \fi
%   \else
%     \lwbox
%   \fi
% \else
%   \usebox0
% \fi
% \end{quote}
% If you have a \xfile{docstrip.cfg} that configures and enables \docstrip's
% TDS installing feature, then some files can already be in the right
% place, see the documentation of \docstrip.
%
% \subsection{Refresh file name databases}
%
% If your \TeX~distribution
% (\teTeX, \mikTeX, \dots) relies on file name databases, you must refresh
% these. For example, \teTeX\ users run \verb|texhash| or
% \verb|mktexlsr|.
%
% \subsection{Some details for the interested}
%
% \paragraph{Attached source.}
%
% The PDF documentation on CTAN also includes the
% \xfile{.dtx} source file. It can be extracted by
% AcrobatReader 6 or higher. Another option is \textsf{pdftk},
% e.g. unpack the file into the current directory:
% \begin{quote}
%   \verb|pdftk iflang.pdf unpack_files output .|
% \end{quote}
%
% \paragraph{Unpacking with \LaTeX.}
% The \xfile{.dtx} chooses its action depending on the format:
% \begin{description}
% \item[\plainTeX:] Run \docstrip\ and extract the files.
% \item[\LaTeX:] Generate the documentation.
% \end{description}
% If you insist on using \LaTeX\ for \docstrip\ (really,
% \docstrip\ does not need \LaTeX), then inform the autodetect routine
% about your intention:
% \begin{quote}
%   \verb|latex \let\install=y% \iffalse meta-comment
%
% File: iflang.dtx
% Version: 2016/05/16 v1.6
% Info: Checks for the current language
%
% Copyright (C) 2007 by
%    Heiko Oberdiek <heiko.oberdiek at googlemail.com>
%    2016
%    https://github.com/ho-tex/oberdiek/issues
%
% This work may be distributed and/or modified under the
% conditions of the LaTeX Project Public License, either
% version 1.3c of this license or (at your option) any later
% version. This version of this license is in
%    http://www.latex-project.org/lppl/lppl-1-3c.txt
% and the latest version of this license is in
%    http://www.latex-project.org/lppl.txt
% and version 1.3 or later is part of all distributions of
% LaTeX version 2005/12/01 or later.
%
% This work has the LPPL maintenance status "maintained".
%
% This Current Maintainer of this work is Heiko Oberdiek.
%
% The Base Interpreter refers to any `TeX-Format',
% because some files are installed in TDS:tex/generic//.
%
% This work consists of the main source file iflang.dtx
% and the derived files
%    iflang.sty, iflang.pdf, iflang.ins, iflang.drv, iflang-test1.tex,
%    iflang-test2.tex, iflang-test3.tex, iflang-test4.tex,
%    iflang-test5.tex.
%
% Distribution:
%    CTAN:macros/latex/contrib/oberdiek/iflang.dtx
%    CTAN:macros/latex/contrib/oberdiek/iflang.pdf
%
% Unpacking:
%    (a) If iflang.ins is present:
%           tex iflang.ins
%    (b) Without iflang.ins:
%           tex iflang.dtx
%    (c) If you insist on using LaTeX
%           latex \let\install=y\input{iflang.dtx}
%        (quote the arguments according to the demands of your shell)
%
% Documentation:
%    (a) If iflang.drv is present:
%           latex iflang.drv
%    (b) Without iflang.drv:
%           latex iflang.dtx; ...
%    The class ltxdoc loads the configuration file ltxdoc.cfg
%    if available. Here you can specify further options, e.g.
%    use A4 as paper format:
%       \PassOptionsToClass{a4paper}{article}
%
%    Programm calls to get the documentation (example):
%       pdflatex iflang.dtx
%       makeindex -s gind.ist iflang.idx
%       pdflatex iflang.dtx
%       makeindex -s gind.ist iflang.idx
%       pdflatex iflang.dtx
%
% Installation:
%    TDS:tex/generic/oberdiek/iflang.sty
%    TDS:doc/latex/oberdiek/iflang.pdf
%    TDS:doc/latex/oberdiek/test/iflang-test1.tex
%    TDS:doc/latex/oberdiek/test/iflang-test2.tex
%    TDS:doc/latex/oberdiek/test/iflang-test3.tex
%    TDS:doc/latex/oberdiek/test/iflang-test4.tex
%    TDS:doc/latex/oberdiek/test/iflang-test5.tex
%    TDS:source/latex/oberdiek/iflang.dtx
%
%<*ignore>
\begingroup
  \catcode123=1 %
  \catcode125=2 %
  \def\x{LaTeX2e}%
\expandafter\endgroup
\ifcase 0\ifx\install y1\fi\expandafter
         \ifx\csname processbatchFile\endcsname\relax\else1\fi
         \ifx\fmtname\x\else 1\fi\relax
\else\csname fi\endcsname
%</ignore>
%<*install>
\input docstrip.tex
\Msg{************************************************************************}
\Msg{* Installation}
\Msg{* Package: iflang 2016/05/16 v1.6 Checks for the current language (HO)}
\Msg{************************************************************************}

\keepsilent
\askforoverwritefalse

\let\MetaPrefix\relax
\preamble

This is a generated file.

Project: iflang
Version: 2016/05/16 v1.6

Copyright (C) 2007 by
   Heiko Oberdiek <heiko.oberdiek at googlemail.com>

This work may be distributed and/or modified under the
conditions of the LaTeX Project Public License, either
version 1.3c of this license or (at your option) any later
version. This version of this license is in
   http://www.latex-project.org/lppl/lppl-1-3c.txt
and the latest version of this license is in
   http://www.latex-project.org/lppl.txt
and version 1.3 or later is part of all distributions of
LaTeX version 2005/12/01 or later.

This work has the LPPL maintenance status "maintained".

This Current Maintainer of this work is Heiko Oberdiek.

The Base Interpreter refers to any `TeX-Format',
because some files are installed in TDS:tex/generic//.

This work consists of the main source file iflang.dtx
and the derived files
   iflang.sty, iflang.pdf, iflang.ins, iflang.drv, iflang-test1.tex,
   iflang-test2.tex, iflang-test3.tex, iflang-test4.tex,
   iflang-test5.tex.

\endpreamble
\let\MetaPrefix\DoubleperCent

\generate{%
  \file{iflang.ins}{\from{iflang.dtx}{install}}%
  \file{iflang.drv}{\from{iflang.dtx}{driver}}%
  \usedir{tex/generic/oberdiek}%
  \file{iflang.sty}{\from{iflang.dtx}{package}}%
  \usedir{doc/latex/oberdiek/test}%
  \file{iflang-test1.tex}{\from{iflang.dtx}{test1}}%
  \file{iflang-test2.tex}{\from{iflang.dtx}{test2}}%
  \file{iflang-test3.tex}{\from{iflang.dtx}{test3}}%
  \file{iflang-test4.tex}{\from{iflang.dtx}{test4}}%
  \file{iflang-test5.tex}{\from{iflang.dtx}{test5}}%
  \nopreamble
  \nopostamble
  \usedir{source/latex/oberdiek/catalogue}%
  \file{iflang.xml}{\from{iflang.dtx}{catalogue}}%
}

\catcode32=13\relax% active space
\let =\space%
\Msg{************************************************************************}
\Msg{*}
\Msg{* To finish the installation you have to move the following}
\Msg{* file into a directory searched by TeX:}
\Msg{*}
\Msg{*     iflang.sty}
\Msg{*}
\Msg{* To produce the documentation run the file `iflang.drv'}
\Msg{* through LaTeX.}
\Msg{*}
\Msg{* Happy TeXing!}
\Msg{*}
\Msg{************************************************************************}

\endbatchfile
%</install>
%<*ignore>
\fi
%</ignore>
%<*driver>
\NeedsTeXFormat{LaTeX2e}
\ProvidesFile{iflang.drv}%
  [2016/05/16 v1.6 Checks for the current language (HO)]%
\documentclass{ltxdoc}
\usepackage{holtxdoc}[2011/11/22]
\begin{document}
  \DocInput{iflang.dtx}%
\end{document}
%</driver>
% \fi
%
%
% \CharacterTable
%  {Upper-case    \A\B\C\D\E\F\G\H\I\J\K\L\M\N\O\P\Q\R\S\T\U\V\W\X\Y\Z
%   Lower-case    \a\b\c\d\e\f\g\h\i\j\k\l\m\n\o\p\q\r\s\t\u\v\w\x\y\z
%   Digits        \0\1\2\3\4\5\6\7\8\9
%   Exclamation   \!     Double quote  \"     Hash (number) \#
%   Dollar        \$     Percent       \%     Ampersand     \&
%   Acute accent  \'     Left paren    \(     Right paren   \)
%   Asterisk      \*     Plus          \+     Comma         \,
%   Minus         \-     Point         \.     Solidus       \/
%   Colon         \:     Semicolon     \;     Less than     \<
%   Equals        \=     Greater than  \>     Question mark \?
%   Commercial at \@     Left bracket  \[     Backslash     \\
%   Right bracket \]     Circumflex    \^     Underscore    \_
%   Grave accent  \`     Left brace    \{     Vertical bar  \|
%   Right brace   \}     Tilde         \~}
%
% \GetFileInfo{iflang.drv}
%
% \title{The \xpackage{iflang} package}
% \date{2016/05/16 v1.6}
% \author{Heiko Oberdiek\thanks
% {Please report any issues at https://github.com/ho-tex/oberdiek/issues}\\
% \xemail{heiko.oberdiek at googlemail.com}}
%
% \maketitle
%
% \begin{abstract}
% This package provides expandible checks for the current language
% based on macro \cs{languagename} or hyphenation patterns.
% \end{abstract}
%
% \tableofcontents
%
% \section{Documentation}
%
% Package \xpackage{babel} defines \cs{iflanguagename}. As first
% argument it takes a language name and executes the second or
% third argument depending on the current language. This
% language test is based on hypenation patterns. However, it
% is possible that different languages or dialects share the
% same patterns. In such cases \cs{iflanguagename} fails.
%
% However, package \xpackage{babel} and some other packages
% such as \xpackage{german} or \xpackage{ngerman} store
% the language name in the macro \cs{languagename} if
% \cs{selectlanguage} is called.
%
% \begin{declcs}{IfLanguageName} \M{lang} \M{then} \M{else}
% \end{declcs}
% Makro \cs{IfLanguageName} compares language \meta{lang} with
% the current setting of macro \cs{languagename}. If both
% contains the same name then the \meta{then} part is called,
% otherwise the \meta{else} part.
%
% The macro is expandable. Thus it can be safely used inside \cs{edef}
% or \cs{csname}. If case of errors like an undefined \cs{languagename}
% the \meta{else} part is executed.
%
% Note: Macro \cs{IfLanguageName} relies on the fact, that
% \cs{languagename} is set correctly:
% \begin{description}
% \item[Package \xpackage{babel}:]
%    \mbox{}\\
%    Full support of \cs{languagename} in its language switching commands.
% \item[Format based on \xpackage{babel} (\xfile{language.dat}):]
%    \mbox{}\\
%    If package \xpackage{babel} is not used (or not yet loaded),
%    then \xpackage{babel}'s
%    \xfile{hyphen.cfg} has set \cs{languagename} to the last
%    language in \xfile{language.dat}, but \cs{language} (current
%    patterns) is zero and points to the first language.
%    Thus the value of \cs{languagename} is basically garbage.
%    Package \xpackage{iflang} warns if \cs{languagename} and
%    \cs{language} do not fit.
%    This can be fixed by loading package \xpackage{babel} previously.
% \item[Format based on \eTeX's \xfile{etex.src} (\xfile{language.def}):]
%    \mbox{}\\
%    Unhappily it does not support
%    \cs{languagename}. Thus this package hooks into \cs{uselanguage}
%    to get \cs{languagename} defined and updated there.
%    At package loading time the changed \cs{uselanguage} has not
%    been called yet. Thus package \xpackage{iflang} tries
%    |USenglish|. This is the definite default language of \xfile{etex.src}.
%    If the current patterns suit this default language, an
%    undefined \cs{languagename} is set to this language. Otherwise
%    a \cs{languagename} remains undefined and a warning is given.
% \end{description}
%
% \begin{declcs}{IfLanguagePatterns} \M{lang} \M{then} \M{else}
% \end{declcs}
% This macro behaves similar to \cs{IfLanguageName}. But the
% language test is based on the current pattern in force (\cs{language}).
% Also this macro is expandable, in case of errors the \meta{else} part
% is called.
%
% The following naming convention for the pattern are supported:
% \begin{description}
% \item[\xpackage{babel}/\xfile{language.dat}]:
%    \cs{l@}\meta{language}
% \item[\xpackage{etex.src}/\xfile{language.def}]:
%    \cs{lang@}\meta{language}
% \end{description}
% Package \xpackage{iflang} looks for \cs{uselanguage} (defined in
% \xfile{etex.src}) to find out the naming convention in use.
%
% \StopEventually{
% }
%
% \section{Implementation}
%
%    \begin{macrocode}
%<*package>
%    \end{macrocode}
%
% \subsection{Reload check and package identification}
%    Reload check, especially if the package is not used with \LaTeX.
%    \begin{macrocode}
\begingroup\catcode61\catcode48\catcode32=10\relax%
  \catcode13=5 % ^^M
  \endlinechar=13 %
  \catcode35=6 % #
  \catcode39=12 % '
  \catcode44=12 % ,
  \catcode45=12 % -
  \catcode46=12 % .
  \catcode58=12 % :
  \catcode64=11 % @
  \catcode123=1 % {
  \catcode125=2 % }
  \expandafter\let\expandafter\x\csname ver@iflang.sty\endcsname
  \ifx\x\relax % plain-TeX, first loading
  \else
    \def\empty{}%
    \ifx\x\empty % LaTeX, first loading,
      % variable is initialized, but \ProvidesPackage not yet seen
    \else
      \expandafter\ifx\csname PackageInfo\endcsname\relax
        \def\x#1#2{%
          \immediate\write-1{Package #1 Info: #2.}%
        }%
      \else
        \def\x#1#2{\PackageInfo{#1}{#2, stopped}}%
      \fi
      \x{iflang}{The package is already loaded}%
      \aftergroup\endinput
    \fi
  \fi
\endgroup%
%    \end{macrocode}
%    Package identification:
%    \begin{macrocode}
\begingroup\catcode61\catcode48\catcode32=10\relax%
  \catcode13=5 % ^^M
  \endlinechar=13 %
  \catcode35=6 % #
  \catcode39=12 % '
  \catcode40=12 % (
  \catcode41=12 % )
  \catcode44=12 % ,
  \catcode45=12 % -
  \catcode46=12 % .
  \catcode47=12 % /
  \catcode58=12 % :
  \catcode64=11 % @
  \catcode91=12 % [
  \catcode93=12 % ]
  \catcode123=1 % {
  \catcode125=2 % }
  \expandafter\ifx\csname ProvidesPackage\endcsname\relax
    \def\x#1#2#3[#4]{\endgroup
      \immediate\write-1{Package: #3 #4}%
      \xdef#1{#4}%
    }%
  \else
    \def\x#1#2[#3]{\endgroup
      #2[{#3}]%
      \ifx#1\@undefined
        \xdef#1{#3}%
      \fi
      \ifx#1\relax
        \xdef#1{#3}%
      \fi
    }%
  \fi
\expandafter\x\csname ver@iflang.sty\endcsname
\ProvidesPackage{iflang}%
  [2016/05/16 v1.6 Checks for the current language (HO)]%
%    \end{macrocode}
%
%    \begin{macrocode}
\begingroup\catcode61\catcode48\catcode32=10\relax%
  \catcode13=5 % ^^M
  \endlinechar=13 %
  \catcode123=1 % {
  \catcode125=2 % }
  \catcode64=11 % @
  \def\x{\endgroup
    \expandafter\edef\csname IfLang@AtEnd\endcsname{%
      \endlinechar=\the\endlinechar\relax
      \catcode13=\the\catcode13\relax
      \catcode32=\the\catcode32\relax
      \catcode35=\the\catcode35\relax
      \catcode61=\the\catcode61\relax
      \catcode64=\the\catcode64\relax
      \catcode123=\the\catcode123\relax
      \catcode125=\the\catcode125\relax
    }%
  }%
\x\catcode61\catcode48\catcode32=10\relax%
\catcode13=5 % ^^M
\endlinechar=13 %
\catcode35=6 % #
\catcode64=11 % @
\catcode123=1 % {
\catcode125=2 % }
\def\TMP@EnsureCode#1#2{%
  \edef\IfLang@AtEnd{%
    \IfLang@AtEnd
    \catcode#1=\the\catcode#1\relax
  }%
  \catcode#1=#2\relax
}
\TMP@EnsureCode{39}{12}% '
\TMP@EnsureCode{40}{12}% (
\TMP@EnsureCode{41}{12}% )
\TMP@EnsureCode{44}{12}% ,
\TMP@EnsureCode{46}{12}% .
\TMP@EnsureCode{47}{12}% /
\TMP@EnsureCode{58}{12}% :
\TMP@EnsureCode{91}{12}% [
\TMP@EnsureCode{93}{12}% ]
\edef\IfLang@AtEnd{\IfLang@AtEnd\noexpand\endinput}
%    \end{macrocode}
%
% \subsection{Tools}
%
% \subsubsection{Provide some basic macros of \LaTeX}
%
%    \begin{macro}{\@firstoftwo}
%    \begin{macrocode}
\expandafter\ifx\csname @firstoftwo\endcsname\relax
  \long\def\@firstoftwo#1#2{#1}%
\fi
%    \end{macrocode}
%    \end{macro}
%    \begin{macro}{\@secondoftwo}
%    \begin{macrocode}
\expandafter\ifx\csname @secondoftwo\endcsname\relax
  \long\def\@secondoftwo#1#2{#2}%
\fi
%    \end{macrocode}
%    \end{macro}
%
% \subsubsection{Expandible existence check for macros}
%
%    \begin{macro}{\IfLang@IfDefined}
%    \begin{macrocode}
\begingroup\expandafter\expandafter\expandafter\endgroup
\expandafter\ifx\csname ifcsname\endcsname\relax
  \expandafter\@firstoftwo
\else
  \expandafter\@secondoftwo
\fi
{%
  \def\IfLang@IfDefined#1{%
    \expandafter\ifx\csname#1\endcsname\relax
      \expandafter\@secondoftwo
    \else
      \expandafter\@firstoftwo
    \fi
  }%
}{%
  \def\IfLang@IfDefined#1{%
    \ifnum\ifcsname#1\endcsname
            \expandafter\ifx\csname#1\endcsname\relax
              1%
            \else
              0%
            \fi
          \else
            1%
          \fi
          =0 %
      \expandafter\@firstoftwo
    \else
      \expandafter\@secondoftwo
    \fi
  }%
}
%    \end{macrocode}
%    \end{macro}
%
% \subsubsection{Macros for messages}
%
%    \begin{macrocode}
\begingroup\expandafter\expandafter\expandafter\endgroup
\expandafter\ifx\csname RequirePackage\endcsname\relax
  \input infwarerr.sty\relax
  \input pdftexcmds.sty\relax
\else
  \RequirePackage{infwarerr}[2007/09/09]%
  \RequirePackage{pdftexcmds}[2016/05/16]%
\fi
%    \end{macrocode}
%
% \subsubsection{Support for \xfile{etex.src}}
%
%    \begin{macro}{\IfLang@prefix}
%    \begin{macrocode}
\begingroup\expandafter\expandafter\expandafter\endgroup
\expandafter\ifx\csname uselanguage\endcsname\relax
  \@PackageInfoNoLine{iflang}{%
    Naming convention for patterns: babel%
  }%
  \def\IfLang@prefix{l@}%
\else
  \@PackageInfoNoLine{iflang}{%
    Naming convention for patterns: etex.src%
  }%
  \def\IfLang@prefix{lang@}%
  \let\IfLang@OrgUseLanguage\uselanguage
  \def\uselanguage#1{%
    \edef\languagename{#1}%
    \IfLang@OrgUseLanguage{#1}%
  }%
%    \end{macrocode}
%    \end{macro}
%    The first \cs{uselanguage} that is executed as last
%    line in \xfile{language.def} cannot patched this way.
%    However, \xfile{language.def} is very strict. It forces
%    the first added and used language to be |USenglish|.
%    Thus, if \cs{languagename} is not defined, we can
%    quite safely assume |USenglish|. As additional
%    safety precaution the actual used patterns are checked.
%    \begin{macrocode}
  \begingroup\expandafter\expandafter\expandafter\endgroup
  \expandafter\ifx\csname languagename\endcsname\relax
    \begingroup\expandafter\expandafter\expandafter\endgroup
    \expandafter\ifx\csname lang@USenglish\endcsname\relax
      \@PackageWarningNoLine{iflang}{%
        \string\lang@USenglish\space is missing%
      }%
    \else
      \ifnum\lang@USenglish=\language
        \def\languagename{USenglish}%
      \else
        \@PackageWarningNoLine{iflang}{%
          \string\languagename\space is not set,\MessageBreak
          current language is unknown%
        }%
      \fi
    \fi
  \fi
\fi
\begingroup\expandafter\expandafter\expandafter\endgroup
\expandafter\ifx\csname languagename\endcsname\relax
  \@PackageInfoNoLine{iflang}{%
    \string\languagename\space is not set%
  }%
\fi
%    \end{macrocode}
%
% \subsection{\cs{IfLanguagePatterns}}
%
%    \begin{macro}{\IfLanguagePatterns}
%    \begin{macrocode}
\def\IfLanguagePatterns#1{%
  \ifnum\IfLang@IfDefined{\IfLang@prefix#1}{%
          \ifnum\csname\IfLang@prefix#1\endcsname=\language
            0%
          \else
            1%
          \fi
        }{1}=0 %
    \expandafter\@firstoftwo
  \else
    \expandafter\@secondoftwo
  \fi
}
%    \end{macrocode}
%    \end{macro}
%
% \subsection{\cs{IfLanguageName}}
%
%    \begin{macrocode}
\begingroup\expandafter\expandafter\expandafter\endgroup
\expandafter\ifx\csname pdf@strcmp\endcsname\relax
  \expandafter\@firstoftwo
\else
  \expandafter\@secondoftwo
\fi
{%
%    \end{macrocode}
%    We do not have \cs{pdf@strcmp} (and \cs{pdfstrcmp}).
%    Thus we must define
%    our own expandable string comparison. The following
%    implementation is based on a \TeX\ pearl from David Kastrup, presented
%    at the conference Bacho\TeX~2005:
%    \url{http://www-stary.gust.org.pl/pearls/2005/david-kastrup/bachotex2005-david-kastrup-pearl1.pdf}
%
%    The orignal code allows macros inside the second string.
%    Because also \cs{languagename} might consists of further
%    macros, we need a variant that allows macros in the first
%    string, too.
%    \begin{macrocode}
  \def\IfLang@StrNil{\relax}%
  \def\IfLang@StrEqual#1{%
    \number\IfLang@StrEqualStart{}{}#1\IfLang@StrNil
  }%
  \def\IfLang@StrEqualStart#1#2#3{%
    \ifx#3\IfLang@StrNil
      \IfLang@StrEqualStop
    \fi
    \ifcat\noexpand#3\relax
      \IfLang@StrExpand{#1}{#2}#3%
    \fi
    \IfLang@StrEqualStart{\if#3#1}{#2\fi}%
  }%
  \def\IfLang@StrEqualStop\fi#1\IfLang@StrEqualStart#2#3#4{%
    \fi
    #2#4\relax'#313 %
  }%
  \def\IfLang@StrExpand#1#2#3\fi\IfLang@StrEqualStart#4#5{%
    \fi
    \IfLang@@StrExpand{#1}{#2}#3%
  }%
  \def\IfLang@@StrExpand#1#2#3\IfLang@StrNil{%
    \expandafter\IfLang@@@StrExpand#3\IfLang@StrNil{#1}{#2}%
  }%
  \def\IfLang@@@StrExpand#1\IfLang@StrNil#2#3{%
    \IfLang@StrEqualStart{#2}{#3}#1\IfLang@StrNil
  }%
%    \end{macrocode}
%    \begin{macro}{\IfLanguageName}
%    \begin{macrocode}
  \def\IfLanguageName#1{%
    \ifnum\IfLang@IfDefined{languagename}{%
            \if\expandafter\IfLang@StrEqual\expandafter%
                           {\languagename}{#1}%
              0%
            \else
              1%
            \fi
          }{1}=0 %
      \expandafter\@firstoftwo
    \else
      \expandafter\@secondoftwo
    \fi
  }%
%    \end{macrocode}
%    \end{macro}
%    \begin{macrocode}
}{%
%    \end{macrocode}
%    \begin{macro}{\IfLanguageName}
%    \begin{macrocode}
  \def\IfLanguageName#1{%
    \ifnum\IfLang@IfDefined{languagename}{%
            \pdf@strcmp{#1}{\languagename}%
          }{1}=0 %
      \expandafter\@firstoftwo
    \else
      \expandafter\@secondoftwo
    \fi
  }%
%    \end{macrocode}
%    \end{macro}
%    \begin{macrocode}
}
%    \end{macrocode}
%
% \subsection{Check plausibility of \cs{languagename}}
%
%    \begin{macrocode}
\begingroup\expandafter\expandafter\expandafter\endgroup
\expandafter\ifx\csname languagename\endcsname\relax
\else
  \IfLanguagePatterns{\languagename}{}{%
    \@PackageWarningNoLine{iflang}{%
      Mismatch between \string\language\space
      (patterns)\MessageBreak
      and setting of \string\languagename
    }%
  }%
\fi
%    \end{macrocode}
%
%    \begin{macrocode}
\IfLang@AtEnd%
%</package>
%    \end{macrocode}
%
% \section{Test}
%
% \subsection{Catcode checks for loading}
%
%    \begin{macrocode}
%<*test1>
%    \end{macrocode}
%    \begin{macrocode}
\catcode`\{=1 %
\catcode`\}=2 %
\catcode`\#=6 %
\catcode`\@=11 %
\expandafter\ifx\csname count@\endcsname\relax
  \countdef\count@=255 %
\fi
\expandafter\ifx\csname @gobble\endcsname\relax
  \long\def\@gobble#1{}%
\fi
\expandafter\ifx\csname @firstofone\endcsname\relax
  \long\def\@firstofone#1{#1}%
\fi
\expandafter\ifx\csname loop\endcsname\relax
  \expandafter\@firstofone
\else
  \expandafter\@gobble
\fi
{%
  \def\loop#1\repeat{%
    \def\body{#1}%
    \iterate
  }%
  \def\iterate{%
    \body
      \let\next\iterate
    \else
      \let\next\relax
    \fi
    \next
  }%
  \let\repeat=\fi
}%
\def\RestoreCatcodes{}
\count@=0 %
\loop
  \edef\RestoreCatcodes{%
    \RestoreCatcodes
    \catcode\the\count@=\the\catcode\count@\relax
  }%
\ifnum\count@<255 %
  \advance\count@ 1 %
\repeat

\def\RangeCatcodeInvalid#1#2{%
  \count@=#1\relax
  \loop
    \catcode\count@=15 %
  \ifnum\count@<#2\relax
    \advance\count@ 1 %
  \repeat
}
\def\RangeCatcodeCheck#1#2#3{%
  \count@=#1\relax
  \loop
    \ifnum#3=\catcode\count@
    \else
      \errmessage{%
        Character \the\count@\space
        with wrong catcode \the\catcode\count@\space
        instead of \number#3%
      }%
    \fi
  \ifnum\count@<#2\relax
    \advance\count@ 1 %
  \repeat
}
\def\space{ }
\expandafter\ifx\csname LoadCommand\endcsname\relax
  \def\LoadCommand{\input iflang.sty\relax}%
\fi
\def\Test{%
  \RangeCatcodeInvalid{0}{47}%
  \RangeCatcodeInvalid{58}{64}%
  \RangeCatcodeInvalid{91}{96}%
  \RangeCatcodeInvalid{123}{255}%
  \catcode`\@=12 %
  \catcode`\\=0 %
  \catcode`\%=14 %
  \LoadCommand
  \RangeCatcodeCheck{0}{36}{15}%
  \RangeCatcodeCheck{37}{37}{14}%
  \RangeCatcodeCheck{38}{47}{15}%
  \RangeCatcodeCheck{48}{57}{12}%
  \RangeCatcodeCheck{58}{63}{15}%
  \RangeCatcodeCheck{64}{64}{12}%
  \RangeCatcodeCheck{65}{90}{11}%
  \RangeCatcodeCheck{91}{91}{15}%
  \RangeCatcodeCheck{92}{92}{0}%
  \RangeCatcodeCheck{93}{96}{15}%
  \RangeCatcodeCheck{97}{122}{11}%
  \RangeCatcodeCheck{123}{255}{15}%
  \RestoreCatcodes
}
\Test
\csname @@end\endcsname
\end
%    \end{macrocode}
%    \begin{macrocode}
%</test1>
%    \end{macrocode}
%
% \subsection{Test with \LaTeX}
%
%    \begin{macrocode}
%<*test2|test3>
%    \end{macrocode}
%    \begin{macrocode}
\NeedsTeXFormat{LaTeX2e}
%<test3>\let\pdfstrcmp\relax
\nofiles
\documentclass{minimal}
\usepackage{qstest}
\IncludeTests{*}
\LogTests{log}{*}{*}
\usepackage[english,naustrian,ngerman]{babel}
\usepackage{iflang}
\begin{document}
\begin{qstest}{IfLanguagePatterns}{language, pattern}
  \def\test#1#2{%
    \Expect*{\IfLanguagePatterns{#1}{true}{false}}{#2}%
  }%
  \test{ngerman}{true}%
  \test{naustrian}{true}%
  \test{english}{false}%
  \test{foobar}{false}%
\end{qstest}
\begin{qstest}{IfLanguageName}{language, name}
  \def\test#1#2{%
    \Expect*{\IfLanguageName{#1}{true}{false}}{#2}%
  }%
  \test{ngerman}{true}%
  \test{naustrian}{false}%
  \selectlanguage{naustrian}%
  \test{ngerman}{false}%
  \test{naustrian}{true}%
  \test{foobar}{false}%
  %
  \def\languagename{naustrian}%
  \test{naustrian}{true}%
  \test{ngerman}{false}%
  %
  \edef\languagename{\string naustrian}%
  \test{naustrian}{true}%
  \test{ngerman}{false}%
  %
  \def\languagename{naustrian}%
  \makeatletter
  \@onelevel@sanitize\languagename
  \test{naustrian}{true}%
  \test{ngerman}{false}%
  %
  \def\languagename{naustrian}%
  \def\xaustrian{naustrian}%
  \def\xgerman{ngerman}%
  \test{\xaustrian}{true}%
  \test{\xgerman}{false}%
  %
  \def\languagename{\xaustrian}%
  \test{naustrian}{true}%
  \test{ngerman}{false}%
  \test{\xaustrian}{true}%
  \test{\xgerman}{false}%
  \test{\languagename}{true}%
  \test{\languagename\space}{false}%
  %
  \def\languagename{\empty\xaustrian\empty}%
  \test{naustrian}{true}%
  \test{ngerman}{false}%
  \test{\empty\xaustrian\empty}{true}%
  \test{\empty\xgerman\empty}{false}%
\end{qstest}
\begin{qstest}{IfDefined}{defined}
  \makeatletter
  \let\foobar\relax
  \Expect*{\IfLang@IfDefined{foobar}{true}{false}}{false}%
  \Expect*{\ifx\foobar\relax true\else false\fi}{true}%
  \let\foobar\UNDEFINED
  \Expect*{\IfLang@IfDefined{foobar}{true}{false}}{false}%
  \Expect*{\ifx\foobar\relax true\else false\fi}{false}%
  \Expect*{\ifx\foobar\UNDEFINED true\else false\fi}{true}%
\end{qstest}
\end{document}
%    \end{macrocode}
%    \begin{macrocode}
%</test2|test3>
%    \end{macrocode}
%
% \subsection{Test with \plainTeX\ and \eTeX}
%
%    \begin{macrocode}
%<*test4>
%    \end{macrocode}
%    \begin{macrocode}
%% Format `etex' based on `language.def'
\input iflang.sty
\catcode64=12

\def\TestGeneric#1#2#3{%
  \begingroup
    \edef\x{#1{#2}{true}{false}}%
    \edef\y{#3}%
    \ifx\x\y
    \else
      \errmessage{Failed test: \string#1{#2} <> #3}%
    \fi
  \endgroup
}
\def\TestPatterns{\TestGeneric\IfLanguagePatterns}
\def\TestName{\TestGeneric\IfLanguageName}

\TestPatterns{USenglish}{true}
\TestPatterns{ngerman}{false}

\TestName{USenglish}{true}
\TestName{ngerman}{false}

\uselanguage{ngerman}

\TestPatterns{USenglish}{false}
\TestPatterns{ngerman}{true}

\TestName{USenglish}{false}
\TestName{ngerman}{true}

\csname @@end\endcsname
\end
%    \end{macrocode}
%    \begin{macrocode}
%</test4>
%    \end{macrocode}
%
% \subsection{Test with \plainTeX\ and without \eTeX/\pdfTeX}
%
%    \begin{macrocode}
%<*test5>
%    \end{macrocode}
%    \begin{macrocode}
%% Format `tex' (vanilla plain-TeX)
\let\ifcsname\UNDEFINED
\let\pdfstrcmp\UNDEFINED
\input iflang.sty
\catcode64=11

\def\TestDefined#1{%
  \IfLang@IfDefined{foobar}{}{}%
  \ifx\foobar#1%
  \else
    \errmessage{Failed test: \string\foobar <> \string#1}%
  \fi
}
\let\foobar\relax
\TestDefined\relax
\let\foobar\UNDEFINED
\TestDefined\relax

\def\strip@prefix#1>{}
\def\@onelevel@sanitize#1{%
  \edef#1{\expandafter\strip@prefix\meaning#1}%
}
\def\TestCompare#1#2#3{%
  \begingroup
    \edef\x{%
      \if\IfLang@StrEqual{#1}{#2}%
        true%
      \else
        false%
      \fi
    }%
    \def\expect{#3}%
    \ifx\x\expect
    \else
      \def\a{#1}%
      \@onelevel@sanitize\a
      \def\b{#2}%
      \@onelevel@sanitize\b
      \errmessage{Failed test: `\a'='\b' <> \expect}%
    \fi
  \endgroup
}
\TestCompare{junk}{junk}{true}
\TestCompare{}{}{true}
\TestCompare{a}{b}{false}
\TestCompare{aa}{bb}{false}
\def\a{ax}
\def\b{bx}
\def\c{\a\b}
\def\d{\c\b}
\def\exch#1#2{#2#1}
\def\gobble#1{}
\TestCompare{\gobble a}{}{true}
\TestCompare{}{\gobble a}{true}
\TestCompare{a\exch xyb}{ayxb}{true}
\TestCompare{\c}{\c}{true}
\TestCompare{\d}{\c\b}{true}

\csname @@end\endcsname
\end
%    \end{macrocode}
%    \begin{macrocode}
%</test5>
%    \end{macrocode}
%
% \section{Installation}
%
% \subsection{Download}
%
% \paragraph{Package.} This package is available on
% CTAN\footnote{\url{http://ctan.org/pkg/iflang}}:
% \begin{description}
% \item[\CTAN{macros/latex/contrib/oberdiek/iflang.dtx}] The source file.
% \item[\CTAN{macros/latex/contrib/oberdiek/iflang.pdf}] Documentation.
% \end{description}
%
%
% \paragraph{Bundle.} All the packages of the bundle `oberdiek'
% are also available in a TDS compliant ZIP archive. There
% the packages are already unpacked and the documentation files
% are generated. The files and directories obey the TDS standard.
% \begin{description}
% \item[\CTAN{install/macros/latex/contrib/oberdiek.tds.zip}]
% \end{description}
% \emph{TDS} refers to the standard ``A Directory Structure
% for \TeX\ Files'' (\CTAN{tds/tds.pdf}). Directories
% with \xfile{texmf} in their name are usually organized this way.
%
% \subsection{Bundle installation}
%
% \paragraph{Unpacking.} Unpack the \xfile{oberdiek.tds.zip} in the
% TDS tree (also known as \xfile{texmf} tree) of your choice.
% Example (linux):
% \begin{quote}
%   |unzip oberdiek.tds.zip -d ~/texmf|
% \end{quote}
%
% \paragraph{Script installation.}
% Check the directory \xfile{TDS:scripts/oberdiek/} for
% scripts that need further installation steps.
% Package \xpackage{attachfile2} comes with the Perl script
% \xfile{pdfatfi.pl} that should be installed in such a way
% that it can be called as \texttt{pdfatfi}.
% Example (linux):
% \begin{quote}
%   |chmod +x scripts/oberdiek/pdfatfi.pl|\\
%   |cp scripts/oberdiek/pdfatfi.pl /usr/local/bin/|
% \end{quote}
%
% \subsection{Package installation}
%
% \paragraph{Unpacking.} The \xfile{.dtx} file is a self-extracting
% \docstrip\ archive. The files are extracted by running the
% \xfile{.dtx} through \plainTeX:
% \begin{quote}
%   \verb|tex iflang.dtx|
% \end{quote}
%
% \paragraph{TDS.} Now the different files must be moved into
% the different directories in your installation TDS tree
% (also known as \xfile{texmf} tree):
% \begin{quote}
% \def\t{^^A
% \begin{tabular}{@{}>{\ttfamily}l@{ $\rightarrow$ }>{\ttfamily}l@{}}
%   iflang.sty & tex/generic/oberdiek/iflang.sty\\
%   iflang.pdf & doc/latex/oberdiek/iflang.pdf\\
%   test/iflang-test1.tex & doc/latex/oberdiek/test/iflang-test1.tex\\
%   test/iflang-test2.tex & doc/latex/oberdiek/test/iflang-test2.tex\\
%   test/iflang-test3.tex & doc/latex/oberdiek/test/iflang-test3.tex\\
%   test/iflang-test4.tex & doc/latex/oberdiek/test/iflang-test4.tex\\
%   test/iflang-test5.tex & doc/latex/oberdiek/test/iflang-test5.tex\\
%   iflang.dtx & source/latex/oberdiek/iflang.dtx\\
% \end{tabular}^^A
% }^^A
% \sbox0{\t}^^A
% \ifdim\wd0>\linewidth
%   \begingroup
%     \advance\linewidth by\leftmargin
%     \advance\linewidth by\rightmargin
%   \edef\x{\endgroup
%     \def\noexpand\lw{\the\linewidth}^^A
%   }\x
%   \def\lwbox{^^A
%     \leavevmode
%     \hbox to \linewidth{^^A
%       \kern-\leftmargin\relax
%       \hss
%       \usebox0
%       \hss
%       \kern-\rightmargin\relax
%     }^^A
%   }^^A
%   \ifdim\wd0>\lw
%     \sbox0{\small\t}^^A
%     \ifdim\wd0>\linewidth
%       \ifdim\wd0>\lw
%         \sbox0{\footnotesize\t}^^A
%         \ifdim\wd0>\linewidth
%           \ifdim\wd0>\lw
%             \sbox0{\scriptsize\t}^^A
%             \ifdim\wd0>\linewidth
%               \ifdim\wd0>\lw
%                 \sbox0{\tiny\t}^^A
%                 \ifdim\wd0>\linewidth
%                   \lwbox
%                 \else
%                   \usebox0
%                 \fi
%               \else
%                 \lwbox
%               \fi
%             \else
%               \usebox0
%             \fi
%           \else
%             \lwbox
%           \fi
%         \else
%           \usebox0
%         \fi
%       \else
%         \lwbox
%       \fi
%     \else
%       \usebox0
%     \fi
%   \else
%     \lwbox
%   \fi
% \else
%   \usebox0
% \fi
% \end{quote}
% If you have a \xfile{docstrip.cfg} that configures and enables \docstrip's
% TDS installing feature, then some files can already be in the right
% place, see the documentation of \docstrip.
%
% \subsection{Refresh file name databases}
%
% If your \TeX~distribution
% (\teTeX, \mikTeX, \dots) relies on file name databases, you must refresh
% these. For example, \teTeX\ users run \verb|texhash| or
% \verb|mktexlsr|.
%
% \subsection{Some details for the interested}
%
% \paragraph{Attached source.}
%
% The PDF documentation on CTAN also includes the
% \xfile{.dtx} source file. It can be extracted by
% AcrobatReader 6 or higher. Another option is \textsf{pdftk},
% e.g. unpack the file into the current directory:
% \begin{quote}
%   \verb|pdftk iflang.pdf unpack_files output .|
% \end{quote}
%
% \paragraph{Unpacking with \LaTeX.}
% The \xfile{.dtx} chooses its action depending on the format:
% \begin{description}
% \item[\plainTeX:] Run \docstrip\ and extract the files.
% \item[\LaTeX:] Generate the documentation.
% \end{description}
% If you insist on using \LaTeX\ for \docstrip\ (really,
% \docstrip\ does not need \LaTeX), then inform the autodetect routine
% about your intention:
% \begin{quote}
%   \verb|latex \let\install=y\input{iflang.dtx}|
% \end{quote}
% Do not forget to quote the argument according to the demands
% of your shell.
%
% \paragraph{Generating the documentation.}
% You can use both the \xfile{.dtx} or the \xfile{.drv} to generate
% the documentation. The process can be configured by the
% configuration file \xfile{ltxdoc.cfg}. For instance, put this
% line into this file, if you want to have A4 as paper format:
% \begin{quote}
%   \verb|\PassOptionsToClass{a4paper}{article}|
% \end{quote}
% An example follows how to generate the
% documentation with pdf\LaTeX:
% \begin{quote}
%\begin{verbatim}
%pdflatex iflang.dtx
%makeindex -s gind.ist iflang.idx
%pdflatex iflang.dtx
%makeindex -s gind.ist iflang.idx
%pdflatex iflang.dtx
%\end{verbatim}
% \end{quote}
%
% \section{Catalogue}
%
% The following XML file can be used as source for the
% \href{http://mirror.ctan.org/help/Catalogue/catalogue.html}{\TeX\ Catalogue}.
% The elements \texttt{caption} and \texttt{description} are imported
% from the original XML file from the Catalogue.
% The name of the XML file in the Catalogue is \xfile{iflang.xml}.
%    \begin{macrocode}
%<*catalogue>
<?xml version='1.0' encoding='us-ascii'?>
<!DOCTYPE entry SYSTEM 'catalogue.dtd'>
<entry datestamp='$Date$' modifier='$Author$' id='iflang'>
  <name>iflang</name>
  <caption>Expandable checks for the current language.</caption>
  <authorref id='auth:oberdiek'/>
  <copyright owner='Heiko Oberdiek' year='2007'/>
  <license type='lppl1.3'/>
  <version number='1.6'/>
  <description>
    This package provides expandable checks for the current language
    based on macro <tt>\languagename</tt> or hyphenation patterns.
    <p/>
    The package is part of the <xref refid='oberdiek'>oberdiek</xref> bundle.
  </description>
  <documentation details='Package documentation'
      href='ctan:/macros/latex/contrib/oberdiek/iflang.pdf'/>
  <ctan file='true' path='/macros/latex/contrib/oberdiek/iflang.dtx'/>
  <miktex location='oberdiek'/>
  <texlive location='oberdiek'/>
  <install path='/macros/latex/contrib/oberdiek/oberdiek.tds.zip'/>
</entry>
%</catalogue>
%    \end{macrocode}
%
% \section{Acknowledgement}
% I wish to thank:
% \begin{description}
% \item[Markus Kohm] Useful hints for version 1.2.
% \end{description}
%
% \begin{History}
%   \begin{Version}{2007/04/10 v1.0}
%   \item
%     First public version.
%   \end{Version}
%   \begin{Version}{2007/04/11 v1.1}
%   \item
%     Line ends sanitized.
%   \end{Version}
%   \begin{Version}{2007/04/12 v1.2}
%   \item
%     Initialization of \cs{languagename} in case of \xfile{etex.src}.
%   \item
%     Some sanity tests added.
%   \item
%     Documentation improved.
%   \end{Version}
%   \begin{Version}{2007/04/26 v1.3}
%   \item
%     Use of package \xpackage{infwarerr}.
%   \end{Version}
%   \begin{Version}{2007/09/09 v1.4}
%   \item
%     Bug fix: \cs{IfLang@StrEqual} $\rightarrow$ \cs{IfLangStrEqual}
%     (Gabriele Balducci).
%   \item
%     Catcode section rewritten.
%   \end{Version}
%   \begin{Version}{2007/11/11 v1.5}
%   \item
%     Use of package \xpackage{pdftexcmds} for \LuaTeX\ support.
%   \end{Version}
%   \begin{Version}{2016/05/16 v1.6}
%   \item
%     Documentation updates.
%   \end{Version}
% \end{History}
%
% \PrintIndex
%
% \Finale
\endinput
|
% \end{quote}
% Do not forget to quote the argument according to the demands
% of your shell.
%
% \paragraph{Generating the documentation.}
% You can use both the \xfile{.dtx} or the \xfile{.drv} to generate
% the documentation. The process can be configured by the
% configuration file \xfile{ltxdoc.cfg}. For instance, put this
% line into this file, if you want to have A4 as paper format:
% \begin{quote}
%   \verb|\PassOptionsToClass{a4paper}{article}|
% \end{quote}
% An example follows how to generate the
% documentation with pdf\LaTeX:
% \begin{quote}
%\begin{verbatim}
%pdflatex iflang.dtx
%makeindex -s gind.ist iflang.idx
%pdflatex iflang.dtx
%makeindex -s gind.ist iflang.idx
%pdflatex iflang.dtx
%\end{verbatim}
% \end{quote}
%
% \section{Catalogue}
%
% The following XML file can be used as source for the
% \href{http://mirror.ctan.org/help/Catalogue/catalogue.html}{\TeX\ Catalogue}.
% The elements \texttt{caption} and \texttt{description} are imported
% from the original XML file from the Catalogue.
% The name of the XML file in the Catalogue is \xfile{iflang.xml}.
%    \begin{macrocode}
%<*catalogue>
<?xml version='1.0' encoding='us-ascii'?>
<!DOCTYPE entry SYSTEM 'catalogue.dtd'>
<entry datestamp='$Date$' modifier='$Author$' id='iflang'>
  <name>iflang</name>
  <caption>Expandable checks for the current language.</caption>
  <authorref id='auth:oberdiek'/>
  <copyright owner='Heiko Oberdiek' year='2007'/>
  <license type='lppl1.3'/>
  <version number='1.6'/>
  <description>
    This package provides expandable checks for the current language
    based on macro <tt>\languagename</tt> or hyphenation patterns.
    <p/>
    The package is part of the <xref refid='oberdiek'>oberdiek</xref> bundle.
  </description>
  <documentation details='Package documentation'
      href='ctan:/macros/latex/contrib/oberdiek/iflang.pdf'/>
  <ctan file='true' path='/macros/latex/contrib/oberdiek/iflang.dtx'/>
  <miktex location='oberdiek'/>
  <texlive location='oberdiek'/>
  <install path='/macros/latex/contrib/oberdiek/oberdiek.tds.zip'/>
</entry>
%</catalogue>
%    \end{macrocode}
%
% \section{Acknowledgement}
% I wish to thank:
% \begin{description}
% \item[Markus Kohm] Useful hints for version 1.2.
% \end{description}
%
% \begin{History}
%   \begin{Version}{2007/04/10 v1.0}
%   \item
%     First public version.
%   \end{Version}
%   \begin{Version}{2007/04/11 v1.1}
%   \item
%     Line ends sanitized.
%   \end{Version}
%   \begin{Version}{2007/04/12 v1.2}
%   \item
%     Initialization of \cs{languagename} in case of \xfile{etex.src}.
%   \item
%     Some sanity tests added.
%   \item
%     Documentation improved.
%   \end{Version}
%   \begin{Version}{2007/04/26 v1.3}
%   \item
%     Use of package \xpackage{infwarerr}.
%   \end{Version}
%   \begin{Version}{2007/09/09 v1.4}
%   \item
%     Bug fix: \cs{IfLang@StrEqual} $\rightarrow$ \cs{IfLangStrEqual}
%     (Gabriele Balducci).
%   \item
%     Catcode section rewritten.
%   \end{Version}
%   \begin{Version}{2007/11/11 v1.5}
%   \item
%     Use of package \xpackage{pdftexcmds} for \LuaTeX\ support.
%   \end{Version}
%   \begin{Version}{2016/05/16 v1.6}
%   \item
%     Documentation updates.
%   \end{Version}
% \end{History}
%
% \PrintIndex
%
% \Finale
\endinput
|
% \end{quote}
% Do not forget to quote the argument according to the demands
% of your shell.
%
% \paragraph{Generating the documentation.}
% You can use both the \xfile{.dtx} or the \xfile{.drv} to generate
% the documentation. The process can be configured by the
% configuration file \xfile{ltxdoc.cfg}. For instance, put this
% line into this file, if you want to have A4 as paper format:
% \begin{quote}
%   \verb|\PassOptionsToClass{a4paper}{article}|
% \end{quote}
% An example follows how to generate the
% documentation with pdf\LaTeX:
% \begin{quote}
%\begin{verbatim}
%pdflatex iflang.dtx
%makeindex -s gind.ist iflang.idx
%pdflatex iflang.dtx
%makeindex -s gind.ist iflang.idx
%pdflatex iflang.dtx
%\end{verbatim}
% \end{quote}
%
% \section{Catalogue}
%
% The following XML file can be used as source for the
% \href{http://mirror.ctan.org/help/Catalogue/catalogue.html}{\TeX\ Catalogue}.
% The elements \texttt{caption} and \texttt{description} are imported
% from the original XML file from the Catalogue.
% The name of the XML file in the Catalogue is \xfile{iflang.xml}.
%    \begin{macrocode}
%<*catalogue>
<?xml version='1.0' encoding='us-ascii'?>
<!DOCTYPE entry SYSTEM 'catalogue.dtd'>
<entry datestamp='$Date$' modifier='$Author$' id='iflang'>
  <name>iflang</name>
  <caption>Expandable checks for the current language.</caption>
  <authorref id='auth:oberdiek'/>
  <copyright owner='Heiko Oberdiek' year='2007'/>
  <license type='lppl1.3'/>
  <version number='1.6'/>
  <description>
    This package provides expandable checks for the current language
    based on macro <tt>\languagename</tt> or hyphenation patterns.
    <p/>
    The package is part of the <xref refid='oberdiek'>oberdiek</xref> bundle.
  </description>
  <documentation details='Package documentation'
      href='ctan:/macros/latex/contrib/oberdiek/iflang.pdf'/>
  <ctan file='true' path='/macros/latex/contrib/oberdiek/iflang.dtx'/>
  <miktex location='oberdiek'/>
  <texlive location='oberdiek'/>
  <install path='/macros/latex/contrib/oberdiek/oberdiek.tds.zip'/>
</entry>
%</catalogue>
%    \end{macrocode}
%
% \section{Acknowledgement}
% I wish to thank:
% \begin{description}
% \item[Markus Kohm] Useful hints for version 1.2.
% \end{description}
%
% \begin{History}
%   \begin{Version}{2007/04/10 v1.0}
%   \item
%     First public version.
%   \end{Version}
%   \begin{Version}{2007/04/11 v1.1}
%   \item
%     Line ends sanitized.
%   \end{Version}
%   \begin{Version}{2007/04/12 v1.2}
%   \item
%     Initialization of \cs{languagename} in case of \xfile{etex.src}.
%   \item
%     Some sanity tests added.
%   \item
%     Documentation improved.
%   \end{Version}
%   \begin{Version}{2007/04/26 v1.3}
%   \item
%     Use of package \xpackage{infwarerr}.
%   \end{Version}
%   \begin{Version}{2007/09/09 v1.4}
%   \item
%     Bug fix: \cs{IfLang@StrEqual} $\rightarrow$ \cs{IfLangStrEqual}
%     (Gabriele Balducci).
%   \item
%     Catcode section rewritten.
%   \end{Version}
%   \begin{Version}{2007/11/11 v1.5}
%   \item
%     Use of package \xpackage{pdftexcmds} for \LuaTeX\ support.
%   \end{Version}
%   \begin{Version}{2016/05/16 v1.6}
%   \item
%     Documentation updates.
%   \end{Version}
% \end{History}
%
% \PrintIndex
%
% \Finale
\endinput
|
% \end{quote}
% Do not forget to quote the argument according to the demands
% of your shell.
%
% \paragraph{Generating the documentation.}
% You can use both the \xfile{.dtx} or the \xfile{.drv} to generate
% the documentation. The process can be configured by the
% configuration file \xfile{ltxdoc.cfg}. For instance, put this
% line into this file, if you want to have A4 as paper format:
% \begin{quote}
%   \verb|\PassOptionsToClass{a4paper}{article}|
% \end{quote}
% An example follows how to generate the
% documentation with pdf\LaTeX:
% \begin{quote}
%\begin{verbatim}
%pdflatex iflang.dtx
%makeindex -s gind.ist iflang.idx
%pdflatex iflang.dtx
%makeindex -s gind.ist iflang.idx
%pdflatex iflang.dtx
%\end{verbatim}
% \end{quote}
%
% \section{Catalogue}
%
% The following XML file can be used as source for the
% \href{http://mirror.ctan.org/help/Catalogue/catalogue.html}{\TeX\ Catalogue}.
% The elements \texttt{caption} and \texttt{description} are imported
% from the original XML file from the Catalogue.
% The name of the XML file in the Catalogue is \xfile{iflang.xml}.
%    \begin{macrocode}
%<*catalogue>
<?xml version='1.0' encoding='us-ascii'?>
<!DOCTYPE entry SYSTEM 'catalogue.dtd'>
<entry datestamp='$Date$' modifier='$Author$' id='iflang'>
  <name>iflang</name>
  <caption>Expandable checks for the current language.</caption>
  <authorref id='auth:oberdiek'/>
  <copyright owner='Heiko Oberdiek' year='2007'/>
  <license type='lppl1.3'/>
  <version number='1.6'/>
  <description>
    This package provides expandable checks for the current language
    based on macro <tt>\languagename</tt> or hyphenation patterns.
    <p/>
    The package is part of the <xref refid='oberdiek'>oberdiek</xref> bundle.
  </description>
  <documentation details='Package documentation'
      href='ctan:/macros/latex/contrib/oberdiek/iflang.pdf'/>
  <ctan file='true' path='/macros/latex/contrib/oberdiek/iflang.dtx'/>
  <miktex location='oberdiek'/>
  <texlive location='oberdiek'/>
  <install path='/macros/latex/contrib/oberdiek/oberdiek.tds.zip'/>
</entry>
%</catalogue>
%    \end{macrocode}
%
% \section{Acknowledgement}
% I wish to thank:
% \begin{description}
% \item[Markus Kohm] Useful hints for version 1.2.
% \end{description}
%
% \begin{History}
%   \begin{Version}{2007/04/10 v1.0}
%   \item
%     First public version.
%   \end{Version}
%   \begin{Version}{2007/04/11 v1.1}
%   \item
%     Line ends sanitized.
%   \end{Version}
%   \begin{Version}{2007/04/12 v1.2}
%   \item
%     Initialization of \cs{languagename} in case of \xfile{etex.src}.
%   \item
%     Some sanity tests added.
%   \item
%     Documentation improved.
%   \end{Version}
%   \begin{Version}{2007/04/26 v1.3}
%   \item
%     Use of package \xpackage{infwarerr}.
%   \end{Version}
%   \begin{Version}{2007/09/09 v1.4}
%   \item
%     Bug fix: \cs{IfLang@StrEqual} $\rightarrow$ \cs{IfLangStrEqual}
%     (Gabriele Balducci).
%   \item
%     Catcode section rewritten.
%   \end{Version}
%   \begin{Version}{2007/11/11 v1.5}
%   \item
%     Use of package \xpackage{pdftexcmds} for \LuaTeX\ support.
%   \end{Version}
%   \begin{Version}{2016/05/16 v1.6}
%   \item
%     Documentation updates.
%   \end{Version}
% \end{History}
%
% \PrintIndex
%
% \Finale
\endinput
