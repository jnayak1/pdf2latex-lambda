% utf8-t1.tex
%%%%%%%%%%%%%%%%%%%%%%%%%%%%%%%%%%%%%%%%%%%%%%%%%%%%%%%%%%%%%%%%%%%
% This file implements the conversion from UTF8 to Cork
% encoding (used by DC (EC) fonts).
% The conversion is done by encTeX v. Dec 2002 or higher.
%
% Copyright (C) 2002-2003 Petr Olsak
% Copyright (C) 2003 David Necas (Yeti)
%
% This program is free software; you can redistribute it and/or modify
% it under the terms of the GNU General Public License as published by
% the Free Software Foundation; either version 2 of the License, or
% (at your option) any later version.
%
% This program is distributed in the hope that it will be useful,
% but WITHOUT ANY WARRANTY; without even the implied warranty of
% MERCHANTABILITY or FITNESS FOR A PARTICULAR PURPOSE.  See the
% GNU General Public License for more details.
%
% You should have received a copy of the GNU General Public License
% along with this program; if not, write to the Free Software
% Foundation, Inc., 59 Temple Place, Suite 330, Boston, MA  02111-1307  USA

\ifx\mubyte\undefined
   \errhelp{Sorry, you can't use this file without encTeX ver. Jan. 2002.}
   \errmessage{The encTeX extension of TeX is not found}
   \endinput \fi

% first, we set the identity mapping in xord/xchr:

\bgroup
\ifx\xordcode\undefined
   \errhelp{May be, you are using this file from csplain which disables
            the \xordcode primitive. Use
            initex \let\enc=u \input csplain.ini
            for csplain generation instead this file.
            If you are using ISO8859-2 input encoding in csplain,
            you can skip this error message.}
   \errmessage{I can't set the xord/xchr to identity mapping}
   \def\xchrcode\count255=\count255{} \def\xordcode\count255=\count255{}
\fi
\count255=128
\loop \xordcode\count255=\count255
      \xchrcode\count255=\count255
      \advance\count255 by 1
\ifnum \count255<256 \repeat
\egroup


% we remove the current conversion table, if exists:

{\catcode`\^^@=12
\gdef\clearmubytes{\bgroup \count255=1
   \loop \uccode`X=\count255
       \uppercase{\mubyte XX\endmubyte}%
       \advance\count255 by1
       \ifnum\count255<256 \repeat
   \mubyte ^^@^^@\endmubyte
   \egroup}
}
\clearmubytes

% include these first, so we can redefine some characters better later here

\input utf8cseq
\input utf8math
\input utf8unkn

% now, the conversion table is created:
\mubyte ^^80 ^^c4^^82\endmubyte % latin capital letter a with breve
\mubyte ^^81 ^^c4^^84\endmubyte % latin capital letter a with ogonek
\mubyte ^^82 ^^c4^^86\endmubyte % latin capital letter c with acute
\mubyte ^^83 ^^c4^^8c\endmubyte % latin capital letter c with caron
\mubyte ^^84 ^^c4^^8e\endmubyte % latin capital letter d with caron
\mubyte ^^85 ^^c4^^9a\endmubyte % latin capital letter e with caron
\mubyte ^^86 ^^c4^^98\endmubyte % latin capital letter e with ogonek
\mubyte ^^87 ^^c4^^9e\endmubyte % latin capital letter g with breve
\mubyte ^^88 ^^c4^^b9\endmubyte % latin capital letter l with acute
\mubyte ^^89 ^^c4^^bd\endmubyte % latin capital letter l with caron
\mubyte ^^8a ^^c5^^81\endmubyte % latin capital letter l with stroke
\mubyte ^^8b ^^c5^^83\endmubyte % latin capital letter n with acute
\mubyte ^^8c ^^c5^^87\endmubyte % latin capital letter n with caron
\mubyte ^^8d ^^c5^^8a\endmubyte % latin capital letter eng (sami)
\mubyte ^^8e ^^c5^^90\endmubyte % latin capital letter o with double acute
\mubyte ^^8f ^^c5^^94\endmubyte % latin capital letter r with acute
\mubyte ^^90 ^^c5^^98\endmubyte % latin capital letter r with caron
\mubyte ^^91 ^^c5^^9a\endmubyte % latin capital letter s with acute
\mubyte ^^92 ^^c5^^a0\endmubyte % latin capital letter s with caron
\mubyte ^^93 ^^c5^^9e\endmubyte % latin capital letter s with cedilla
\mubyte ^^94 ^^c5^^a4\endmubyte % latin capital letter t with caron
\mubyte ^^95 ^^c5^^a2\endmubyte % latin capital letter t with cedilla
\mubyte ^^96 ^^c5^^b0\endmubyte % latin capital letter u with double acute
\mubyte ^^97 ^^c5^^ae\endmubyte % latin capital letter u with ring above
\mubyte ^^98 ^^c5^^b8\endmubyte % latin capital letter y with diaeresis
\mubyte ^^99 ^^c5^^b9\endmubyte % latin capital letter z with acute
\mubyte ^^9a ^^c5^^bd\endmubyte % latin capital letter z with caron
\mubyte ^^9b ^^c5^^bb\endmubyte % latin capital letter z with dot above
\mubyte ^^9c ^^c4^^b2\endmubyte % latin capital ligature ij
\mubyte ^^9d ^^c4^^b0\endmubyte % latin capital letter i with dot above
\mubyte ^^9e ^^c4^^91\endmubyte % latin small letter d with stroke
\mubyte ^^9f ^^c2^^a7\endmubyte % section sign
\mubyte ^^a0 ^^c4^^83\endmubyte % latin small letter a with breve
\mubyte ^^a1 ^^c4^^85\endmubyte % latin small letter a with ogonek
\mubyte ^^a2 ^^c4^^87\endmubyte % latin small letter c with acute
\mubyte ^^a3 ^^c4^^8d\endmubyte % latin small letter c with caron
\mubyte ^^a4 ^^c4^^8f\endmubyte % latin small letter d with caron
\mubyte ^^a5 ^^c4^^9b\endmubyte % latin small letter e with caron
\mubyte ^^a6 ^^c4^^99\endmubyte % latin small letter e with ogonek
\mubyte ^^a7 ^^c4^^9f\endmubyte % latin small letter g with breve
\mubyte ^^a8 ^^c4^^ba\endmubyte % latin small letter l with acute
\mubyte ^^a9 ^^c4^^be\endmubyte % latin small letter l with caron
\mubyte ^^aa ^^c5^^82\endmubyte % latin small letter l with stroke
\mubyte ^^ab ^^c5^^84\endmubyte % latin small letter n with acute
\mubyte ^^ac ^^c5^^88\endmubyte % latin small letter n with caron
\mubyte ^^ad ^^c5^^8b\endmubyte % latin small letter eng (sami)
\mubyte ^^ae ^^c5^^91\endmubyte % latin small letter o with double acute
\mubyte ^^af ^^c5^^95\endmubyte % latin small letter r with acute
\mubyte ^^b0 ^^c5^^99\endmubyte % latin small letter r with caron
\mubyte ^^b1 ^^c5^^9b\endmubyte % latin small letter s with acute
\mubyte ^^b2 ^^c5^^a1\endmubyte % latin small letter s with caron
\mubyte ^^b3 ^^c5^^9f\endmubyte % latin small letter s with cedilla
\mubyte ^^b4 ^^c5^^a5\endmubyte % latin small letter t with caron
\mubyte ^^b5 ^^c5^^a3\endmubyte % latin small letter t with cedilla
\mubyte ^^b6 ^^c5^^b1\endmubyte % latin small letter u with double acute
\mubyte ^^b7 ^^c5^^af\endmubyte % latin small letter u with ring above
\mubyte ^^b8 ^^c3^^bf\endmubyte % latin small letter y with diaeresis
\mubyte ^^b9 ^^c5^^ba\endmubyte % latin small letter z with acute
\mubyte ^^ba ^^c5^^be\endmubyte % latin small letter z with caron
\mubyte ^^bb ^^c5^^bc\endmubyte % latin small letter z with dot above
\mubyte ^^bc ^^c4^^b3\endmubyte % latin small ligature ij
\mubyte ^^bd ^^c2^^a1\endmubyte % inverted exclamation mark
\mubyte ^^be ^^c2^^bf\endmubyte % inverted question mark
\mubyte ^^bf ^^c2^^a3\endmubyte % pound sign
\mubyte ^^c0 ^^c3^^80\endmubyte % latin capital letter a with grave
\mubyte ^^c1 ^^c3^^81\endmubyte % latin capital letter a with acute
\mubyte ^^c2 ^^c3^^82\endmubyte % latin capital letter a with circumflex
\mubyte ^^c3 ^^c3^^83\endmubyte % latin capital letter a with tilde
\mubyte ^^c4 ^^c3^^84\endmubyte % latin capital letter a with diaeresis
\mubyte ^^c5 ^^c3^^85\endmubyte % latin capital letter a with ring above
\mubyte ^^c6 ^^c3^^86\endmubyte % latin capital letter ae (ash)
\mubyte ^^c7 ^^c3^^87\endmubyte % latin capital letter c with cedilla
\mubyte ^^c8 ^^c3^^88\endmubyte % latin capital letter e with grave
\mubyte ^^c9 ^^c3^^89\endmubyte % latin capital letter e with acute
\mubyte ^^ca ^^c3^^8a\endmubyte % latin capital letter e with circumflex
\mubyte ^^cb ^^c3^^8b\endmubyte % latin capital letter e with diaeresis
\mubyte ^^cc ^^c3^^8c\endmubyte % latin capital letter i with grave
\mubyte ^^cd ^^c3^^8d\endmubyte % latin capital letter i with acute
\mubyte ^^ce ^^c3^^8e\endmubyte % latin capital letter i with circumflex
\mubyte ^^cf ^^c3^^8f\endmubyte % latin capital letter i with diaeresis
\mubyte ^^d0 ^^c3^^90\endmubyte % latin capital letter eth (icelandic)
\mubyte ^^d1 ^^c3^^91\endmubyte % latin capital letter n with tilde
\mubyte ^^d2 ^^c3^^92\endmubyte % latin capital letter o with grave
\mubyte ^^d3 ^^c3^^93\endmubyte % latin capital letter o with acute
\mubyte ^^d4 ^^c3^^94\endmubyte % latin capital letter o with circumflex
\mubyte ^^d5 ^^c3^^95\endmubyte % latin capital letter o with tilde
\mubyte ^^d6 ^^c3^^96\endmubyte % latin capital letter o with diaeresis
\mubyte ^^d7 ^^c5^^92\endmubyte % latin capital ligature oe
\mubyte ^^d8 ^^c3^^98\endmubyte % latin capital letter o with stroke
\mubyte ^^d9 ^^c3^^99\endmubyte % latin capital letter u with grave
\mubyte ^^da ^^c3^^9a\endmubyte % latin capital letter u with acute
\mubyte ^^db ^^c3^^9b\endmubyte % latin capital letter u with circumflex
\mubyte ^^dc ^^c3^^9c\endmubyte % latin capital letter u with diaeresis
\mubyte ^^dd ^^c3^^9d\endmubyte % latin capital letter y with acute
\mubyte ^^de ^^c3^^9e\endmubyte % latin capital letter thorn (icelandic)
\mubyte ^^e0 ^^c3^^a0\endmubyte % latin small letter a with grave
\mubyte ^^e1 ^^c3^^a1\endmubyte % latin small letter a with acute
\mubyte ^^e2 ^^c3^^a2\endmubyte % latin small letter a with circumflex
\mubyte ^^e3 ^^c3^^a3\endmubyte % latin small letter a with tilde
\mubyte ^^e4 ^^c3^^a4\endmubyte % latin small letter a with diaeresis
\mubyte ^^e5 ^^c3^^a5\endmubyte % latin small letter a with ring above
\mubyte ^^e6 ^^c3^^a6\endmubyte % latin small letter ae (ash)
\mubyte ^^e7 ^^c3^^a7\endmubyte % latin small letter c with cedilla
\mubyte ^^e8 ^^c3^^a8\endmubyte % latin small letter e with grave
\mubyte ^^e9 ^^c3^^a9\endmubyte % latin small letter e with acute
\mubyte ^^ea ^^c3^^aa\endmubyte % latin small letter e with circumflex
\mubyte ^^eb ^^c3^^ab\endmubyte % latin small letter e with diaeresis
\mubyte ^^ec ^^c3^^ac\endmubyte % latin small letter i with grave
\mubyte ^^ed ^^c3^^ad\endmubyte % latin small letter i with acute
\mubyte ^^ee ^^c3^^ae\endmubyte % latin small letter i with circumflex
\mubyte ^^ef ^^c3^^af\endmubyte % latin small letter i with diaeresis
\mubyte ^^f0 ^^c3^^b0\endmubyte % latin small letter eth (icelandic)
\mubyte ^^f1 ^^c3^^b1\endmubyte % latin small letter n with tilde
\mubyte ^^f2 ^^c3^^b2\endmubyte % latin small letter o with grave
\mubyte ^^f3 ^^c3^^b3\endmubyte % latin small letter o with acute
\mubyte ^^f4 ^^c3^^b4\endmubyte % latin small letter o with circumflex
\mubyte ^^f5 ^^c3^^b5\endmubyte % latin small letter o with tilde
\mubyte ^^f6 ^^c3^^b6\endmubyte % latin small letter o with diaeresis
\mubyte ^^f7 ^^c5^^93\endmubyte % latin small ligature oe
\mubyte ^^f8 ^^c3^^b8\endmubyte % latin small letter o with stroke
\mubyte ^^f9 ^^c3^^b9\endmubyte % latin small letter u with grave
\mubyte ^^fa ^^c3^^ba\endmubyte % latin small letter u with acute
\mubyte ^^fb ^^c3^^bb\endmubyte % latin small letter u with circumflex
\mubyte ^^fc ^^c3^^bc\endmubyte % latin small letter u with diaeresis
\mubyte ^^fd ^^c3^^bd\endmubyte % latin small letter y with acute
\mubyte ^^fe ^^c3^^be\endmubyte % latin small letter thorn (icelandic)
\mubyte ^^ff ^^c3^^9f\endmubyte % latin small letter sharp s (german)

% Non-characters
\chardef\erqq="11
\mubyte \erqq ^^e2^^80^^9d\endmubyte % right double quotation mark
\chardef\erq="27
\mubyte \erq ^^e2^^80^^99\endmubyte % right single quotation mark
% there's no \elqq, it's the same Unicode character as \crqq
\chardef\flq="0E
\chardef\frq="0F
\mubyte \flq ^^e2^^80^^b9\endmubyte % single left-pointing angle quotation mark
\mubyte \frq ^^e2^^80^^ba\endmubyte % single right-pointing angle quotation mark
\chardef\endash="15
\chardef\emdash="16
\mubyte \endash ^^e2^^80^^93\endmubyte % en dash
\mubyte \emdash ^^e2^^80^^94\endmubyte % em dash
\chardef\utfligatureff="1B
\chardef\utfligaturefi="1C
\chardef\utfligaturefl="1D
\chardef\utfligatureffi="1E
\chardef\utfligatureffl="1F
\mubyte \utfligatureff ^^ef^^ac^^80\endmubyte % latin small ligature ff
\mubyte \utfligaturefi ^^ef^^ac^^81\endmubyte % latin small ligature fi
\mubyte \utfligaturefl ^^ef^^ac^^82\endmubyte % latin small ligature fl
\mubyte \utfligatureffi ^^ef^^ac^^83\endmubyte % latin small ligature ffi
\mubyte \utfligatureffl ^^ef^^ac^^84\endmubyte % latin small ligature ffl

% You can add more UTF-8 codes here. You can map these codes to
% control sequences (see encdoc.tex for more datails) so,
% the number of UTF-8 codes examined by TeX is unlimited.

% ...

\mubytein=1 \mubyteout=3

%% for compatibility with hyphen.lan file:
\let\csaccents=\relax \let\cmaccents=\relax

% now we still have to deal with accents
\input t1macro \input encmacro

% This needed for hyphenation patterns.

% (1) Czech/Slovak alphabet
%            input TeX   lc   uc    sf cat prn        sequence
\setcharcode  ?  "C1  "E1  "C1   999  11  1  \texaccent \'A
\setcharcode  ?  "E1  "E1  "C1  1000  11  1  \texaccent \'a
\setcharcode  ?  "C4  "E4  "C4   999  11  1  \texaccent \"A
\setcharcode  ?  "E4  "E4  "C4  1000  11  1  \texaccent \"a
\setcharcode  ?  "83  "A3  "83   999  11  1  \texaccent \v C
\setcharcode  ?  "A3  "A3  "83  1000  11  1  \texaccent \v c
\setcharcode  ?  "84  "A4  "84   999  11  1  \texaccent \v D
\setcharcode  ?  "A4  "A4  "84  1000  11  1  \texaccent \v d
\setcharcode  ?  "C9  "E9  "C9   999  11  1  \texaccent \'E
\setcharcode  ?  "E9  "E9  "C9  1000  11  1  \texaccent \'e
\setcharcode  ?  "85  "A5  "85   999  11  1  \texaccent \v E
\setcharcode  ?  "A5  "A5  "85  1000  11  1  \texaccent \v e
\setcharcode  ?  "CD  "ED  "CD   999  11  1  \texaccent \'I
\setcharcode  ?  "ED  "ED  "CD  1000  11  1  \texaccent \'i  \texaccent \'\i
\setcharcode  ?  "88  "A8  "88   999  11  1  \texaccent \'L
\setcharcode  ?  "A8  "A8  "88  1000  11  1  \texaccent \'l
\setcharcode  ?  "89  "A9  "89   999  11  1  \texaccent \v L
\setcharcode  ?  "A9  "A9  "89  1000  11  1  \texaccent \v l
\setcharcode  ?  "8C  "AC  "8C   999  11  1  \texaccent \v N
\setcharcode  ?  "AC  "AC  "8C  1000  11  1  \texaccent \v n
\setcharcode  ?  "D3  "F3  "D3   999  11  1  \texaccent \'O
\setcharcode  ?  "F3  "F3  "D3  1000  11  1  \texaccent \'o
\setcharcode  ?  "D4  "F4  "D4   999  11  1  \texaccent \^O
\setcharcode  ?  "F4  "F4  "D4  1000  11  1  \texaccent \^o
\setcharcode  ?  "D6  "F6  "D6   999  11  1  \texaccent \"O
\setcharcode  ?  "F6  "F6  "D6  1000  11  1  \texaccent \"o
\setcharcode  ?  "8F  "AF  "8F   999  11  1  \texaccent \'R
\setcharcode  ?  "AF  "AF  "8F  1000  11  1  \texaccent \'r
\setcharcode  ?  "90  "B0  "90   999  11  1  \texaccent \v R
\setcharcode  ?  "B0  "B0  "90  1000  11  1  \texaccent \v r
\setcharcode  ?  "92  "B2  "92   999  11  1  \texaccent \v S
\setcharcode  ?  "B2  "B2  "92  1000  11  1  \texaccent \v s
\setcharcode  ?  "94  "B4  "94   999  11  1  \texaccent \v T
\setcharcode  ?  "B4  "B4  "94  1000  11  1  \texaccent \v t
\setcharcode  ?  "DA  "FA  "DA   999  11  1  \texaccent \'U
\setcharcode  ?  "FA  "FA  "DA  1000  11  1  \texaccent \'u
\setcharcode  ?  "97  "B7  "97   999  11  1  \texaccent \r U
\setcharcode  ?  "B7  "B7  "97  1000  11  1  \texaccent \r u
\setcharcode  ?  "DC  "FC  "DC   999  11  1  \texaccent \"U
\setcharcode  ?  "FC  "FC  "DC  1000  11  1  \texaccent \"u
\setcharcode  ?  "DD  "FD  "DD   999  11  1  \texaccent \'Y
\setcharcode  ?  "FD  "FD  "DD  1000  11  1  \texaccent \'y
\setcharcode  ?  "9A  "BA  "9A   999  11  1  \texaccent \v Z
\setcharcode  ?  "BA  "BA  "9A  1000  11  1  \texaccent \v z

% (2) Non Czech/Slovak alphabet
%            input TeX   lc   uc    sf cat prn        sequence
\setcharcode  ?  "81  "A1  "81   999  11  1  \texaccent \og A
\setcharcode  ?  "A1  "A1  "81  1000  11  1  \texaccent \og a
\setcharcode  ?  "8A  "AA  "8A   999  11  1  \texmacro  \L
\setcharcode  ?  "AA  "AA  "8A  1000  11  1  \texmacro  \l
\setcharcode  ?  "91  "B1  "91   999  11  1  \texaccent \'S
\setcharcode  ?  "B1  "B1  "91  1000  11  1  \texaccent \'s
\setcharcode  ?  "93  "B3  "93   999  11  1  \texaccent \c S
\setcharcode  ?  "B3  "B3  "93  1000  11  1  \texaccent \c s
\setcharcode  ?  "99  "B9  "99   999  11  1  \texaccent \'Z
\setcharcode  ?  "B9  "B9  "99  1000  11  1  \texaccent \'z
\setcharcode  ?  "9B  "BB  "9B   999  11  1  \texaccent \.Z
\setcharcode  ?  "BB  "BB  "9B  1000  11  1  \texaccent \.z
\setcharcode  ?  "C2  "E2  "C2   999  11  1  \texaccent \^A
\setcharcode  ?  "E2  "E2  "C2  1000  11  1  \texaccent \^a
\setcharcode  ?  "80  "A0  "80   999  11  1  \texaccent \u A
\setcharcode  ?  "A0  "A0  "80  1000  11  1  \texaccent \u a
\setcharcode  ?  "82  "A2  "82   999  11  1  \texaccent \'C
\setcharcode  ?  "A2  "A2  "82  1000  11  1  \texaccent \'c
\setcharcode  ?  "C7  "E7  "C7   999  11  1  \texaccent \c C
\setcharcode  ?  "E7  "E7  "C7  1000  11  1  \texaccent \c c
\setcharcode  ?  "86  "A6  "86   999  11  1  \texaccent \og E
\setcharcode  ?  "A6  "A6  "86  1000  11  1  \texaccent \og e
\setcharcode  ?  "CB  "EB  "CB   999  11  1  \texaccent \"E
\setcharcode  ?  "EB  "EB  "CB  1000  11  1  \texaccent \"e
\setcharcode  ?  "CE  "EE  "CE   999  11  1  \texaccent \^I
\setcharcode  ?  "EE  "EE  "CE  1000  11  1  \texaccent \^i  \texaccent \^\i
\setcharcode  ?  "D0  "F0  "D0   999  11  1  \texmacro \Dslash
\setcharcode  ?  "F0  "F0  "D0  1000  11  1  \texmacro \dslash
\setcharcode  ?  "8B  "AB  "8B   999  11  1  \texaccent \'N
\setcharcode  ?  "AB  "AB  "8B  1000  11  1  \texaccent \'n
\setcharcode  ?  "8E  "AE  "8E   999  11  1  \texaccent \H O
\setcharcode  ?  "AE  "AE  "8E  1000  11  1  \texaccent \H o
\setcharcode  ?  "96  "B6  "96   999  11  1  \texaccent \H U
\setcharcode  ?  "B6  "B6  "96  1000  11  1  \texaccent \H u
\setcharcode  ?  "95  "B5  "95   999  11  1  \texaccent \c T
\setcharcode  ?  "B5  "B5  "95  1000  11  1  \texaccent \c t
\setcharcode  ?  "FF  "FF  "DF  1000  11  1  \texmacro  \ss

\redefaccent \'
\redefaccent \v
\redefaccent \"
\redefaccent \^
\redefaccent \r

% finally, we can forbid the encTeX primitives in document
% (this is commented out here because it is only an example):

% \let\xordcode=\undefined \let\xchrcode=\undefined\let
% \let\xprncode=\undefined
% \let\mubytein=\undefined \let\mubyteout=\undefined
% \let\mubyte=\undefined   \let\endmubyte=\undefined

\endinput

