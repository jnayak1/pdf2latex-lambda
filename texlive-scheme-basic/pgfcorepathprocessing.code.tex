% Copyright 2006 by Till Tantau
%
% This file may be distributed and/or modified
%
% 1. under the LaTeX Project Public License and/or
% 2. under the GNU Public License.
%
% See the file doc/generic/pgf/licenses/LICENSE for more details.

\ProvidesFileRCS $Header: /cvsroot/pgf/pgf/generic/pgf/basiclayer/pgfcorepathprocessing.code.tex,v 1.8 2008/10/09 16:46:56 tantau Exp $





% Extract start and end vectors of a path
%
% #1 = a macro that stores a path
%
% Description:
%
% This command will extract the first and second coordinate on the
% path as well as the last and second-last. These coordinates will be
% put in the pgfpoint-macros \pgfpointfirstonpath,
% \pgfpointsecondonpath, \pgfpointsecondlastonpath, and
% \pgfpointlastonpath. If the path is empty, these macros are set to
% \pgfpointorigin. 
%
% Example:
%
% \pgfsyssoftpath@moveto{0}{0}
% \pgfsyssoftpath@lineto{10}{10}
% \pgfsyssoftpath@curveto{20}{20}{30}{30}{40}{40}
% \pgfsyssoftpath@getcurrentpath\mypath
% \pgfprocesspathextractpoints\mypath
% 
% % \pgfpointfirstonpath is now \pgfpoint{0}{0}
% % \pgfpointsecondonpath is now \pgfpoint{10bp}{10bp}
% % \pgfpointsecondlastonpath is now \pgfpoint{30bp}{30bp}
% % \pgfpointlastonpath is now \pgfpoint{40bp}{40bp}

\def\pgfprocesspathextractpoints#1{%
  \ifx#1\pgfutil@empty%
    \let\pgfpointfirstonpath=\pgfpointorigin%
    \let\pgfpointsecondonpath=\pgfpointorigin%
    \let\pgfpointsecondlastonpath=\pgfpointorigin%
    \let\pgfpointlastonpath=\pgfpointorigin%
  \else%
    \expandafter\pgf@extractprocessorfirst#1\pgf@stop%
  \fi%
}

\def\pgf@extractprocessorfirst#1#2#3#4{%
  \def\pgfpointfirstonpath{\pgfqpoint{#2}{#3}}%
  \ifx#4\pgf@stop%
    % pretty short path...
    \let\pgfpointsecondonpath=\pgfpointfirstonpath%
    \let\pgfpointsecondlastonpath=\pgfpointfirstonpath%
    \let\pgfpointlastonpath=\pgfpointfirstonpath%
    \let\next=\pgfutil@gobble%
  \else
    \let\next=\pgf@extractprocessorsecond%
  \fi%
  \next#4%
}

\def\pgf@extractprocessorsecond#1{%
  \ifx#1\pgfsyssoftpath@movetotoken%
    % Reset things...
    \let\pgf@next=\pgf@extractprocessorfirst%
  \else%
    \let\pgf@next=\pgf@@extractprocessorsecond%
  \fi%
  \pgf@next#1%
}
\def\pgf@@extractprocessorsecond#1#2#3#4{%
  \def\pgfpointsecondonpath{\pgfqpoint{#2}{#3}}%
  \let\pgfpointsecondlastonpath=\pgfpointfirstonpath%
  \let\pgfpointlastonpath=\pgfpointsecondonpath%
  \ifx#4\pgf@stop%
    % short path...
    \let\next=\pgfutil@gobble%
  \else
    \let\next=\pgf@extractprocessorother%
  \fi%
  \next#4%
}

\def\pgf@extractprocessorother#1{%
  \ifx#1\pgfsyssoftpath@movetotoken%
    % Reset things...
    \let\pgf@next=\pgf@extractprocessorfirst%
  \else%
    \let\pgf@next=\pgf@@extractprocessorother%
  \fi%
  \pgf@next#1%
}
\def\pgf@@extractprocessorother#1#2#3#4{%
  \let\pgfpointsecondlastonpath=\pgfpointlastonpath%
  \def\pgfpointlastonpath{\pgfqpoint{#2}{#3}}%
  \ifx#4\pgf@stop%
    % end!
    \let\next=\pgfutil@gobble%
  \fi%
  \next#4%
}



% Check for closepath and rectangles
%
% #1 = a macro that stores a path
% #2 = command to be executed when a closepath or a rectangle is encountered
%
% Example:
%
% \pgfsyssoftpath@moveto{0}{0}
% \pgfsyssoftpath@lineto{10}{10}
% \pgfsyssoftpath@curveto{20}{20}{30}{30}{40}{40}
% \pgfsyssoftpath@getcurrentpath\mypath
% \pgfprocesscheckclosed{\mypath}{\pgf@drawarrowsfalse}

\def\pgfprocesscheckclosed#1#2{%
  {%
    \global\let\pgf@proc@todo=\relax%
    \let\pgfsyssoftpath@movetotoken=\pgfutil@gobbletwo%
    \let\pgfsyssoftpath@linetotoken=\pgfutil@gobbletwo%
    \let\pgfsyssoftpath@curvetosupportatoken=\pgfutil@gobbletwo%
    \let\pgfsyssoftpath@curvetosupportbtoken=\pgfutil@gobbletwo%
    \let\pgfsyssoftpath@curvetotoken=\pgfutil@gobbletwo%
    \def\pgfsyssoftpath@rectcornertoken{\gdef\pgf@proc@todo{#2}\pgfutil@gobbletwo}%
    \def\pgfsyssoftpath@rectsizetoken{\gdef\pgf@proc@todo{#2}\pgfutil@gobbletwo}%
    \def\pgfsyssoftpath@closepathtoken{\gdef\pgf@proc@todo{#2}\pgfutil@gobbletwo}%
    #1%
  }%
  \pgf@proc@todo%
}

  
% Resolve specialround
%
% #1 = a macro that stores a path
% #2 = a macro that stores the new path
%
% Description:
%
% Resolves all specialround tokens. When such a token is encountered,
% the objective is to replace the next corner by a rounded corner. 
%

\def\pgfprocessround#1#2{%
  % first, check whether #1 contains \pgfsyssoftpath@specialroundtoken.
  % if not, do quick assignment!
  \expandafter\pgf@processroundcheck\expandafter#1\expandafter#2#1\pgfsyssoftpath@specialroundtoken\pgf@stop%
}

\def\pgf@processroundcheck#1#2#3\pgfsyssoftpath@specialroundtoken#4\pgf@stop{%
  \def\pgf@test{#4}%
  \ifx\pgf@test\pgfutil@empty%
    % ok, nothing to be done!
    \let#2=#1%
  \else%
    % very well...
    \pgf@@processround#1#2%
  \fi%
}


\def\pgf@@processround#1#2{%  
  \let\pgfprocess@newpath\pgfutil@empty%
  \let\pgfprocess@moveto\pgfutil@empty%
  \let\pgfprocess@firstto\pgfutil@empty%
  \let\pgfprocess@segment\pgfutil@empty%
  \pgf@xa=0pt%
  \pgf@ya=0pt%
  \expandafter\pgfprocess@round#1\pgf@stop{0pt}{0pt}%
  \expandafter\pgfutil@g@addto@macro\expandafter\pgfprocess@newpath\expandafter{\pgfprocess@moveto}%
  \expandafter\pgfutil@g@addto@macro\expandafter\pgfprocess@newpath\expandafter{\pgfprocess@segment}%
  \let#2=\pgfprocess@newpath%
}

\def\pgfprocess@round#1#2#3{%
  \ifx#1\pgf@stop%
    \let\next=\relax%
  \else%
    \ifx#1\pgfsyssoftpath@specialroundtoken%
      \def\pgfprocess@savex{#2}%
      \def\pgfprocess@savey{#3}%
      \let\next=\pgfprocess@specialround%
      \ifdim#2=0pt%
        \ifdim#3=0pt%
          \let\next=\pgfprocess@round% skip!
        \fi%
      \fi%
    \else%
      \ifx\pgfprocess@firstto\pgfutil@empty%
        \def\pgfprocess@firstto{{#2}{#3}}%
      \fi%
      \pgf@xa=#2%
      \pgf@ya=#3%
      \ifx#1\pgfsyssoftpath@movetotoken%
        % ok, move old segment to newpath
        \expandafter\pgfutil@g@addto@macro\expandafter\pgfprocess@newpath\expandafter{\pgfprocess@moveto}%
        \expandafter\pgfutil@g@addto@macro\expandafter\pgfprocess@newpath\expandafter{\pgfprocess@segment}%
        \def\pgfprocess@moveto{#1{#2}{#3}}%
        \let\pgfprocess@firstto=\pgfutil@empty%
        \let\pgfprocess@segment=\pgfutil@empty%
      \else%
        \pgfutil@g@addto@macro\pgfprocess@segment{#1{#2}{#3}}%
      \fi%
      \let\next=\pgfprocess@round%
    \fi%
  \fi%
  \next%  
}

\def\pgfprocess@specialround#1#2#3{%
  \ifx\pgfprocess@firstto\pgfutil@empty%
    \def\pgfprocess@firstto{{#2}{#3}}%
  \fi%
  \let\next=\pgfprocess@round% default: do nothing
  \ifx#1\pgfsyssoftpath@linetotoken%
    % Ok, round a lineto
    \let\next=\pgfprocess@specialroundcope%
  \else%
    \ifx#1\pgfsyssoftpath@curvetosupportatoken%
      % Ok, round a curveto. 
      \let\next=\pgfprocess@specialroundcurveto%
    \else%
      \ifx#1\pgfsyssoftpath@closepathtoken%
        % Ok, round a closepath (this is tricky...)
        \let\next=\pgfprocess@specialroundclose%
      \fi%
    \fi%
  \fi%
  \next#1{#2}{#3}%
}%

\def\pgfprocess@specialroundcope#1#2#3#4#5#6{%
  \let\next=\pgfprocess@round% do nothing by default
  \let\pgfprocess@continueafterrounding=\pgfprocess@round%
  \ifx#4\pgfsyssoftpath@linetotoken%
    % ok, let's get kicking...
    \pgf@xc=#5%
    \pgf@yc=#6%
    \let\next=\pgfprocess@@specialround%
  \else%
    \ifx#4\pgfsyssoftpath@curvetosupportatoken%  
      \pgf@xc=#5%
      \pgf@yc=#6%
      \let\next=\pgfprocess@@specialround%
    \else%
      \ifx#4\pgfsyssoftpath@closepathtoken%  
        \pgf@xc=#5%
        \pgf@yc=#6%
        \let\next=\pgfprocess@@specialround%
      \else
        \ifx#4\pgfsyssoftpath@specialroundtoken%  
          \let\next=\pgfprocess@@findrightcoordinates%
        \fi%
      \fi%
    \fi%
  \fi%
  \next#1{#2}{#3}#4{#5}{#6}%
}

\def\pgfprocess@@findrightcoordinates#1#2#3#4#5#6#7#8#9{%
  \pgf@xc=#8%
  \pgf@yc=#9%
  \pgfprocess@@specialround#1{#2}{#3}#4{#5}{#6}#7{#8}{#9}%
}

\def\pgfprocess@@specialround#1#2#3{%
  \pgfutil@g@addto@macro\pgfprocess@segment{#1}%
  % calculate start coordinate of the curve
  \pgf@xb=#2%
  \pgf@yb=#3%
  %
  \edef\pgf@marshal%
  {\noexpand\pgfpointlineatdistance{\pgfprocess@savex}%
    {\noexpand\pgfqpoint{\the\pgf@xb}{\the\pgf@yb}}%
    {\noexpand\pgfqpoint{\the\pgf@xa}{\the\pgf@ya}}}%
  \pgf@process{\pgf@marshal}%
  \pgf@xa=\pgf@x% save start point of added curve
  \pgf@ya=\pgf@y%
  \edef\pgfprocess@addition{{\the\pgf@x}{\the\pgf@y}}%
  \expandafter\pgfutil@g@addto@macro\expandafter\pgfprocess@segment\expandafter{\pgfprocess@addition}%
  %
  % now add curve
  %
  \edef\pgf@marshal%
  {\noexpand\pgfpointlineatdistance{\pgfprocess@savey}%
    {\noexpand\pgfqpoint{\the\pgf@xb}{\the\pgf@yb}}%
    {\noexpand\pgfqpoint{\the\pgf@xc}{\the\pgf@yc}}}%
  \pgf@process{\pgf@marshal}%
  \pgf@xc=\pgf@x% save end point of added curve
  \pgf@yc=\pgf@y%
  % calculate first support:
  {%
    \pgf@x=\pgf@xb%
    \pgf@y=\pgf@yb%
    \advance\pgf@x by-\pgf@xa%
    \advance\pgf@y by-\pgf@ya%
    \advance\pgf@xa by 0.5522847\pgf@x%
    \advance\pgf@ya by 0.5522847\pgf@y%
    \edef\pgfprocess@addition{\noexpand\pgfsyssoftpath@curvetosupportatoken{\the\pgf@xa}{\the\pgf@ya}}%
    \expandafter\pgfutil@g@addto@macro\expandafter\pgfprocess@segment\expandafter{\pgfprocess@addition}%
  }%
  % calculate second support:
  {%
    \pgf@x=\pgf@xb%
    \pgf@y=\pgf@yb%
    \advance\pgf@x by-\pgf@xc%
    \advance\pgf@y by-\pgf@yc%
    \advance\pgf@xc by 0.5522847\pgf@x%
    \advance\pgf@yc by 0.5522847\pgf@y%
    \edef\pgfprocess@addition{\noexpand\pgfsyssoftpath@curvetosupportbtoken{\the\pgf@xc}{\the\pgf@yc}}%
    \expandafter\pgfutil@g@addto@macro\expandafter\pgfprocess@segment\expandafter{\pgfprocess@addition}%
  }%
  % add end:
  \edef\pgfprocess@addition{\noexpand\pgfsyssoftpath@curvetotoken{\the\pgf@xc}{\the\pgf@yc}}%
  \expandafter\pgfutil@g@addto@macro\expandafter\pgfprocess@segment\expandafter{\pgfprocess@addition}%
  %
  % continue!
  %
  \pgf@xa=#2%
  \pgf@ya=#3%
  \pgfprocess@continueafterrounding%
}

\def\pgfprocess@specialroundcurveto#1#2#3#4#5#6{%
  \pgfutil@g@addto@macro\pgfprocess@segment{#1{#2}{#3}#4{#5}{#6}}% supports
  \pgf@xa=#5%
  \pgf@ya=#6%
  \pgfprocess@specialroundcope}

\def\pgfprocess@specialroundclose{%
  \let\next=\pgfprocess@round%
  \ifx\pgfprocess@moveto\pgfutil@empty%
  \else%
    \ifx\pgfprocess@firstto\pgfutil@empty%
    \else%
      \expandafter\pgfprocess@@specialclosetarget\pgfprocess@firstto%
      \let\pgfprocess@continueafterrounding=\pgfprocess@cleanupafterclose%
      \edef\next{%
        \noexpand\pgfprocess@@specialround%
        \expandafter\expandafter\expandafter\noexpand\expandafter\pgfprocess@@specialclosemoveto\pgfprocess@moveto%
      }%
    \fi%
  \fi%  
  \next%
}

\def\pgfprocess@@specialclosetarget#1#2{%
  \pgf@xc=#1%
  \pgf@yc=#2%
}

\def\pgfprocess@@specialclosemoveto#1#2#3{%
  \pgfsyssoftpath@linetotoken{#2}{#3}%
}


\def\pgfprocess@cleanupafterclose#1#2#3{%
  \edef\pgfprocess@moveto{\noexpand\pgfsyssoftpath@movetotoken{\the\pgf@xc}{\the\pgf@yc}}%
  \edef\pgf@marshal{\noexpand\pgfprocess@round\noexpand\pgfsyssoftpath@closepathtoken{\the\pgf@xc}{\the\pgf@yc}}%
  \pgf@marshal%  
}


% Replace start and end coordinates of a path
%
% #1 = new start coordinate
% #2 = new end coordinate
%
% Description:
%
% This command will replace the first and last coordinates on a path
% by the given coordinates.
%
% Example:
%
% \pgfsyssoftpath@moveto{0}{0}
% \pgfsyssoftpath@lineto{10}{10}
% \pgfsyssoftpath@curveto{20}{20}{30}{10}{40}{0}
% \pgfsyssoftpath@getcurrentpath\mypath
% \pgfsetlinewidth{2pt}
% \pgfusepath{stroke}
% \pgfprocesspathextractpoints{\mypath}
% \pgfprocesspathreplacestartandend{\mypath}
%   {\pgfpointlineatdistance{2pt}{\pgfpointfirstonpath}{\pgfpointsecondonpath}}%
%   {\pgfpointlineatdistance{2pt}{\pgfpointlastonpath}{\pgfpointsecondlastonpath}}%
% \pgfsyssoftpath@setcurrentpath\mypath
% \pgfsetlinewidth{1pt}
% \color{red}
% \pgfusepath{stroke}

\def\pgfprocesspathreplacestartandend#1#2#3{%
  \ifx#1\pgfutil@empty%
  \else%
  {%
    % We (mis)use the current path to store (in a buffered manner) the
    % rest path.
    \pgfsyssoftpath@getcurrentpath\pgf@replace@origpath%
    \let\pgf@prepath=\pgfutil@empty%
    \pgf@process{#3}%
    \pgf@xa=\pgf@x%
    \pgf@ya=\pgf@y%
    \pgf@process{#2}%
    \expandafter\pgf@replaceprocessfirst#1\pgf@stop%
    \pgfsyssoftpath@getcurrentpath\pgf@restpath%
    \expandafter\expandafter\expandafter\expandafter\expandafter\expandafter\expandafter\gdef%
    \expandafter\expandafter\expandafter\expandafter\expandafter\expandafter\expandafter#1%
    \expandafter\expandafter\expandafter\expandafter\expandafter\expandafter\expandafter{%
      \expandafter\expandafter\expandafter\pgf@prepath%
      \expandafter\pgf@newstart\pgf@restpath}%
    % Restore current path
    \pgfsyssoftpath@setcurrentpath\pgf@replace@origpath%
  }%
  \fi%
}

\def\pgf@replaceprocessfirst#1#2#3#4{%
  \def\pgf@origstart{#1{#2}{#3}}%
  \edef\pgf@newstart{\noexpand#1{\the\pgf@x}{\the\pgf@y}}%
  \pgfsyssoftpath@setcurrentpath\pgfutil@empty%
  \ifx#4\pgf@stop%
    \let\pgf@next=\pgfutil@gobble%
  \else%
    \let\pgf@next=\pgf@replaceprocessother%
    \let\pgf@@next=\pgf@@replaceprocessother%
  \fi%
  \pgf@next#4%
}

\def\pgf@replaceprocessother#1{%
  \ifx#1\pgfsyssoftpath@movetotoken%
    \pgf@replace@handle@move%
  \fi%
  \pgf@@next#1%
}
\def\pgf@@replaceprocessother#1#2#3#4{%
  \ifx#4\pgf@stop%
    \edef\pgf@restpathaddition{{\noexpand#1{\the\pgf@xa}{\the\pgf@ya}}}%
    \expandafter\pgfsyssoftpath@addtocurrentpath\pgf@restpathaddition%
    \let\pgf@next=\pgfutil@gobble%
  \else%
    \pgfsyssoftpath@addtocurrentpath{#1{#2}{#3}}%
  \fi%
  \pgf@next#4%
}  

\def\pgf@replace@handle@move{%
  \pgfsyssoftpath@getcurrentpath\pgf@restpath%
  % Arggghh! Must reset everything!
  \expandafter\expandafter\expandafter\expandafter\expandafter\expandafter\expandafter\gdef%
  \expandafter\expandafter\expandafter\expandafter\expandafter\expandafter\expandafter\pgf@prepath%
  \expandafter\expandafter\expandafter\expandafter\expandafter\expandafter\expandafter{%
    \expandafter\expandafter\expandafter\pgf@prepath%
    \expandafter\pgf@origstart\pgf@restpath}%
  % Redo first.
  \let\pgf@@next=\pgf@replaceprocessfirst%
}
  
\endinput
